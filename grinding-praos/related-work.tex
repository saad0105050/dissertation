\section{Related work}\label{sec:related-work}


Below, we survey the coin-flipping and beacon literature in more detail.

\paragraph{Collective coin-flipping as a Boolean function.} 
The (single round) collective coin-flipping in the full information
model is a classic problem introduced in the seminal work
of~\citet{BL85}.  In this problem, $n$-players communicate over a
single broadcast channel.  First, each honest player submits a
uniformly random bit, and then each adversarial player (strategically)
submits an arbitrary bit.  The output of the protocol is a single bit,
computed as a Boolean function $F$ applied to the submitted bits.  If
$F$ is the majority function, an adversarial coalition of size
$O(\sqrt{n})$ can arbitrarily influence the output of the
protocol~\cite{BL85}.  Later,~\citet*{KKL} showed that for
\emph{every} Boolean function $F$ in this setting, there is an
adversarial coalition of size $O(n/(\log n))$ that can bias the
output.  It was conjectured by~\citet{Friedgut} that Boolean functions
on the continuous cube $[0,1]^n$ (or equivalently, the discrete cube
$[n]^n$) can be biased by a coalition of size $o(n)$.
See~\cite{Hatami} and~\cite{Kalai} for advances on this line of
research.  As for multi-round coin-flipping protocols, \citet*{RSZ}
showed that any $o(\log^* n)$-round coin-flipping protocol can be
biased by a coalition of size $o(n)$.

Cryptography is a powerful tool for collective coin-flipping. While the most directly relevant related work is AlgoRand---which uses a mechanism similar to ours---we survey some other work first to set the stage. 

\paragraph{PVSS-based beacons.} 
Publicly verifiable secret sharing (PVSS) techniques have been
successfully used to design unbiasable, multi-round coin-flipping
protocols, such as Ouroboros~\cite{Ouroboros},
RandHound~\cite{RandHound}, Scrape~\cite{Scrape}, and
HydRand~\cite{HydRand}.  The coin-flipping in these protocols have two
(logical) rounds: first, a round where the players broadcast
commitments to their inputs, followed by a round where these
commitments are revealed. An advantage of these protocols is that they
can provide extremely strong guarantees on the quality of the
randomness---when the protocols are secure, they produce output values
that are indistinguishable from uniform. A difficulty is that they
demand $O(n^2)$ messages to be broadcast, which is intractable in
practice. A standard technique to mitigate this is to first elect a small
committee of players which then carry out the full PVSS. Of course,
such a procedure is immediately subject to attacks by an adaptive
adversary who can corrupt the committee once it is determined.

\paragraph{VDF-based beacons.} 
Verifiable Delay Functions (VDF)
\cite{vdf-boneh,vdf-wesolowski,vdf-pietrzak} may be used to construct
unpredictable, multi-round coin-flipping protocols (for a sketch of
such a possible construction see~\cite{vdf-ethereum}). Using VDFs in
this context is orthogonal to our techniques: a VDF can delay the
revelation of the beacon outcome to temper the opportunities the
adversary has to grind the beacon output. In general, tuning the
hardness parameter of the VDF to a high level aids this security
objective but naturally interferes with the availability (or liveness)
of the beacon.
% (in the sense of the immediate availability of the
% beacon outputs).
For this reason, a fine-grained analysis of the
randomness properties of the beacon is a precondition to the effective
use of a VDF in a practical beacon algorithm. Moreover, employing a
VDF will require a less cryptographic assumption (moderate hardness).
We leave the combination of VDF techniques with our randomness
generation algorithm as an interesting future research direction.

\paragraph{Threshold signatures.}
DFINITY~\cite{Dfinity} uses a one-round coin-flipping protocol based
on non-interactive $(t,n)$-Threshold BLS signatures~\cite{BLS}. 
In the setup phase, a random subset of $n$
players---a ``group''---is selected from a universe of $N$
players  
and 
key-pairs for the group members are established. 
After the setup, 
each player broadcasts his share 
and any player can
recover the (unique) beacon output from any $t$ shares.
%If there are $f$ adversarial players, taking
%$t \in [f + 1, n-f]$ ensures that the adversarial players cannot
%prevent a recovery and, importantly, they cannot predict the beacon
%output.  As for the iterated beacon, a random group is selected at
%each epoch from a family of groups configured during the setup phase.
%
%Note that this beacon is secure only against static
%adversaries.  Specifically, $n$ depends on the size of the adversarial
%coalition during the setup phase.  This is necessary: since groups are
%pre-configured, an adaptive adversary can enlarge his coalition over
%time and corrupt the group majority 
% violate the constraint $t \in [f+1, n-f]$ (by increasing $f$)
%at a future round with a constant probability. In contrast, our
%protocol is secure as long as the adversary controls less than $N/2$
%players.  In addition, our protocol is simpler: there is no committee
%or group selection.
The main disadvantage with respect to our techniques is of course
the complexity of the group setup which requires $\Omega(n^2)$ communication
for a single distributed key generation process between $n$ players.

\paragraph{Algorand.} 
Algorand~\cite{Algorand} uses a Byzantine Agreement-based multi-round
coin-flipping protocol where the players are equipped with verifiable
random functions (VRF).  While the multi-round beacon output is
uniform except with a negligible probability, the beacon output of a
single-round is uniform with only constant probability---thus
conditional min entopy is small. As mentioned above, our protocol can
be viewed as a parallelization of the AlgoRand technique; indeed, a
single-round of AlgoRand coin-flipping is equivalent to setting
$k = 1$ in our protocol.  In contrast, our one-round protocol
guarantees a ``small'' loss in min-entropy except with a negligible
probability.


\paragraph{VRF-based eventual consensus Proof-of-Stake blockchains.}
Many proof-of-stake blockchain
protocols, such as Ouroboros Praos~\cite{Praos}, 
Genesis~\cite{Genesis}, and Snow White~\cite {SnowWhite}, 
use VRFs 
in their coin-flipping protocols. 
This is a multi-round protocol 
where each ``block'' contains a nonce.  
The coin-flipping output is a cryptographic hash 
of the XOR of the nonces 
recorded in the ``common prefix'' of all 
blockchains held by the honest players 
at the end of the epoch. 
(Thus, these protocols do not use a broadcast channel.)
As we show in Appendix~\ref{sec:praos}, 
the min-entropy loss for this beacon 
is linear in the security parameter 
for an adversarial stake $10\%$ or higher.
Our treatment of this beacon 
is a first in the literature. 

\paragraph{Nakamoto-style PoW beacons.} 
\citet{ZuckermanBeacon} establish a lower bound for a class of
Nakamoto-style proof-of-work based beacons and discuss approaches for
a game-theoretic analysis of beacon protocols.
