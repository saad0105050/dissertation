
\section{An expression for the grinding power} 
Let $w = xy$ be a characteristic string. 
Define
\begin{equation}\label{eq:S_y}
    % S(d, z) 
    % \defeq \sum_{i \geq z}^{d - z}{ \binom{d - z}{i} } 
    S(y) 
    \defeq \sum_{i \geq \#_\h(y)}^{\#_\A(y)}{ \binom{\#_\A(y)}{i} } 
    % \,.
\end{equation}
and notice that it is the number of viable blockchains $\Chain$
in an execution on $w$ 
so that the last honest block in $\Chain$ occurs in slot $|x|$. 
By summing over all honest slots, 
the total number of viable blockchains for $w = w_1 w_2 \cdots w_n$,
i.e., the grinding power for $w$, is 
\begin{equation}\label{eq:g_praos}
    % g(w) 
    % = \sum_{\substack{
    %     x,y : w = xy, \\ 
    %     |x| \geq 0, \text{and}\\ 
    %     \text{slot $|x|$ is honest} }
    %   } w_{|x|}\, S(y)
    % \,.  
    g(w_1 w_2 \cdots w_n) 
    = \sum_{ \text{$t \in [n] : w_t$ is honest} }
      w_t\, S(w_{t+1} \cdots w_n)
    \,.  
\end{equation}


In our exposition, a special role will be played by
the number of characteristic strings, 
of length $d$, containing $z$ honest indices.
For the sake of convenience, 
we overload the function $S(\cdot)$ as 
\begin{equation}\label{eq:S_d_z}
  S(d, z) 
    \defeq \sum_{i \geq z}^{d-z}{ \binom{d-z}{i} } 
    \quad\text{and}\quad
  S(d) \defeq \sum_{z = 0}^{d/2} S(d,z)
\end{equation}
to denote this number.

% \paragraph{Asymptotic bounds on $S(d,z)$.} 
When $z \leq d/3$, the sum $S(d,z)$ in~\eqref{eq:S_d_z} 
will contain the central binomial coefficient $${d-z \choose (d-z)/2} = \Theta(2^{d-z}/\sqrt{d-z})\,.$$ 
It follows that 
\begin{equation} \label{eq:S_d_z_small}
    2^{d-z-1} 
    \leq 
    S(d,z)
    \leq 2^{d-z}\,, 
    \qquad \text{if}\quad 0 \leq z \leq d/3 
    \,.
\end{equation}
On the other hand, for any $0 \leq t < N/2$
\begin{align}\label{eq:binomial-sum-bound}
\frac{2^{H(t/N)N} }{\sqrt{8 (t/N) (1-t/N)} }
\leq 
\sum_{k=0}^t{ {N \choose k} } 
&\leq 
2^{H(t/N)N} \, , 
\end{align}
where $H : [0, 1] \rightarrow [0, 1]$ is the binary entropy function defined as
\begin{align*}
    H(0) &= H(1) = 0\, , \\
    H(\alpha) &= -\alpha \log_2(\alpha) - (1-\alpha) \log_2(1-\alpha)\,, 
        \qquad \text{if}\quad 0 < \alpha < 1\, .
\end{align*}
Thus we can bound $S(d,z)$ using (\ref{eq:binomial-sum-bound}) as follows:
\begin{equation} \label{eq:S_d_z_large}
    \frac{2^{(d-z)H(z/(d-z))}(d-z)}{\sqrt{8 z(d-2z) }}
    \leq
    S(d, z) 
    \leq
    2^{(d-z)H(z/(d-z))}
    \,,\qquad \text{if} \quad d/3 < z \leq d/2 
    \,.
\end{equation}

Let $\alpha \in [0, 0.5), n \in \NN$, and $W \sim \mathcal{L}_n(\alpha)$.
We will derive tail bounds on $g(W)$ by bounding the second moment of $S(d,z)$ 
for $d \in [n]$ and $z \in [d]$.




\section{The Moment Generating Function of the grinding power}
Let $\epsilon \in (0, 1)$ and write $\alpha = (1 - \epsilon)/2$, and 
think of $\epsilon$ as the ``adversarial bias.'' 
Let $W \sim \mathcal{L}_n(\alpha)$ and 
Define 
\begin{align}\label{eq:suffix-prob-binomial}
    B_{n, \epsilon}(k) 
    &= \Pr[ \#_\A(W_1 \ldots W_n) = k ]  
    = 2^{-n}\binom{n}{k} (1-\epsilon)^k (1 + \epsilon)^{n-k} 
    =  \binom{n}{k} 
        \left(\frac{1-\epsilon}{2}\right)^n 
        \left(\frac{1 + \epsilon}{1-\epsilon} \right)^{n-k} 
        \,.
\end{align}


Recall that for any characteristic string $w = w_1 w_2 \ldots$, 
we have $w_0 = 1$ by convention. 
Let $\lambda$ be a positive real, $\lambda \geq 1$. 
% (For our application, we would only use $\lamba = 2$.)
We can express the $\lambda$th moment of the grinding power as
\begin{align}\label{eq:grinding-power-moment-praos}
    \Exp\, g(W)^\lambda 
    &= \sum_{t = 0}^n \Pr[W_t \neq \A] \cdot \Exp (W_t\, S(W_{t+1}\ldots W_n))^\lambda \nonumber \\
    &= \sum_{t = 0}^n
    \sum_{k \geq 1} 
      \Pr[W_{t} = k] \Pr[\text{$W_{t+1}\ldots W_n$ is $\Aheavy$}]\cdot (k S(n - t) )^\lambda \nonumber \\
    &= A(\epsilon, \lambda) 
      \sum_{t = 0}^n 
      \Pr[\text{$W_{t+1}\ldots W_n$ is $\Aheavy$}]\cdot S(n-t)^\lambda \nonumber \\
    &=  %\Pr[\text{$W_1$ is honest}]\,
        A(\epsilon, \lambda)
        \cdot 
        \sum_{d=1}^n \sum_{z = 0}^{ d/2 } B_{d, \epsilon}(d - z) S(d,z)^\lambda
\end{align}
where 
\begin{equation}
  A(\epsilon, \lambda) \defeq \sum_{k\geq 1} \Pr[W_1 = k] \cdot k^\lambda
  % \,,
  \label{eq:A}
\end{equation}
and, by convention, 
the inner sum in~\eqref{eq:grinding-power-moment-praos} 
runs for $z = 0, 1, \ldots, \lfloor d/2 \rfloor$.

The first equality in the derivation of~\eqref{eq:grinding-power-moment-praos} follows from 
% the subadditivity of the convex function 
% $x \mapsto x^\lambda$ for $\lambda \geq 1, x \geq 0$,  
~\eqref{eq:g_praos}. 
The second equality follows 
since only $\Aheavy$ suffixes of $w$ contribute to $g(w)$ 
and that the symbols in $W$ are independent. 
The third equality follows from rearranging the sums. 
The last equality follows from~\eqref{eq:suffix-prob-binomial} 
and~\eqref{eq:S_d_z}.

Our goal \footnote{
It turns out that the the second moment of $g(W)$ grows 
much faster than its first moment. In fact, 
Chebyshev's inequality on $g(W)$ fares only marginally better 
than what we get via Markov's inequality on $g(W)^2$. 
% Thus we defer the analysis of the first moment in 
% % Appendix~\ref{app:grinding-praos}.
% the full version~\cite{GrindingFullVersion}.
}
is to develop a tight upper bound on $\Exp g(W)^2$ 
which, when used in Markov's inequality, 
would give us a tail bound on $g(W)$. 

\begin{lemma}[Tail bound on $g$]\label{lemma:praos-tail-gamma}
  Let $\epsilon, \EpsP \in (0, 1)$ and $n, k, s \in \NN$. 
  Let $W \sim \mathcal{L}_n((1-\epsilon)/2)$.
  % Let $m(W)$ denote the right-hand side of~\eqref{eq:grinding-praos-second-moment}. 
  Let $E$ be the event that 
  $\Blockchain$ remains $(\EpsP, k, s)$-secure during 
  the $n$ slots represented by $W$. 
  Let $m(W)$ be an upper bound on $\Exp[ g(W)^2 \mid E]$.
  Let $\gamma = (m(W)/\EpsP)^{1/2}$.
  Then 
  $
      \Pr\left[g(W) \geq \gamma\right] \leq 2 \EpsP
      % \,.
  $.
\end{lemma}
\begin{proof}
  % Recall the event $E$ defined in Lemma~\ref{lemma:grinding-praos-second-moment}.
  Let $\gamma \in \NN$. 
  Since $g(W)$ is non-negative, 
  $$
    % p_\gamma 
    \Pr[g(W) \geq \gamma]
    \leq (1 - \Pr[E]) + \Pr[g(W)^2 \geq \gamma^2 \mid E] 
    \leq \EpsP + m(W)/\gamma^2
  $$ 
  by Markov's inequality on the non-negative random variable $g(W)^2$. 
  We set $\gamma$ to be large enough so that the above quantity is at most $2 \EpsP$. 
  Equivalently, 
  $m(W)/\gamma^2$ would be at most $\EpsP$, 
  or $\gamma$ is at least $\sqrt{m(W)/\EpsP}$. 
\end{proof}


\section{The second moment of \texorpdfstring{$g(W)$}{the grinding power}}

We begin by stating three helpful claims 
whose proofs will be presented later.

\begin{claim}\label{claim:multiple-honest-blocks}
  For $\epsilon \in (0, 1)$, 
  $
    A(\epsilon, 2) 
    % \leq 
    % \frac{(1 + \epsilon)^2 (5 + \epsilon)}{16 \left( 1 - e^{(1 + \epsilon)/2} \right)}
    % \leq 2^{- 0.3 + 1.85 \epsilon}
    % \leq 2^{- 1.3 + 0.41 \epsilon}
    % \leq 1
    \leq 0.8 (1 + 2 \epsilon)
    \,.
  $
\end{claim}


% $\Exp H_m^2 \leq \Exp \GeomOneMinusAlpha^2 = (2 - \alpha)/\alpha^2$. 
% The last equality is a known fact; see~\eqref{eq:geom-moments-1-2} 
% which gives the second moment of $\GeomAlpha$. 
% Therefore,
% \begin{align}
%   \sum_{k\geq 1} \Pr[W_1 = k]\, k^\lambda 
%   &\leq \frac{1 - \alpha}{1 - e^{-(1 - \alpha)}} \frac{2 - \alpha}{\alpha^2} \nonumber \\
%   % &= \frac{1+\epsilon}{2} \frac{3 + \epsilon}{2} \frac{4}{(1 - \epsilon)^2}
%   &= \frac{(1+\epsilon)(3 + \epsilon)}{(1 - \epsilon)^2\left( 1 - e^{-(1 + \epsilon)/2} \right)}
%   \,.
% \end{align}



\begin{claim}\label{claim:t1star-variance-exact}
  Let $\epsilon \in (0,1)$ and 
  let $d,z$ be non-negative integers where $d \geq 2$ and $z \leq d/3$. 
  \begin{align*}
    B_{d, \epsilon}(d-z) S(d,z)^2
    &\leq \begin{dcases} 
    \left( (5-3 \epsilon)/2 \right)^d \,,
        &\quad \text{if}\quad
        0 < \epsilon < 1/3 \,, \\
    \left( 2^{2/3} \phi(\epsilon) \right)^d \,,
    %\frac{6}{\sqrt{5 d}}
        &\quad \text{if}\quad
        1/3 \leq \epsilon < 1 \, ,
    \end{dcases}
  \end{align*}
  where
  \begin{align}
    \phi(\epsilon) 
    &\triangleq \frac{3}{2} (1+\epsilon)^{1/3} (1-\epsilon)^{2/3}\label{eq:phi_eps} 
    \,.
  \end{align}
\end{claim}

\begin{claim}\label{claim:t2star-variance-exact}
  Let $\epsilon \in (0,1)$ and 
  let $d,z$ be non-negative integers where $d/3 \leq z \leq d/2$. 
  \begin{align*}
    B_{d, \epsilon}(d-z) S(d,z)^2
    &\leq \begin{cases} 
    \left( 2^{2/3} \phi(\epsilon) \right)^d \,,
        &\quad\text{if}\quad 0 < \epsilon \leq 0.6\,, \\
    \left( (5/3) \phi(\epsilon)  \right)^d \,,
        &\quad\text{if}\quad 0.6\leq \epsilon < 0.81\,, \\
    1 \,,
        &\quad\text{if}\quad 0.81 \leq \epsilon < 1
        \,.
    \end{cases}
  \end{align*}
\end{claim}
% We defer the proofs of these claims in the next subsection .



\begin{lemma}[Second moment of $g$]\label{lemma:grinding-praos-second-moment}
  Let $\epsilon, \EpsP \in (0, 1)$ and $n, k, s \in \NN$. 
  Let $W \sim \mathcal{L}_n((1 - \epsilon)/2)$. 
  Let $E$ denote the event that 
  the blockchain protocol $\Blockchain$ 
  remains $(\EpsP, k, s)$-secure throughout 
  the $n$ slots represented by $W$. 
  Then
  \begin{equation}\label{eq:grinding-praos-second-moment}
    \Exp[ g(W)^2 \mid E] 
    \leq 
      % s^2/2 
      % s^2 
      % 2^{- 1.3 + 1.85 \epsilon} 
      % A(\epsilon,2)
      0.4 s^2 (1 + 2\epsilon)
      \cdot \begin{cases}
      \left( (5-3 \epsilon)/2 \right)^s & \text{if}\quad 0 < \epsilon \leq 1/3\,, \\
      \left( 2^{2/3} \phi(\epsilon) \right)^s & \text{if}\quad 1/3 < \epsilon \leq 3/5\,, \\
      \left( (5/3) \phi(\epsilon) \right)^s & \text{if}\quad 3/5 < \epsilon \leq 0.81\,, \\
      1,  & \text{if}\quad 0.81 < \epsilon \leq 1\,,
    \end{cases}    
  \end{equation}
  where $\phi(\epsilon)$ is defined in (\ref{eq:phi_eps}).
\end{lemma}
\begin{proof}
  Conditioned on the event $E$, 
  every viable blockchain at the end of slot $n$ 
  must contain at least one honestly-generated block from the last $s$ slots. 
  Thus, 
  in ~\eqref{eq:grinding-power-moment-praos}, 
  it suffices to consider the suffixes of $W$ of length at most $s$. 
  The factor $s^2/2$ is an upper bound on 
  the number of terms in the double sum in~\eqref{eq:grinding-power-moment-praos}. 
  Note that the bounds in 
  Claims~\ref{claim:t1star-variance-exact} 
  and~\ref{claim:t2star-variance-exact} are monotone increasing in $d$. 
  Hence we use the upper bounds on $B_{s, \epsilon}(s-z) S(s,z)^2$ for 
  different ranges of $z$ and $\epsilon$. 
  Furthermore, by Claim~\ref{claim:multiple-honest-blocks}, 
  % $s^2/2 \cdot A(\epsilon,2) \leq s^2 2^{-1.3 + 1.85 \epsilon}$.
  $A(\epsilon, 2)$ is at most $0.8 + 1.6 \epsilon$. 
  The lemma follows by observing that $s^2/2 \cdot (0.8 + 1.6 \epsilon) = 0.4 s^2 (1 + 2 \epsilon)$.
  % {\color{blue}Explain the case when the moment is below one.}
\end{proof}




