
\newcommand{\FuncBeacon}{\mathcal{F}_\mathsf{beacon}}
\newcommand{\ProtocolBeacon}{\mathcal{\Pi}_\mathsf{beacon}}
% \newcommand\Litem[1]{\item{\bfseries #1\enspace}}




\section{Collective coin-flipping}

    A \emph{coin-flipping protocol $\Pi$ of dimension $\kappa$} 
    takes, as input, a uniformly random $\kappa$-bit string $\eta_0$
    and outputs a $\kappa$-bit string $\eta$. 
    The goal of the protocol is to make the distribution of $\eta$ 
    as close to uniform as possible.
    
    \begin{definition}[Coin-flipping with $(\epsilon, \rho)$-security]\label{def:coin-flipping-security}
        Let $\kappa \in \NN, \rho, \epsilon \in \RR, \rho \in [0, \kappa]$, and $\epsilon \in (0,1)$.
        Let $\Pi$ be an coin-flipping protocol with dimension $\kappa$ 
        and output $\eta$. 
        We say that \emph{$\Pi$ is $(\epsilon, \rho)$-secure} if 
        $\MinEntropy(\eta) \geq \kappa - \rho$ 
        except with probability $\epsilon$. 
        Here, $\MinEntropy$ is the min-entropy function.
    \end{definition}
    
    Suppose a coin-flipping protocol $\Pi$ is $(\epsilon, \rho)$-secure. 
    Then 
    the min-entropy loss of its output distribution is at most $\rho$. 
    On the other hand, let $T$ be an arbitrary set of $\kappa$-bit Boolean strings. 
    The probability that $\eta \in T$ is at most $2^\rho \cdot |T| \cdot 2^{-\kappa}$. 
    The amplification factor $2^\rho$ can be interpreted as 
    the adversarial participants 
    choosing the designated output $\eta$ from 
    a set containing $2^\rho$ uniformly random strings; 
    {\color{red} strings in this set may be correlated with each other}. 
    An unbiased protocol will necessarily have $\rho = 0$.


    \paragraph{Protocol participants.} 
    Let $\Players$ be the set of protocol participants. 
    With each party $u \in \Players$ is associated a positive real $\sigma_u \in [0, 1]$ 
    so that $\sum_{u \in \Players} \sigma_u = 1$. 

    \paragraph{The adversary.} 
    Let $\alpha \in [0,1]$. 
    The adversary can corrupt any player instantly subject to 
    a budget $\alpha$: that is, 
    at any moment, the set of players $A \subset \Players$ controlled by the adversary 
    always satisfy $\sum_{u \in A} \sigma_u \leq \alpha$. 
    This adversary is called an \emph{$\alpha$-dominated adversary}. 
    If a player is ever corrupted by the adversary, 
    he remains under his control for the rest of the protocol. 
    The adversary is Byzantine: 
    once he controls a player, he can send arbitrary messages on behalf of the player.

    \paragraph{Communication model.} 
    We assume a synchronous communication where 
    the adversary is responsible for delivering messages in any order he chooses.

    \begin{definition}[$(\ell, n)$-coin-flipping with dimension $\kappa$]\label{def:coin-flipping}
        Let $\ell, n, \kappa \in \NN$ and 
        let $\eta_0 \in \{0,1\}^\kappa$ be a uniformly random \emph{initial value}. 
        The protocol has $r = \ell + n$ rounds. 
        \begin{itemize}
            \item (Committees.) 
            With every round $i \in [r]$ is associated 
            a \emph{committee} $C_i$ of players. 
            (The committee election scheme is left unspecified at this moment. 
            See Figure~\ref{fig:leader-election-schemes} below.)
            The committee membership is private unless 
            a member discloses it. 

            \item (Nonces.) 
            Before the protocol commences, 
            for every $i \in [r]$, 
            each member $u \in C_i$ is given a \emph{nonce} $y_{i,u} \in \{0,1\}^\kappa$. 
            The nonces of the honest members are uniformly random. 
            The adversarial members may choose their nonces arbitrarily at this point 
            but once the protocol commences, 
            they cannot change the value of their nonces. 

            \item (Adversarial knowledge and the lookahead.) 
            The adversarial players know, in advance, 
            about all adversarial committe members and 
            their nonces across all rounds.  
            They also know which rounds contain an honest committee member.
            Furthermore, at round $i$, 
            adversarial players can see 
            the honest nonces for rounds $i, i+1, \ldots, \min\{i + \ell - 1, r\}$.

            \item (Announcement.)
            With each round $i$ is associated 
            a publicly visible set $Y_i$, initially empty. 
            At round $i$, 
            an honest committee member inserts his nonce into $Y_i$. 
            An adversarial committee member $u$ may strategically choose to insert or not. 

            \item (Output.)
            At the end of round $r$, 
            let $m_i$ 
            be the lexicographic minimum of all strings recorded in $Y_i, i \in [r]$.
            (If $Y_i$ is empty then set $m_i = 0^\kappa$.)
            The \emph{output} of the protocol is 
            $
                \eta = \eta_0 \oplus m_1 \oplus \cdots m_{\ell + n}
                % \,.
            $. 
        \end{itemize}
    \end{definition}

    Since the adversary cannot change the nonces, 
    all he can do to bias the output 
    is to strategically vary the size of the sets $Y_i, i \in [\ell + n]$.

    \begin{figure}[h]
        \begin{framed}
            \begin{center}
                Two committee election schemes                
            \end{center}
            \begin{enumerate}[label=\textbf{Scheme \Alph*:},ref=\Alph*,leftmargin=6em]
                \item \label{lottery:public}
                (\textbf{Singleton committee.})
                All players use a common random string to 
                publicly sample, 
                independently for each round $i \in [\ell + n]$, 
                a single player $u_i \in \Players$ and  
                set $C_i = \{u_i\}$. 
                For all players $u \in \Players$, 
                the probability that $u = u_i$ is $\sigma_u$.

                \item \label{lottery:private}
                (\textbf{Private lottery.})
                Independently for each round $i \in [\ell + n]$, 
                player $u \in \Players$ 
                independently inserts himself 
                into the committee $C_i$ 
                with probability $\sigma_u$. 
            \end{enumerate}
        \end{framed}
        \caption{Two committee election schemes for an $(\ell, n)$-coin-flipping protocol}
        \label{fig:leader-election-schemes}
    \end{figure}


    \begin{theorem}[Main theorem]
        Let $\kappa, \gamma, \ell, n \in \NN$ and $\epsilon \in (0,1)$.
        Let $\Pi$ be an $(\ell, n)$-coin-flipping protocol with dimension $\kappa$ 
        and output $\eta$ whose committees are elected using Scheme A.
        Then $\Pi$ is $(\EpsP, \gamma)$-secure where {\color{red} to do}.
    \end{theorem}

\section{Eventual consensus PoS blockchains using the longest-chain rule}
    Let $\Blockchain$ be an eventual consensus PoS blockchain protocol 
    under the longest-chain rule. 
    The protocol $\Blockchain$ advances in discrete rounds 
    which we call \emph{slots}.
    Every participant $u$ in $\Blockchain$ 
    maintains a blockchain $\Chain_u$ 
    and updates it at every slot using the following simple rule: 
    \begin{enumerate}
      \item If a longer blockchain $\Chain$ is available, 
      $u$ sets $\Chain_u \leftarrow \Chain$.

      \item If $u$ is assigned to create a block at this slot, 
      $u$ adds a new block to $\Chain_u$ and broadcasts immediately.
    \end{enumerate}
%     \paragraph{Time, slots, and synchrony}

%     \paragraph{Transaction ledger properties.} 
%     A blockchain can implement a \emph{transaction ledger} in which ``transactions'' 
%     can be divided into linearly ordered ``blocks,'' giving a partial order in the transactions. 
%     A ledger must satisfy the following properties:
%     \begin{description}[labelindent=0.5cm]
%         \item[Persistence with parameter $k \in \NN$:] A block is declared \emph{stable} 
%         if it is at least $k$ blocks deep from the end of the ledger. 
%         If an honest node (at slot $t$) reports a block $B$ at slot $b$ to be stable, 
%         all honest nodes, queried at any slot $t^\prime \geq t$, must also report $B$ to be stable at slot $b$.

%         \item[Liveness with parameters $u,k \in \NN$:] Suppose a transaction is issued at slot $t$ and 
%         all honest nodes attempt to include it in the ledger for $u$ consecutive slots. 
%         Then the transaction will be stable in the views of all honest nodes at slot $t^\prime > t+u + 1$.
%     \end{description}

%     The above properties can be derived from the followoing elementary properties of the blockchain 
%     that implements the ledger.
%     However, it would be convenient for us to lay down some notation first. 
    Consider a blockchain $\Chain$ and suppose its most recent block is issued from some slot $s \in \NN$. 
    The ``trimmed chain'' $\Chain\TrimSlot{k}$ is defined as 
    the blockchain obtained from $\Chain$ by deleting all blocks (from $\Chain$) 
    corresponding to the last $k$ slots, i.e., slots $s, s - 1, \ldots, s - k + 1$. 
    In addition, we use the expression $\Chain_1 \Prefix \Chain_2$ to mean that 
    the chain $\Chain_1$ is a prefix of chain $\Chain_2$. 
    Furthermore, given a blockchain $\Chain$ and two slots $t_1$ and $t_2 \geq t_1$, 
    $\Chain[t_1 : t_2]$ denotes the chain segment containing all blocks from $\Chain$ 
    that are issued from slots $t \in [t_1, t_2]$.

    \begin{definition}[Common Prefix property with parameter $k \in \NN$]\label{def:cp}        
        Let $\Chain_1$ and $\Chain_2$ be two blockchains adopted by two honest players 
        at the onset of rounds (i.e., slots) $r_1$ and $r_2$, respectively, with $r_1 \leq r_2$. 
        Then $\Chain_1\TrimSlot{k} \Prefix \Chain_2$. 
    \end{definition}
    We use the shorthand $\kSlotCP$ for referring to the Common Prefix property defined above. 
    Observe that we defined this property in terms of elapsed time, i.e., slots; 
    traditionally (cf. \cite{C:GarKiaLeo17}), it has been defined in terms of the number of deleted blocks. 


    \begin{definition}[Existential Chain Quality property with parameter $s \in \NN$]\label{def:ECQ}        
        Consider slots $t_1, t_2, t$ satisfying $t_1 + s \leq t_2 \leq t$. 
        Let $\Chain$ be the blockchain held by an honest party at slot $t$. 
        Then $\Chain[t_1:t_2]$ contains at least one 
        honestly gneerated block.
    \end{definition}
    We use the shorthand $\sECQ$ for referring to the ECQ property defined above. 

    \begin{definition}[Blockchain protocol with $(\EpsP, k,s)$-security]\label{def:blockchain-security}
        Let $\EpsP \in \RR$ and $k, s \in \NN$. 
        A PoS blockchain protocol is $(\epsilon, k, s)$-secure if 
        it satisfies $\kSlotCP$ and $\sECQ$ property 
        except with probability at most $\epsilon$.
    \end{definition}


\section{Verifiable Random Functions}
% \paragraph{Verifiable Random Functions.}
\newcommand{\Gen}{\mathsf{gen}}
\newcommand{\Prove}{\mathsf{prove}}
\newcommand{\Verify}{\mathsf{verify}}
% \newcommand{\pk}{\mathsf{pk}}
% \newcommand{\sk}{\mathsf{sk}}
\begin{definition}[Verifiable Random Function]\label{def:VRF}
  A family $\mathcal{F}$ 
  of functions $F : \{0,1\}^\ell \rightarrow \{0,\}^k$ 
  is a family of VRFs if there exist algorithms 
  $(\Gen, \Prove, \Verify)$ 
  so that the following holds: 
  $\Gen(1^k)$ outputs a pair of keys $(\pk, \sk)$; 
  $\Prove_\sk(x)$ outputs a pair $(F_\sk(x), \pi_\sk(x))$ 
  where 
  $F_\sk \in \mathcal{F}$, $F_\sk(x)$ is the function value, and 
  $\pi_\sk(x)$ is the proof of correctness; and 
  $\Verify_\pk(x,y,\pi_\sk(x))$ efficiently verifies 
  that $y = F_\sk(x)$ using the proof $\pi_\sk(x)$, 
  outputting 1 if $y$ is valid and 0 otherwise. 
  Additionally, we require the following properties:

  \begin{enumerate}
    \item \emph{Uniqueness:} 
    No values $(\pk, x, y, y', \pi, \pi')$  can satisfy both 
    $\Verify_\pk(x,y,\pi) = 1$ and $\Verify_\pk(x,y',\pi') = 1$ 
    unless $y = y'$.

    \item \emph{Provability:} 
    If $(y, \pi) = \Prove_\sk(x)$ then $\Verify_\pk(x,y, \pi) = 1$.

    \item \emph{Pseudorandomness:} 
    To all probabilistic polynomial-time (PPT) algorithm 
    which runs $\Poly(k)$ steps when its first input is $1^k$ 
    and does not query the $\Prove$ oracle on $x$, 
    the distribution of $F_\sk(x)$ appears uniform in $\{0,1\}^k$, 
    except with a probability negligible in $k$.
  \end{enumerate}

\end{definition}
