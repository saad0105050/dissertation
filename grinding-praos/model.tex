

In this chapter, 
we recall the eventual consensus PoS blockchains under the longest-chain rule 
(see \Section~\ref{sec:model-multihonest}). 
% We repeat the necessary components here for the sake of self-containment. 
Recall the consistency property $\kSlotCP$. 
In this section, we introduce a new security property, called the ``Existential Chain Quality Property'' ($\ECQ$), 
which captures the desired behavior that a long enough segment of an honestly held blockchain 
must contain an honest block. 
We use these two properties to define a succinct vocabulary for blockchain security, 
namely, the $(\EpsP, k, s)$-security; this is standard.

Next, we shift our focus toward the leader election schemes. 
Although Praos uses private leader elections, 
we take the opportunity to define the public leader elections as well. 


% \section{Eventual consensus PoS blockchains using the longest-chain rule}
    Let $\Blockchain$ be an eventual consensus PoS blockchain protocol 
    under the longest-chain rule, 
    as in \Section~\ref{sec:model-multihonest}. 
    The protocol $\Blockchain$ advances in discrete rounds 
    which we call \emph{slots}.
    Every participant $u$ in $\Blockchain$ 
    maintains a blockchain $\Chain_u$ 
    and updates it at every slot using the following simple rule: 
    \begin{enumerate}
      \item If a longer blockchain is available in his view, 
      $u$ sets $\Chain_u \leftarrow \Chain$ where 
      $\Chain$ is a longest blockchains. 
      (There could be multiple longest chains.)

      \item If $u$ is assigned to create a block at this slot 
      (i.e., he is a \emph{leader}),
      $u$ adds a new block to $\Chain_u$ and broadcasts immediately.
    \end{enumerate}

    Consider a blockchain $\Chain$ and suppose its most recent block is issued from some slot $s \in \NN$. 
    The ``trimmed chain'' $\Chain\TrimSlot{k}$ is defined as 
    the blockchain obtained from $\Chain$ by deleting all blocks (from $\Chain$) 
    corresponding to the last $k$ slots, i.e., slots $s, s - 1, \ldots, s - k + 1$. 
    In addition, we use the expression $\Chain_1 \Prefix \Chain_2$ to mean that 
    the chain $\Chain_1$ is a prefix of chain $\Chain_2$. 
    Furthermore, given a blockchain $\Chain$ and two slots $t_1$ and $t_2 \geq t_1$, 
    $\Chain[t_1 : t_2]$ denotes the chain segment containing all blocks from $\Chain$ 
    that are issued from slots $t \in [t_1, t_2]$.

    Recall the common prefix property (CP) (in particular, the $\kSlotCP$ property) 
    from Definition~\ref{def:cp-slot-mh}. 
    Below, we mention another important property, 
    the \emph{existential honest chain quality property}: 


    \begin{definition}[Existential Chain Quality property with parameter $s \in \NN$]\label{def:ECQ}        
        Consider slots $t_1, t_2, t$ satisfying $t_1 + s \leq t_2 \leq t$. 
        Let $\Chain$ be the blockchain held by an honest party at slot $t$. 
        Then $\Chain[t_1:t_2]$ contains at least one 
        honestly generated block.
    \end{definition}
    We use the shorthand $\sECQ$ for referring to the $\ECQ$ property defined above. 
    Although the analysis of Praos does not contain a proof of this property, 
    the proof should be very similar to the one in Ouroboros Genesis~\cite{Genesis} 
    which is built upon the Praos protocol.
    Next, we can quantify the security of the blockchain protocol in terms of these two properties:

    \begin{definition}[$(\EpsP, k,s)$-security]\label{def:blockchain-security}
        Let $\EpsP \in \RR$ and $k, s \in \NN$. 
        An eventual consensus blockchain protocol is $(\EpsP, k, s)$-secure if 
        it satisfies $\kSlotCP$ and $\sECQ$ property 
        except with probability at most $\EpsP$.
    \end{definition}





\Paragraph{A (more) precise definition of VRFs.}
Below, we present a definition of VRFs that is good enough for our exposition. 
For a more complete definition, see~\cite{VRFMicali,VRFDodis}.

\newcommand{\Gen}{\mathsf{gen}}
\newcommand{\Prove}{\mathsf{prove}}
\newcommand{\Verify}{\mathsf{verify}}
% \newcommand{\pk}{\mathsf{pk}}
% \newcommand{\sk}{\mathsf{sk}}
\begin{definition}[Verifiable Random Function (VRF)]\label{def:VRF}
  A family $\mathcal{F}$ 
  of functions $F : \{0,1\}^\ell \rightarrow \{0,\}^k$ 
  is a family of VRFs if there exist algorithms 
  $(\Gen, \Prove, \Verify)$ 
  so that the following holds: 
  $\Gen(1^k)$ outputs a pair of keys $(\pk, \sk)$; 
  $\Prove_\sk(x)$ outputs a pair $(F_\sk(x), \pi_\sk(x))$ 
  where 
  $F_\sk \in \mathcal{F}$, $F_\sk(x)$ is the function value, and 
  $\pi_\sk(x)$ is the proof of correctness; and 
  $\Verify_\pk(x,y,\pi_\sk(x))$ efficiently verifies 
  that $y = F_\sk(x)$ using the proof $\pi_\sk(x)$, 
  outputting 1 if $y$ is valid and 0 otherwise. 
  Additionally, we require the following properties:

  \begin{enumerate}
    \item \emph{Uniqueness:} 
    No values $(\pk, x, y, y', \pi, \pi')$  can satisfy both 
    $\Verify_\pk(x,y,\pi) = 1$ and $\Verify_\pk(x,y',\pi') = 1$ 
    unless $y = y'$.

    \item \emph{Provability:} 
    If $(y, \pi) = \Prove_\sk(x)$ then $\Verify_\pk(x,y, \pi) = 1$.

    \item \emph{Pseudorandomness:} 
    To all probabilistic polynomial-time (PPT) algorithm 
    which runs $\Poly(k)$ steps when its first input is $1^k$ 
    and does not query the $\Prove$ oracle on $x$, 
    the distribution of $F_\sk(x)$ appears uniform in $\{0,1\}^k$, 
    except with a probability negligible in $k$.
  \end{enumerate}

\end{definition}


\section{Lottery-based leader election schemes}\label{sec:leader-election-public-private}
Let $\alpha \in (0,1)$ and consider the players $\Players = \Players(\alpha)$
Let $C_i$ be the set of leaders for slot $i$. 
We consider two types of leader election schemes for an epoch: 
% \begin{enumerate}[label=\textbf{Scheme \Alph*:},ref=\Alph*,leftmargin=3em]
\begin{description}
    \item[\textbf{$\PublicLeaderElection(\alpha)$: Publicly elect a single leader per slot.}] \label{lottery:public}
    At the onset of an epoch, 
    all players use a common random string to 
    publicly sample, 
    independently for each round $i \in [\ell + n]$, 
    a single player $\ell_i \in \Players$ and  
    set $C_i = \{\ell_i\}$. 
    For all players $u \in \Players$, 
    the probability that $u = \ell_i$ is $\sigma_u$. 
    Thus, the leader schedule for an epoch 
    is public knowledge before an epoch commences. 

    \item[\textbf{$\PrivateLeaderElection(\alpha)$: Privately elect zero or more leaders per slot.}] \label{lottery:private}
    Independently for each round $i \in [\ell + n]$, 
    player $u \in \Players$ 
    privately and independently 
    evaluates a Boolean random variable $D$
    which has expectation $\sigma_u$; 
    if $D = 1$, $u$ 
    inserts himself into the set $C_i$ 
    and announces this fact during slot $i$. 
\end{description}
\noindent
Ouroboros uses a public leader election 
at the outset of an epoch 
and, therefore, it is secure only against a static adversary. 
SnowWhtie and Praos (and Bitcoin as well) uses a private leader election 
and therefore, must prove security against an adaptive adversary. 




\section{Randomness beacons and the loss in min-entropy}
A beacon protocol takes a uniformly random initial string $\eta_0$ 
and produces a series of strings $\eta_1, \eta_2, \ldots$; 
these strings are called the outputs of the beacon.


The common, public random string $\eta_{t-1}$ 
is used in various computations in epoch $t$, 
including the leader election process.

Specifically, $\eta_{t-1}$ is used by all participants in epoch $t$ 
(in combination with their respective private keys) 
to randomly select a player-specific VRF (see below) 
and use it to elect leaders from among themselves.
Therefore, the ``randomness''
of $\eta_1, \eta_2, \ldots$ is a critical factor underlying 
the security of the blockchain protocol; 
we explore this next.

% \Paragraph{Beacon output quality.}
Let $n \in \NN$ and let $A$ be the set of $n$-bit Boolean strings.
Suppose a random variable $X$ takes values from $A$
and its probability distribution is $\mathcal{X}$. 
One way to measure how ``random'' the outcomes of $X$ are 
is to measure the \emph{min-entropy} of $X$:

\begin{definition}[Entropy and Min-entropy]
    Let $X$ be a random variable (taking values from set $A$) 
    and let $\mathcal{X} : A \rightarrow [0,1]$ be its probability distribution.
    The \emph{entropy} of $X$ is defined as 
    \begin{equation}\label{def:entropy}
        H(X) = \sum_{x \in A} \mathcal{X}(x) \log_2 (1/\mathcal{X}(x))
        \,.
    \end{equation}
    The \emph{min-entropy} of $X$ is defined as 
    \begin{equation}\label{def:min-entropy}
        H_\infty(X) = - \log_2 \max_{x \in A} \mathcal{X}(x)
        \,.
    \end{equation}
    We also write $H_\infty(\mathcal{X})$ to denote the min-entropy of $X$.
\end{definition}
\noindent

In the above definition, 
if $A = \{0,1\}$ and $X$ is a Boolean random variable with expectation $\alpha \in (0,1)$, 
the entropy of $X$ is given by 
the \emph{binary entropy function} $H: [0,1] \rightarrow [0,1]$ defined as 
\begin{align}\label{eq:binary-entropy}
    H(\alpha) &= -\alpha \log_2(\alpha) - (1-\alpha) \log_2(1-\alpha)
\end{align}
where, By convention, we set $H(0) = H(1) = 0$. 
Since $H$ is a linear combination of logarithms, it is concave. 

Let $U$ be a uniformly distributed random variable on $A$; 
notice that $H_\infty(X) \leq H_\infty(U) = \log_2(|A|)$. 
If $X$ has a low min-entropy, 
it indicates that the distribution of $X$ is ``far'' from being uniformly random. 
Specifically, \emph{the min-entropy loss in $X$} 
is defined as the difference between $H_\infty(X)$ and $H_\infty(U) = \log_2(|A|)$.


\Paragraph{Grinding power and beacon security.}
Let $n \in \NN$ and $A \subseteq \{0,1\}^n$.
Let $\eta_0$ be uniformly random in $\{0,1\}^n$ 
and it is used as the initial value for the beacon protocol. 
Let $X_t \in A$ be the random variable 
representing the beacon output at epoch $t$, 
i.e., $X_t = \eta_t, t \in 1, 2, \ldots$\,. 
Let $\mathcal{X}_t$ be the distribution of $X_t$. 

\begin{definition}[Grinding power, informal]\label{def:grinding-power-num-choices}
The \emph{grinding power} $g$ of the adversary $\Adversary$ 
for the beacon output $\eta$
is the number of choices $\Adversary$ has for $\eta$. 
The grinding power of $\Adversary$ over 
the lifetime of the beacon protocol
is the maximum across all beacon outputs.
\end{definition}

Consider a simple one-round game played by the adversary $\Adversary$. 
Assume that $\Adversary$ has a \emph{target} $\eta^* \in \{0,1\}^n$. 
$\Adversary$ wins if $X = \eta^*$; he loses otherwise. 
Let $p_\Adversary$ be his winning probability. 
If $X$ were uniform in $\{0,1\}^n$, 
$p_\Adversary$ would have been $2^{-n}$. 
If the grinding power $g$ is larger than one, 
$X$ is not uniform since 
$\Adversary$ can reset it $g$ times 
and inflate his winning probability to $g 2^{-n}$. 
Thus, 
$$ 
    \max_{\eta \in \{0,1\}^n} \mathcal{X}(\eta) \geq \mathcal{X}(\eta^*) = g 2^{-n}
$$ 
and, consequently, 
$$
    H_\infty(X) \geq -\log_2 (g 2^{-n}) = n - \log_2 g
    \,.
$$
Thus, we have proved the following fact:

\begin{fact}\label{fact:min-entropy-grinding-power}
  For any beacon with grinding power $g$, 
  the min-entropy loss in its output is at most $\log_2 g$.
\end{fact}

\begin{definition}[Beacon security]\label{def:beacon-security}
    Let $n,k,s \in \NN, \rho \in \RR_+$ and $\EpsP, \EpsG \in (0,1)$.
    Let $\Pi_\Beacon$ be a beacon protocol for $n$-bit Boolean strings 
    used in conjunction with a blockchain protocol $\Pi$. 
    Let $\mathcal{B}$ be the distribution of the beacon output.
    Let $E$ be the event that $\Pi$ remains $(\EpsP, k, s)$-secure over the lifetime of the protocol. 
    Condition on $E$. 
    Suppose that except with probability $\EpsG$, 
    the (conditional) min-entropy 
    $H_\infty(\mathcal{B} \mid E)$ is at least $n - \rho$. 
    Then we say that $\Pi_\Beacon$ is \emph{$(\EpsP + \EpsG, \rho)$-secure}.
\end{definition}



Note that the output of the beacon at epoch $L = 2, 3, \ldots$  
is a sequential composition of $L$ independent instances of $\Pi_\Beacon$; 
the output from an epoch is used as the initial value for the next epoch. 
Specifically, 
the beacon output for epoch $t \in [L]$ is 
\begin{align}\label{eq:beacon-iterated}
    \eta_t = \Pi_\Beacon(\eta_{t-1}) = \Pi_\Beacon^t(\eta_0)\,.
\end{align}
where $\eta_0$ is the uniformly random initial value for the first epoch 
and the superscript $t$ denotes a $t$-fold sequential composition.


\begin{lemma}[Security, iterated beacon]\label{lemma:beacon-composition}
Let $\EpsP \in (0,1)$ and $\rho \in \RR_+$.
Suppose $\Pi_\Beacon$ remains $(\EpsP, \rho)$-secure over a single epoch. 
When it is iterated $L$ times as in~\eqref{eq:beacon-iterated}, 
it remains $\left(L 2^\rho \EpsP, \rho\right)$-secure 
over the $L$ epochs.
\end{lemma}
\begin{proof}
    Let $\Pi_t$ be the instance of $\Pi_\Beacon$ at epoch $t$. 
    We will prove the claim by an induction on $t \in [L]$. 

    Each protocol instance $\Pi_t, t \in [L]$ 
    is an independent instance of $\Pi_\Beacon$. 
    Our inductive hypothesis is that for $t \in [L]$, 
    all previous instances 
    $\Pi_\tau, \tau \in [t-1]$ remain $(\EpsP, \rho)$-secure in epoch $\tau$. 
    Thus, there are at most $2^\rho$ choices for $\eta_{t - 1}$, 
    resulting from the grinding in epoch $t-1$.
    By a union bound on these choices, $\Pi_t$ is $(2^\rho \EpsP, \rho)$-secure. 
    A union bound over $t \in [L]$ shows that 
    except with probability at most $L\cdot 2^\rho \EpsP$, 
    every $\Pi_t, t \in [L]$ is $(2^\rho \EpsP, \rho)$-secure.
\end{proof}


In \Section~\ref{sec:praos}, we characterize the grinding power 
for the beacon protocol of Ouroboros Praos~\cite{Praos} and Snow White~\cite{SnowWhite}.



% The length of each epoch is $R \geq n + k$. 

% \begin{definition}[Praos beacon with dimension $\kappa$]\label{def:praos-beacon}~
%   \begin{itemize}
%     \item Before the protocol commences, 
%     all participants are given a uniformly random string $\eta_0 \in \{0,\}^k$. 

%     \item Before epoch $t = 1, 2, \ldots$ commences, 
%     a string $\eta_{t-1}$ is given to all players; 
%     this is called the \emph{initial value} for this epoch. 
%     In fact, for $t \geq 2$, $\eta_{t-1}$ is the beacon output from the previous epoch; see below.

%     \item Inside slot $i$ in epoch $t$, 
%     when a slot leader $u$ issues a new block, 
%     he also includes a \emph{nonce input} 
%     $\VRF_{\PrivateKey(u)}(t \Vert i \Vert \eta_{t-1}) \in \{0,1\}^\kappa$. 
%     Here, $\Vert$ denotes concatenation.
    
%     \item After the epoch is complete, 
%     every honest participant computes the beacon output $\eta_t \in \{0,1\}^\kappa$ as follows: 
%     he gathers all nonce inputs 
%     from a specified prefix of his blockchain, 
%     XORs them together, and then applies a cryptographic hash.
%   \end{itemize}
%   The strings $\eta_1, \eta_2, \ldots$ are called the beacon outputs.
% \end{definition}


