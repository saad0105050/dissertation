
\newcommand{\FuncBeacon}{\mathcal{F}_\mathsf{beacon}}
\newcommand{\ProtocolBeacon}{\mathcal{\Pi}_\mathsf{beacon}}
% \newcommand\Litem[1]{\item{\bfseries #1\enspace}}




% \section{Collective coin-flipping}

%     A \emph{coin-flipping protocol $\Pi$ of dimension $\kappa$} 
%     takes, as input, a uniformly random $\kappa$-bit string $\eta_0$
%     and outputs a $\kappa$-bit string $\eta$. 
%     The goal of the protocol is to make the distribution of $\eta$ 
%     as close to uniform as possible.
    
%     \begin{definition}[Coin-flipping with $(\epsilon, \rho)$-security]\label{def:coin-flipping-security}
%         Let $\kappa \in \NN, \rho, \epsilon \in \RR, \rho \in [0, \kappa]$, and $\epsilon \in (0,1)$.
%         Let $\Pi$ be an coin-flipping protocol with dimension $\kappa$ 
%         and output $\eta$. 
%         We say that \emph{$\Pi$ is $(\epsilon, \rho)$-secure} if 
%         $\MinEntropy(\eta) \geq \kappa - \rho$ 
%         except with probability $\epsilon$. 
%         Here, $\MinEntropy$ is the min-entropy function.
%     \end{definition}
    
%     Suppose a coin-flipping protocol $\Pi$ is $(\epsilon, \rho)$-secure. 
%     Then 
%     the min-entropy loss of its output distribution is at most $\rho$. 
%     On the other hand, let $T$ be an arbitrary set of $\kappa$-bit Boolean strings. 
%     The probability that $\eta \in T$ is at most $2^\rho \cdot |T| \cdot 2^{-\kappa}$. 
%     The amplification factor $2^\rho$ can be interpreted as 
%     the adversarial participants 
%     choosing the designated output $\eta$ from 
%     a set containing $2^\rho$ uniformly random strings; 
%     {\color{red} strings in this set may be correlated with each other}. 
%     An unbiased protocol will necessarily have $\rho = 0$.


%     \paragraph{Protocol participants.} 
%     Let $\Players$ be the set of protocol participants. 
%     With each party $u \in \Players$ is associated a positive real $\sigma_u \in [0, 1]$ 
%     so that $\sum_{u \in \Players} \sigma_u = 1$. 

%     \paragraph{The adversary.} 
%     Let $\alpha \in [0,1]$. 
%     The adversary can corrupt any player instantly subject to 
%     a budget $\alpha$: that is, 
%     at any moment, the set of players $A \subset \Players$ controlled by the adversary 
%     always satisfy $\sum_{u \in A} \sigma_u \leq \alpha$. 
%     This adversary is called an \emph{$\alpha$-dominated adversary}. 
%     If a player is ever corrupted by the adversary, 
%     he remains under his control for the rest of the protocol. 
%     The adversary is Byzantine: 
%     once he controls a player, he can send arbitrary messages on behalf of the player.

%     \paragraph{Communication model.} 
%     We assume a synchronous communication where 
%     the adversary is responsible for delivering messages in any order he chooses.

%     \begin{definition}[$(\ell, n)$-coin-flipping with dimension $\kappa$]\label{def:coin-flipping}
%         Let $\ell, n, \kappa \in \NN$ and 
%         let $\eta_0 \in \{0,1\}^\kappa$ be a uniformly random \emph{initial value}. 
%         The protocol has $r = \ell + n$ rounds. 
%         \begin{itemize}
%             \item (Committees.) 
%             With every round $i \in [r]$ is associated 
%             a \emph{committee} $C_i$ of players. 
%             (The committee election scheme is left unspecified at this moment. 
%             See Figure~\ref{fig:leader-election-schemes} below.)
%             The committee membership is private unless 
%             a member discloses it. 

%             \item (Nonces.) 
%             Before the protocol commences, 
%             for every $i \in [r]$, 
%             each member $u \in C_i$ is given a \emph{nonce} $y_{i,u} \in \{0,1\}^\kappa$. 
%             The nonces of the honest members are uniformly random. 
%             The adversarial members may choose their nonces arbitrarily at this point 
%             but once the protocol commences, 
%             they cannot change the value of their nonces. 

%             \item (Adversarial knowledge and the lookahead.) 
%             The adversarial players know, in advance, 
%             about all adversarial committe members and 
%             their nonces across all rounds.  
%             They also know which rounds contain an honest committee member.
%             Furthermore, at round $i$, 
%             adversarial players can see 
%             the honest nonces for rounds $i, i+1, \ldots, \min\{i + \ell - 1, r\}$.

%             \item (Announcement.)
%             With each round $i$ is associated 
%             a publicly visible set $Y_i$, initially empty. 
%             At round $i$, 
%             an honest committee member inserts his nonce into $Y_i$. 
%             An adversarial committee member $u$ may strategically choose to insert or not. 

%             \item (Output.)
%             At the end of round $r$, 
%             let $m_i$ 
%             be the lexicographic minimum of all strings recorded in $Y_i, i \in [r]$.
%             (If $Y_i$ is empty then set $m_i = 0^\kappa$.)
%             The \emph{output} of the protocol is 
%             $
%                 \eta = \eta_0 \oplus m_1 \oplus \cdots m_{\ell + n}
%                 % \,.
%             $. 
%         \end{itemize}
%     \end{definition}

%     Since the adversary cannot change the nonces, 
%     all he can do to bias the output 
%     is to strategically vary the size of the sets $Y_i, i \in [\ell + n]$.

%     \begin{figure}[h]
%         \begin{framed}
%             \begin{center}
%                 Two committee election schemes                
%             \end{center}
%             \begin{enumerate}[label=\textbf{Scheme \Alph*:},ref=\Alph*,leftmargin=6em]
%                 \item \label{lottery:public}
%                 (\textbf{Singleton committee.})
%                 All players use a common random string to 
%                 publicly sample, 
%                 independently for each round $i \in [\ell + n]$, 
%                 a single player $u_i \in \Players$ and  
%                 set $C_i = \{u_i\}$. 
%                 For all players $u \in \Players$, 
%                 the probability that $u = u_i$ is $\sigma_u$.

%                 \item \label{lottery:private}
%                 (\textbf{Private lottery.})
%                 Independently for each round $i \in [\ell + n]$, 
%                 player $u \in \Players$ 
%                 independently inserts himself 
%                 into the committee $C_i$ 
%                 with probability $\sigma_u$. 
%             \end{enumerate}
%         \end{framed}
%         \caption{Two committee election schemes for an $(\ell, n)$-coin-flipping protocol}
%         \label{fig:leader-election-schemes}
%     \end{figure}


%     \begin{theorem}[Main theorem]
%         Let $\kappa, \gamma, \ell, n \in \NN$ and $\epsilon \in (0,1)$.
%         Let $\Pi$ be an $(\ell, n)$-coin-flipping protocol with dimension $\kappa$ 
%         and output $\eta$ whose committees are elected using Scheme A.
%         Then $\Pi$ is $(\EpsP, \gamma)$-secure where {\color{red} to do}.
%     \end{theorem}

\section{Eventual consensus PoS blockchains using the longest-chain rule}
    Let $\Blockchain$ be an eventual consensus PoS blockchain protocol 
    under the longest-chain rule, 
    as in \Section~\ref{sec:model-multihonest}. 
    The protocol $\Blockchain$ advances in discrete rounds 
    which we call \emph{slots}.
    Every participant $u$ in $\Blockchain$ 
    maintains a blockchain $\Chain_u$ 
    and updates it at every slot using the following simple rule: 
    \begin{enumerate}
      \item If a longer blockchain is available in his view, 
      $u$ sets $\Chain_u \leftarrow \Chain$ where 
      $\Chain$ is a longest blockchains. 
      (There could be multiple longest chains.)

      \item If $u$ is assigned to create a block at this slot 
      (i.e., he is a \emph{leader}),
      $u$ adds a new block to $\Chain_u$ and broadcasts immediately.
    \end{enumerate}

    Consider a blockchain $\Chain$ and suppose its most recent block is issued from some slot $s \in \NN$. 
    The ``trimmed chain'' $\Chain\TrimSlot{k}$ is defined as 
    the blockchain obtained from $\Chain$ by deleting all blocks (from $\Chain$) 
    corresponding to the last $k$ slots, i.e., slots $s, s - 1, \ldots, s - k + 1$. 
    In addition, we use the expression $\Chain_1 \Prefix \Chain_2$ to mean that 
    the chain $\Chain_1$ is a prefix of chain $\Chain_2$. 
    Furthermore, given a blockchain $\Chain$ and two slots $t_1$ and $t_2 \geq t_1$, 
    $\Chain[t_1 : t_2]$ denotes the chain segment containing all blocks from $\Chain$ 
    that are issued from slots $t \in [t_1, t_2]$.

    Recall the common prefix property (CP) (in particular, the $\kSlotCP$ property) 
    from Definition~\ref{def:cp-slot-mh}. 
    Below, we mention another important property, 
    the \emph{existential honest chain quality property}: 


    \begin{definition}[Existential Chain Quality property with parameter $s \in \NN$]\label{def:ECQ}        
        Consider slots $t_1, t_2, t$ satisfying $t_1 + s \leq t_2 \leq t$. 
        Let $\Chain$ be the blockchain held by an honest party at slot $t$. 
        Then $\Chain[t_1:t_2]$ contains at least one 
        honestly gneerated block.
    \end{definition}
    We use the shorthand $\sECQ$ for referring to the ECQ property defined above. 
    Next, we can quantify the security of the blockchain protocol in terms of these two properties:

    \begin{definition}[$(\EpsP, k,s)$-security]\label{def:blockchain-security}
        Let $\EpsP \in \RR$ and $k, s \in \NN$. 
        A PoS blockchain protocol is $(\EpsP, k, s)$-secure if 
        it satisfies $\kSlotCP$ and $\sECQ$ property 
        except with probability at most $\EpsP$.
    \end{definition}



\section{Leader election schemes and static/adaptive adversaries}\label{sec:leader-election-public-private}\label{sec:static-dynamic-adversary}

One can consider two types of adversaries in terms of 
how fast they can corrupt an honest player. the so-called \emph{static adversary} 
Suppose that an epoch is $R$ slots long. 
and the \emph{dynamic (or adaptive) adversary}. 
A \emph{static adversary} takes at least $R$ slots to corrupt an honest player 
but an \emph{adaptive adversary} can instantly corrupt an honest player. 
Ouroboros~\cite {Ouroboros} is secure against a static adversary 
while Praos~\cite {Praos} and SnowWhite~\cite{SnowWhite} 
are secure against an adaptive adversary.
The adversarial model critically depends on how a slot-leader is elected. 

Let
$\Players$ be the set of participants 
where each player $i \in \Players$ is associated with a positive real number 
$\sigma_i \in [0,1]$, called his \emph{relative stake ratio}, or just \emph{stake} in short. 
The stake ratios satisfy $\sum_{i \in \Players} \sigma_i = 1$ 
and, for this analysis, remains constant over the course of the execution of the protocol. 
(As is common in the literature (e.g., in \cite{Ouroboros,Praos,SnowWhite}), 
we can handle a change in stake separately from this analysis and then hook these two together.)

Let $\HonestPlayers, \DishonestPlayers \subseteq \Players$ 
denote the set of honest and dishonest players, respectively;
these sets are disjoint and $\Players = \HonestPlayers \Union \DishonestPlayers$.
Given a real $\alpha \in (0,1/2)$, 
we use the notation $\Players(\alpha)$ 
to denote the fact that the total dishonest stake bounded by $\alpha$: i.e., 
$\sum_{i \in \DishonestPlayers} \sigma_i \leq \alpha$
$\sum_{i \in \HonestPlayers} \sigma_i \geq 1 - \alpha$. 
We do not, however, make any assumptions on the number of participants.


In every slot, 
one or more participants are elected as a ``leader'' who can add block to a chain. 
(The claim ``player $u$ won the leader election for slot $i$ at epoch $e$'' 
can be proved using digital signatures and VRFs (Definition~\ref{def:VRF}.)
Let $C_i$ be the set of leaders for slot $i$. 
We consider two types of leader election schemes for an epoch: 
% \begin{enumerate}[label=\textbf{Scheme \Alph*:},ref=\Alph*,leftmargin=3em]
\begin{description}
    \item[\textbf{$\PublicLeaderElection(\alpha)$: Publicly elect a single leader per slot.}] \label{lottery:public}
    At the onset of an epoch, 
    all players use a common random string to 
    publicly sample, 
    independently for each round $i \in [\ell + n]$, 
    a single player $\ell_i \in \Players$ and  
    set $C_i = \{\ell_i\}$. 
    For all players $u \in \Players$, 
    the probability that $u = \ell_i$ is $\sigma_u$. 
    Thus, the leader schedule for an epoch 
    is public knowledge before an epoch commences. 

    \item[\textbf{$\PrivateLeaderElection(\alpha)$: Privately elect zero or more leaders per slot.}] \label{lottery:private}
    Independently for each round $i \in [\ell + n]$, 
    player $u \in \Players$ 
    privately and independently 
    evaluates a Boolean random variable $D$
    which has expectation $\sigma_u$; 
    if $D = 1$, $u$ 
    inserts himself into the set $C_i$ 
    and announces this fact during slot $i$. 
\end{description}
\noindent
Ouroboros uses a public leader election 
at the outset of an epoch 
and, therefore, it is secure only against a static adversary. 
SnowWhtie and Praos (and Bitcoin as well) uses a private leader election 
and therefore, must prove security against an adaptive adversary. 


\section{The grinding problem for resettable randomness beacons}\label{beacon-model}
Before an epoch (say, epoch $t$) commences, 
all participants collectively agree on a random string $\eta_{t-1}$; 
this is called the output of the \emph{beacon protocol}, $\Pi_\Beacon$, 
for epoch $t - 1$. 
In particular, $\Pi_\Beacon$ is initialized with a fresh random string $\eta_0$ 
(called the \emph{initial value})
and it periodically outputs, for each epoch $t = 1, 2, \ldots$\,, 
a random string $\eta_t$. 
The common, public random string $\eta_{t-1}$ 
is used in various computations in epoch $t$. 
But none is more important than the leader election process 
based on VRFs.

\newcommand{\Gen}{\mathsf{gen}}
\newcommand{\Prove}{\mathsf{prove}}
\newcommand{\Verify}{\mathsf{verify}}
% \newcommand{\pk}{\mathsf{pk}}
% \newcommand{\sk}{\mathsf{sk}}
\begin{definition}[Verifiable Random Function (VRF)]\label{def:VRF}
  A family $\mathcal{F}$ 
  of functions $F : \{0,1\}^\ell \rightarrow \{0,\}^k$ 
  is a family of VRFs if there exist algorithms 
  $(\Gen, \Prove, \Verify)$ 
  so that the following holds: 
  $\Gen(1^k)$ outputs a pair of keys $(\pk, \sk)$; 
  $\Prove_\sk(x)$ outputs a pair $(F_\sk(x), \pi_\sk(x))$ 
  where 
  $F_\sk \in \mathcal{F}$, $F_\sk(x)$ is the function value, and 
  $\pi_\sk(x)$ is the proof of correctness; and 
  $\Verify_\pk(x,y,\pi_\sk(x))$ efficiently verifies 
  that $y = F_\sk(x)$ using the proof $\pi_\sk(x)$, 
  outputting 1 if $y$ is valid and 0 otherwise. 
  Additionally, we require the following properties:

  \begin{enumerate}
    \item \emph{Uniqueness:} 
    No values $(\pk, x, y, y', \pi, \pi')$  can satisfy both 
    $\Verify_\pk(x,y,\pi) = 1$ and $\Verify_\pk(x,y',\pi') = 1$ 
    unless $y = y'$.

    \item \emph{Provability:} 
    If $(y, \pi) = \Prove_\sk(x)$ then $\Verify_\pk(x,y, \pi) = 1$.

    \item \emph{Pseudorandomness:} 
    To all probabilistic polynomial-time (PPT) algorithm 
    which runs $\Poly(k)$ steps when its first input is $1^k$ 
    and does not query the $\Prove$ oracle on $x$, 
    the distribution of $F_\sk(x)$ appears uniform in $\{0,1\}^k$, 
    except with a probability negligible in $k$.
  \end{enumerate}

\end{definition}

Specifically, $\eta_{t-1}$ is used by all participants in epoch $t$ 
(in combination with their respective private keys) 
to randomly select a player-specific VRF (see below) 
and use it to elect leaders from among themselves.
Therefore, the ``randomness''
of $\eta_1, \eta_2, \ldots$ is critical factor underlying 
the security of the blockchain protocol; 
we explore this next.

\section{The quality of the beacon output; the grinding problem}

Let $n \in \NN$ and let $A$ be the set of $n$-bit Boolean strings.
Suppose a random variable $X$ takes values from $A$
and its probability distribution is $\mathcal{X}$. 
One way to measure how ``random'' the outcomes of $X$ are 
is to measure the \emph{min-entropy} of $X$:

\begin{definition}[Entropy and Min-entropy]
    Let $X$ be a random variable (taking values from set $A$) 
    and let $\mathcal{X} : A \rightarrow [0,1]$ be its probability distribution.
    The \emph{entropy} of $X$ is defined as 
    \begin{equation}\label{def:entropy}
        H(X) = \sum_{x \in A} \mathcal{X}(x) \log_2 (1/\mathcal{X}(x))
        \,.
    \end{equation}
    The \emph{min-entropy} of $X$ is defined as 
    \begin{equation}\label{def:min-entropy}
        H_\infty(X) = - \log_2 \max_{x \in A} \mathcal{X}(x)
        \,.
    \end{equation}
    We also write $H_\infty(\mathcal{X})$ to denote the min-entropy of $X$.
\end{definition}
\noindent

In the above definition, 
if $A = \{0,1\}$ and $X$ is a Boolean random variable with expectation $\alpha \in (0,1)$, 
the entropy of $X$ is given by 
the \emph{binary entropy function} $H: [0,1] \rightarrow [0,1]$ defined as 
\begin{align}\label{eq:binary-entropy}
    H(\alpha) &= -\alpha \log_2(\alpha) - (1-\alpha) \log_2(1-\alpha)
\end{align}
where, By convention, we set $H(0) = H(1) = 0$. 
Since $H$ is a linear combination of logarithms, it is concave. 

Let $U$ be a uniformly distributed random variable on $A$; 
notice that $H_\infty(X) \leq H_\infty(U) = \log_2(|A|)$. 
If $X$ has a low min-entropy, 
it indicates that the distribution of $X$ is ``far'' from being uniformly random. 
Specifically, \emph{the min-entropy loss in $X$} 
is defined as the difference between $H_\infty(X)$ and $H_\infty(U) = \log_2(|A|)$.


\subsection{The grinding problem}
In eventual consensus PoS, 
it takes some time for the blockchains to settle. 
Recall that the beacon protocol is also carried out through the blockchains 
and the adversary has the power to force an honest player 
to select any particular chain (see Section~\ref{sec:lcr-model}).
Thus, in principle, 
the adversary has some choice over the final value of the beacon output.
In particular, he can essentially \emph{reset} the beacon value: 
He could test multiple potential beacon outputs to see if one is favorable for him, 
e.g., whether it would make the upcoming leader election (in the next epoch) 
more favorable to him. 
If a potential beacon output is not favorable then he can, in principle, reset the beacon, 
i.e., try over and over again before settling for a beacon output 
(by making the corresponding blockchain to prevail). 
The maximum number of potential resets is called his \emph{grinding power}.

Specifically, let $n \in \NN$ and $A \subseteq \{0,1\}^n$.
Let $\eta_0$ be uniformly random in $\{0,1\}^n$ 
and it is used as the initial value for the beacon protocol. 
Let $X_t \in A$ be the random variable 
representing the beacon output at epoch $t$, 
i.e., $X_t = \eta_t, t \in 1, 2, \ldots$\,. 
Let $\mathcal{X}_t$ be the distribution of $X_t$. 

\begin{definition}[Grinding power, informal]
The \emph{grinding power} $g$ of the adversary $\Adversary$ 
for the beacon output $\eta$
is the number of choices $\Adversary$ has for $\eta$. 
The grinding power of $\Adversary$ over 
the lifetime of the beacon protocol
is the maximum over all beacon outputs.
\end{definition}

Consider a simple one-round game played by the adversary $\Adversary$. 
Assume that $\Adversary$ has a \emph{target} $\eta^* \in \{0,1\}^n$. 
$\Adversary$ wins if $X = \eta^*$; he loses otherwise. 
Let $p_\Adversary$ be his winning probability. 
If $X$ were uniform in $\{0,1\}^n$, 
$p_\Adversary$ would have been $2^{-n}$. 
If the grinding power $g$ is larger than one, 
$X$ is not uniform since 
$\Adversary$ can reset it $g$ times 
and inflate his winning probability to $g 2^{-n}$. 
Thus, 
$$ 
    \max_{\eta \in \{0,1\}^n} \mathcal{X}(\eta) \geq \mathcal{X}(\eta^*) = g 2^{-n}
$$ 
and, consequently, 
$$
    H_\infty(X) \geq -\log_2 (g 2^{-n}) = n - \log_2 g
    \,.
$$
Thus, we have proved the following fact:

\begin{fact}\label{fact:min-entropy-grinding-power}
  For any beacon with grinding power $g$, 
  the min-entropy loss in its output is at most $\log_2 g$.
\end{fact}

\begin{definition}[Beacon security]\label{def:beacon-security}
    Let $n,k,s \in \NN, \rho \in \RR_+$ and $\EpsP, \EpsG \in (0,1)$.
    Let $\Pi_\Beacon$ be a beacon protocol for $n$-bit Boolean strings 
    used in conjunction with a blockchain protocol $\Pi$. 
    Let $\mathcal{B}$ be the distribution of the beacon output.
    Let $E$ be the event that $\Pi$ remains $(\EpsP, k, s)$-secure over the lifetime of the protocol. 
    Condition on $E$. 
    Suppose that except with probability $\EpsG$, 
    the (conditional) min-entropy 
    $H_\infty(\mathcal{B} \mid E)$ is at least $n - \rho$. 
    Then we say that $\Pi_\Beacon$ is \emph{$(\EpsP + \EpsG, \rho)$-secure}.
\end{definition}



Note that the output of the beacon at epoch $L = 2, 3, \ldots$  
is a sequential composition of $L$ independent instances of $\Pi_\Beacon$; 
the output from an epoch is used as the initial value for the next epoch. 
Specifically, 
the beacon output for epoch $t \in [L]$ is 
\begin{align}\label{eq:beacon-iterated}
    \eta_t = \Pi_\Beacon(\eta_{t-1}) = \Pi_\Beacon^t(\eta_0)\,.
\end{align}
where $\eta_0$ is the uniformly random initial value for the first epoch 
and the superscript $t$ denotes a $t$-fold sequential composition.


\begin{lemma}[Iterated beacon]\label{lemma:beacon-composition}
Let $\EpsP \in (0,1)$ and $\rho \in \RR_+$.
Suppose $\Pi_\Beacon$ remains $(\EpsP, \rho)$-secure over a single epoch. 
When it is iterated $L$ times as in~\eqref{eq:beacon-iterated}, 
it remains $\left(L 2^\rho \EpsP, \rho\right)$-secure 
over the $L$ epochs.
\end{lemma}
\begin{proof}
    Let $\Pi_t$ be the instance of $\Pi_\Beacon$ at epoch $t$. 
    We will prove the claim by an induction on $t \in [L]$. 

    Each protocol instance $\Pi_t, t \in [L]$ 
    is an independent instance of $\Pi_\Beacon$. 
    Our inductive hypothesis is that for $t \in [L]$, 
    all previous instances 
    $\Pi_\tau, \tau \in [t-1]$ remain $(\EpsP, \rho)$-secure in epoch $\tau$. 
    Thus, there are at most $2^\rho$ choices for $\eta_{t - 1}$, 
    resulting from the grinding in epoch $t-1$.
    By a union bound on these choices, $\Pi_t$ is $(2^\rho \EpsP, \rho)$-secure. 
    A union bound over $t \in [L]$ shows that 
    except with probability at most $L\cdot 2^\rho \EpsP$, 
    every $\Pi_t, t \in [L]$ is $(2^\rho \EpsP, \rho)$-secure.
\end{proof}


In \Section~\ref{sec:praos}, we characterize the grinding power 
for the beacon protocol of Ouroboros Praos~\cite{Praos} and Snow White~\cite{SnowWhite}.





