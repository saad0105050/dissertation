% \small{
% \begin{enumerate}
%     \item Which problem are we talking about?
%     \item What should a desired solution do?
%     \item What have others done about it?
%     \item What is still missing?
% \end{enumerate}
% }

The heartbeat of a blockchain protocol is the periodic flashes from its \emph{randomness beacon}. 
In every period, the players agree on a random string without resorting to any trust. 
This random string fuels a flurry of activities, 
culminating in the selection of another random string for the next period, ad infinitum. 
% In fact, any distributed protocol that periodically needs random strings must use a randomness beacon. 

It is pivotal that the beacon is \emph{bias-resistant}, 
which means the periodic outputs must be statistically close to the uniform distribution. 
The beacon must be \emph{unpredictable} as well, which means no one can predict the outcomes except with a negligible probability. 
In addition, it has to be \emph{publicly verifiable}, 
which means anyone can use publicly available information to verify each step of the computation. 
Finally, it must provide \emph{liveness}, which means 
\InlineCases{
    \item that the beacon always produces its periodic output 
    and 
    \item everyone receives the same output. 
}

Consider a proof-of-stake (PoS) blockchain protocol 
where an adversary can query the potential beacon output for the next epoch 
before settling for a beacon value that would sway the future events in his favor. 
(He can do so by forcing a specific blockchain prefix to become consistent for this epoch.)
If he can make at most $q$ queries, 
he can inflate the failure probability for the next epoch by at most a factor of $q$. 
This is the so-called \emph{grinding attack}. In fact,  
it is an attack against the bias-resistance property of the beacon. 
The quantity $q$ is called the \emph{grinding power} of the adversary.

The security guarantees of many state-of-the-art PoS blockchain protocols, 
such as Ouroboros Praos~\cite{Praos} and Snow White~\cite{SnowWhite}, 
requires that $q$ be bounded by $\Poly(k)$ 
where $k$ is the security parameter of the blockchain protocol. 
However, we have an open question: 
\emph{how does $q$ grow as a function of the adversarial stake $\alpha$ and the security parameter $k$? 
When is it indeed $\Poly(k)$?}

In this work, we put forth a rigorous analysis of the grinding attack against Praos and Snow White. 
Our analysis shows that it is indeed $\Poly(k)$ if the adversarial stake is at most $10\%$; 
see Theorem~\ref{thm:minentropy-loss-praos-multi-epochs}.

% Or equivalently, what is the min-entropy of the underlying randomness beacon as a function of the adversarial stake?
% What is the quantitative relationship between the security properties (e.g., the min-entropy) of a beacon 
% and those (e.g., epoch length) of the ambient protocol? 

% Specifically, we show that as long as the ambient protocol guarantees eventual consensus and 

% This brings us to the randomness beacons in general. 




\section{A technical overview}
To do


\section{Outline of the exposition}
To do
