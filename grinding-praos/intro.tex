% \small{
% \begin{enumerate}
%     \item Which problem are we talking about?
%     \item What should a desired solution do?
%     \item What have others done about it?
%     \item What is still missing?
% \end{enumerate}
% }

The heartbeat of a blockchain protocol is the periodic flashes from its \emph{randomness beacon}. 
In every period, the players agree on a random string without resorting to any trust. 
This random string fuels a flurry of activities, 
culminating in the selection of another random string for the next period. Ad infinitum. 
% In fact, any distributed protocol that periodically needs random strings must use a randomness beacon. 

It is pivotal that the beacon is \emph{bias-resistant}, 
which means the periodic outputs must be statistically close to the uniform distribution. 
The beacon must be \emph{unpredictable} as well, which means no one can predict the outcomes except with a negligible probability. 
In addition, it has to be \emph{publicly verifiable}, 
which means anyone can use publicly available information to verify each step of the computation. 
Finally, it must provide \emph{liveness}, which means 
\InlineCases{
    \item that the beacon always produces its periodic output 
    and 
    \item everyone receives the same output. 
}

\Paragraph{The grinding attack against the beacon.}
Consider a proof-of-stake (PoS) blockchain protocol 
where 
\InlineCases{
  \item in the middle of an epoch, the adversary can query the potential beacon output for this epoch;
  \item if he does not like the output, he ``resets'' the beacon and query again; and   
  \item he resets the beacon multiple times before settling for a beacon value that would sway the future events in his favor. 
}
Such a beacon is sometimes called a ``resettable leaky beacon;'' 
quoting from~\citet{Praos}:
\quotebox{\singlespacing
  [The beacon's] leakiness stems from the fact that the adversary can complete the blockchain segment
  that determines the beacon value before revealing it to the honest participants, while resettability
  stems from the fact that the adversary can try a bounded number of different blockchain extensions
  that will stabilize the final beacon value to a different preferred value.
}  


This is the so-called \emph{grinding attack}. 
If the adversary can make $q$ queries in this process, 
he can inflate the failure probability for the next epoch by a factor of $q$. 
Thus, it is an attack against the bias-resistance property of the beacon. 
The largest possible value of $q$ is called the \emph{grinding power} of the adversary.
A grinding attack results in a loss of randomness in beacon output.
It turns out that this loss, quantified as a loss in the min-entropy in the beacon output, 
is just the base-2 logarithm of the grinding power.


\Paragraph{The importance of quantifying the grinding power.}
Praos is a successor to Ouroboros classic~\cite{Ouroboros} which uses 
a complicated beacon protocol which is completely resistant against a grinding attack. 
However, Praos opted to forgo this cumbersome protocol in favor of a simple, VRF-based beacon protocol. 
Quoting~\cite{Praos}, 
\quotebox{\singlespacing
  Note that implementing the beacon via hashing VRF values will make feasible a type of ``grinding
  attack'' where the adversary can trade hashing power for a slight bias of the protocol execution
  to its advantage. We show how this bias can be controlled by suitably increasing the relevant
  parameters depending on the hashing power that is available to the adversary.
}


There are some critical issues, however:
\begin{enumerate}
  \item In reality, the bias in the beacon remains ``slight'' as long as 
  the hashing power available to the adversary is \emph{assumed} to be bounded.
  But having a Proof-of-Work-like hashing power bound on the PoS guarantees is problematic 
  as it goes against the Proof-of-Stake philosophy.

  \item Suppose we want to make the grinding power $q$ \emph{not} depend on the adversary's hashing power. 
  But this means $q$ should grow as a function of the adversarial power 
  and the epoch length; 
  a large enough $q$ could make the security bound meaningless. 

  \item There is an ever-present tug-of-war between the consistency security and the grinding insecurity. 
  Specifically, to reduce the risk of a consistency error, we take a large $k$. 
  But a large $k$ also increases the window in which the adversary can perform a grinding attack 
  and thereby inflate the likelihood of a consistency violation in the next epoch. 
  Unfortunately, 
  if we simply assume that the grinding power is ``bounded,'' 
  as does Praos and SnowWhite, 
  we cannot quantify this trade-off: 
  as we do not understand the grinding power as a function of $k$ and the adversarial stake $\alpha$.
  This puts the protocol designers in a handicapped position 
  and keeps the users of the blockchain protocol ill-informed about the actual risk involved. 
\end{enumerate}

Therefore, an urgent open question for eventual consensus PoS blockchains is to understand 
how the grinding power grows as a function of $\alpha$ and $k$. 


\Paragraph{Our contributions and implications.}
In this work, we answer the above question for ``eventual consensus longest-chain rule beacons'' such as those in 
Ouroboros Praos~\cite{Praos} and Snow White~\cite{SnowWhite}. 


In Theorem~\ref{thm:minentropy-loss-praos-multi-epochs}, 
we prove an upper bound on the grinding power 
and show that it is small (i.e., $\Poly(k)$) if the adversarial stake is less $9.55\%$.
However, it grows as $e^{ck}$, where $c \geq 1/2$, 
highlighting the delicate balance between the consistency property 
(i.e., taking a large $k$) 
and the grinding power.

Looking ahead, we would like to have a beacon which is less dependent on $k$. 
We tackle this problem in Part~\ref{part:xorgames} of this Dissertation.



\Paragraph{Outline of the exposition.}
In \Section~\ref{sec:praos-related-work}, 
we present a brief survey of various beacon protocols 
and the closely related problem of collective coin-flipping. 
Then, in \Section~\ref{sec:model-grinding}, 
we formalize the notions of grinding power and the min-entropy loss in the beacon output. 
In \Section~\ref{sec:praos}, 
we analyze the grinding power $g$ in the Praos beacon, 
describe an upper bound $\gamma$ on the grinding power, 
and state tail bounds that say ``$\Pr[g > \gamma]$ is small.'' 
This leads to our main theorem, Theorem~\ref{thm:minentropy-loss-praos-multi-epochs}.
The main workhorse for the proof of the above theorem is 
a tight second-moment bound on $g$, derived in Lemma~\ref{lemma:grinding-praos-second-moment}. 
% The proof of three claims instrumental in the proof of the second moment 
% is given in {\Section}s~\ref{sec:praos-claim-multiple-honest-blocks}--\ref{sec:praos-claim-t2star-variance-exact}.

Although it does not directly take part in the proof of the main theorem, 
we present a tight upper and lower bound of the mean of the grinding power 
in \Section~\ref{sec:praos-mean}. 
We follow this up in \Section~\ref{sec:praos-higher-moments} by some tight asymptotic upper bounds on 
integral higher moments of the grinding power. 
We finish our exposition by proving some simple moment bounds for the grinding power 
in \Section~\ref{sec:praos-simple-moments}. 
Compared to the earlier, asymptotically tight moment bounds, 
these latter moments are easier to obtain but 
they have a worse dependence on the adversarial stake.

% Or equivalently, what is the min-entropy of the underlying randomness beacon as a function of the adversarial stake?
% What is the quantitative relationship between the security properties (e.g., the min-entropy) of a beacon 
% and those (e.g., epoch length) of the ambient protocol? 

% Specifically, we show that as long as the ambient protocol guarantees eventual consensus and 

% This brings us to the randomness beacons in general. 


%%%%%%%%%%%%%%%%%%%%%%%%
% Related Work
\section{Related work}\label{sec:related-work}


Below, we survey the coin-flipping and beacon literature in more detail.

\paragraph{Collective coin-flipping as a Boolean function.} 
The (single round) collective coin-flipping in the full information
model is a classic problem introduced in the seminal work
of~\citet{BL85}.  In this problem, $n$-players communicate over a
single broadcast channel.  First, each honest player submits a
uniformly random bit, and then each adversarial player (strategically)
submits an arbitrary bit.  The output of the protocol is a single bit,
computed as a Boolean function $F$ applied to the submitted bits.  If
$F$ is the majority function, an adversarial coalition of size
$O(\sqrt{n})$ can arbitrarily influence the output of the
protocol~\cite{BL85}.  Later,~\citet*{KKL} showed that for
\emph{every} Boolean function $F$ in this setting, there is an
adversarial coalition of size $O(n/(\log n))$ that can bias the
output.  It was conjectured by~\citet{Friedgut} that Boolean functions
on the continuous cube $[0,1]^n$ (or equivalently, the discrete cube
$[n]^n$) can be biased by a coalition of size $o(n)$.
See~\cite{Hatami} and~\cite{Kalai} for advances on this line of
research.  As for multi-round coin-flipping protocols, \citet*{RSZ}
showed that any $o(\log^* n)$-round coin-flipping protocol can be
biased by a coalition of size $o(n)$.

Cryptography is a powerful tool for collective coin-flipping. While the most directly relevant related work is AlgoRand---which uses a mechanism similar to ours---we survey some other work first to set the stage. 

\paragraph{PVSS-based beacons.} 
Publicly verifiable secret sharing (PVSS) techniques have been
successfully used to design unbiasable, multi-round coin-flipping
protocols, such as Ouroboros~\cite{Ouroboros},
RandHound~\cite{RandHound}, Scrape~\cite{Scrape}, and
HydRand~\cite{HydRand}.  The coin-flipping in these protocols have two
(logical) rounds: first, a round where the players broadcast
commitments to their inputs, followed by a round where these
commitments are revealed. An advantage of these protocols is that they
can provide extremely strong guarantees on the quality of the
randomness---when the protocols are secure, they produce output values
that are indistinguishable from uniform. A difficulty is that they
demand $O(n^2)$ messages to be broadcast, which is intractable in
practice. A standard technique to mitigate this is to first elect a small
committee of players which then carry out the full PVSS. Of course,
such a procedure is immediately subject to attacks by an adaptive
adversary who can corrupt the committee once it is determined.

\paragraph{VDF-based beacons.} 
Verifiable Delay Functions (VDF)
\cite{vdf-boneh,vdf-wesolowski,vdf-pietrzak} may be used to construct
unpredictable, multi-round coin-flipping protocols (for a sketch of
such a possible construction see~\cite{vdf-ethereum}). Using VDFs in
this context is orthogonal to our techniques: a VDF can delay the
revelation of the beacon outcome to temper the opportunities the
adversary has to grind the beacon output. In general, tuning the
hardness parameter of the VDF to a high level aids this security
objective but naturally interferes with the availability (or liveness)
of the beacon.
% (in the sense of the immediate availability of the
% beacon outputs).
For this reason, a fine-grained analysis of the
randomness properties of the beacon is a precondition to the effective
use of a VDF in a practical beacon algorithm. Moreover, employing a
VDF will require a less cryptographic assumption (moderate hardness).
We leave the combination of VDF techniques with our randomness
generation algorithm as an interesting future research direction.

\paragraph{Threshold signatures.}
DFINITY~\cite{Dfinity} uses a one-round coin-flipping protocol based
on non-interactive $(t,n)$-Threshold BLS signatures~\cite{BLS}. 
In the setup phase, a random subset of $n$
players---a ``group''---is selected from a universe of $N$
players  
and 
key-pairs for the group members are established. 
After the setup, 
each player broadcasts his share 
and any player can
recover the (unique) beacon output from any $t$ shares.
%If there are $f$ adversarial players, taking
%$t \in [f + 1, n-f]$ ensures that the adversarial players cannot
%prevent a recovery and, importantly, they cannot predict the beacon
%output.  As for the iterated beacon, a random group is selected at
%each epoch from a family of groups configured during the setup phase.
%
%Note that this beacon is secure only against static
%adversaries.  Specifically, $n$ depends on the size of the adversarial
%coalition during the setup phase.  This is necessary: since groups are
%pre-configured, an adaptive adversary can enlarge his coalition over
%time and corrupt the group majority 
% violate the constraint $t \in [f+1, n-f]$ (by increasing $f$)
%at a future round with a constant probability. In contrast, our
%protocol is secure as long as the adversary controls less than $N/2$
%players.  In addition, our protocol is simpler: there is no committee
%or group selection.
The main disadvantage with respect to our techniques is of course
the complexity of the group setup which requires $\Omega(n^2)$ communication
for a single distributed key generation process between $n$ players.

\paragraph{Algorand.} 
Algorand~\cite{Algorand} uses a Byzantine Agreement-based multi-round
coin-flipping protocol where the players are equipped with verifiable
random functions (VRF).  While the multi-round beacon output is
uniform except with a negligible probability, the beacon output of a
single-round is uniform with only constant probability---thus
conditional min entopy is small. As mentioned above, our protocol can
be viewed as a parallelization of the AlgoRand technique; indeed, a
single-round of AlgoRand coin-flipping is equivalent to setting
$k = 1$ in our protocol.  In contrast, our one-round protocol
guarantees a ``small'' loss in min-entropy except with a negligible
probability.


\paragraph{VRF-based eventual consensus Proof-of-Stake blockchains.}
Many proof-of-stake blockchain
protocols, such as Ouroboros Praos~\cite{Praos}, 
Genesis~\cite{Genesis}, and Snow White~\cite {SnowWhite}, 
use VRFs 
in their coin-flipping protocols. 
This is a multi-round protocol 
where each ``block'' contains a nonce.  
The coin-flipping output is a cryptographic hash 
of the XOR of the nonces 
recorded in the ``common prefix'' of all 
blockchains held by the honest players 
at the end of the epoch. 
(Thus, these protocols do not use a broadcast channel.)
As we show in Appendix~\ref{sec:praos}, 
the min-entropy loss for this beacon 
is linear in the security parameter 
for an adversarial stake $10\%$ or higher.
Our treatment of this beacon 
is a first in the literature. 

\paragraph{Nakamoto-style PoW beacons.} 
\citet{ZuckermanBeacon} establish a lower bound for a class of
Nakamoto-style proof-of-work based beacons and discuss approaches for
a game-theoretic analysis of beacon protocols.



% \section{A technical overview of our analysis}
% To do

