
  We use $\log$ to denote the base-$2$ logarithm and $H()$ to denote the binary entropy function. 
  Since $z \leq d/3$, $S(d,z)$ is at most $2^{d-z}$ and consequently, 
  $S(d, z)^2 B_{d, \epsilon}(d-z)$ is at most
  \begin{align}\label{eq:tz-variance-small-d}
  t_z&\defeq 2^{2(d-z)} \cdot (1-\epsilon)^d 2^{-d} {d \choose z}  \left(\frac{1+\epsilon}{1-\epsilon}\right)^{z} \nonumber \\
  &= 2^d(1-\epsilon)^d {d \choose z} \left(\frac{1+\epsilon}{4(1-\epsilon)}\right)^{z}\,.
  \end{align}
  Our goal is to bound $t_z$ from above. 

  % \paragraph{Case: $\epsilon < 1/3$.} 
  % Now we are dealing with the situation where $d \leq d/3$ and $\epsilon < 1/3$. 
  Let $\alpha \defeq z/d$ and observe that $\alpha \in [0, 1/3]$ since $z \leq d/3$.
  Using $z = \alpha d$ in~\eqref{eq:tz-variance-small-d} and 
  using the simplified bound $\binom{d}{\alpha d} \leq 2^{H(\alpha)d}$ from~\eqref{eq:nck}, 
  we can write
  \begin{align*}
  t_z
  &= 2^{d}(1-\epsilon)^d {d \choose \alpha d} \left(\frac{1+\epsilon}{4(1-\epsilon)}\right)^{\alpha d} \\
  &\leq 2^d
  (1-\epsilon)^d 
  2^{H(\alpha)d} 
  \left(\frac{1+\epsilon}{4(1-\epsilon)}\right)^{\alpha}
  \,.
  \end{align*}
  If we define 
  \begin{align*}
  q(\alpha, \epsilon)
  &\defeq H(\alpha)+1-2\alpha+ \alpha \log \frac{1+\epsilon}{1-\epsilon}+\log(1-\epsilon)
  \,,
  \end{align*}
  we can write $t_z = t_{\alpha d} \leq 2^{q(\alpha, \epsilon)d}$. 
  Thus our goal is to bound $q(\alpha, \epsilon)$ from above. 

  The function $q(\alpha, \epsilon)$ must be a unimodal concave function in $\alpha$ since for a fixed $\epsilon$, 
  it is a linear combination of $\alpha$ and the entropy function which is a unimodal concave function. 
  The largest term in the sequence $\{t_z\}$ corresponds to some $\alpha=\alpha^*$ that maximizes $q(\alpha, \epsilon)$. 
  % In order to bound $t_z$, it suffices to bound $q(\alpha^*, \epsilon)$. 
  The partial derivative of $q(\alpha, \epsilon)$ with respect to $\alpha$ is
  \begin{align*}
  \frac{\partial}{\partial \alpha}q(\alpha, \epsilon) &= -\log \frac{\alpha}{1-\alpha} - 2 + \log \frac{1+\epsilon}{1-\epsilon} 
  \,.
  % \frac{\partial^2}{\partial \alpha^2}q(\alpha, \epsilon) &= -\frac{1}{\alpha} - \frac{1}{1-\alpha} < 0
  % q^{(3)}(\alpha, \epsilon) &= \frac{1}{\alpha^2} - \frac{1}{(1-\alpha)^2} \\
  % q^{(4)}(\alpha, \epsilon) &= -\frac{2}{\alpha^3} - \frac{2}{(1-\alpha)^3} < 0
  \end{align*}
  The solution to the equation $\frac{\partial}{\partial \alpha}q(\alpha, \epsilon) = 0$ is given by
  % \begin{align*}
  % &\log (\frac{1}{\alpha} - 1) = 2 + \log\frac{1-\epsilon}{1
  % +\epsilon}\\
  % \text{or, }& \frac{1}{\alpha}-1 = \frac{4(1-\epsilon)}{1
  % +\epsilon} \\
  % \text{or, }&  \alpha^* \defeq \alpha^*(\epsilon) = \frac{1}{1+\frac{4}{r}} =  \frac{r}{4+r}\,,\\
  % \end{align*}
  $\alpha^* = r/(4 + r)$ where $r \defeq (1+\epsilon)/(1-\epsilon) < (1+1/3)/(1-1/3) = 2$.
  % Since $q(\alpha, \epsilon)$ is a linear combination of an $\alpha$-concave function (entropy) and linear terms in $\alpha$, $\alpha^*$ must maximize $q(\alpha, \epsilon)$. 

  We can substitute the definition of the binary entropy function $H(.)$ 
  in the definition of $q(\alpha, \epsilon)$ to get 
  \begin{align*}
  q(\alpha^*, \epsilon) 
  &= -\log\left[
  \left(\alpha^*\right)^{\alpha^*}(1-{\alpha^*})^{1-{\alpha^*}}
  \right] 
  + \log r^{\alpha^*} + 1 - 2{\alpha^*} + \log(1-\epsilon) \\
  &= -\log\left[ \left(\frac{r}{4+r}\right)^{\frac{r}{4+r}} \left(\frac{4}{4+r}\right)^{\frac{4}{4+r}}\right] 
   - \log \left(r^{-\frac{r}{4+r}} \right) + 1 - \frac{2r}{4+r}+\log(1-\epsilon) \\
  &= -\frac{1}{4+r} \log(4^4) + \log(4+r)+\frac{4-r}{4+r} + \log(1-\epsilon) \\
  &= \log\left[(4+r)(1-\epsilon)\right] - 1 \\
  &= \log\left[\left(4+\frac{1+\epsilon}{1-\epsilon}\right)(1-\epsilon)\right] - 1 \\
  &= \log(5-3\epsilon) - 1  = \log\left( (5 - 3 \epsilon)/2 \right)
  % \\
  % \implies 2^{q(\alpha, \epsilon)}
  % &\leq \frac{5-3 \epsilon}{2}
  \,.
  \end{align*} 
  Since $q(\alpha, \epsilon) \leq q(\alpha^*, \epsilon)$, we have 
  \[
      t_z \leq 2^{q(\alpha, \epsilon)d} \leq 2^{q(\alpha^*, \epsilon)d} = \frac{5 - 3 \epsilon}{2}
      \,.
  \]
  Although this bound holds for $\epsilon \in (0, 1)$, we claim this bound for only $\epsilon < 1/3$ 
  since we can in fact obtain a tighter bound for $\epsilon \geq 1/3$. 


  \paragraph{A tighter bound for $\epsilon \geq 1/3$.} 
  % First consider the case when $\epsilon \geq 1/3$. 
  We start by observing that $t_z$ increases monotonically in $z$ if $\epsilon \geq 1/3$. 
  To see this, note that the ratio of two consecutive terms in the sequence $t_0, t_1, t_2, \ldots$ is
  \[
  % r(z) \defeq
  \frac{t_{z+1}}{t_z} 
  =\frac{ {d \choose z+1} (1+\epsilon) }{ {d \choose z} 4(1-\epsilon) } 
  \geq \frac{(d-z)(1+\epsilon)}{4(z+1) (1-\epsilon)}\, .
  \]
  Since $\epsilon \geq 1/3$, this ratio is at least $(d-z)/2(z+1)$ which is 
  strictly greater than one for $z+1 \leq d/3$. 
  Hence the largest term in the sequence $\{t_z\}$ would be the last one. 
  Therefore,
  \begin{align*}
  t_z
  &\leq t_{d/3} = 2^d (1-\epsilon)^d 
  {d \choose d/3} 
  \left(\frac{1+\epsilon}{4(1-\epsilon)}\right)^{d/3} \\
  &\leq 2^{d/3}
  \frac{2^{H(1/3)d}}{\sqrt{ \pi (d/3) (2/3)}} 
  (1+\epsilon)^{d/3}(1-\epsilon)^{2d/3}  \qquad \text{using Corollary~\ref{coro:nchoosek_1}} \\
  &= \left( 2^{1/3 + H(1/3)} (1+\epsilon)^{1/3} (1-\epsilon)^{2/3} \right)^d \frac{3}{\sqrt{2 \pi d}} \\
  &= \left( 2^{2/3} \phi(\epsilon) \right)^d \frac{3}{\sqrt{2 \pi d}} 
  \qquad \text{using (\ref{eq:phi_eps}) and since }2^{1/3 + H(1/3)} = 3/2^{1/3}\\
  &\leq \left( 2^{2/3} \phi(\epsilon) \right)^d
  \end{align*}
  since $d \geq 2$. 

  It is not hard to check that 
  $2^{2/3} \phi(\epsilon)$ is at most $(5 - 3\epsilon)/2$ for $\epsilon \geq 1/3$. 
  First, check that they are equal at $\epsilon = 1/3$ and 
  at $\epsilon = 1$, $2^{2/3}\phi(1) = 0 < (5 - 3\cdot 1)/2 = 1$.
  Next, we establish the claimed
  inequality for respective logarithms. 
  In particular, let 
  \[
      a(\epsilon) = \ln\left( 2^{2/3} \phi(\epsilon) \right)
      \,,\qquad\text{and}\qquad 
      b(\epsilon) = \ln\left( (5 - 3 \epsilon)/2\right)
      \,.
  \]
  Check that the respective derivatives are 
  $a^\prime(\epsilon) = -(1+3 \epsilon)/3(1 - \epsilon^2)$ 
  and $b^\prime(\epsilon) = - 3/(5 - 3\epsilon)$. 
  Since both derivatives are negative, it remains to check that $|a^\prime(\epsilon)| \geq |b^\prime(\epsilon)|$. 
  To that end, we claim that the ratio $|a^\prime(\epsilon)|/|b^\prime(\epsilon)|$ 
  is strictly positive for $\epsilon > 1/3$. 
  To see this, observe that the ratio equals $(5 + 12 \epsilon - 9 \epsilon^2)/9(1-\epsilon^2)$. 
  For $\epsilon = 1/3 + \delta$ for any $\delta > 0$, 
  the numerator is $5 + 12 (1/3 + \delta) - 9 \epsilon^2$, 
  or $9(1 - \epsilon^2) + 12\delta$. 
  Since the denominator is $9(1 - \epsilon^2)$, the ratio is positive for $\epsilon \in(1/3, 1)$. 

\hfill$\qed$