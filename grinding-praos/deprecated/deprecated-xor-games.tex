

\newcommand{\Paragraph}[1]{\\ \\ \noindent\textbf{{#1}}\ }
% \section{Grinding Power in XOR Target Games in a Special Case}

Observe that the quantity $\GrindingPower$ in the XOR target games depend only on the 
\emph{cardinality} of the sets involved. 
In addition, large sets imply a large grinding power for the player. 
A natural way to curb the grinding power is to generate these sets from a stochastic 
process which assigns a small probability to large sets. 
In this section, we consider stochastic processes that select these cardinalities independently 
from a geometric distribution. 

A geometric distribution with 
success parameter $1-\alpha$, where $\alpha \in (0, 1/2)$, is defined as follows.
In a sequence $1, 2, \ldots$ of independent Bernoulli trials, 
each with a success probability $1 - \alpha$, 
let the random variable $X \in \{1, 2, \ldots\}$ be the index of the first success.
We write $X \sim \GeomAlpha$ and the associated probability mass function is
$\Pr[X = k]\, = \alpha^{k-1}(1-\alpha)$ for $k = 1, 2, \ldots$. 


The exact description of the process is given below.

\begin{claim}[Nonce-sets with geometrically-distributed size]\label{claim:set-size-geometric}
Let $n, \kappa \in \NN$, $\alpha \in (0, 1/2)$ and 
let $C = \{c_1, \ldots, c_n\} \subset \{0,1\}^\kappa$ be a set of uniformly random strings. 
Let $A \subset \{1, 2, \ldots, |C|\}$ be an arbitrary set with $|A|/|C| = \alpha$. 
If $h = \min \{c_i \SuchThat i \not \in A\}$ and 
$X = |\{i \SuchThat c_i \leq h\}|$, 
then $X$ has the geometric distribution $\GeomAlpha$.
\end{claim}
\begin{proof}
\newcommand{\Rank}{\mathrm{rank}}
\newcommand{\Index}{\mathrm{index}}
Let $r_i$ be the rank of the element $c_i$ in $C$ in the lexicographic order and 
define $\Index(r) = i$ if and only if $\Rank(c_i) = r$.
It follows that $X = \Rank(h)$.
For every rank $r = 1, 2, \ldots, |C|$, define $Z_r = 1$ if $\Index(r) \in A$ and 
define $Z_r = 0$ otherwise.
Since elements in $C$ are independent and uniform, for every $r$ it follows that 
$Z_1, \ldots, Z_{|C|}$ are independent with $\Pr[Z_r = 1] = \alpha$. 
Specifically, $\Rank(h) = \min \{r \SuchThat Z_r = 0\} \sim \GeomAlpha$.
\end{proof}


The central goal of this section is to provide bounds on the bad probability 
in an epoch-by-epoch composition of XOR target games 
where the cardinality of each set is an independent copy 
of the random variable $Y = X - 1$ where $X \sim \GeomAlpha$. 
Observe that $Y$ can be zero, which means sets in the XOR target game are allowed to be empty.
In this section, XOR target games always have this property 
although we would not mention it explicitly. 
We start with recording some moment bounds for $X$.

\subsection{Moment bounds for the geometric distribution}
Recall that the polylogarithm function of order $\lambda$ is defined as
\begin{align}\label{eq:polylog}
    \Li_{\lambda}(\alpha) &\defeq \sum_{k=1}^\infty{ \frac{\alpha^k}{k^\lambda}}
\qquad \text{for}\quad \alpha, \lambda \in \RR
\end{align}
Polylogarithm functions of non-positive integer orders can be found using the recursive rule 
\[
\Li_1(\alpha) = -\log(1-\alpha)
\, , \qquad \text{and} \qquad
\dfrac{\partial}{\partial \alpha}{\Li_{\lambda}(\alpha)} = \frac{\Li_{\lambda-1}(\alpha)}{\alpha}
\, ,\quad \lambda \in 0, 1, 2, 3, \cdots
\, .
\] 
This allows us to compute
\begin{align}\label{eq:polylog-1-2}
    \Li_{-1}(\alpha) = \frac{\alpha}{(1-\alpha)^2}\, , \qquad \text{and} \qquad 
    \Li_{-2}(\alpha) = \frac{\alpha (1+\alpha)}{(1-\alpha)^3} 
    \, .
\end{align}
% The polylogarithm function lets us succintly express the moments of the random variable $X$. 
For any real $\lambda > 0$, we can express the moments of $X$ in terms of polylogarithm functions, as follows:
\begin{align}\label{eq:geom-moment}
\Exp[X^\lambda]
= \sum_{x=1}^\infty{  x^\lambda (1-\alpha) \alpha^{x-1} } 
= \frac{1-\alpha}{\alpha} \sum_{x=1}^\infty{ \frac{\alpha^{x}}{x^{-\lambda} } } 
= \frac{1-\alpha}{\alpha}\Li_{-\lambda}(\alpha)
\end{align}
It is also easy to check that 
\begin{align}\label{eq:geom-moment-x-minus-one}
\Exp[(X-1)^\lambda]
&= \alpha\,\Exp[X^\lambda]
\end{align}
The above formulae directly gives us the first and second moments of $X$, namely
\begin{align}\label{eq:geom-moments-1-2}
\Exp [X] = \frac{1}{1-\alpha}
\, , \quad 
\Exp[X^2] 
%= \frac{\alpha}{(1-\alpha)^2} + \frac{1}{(1-\alpha)^2}
= \frac{1 + \alpha}{(1-\alpha)^2} 
\, \quad \text{and}\quad
\Var[X] = \frac{\alpha}{(1-\alpha)^2}
\, .
\end{align}
However, for applications in this paper, we will need upper bounds on 
the fractional moments of $X$. 

\begin{proposition}[Fractional moment bound]\label{prop:geom-moment-bound}
For any real $\lambda \geq 1, \alpha \in (0, 1/2)$, and $X \sim \GeomAlpha$,
\begin{equation}\label{eq:geom-moment-bound}
    \Exp[X^\lambda] \leq \frac{\Gamma(1+\lambda)}{(1-\alpha)^\lambda} 
    \, ,
\end{equation}
and an equality is achieved with $\lambda = 1$ and $\alpha = 1/2$.
\end{proposition}
When $\lambda \in (1,2]$, we have the upper bound $\lambda = 1 + (\lambda - 1) \leq e^{\lambda - 1}$. 
Moreover, we can use simple linear upper bounds to the convex Gamma function and 
derive two useful approximations:
\begin{align}
    \Exp[X^\lambda] 
    &\leq \frac{\lambda}{(1-\alpha)^\lambda} \leq \frac{e^{\lambda-1}}{(1-\alpha)^\lambda}
        \,,\qquad\text{for}\quad 1 \leq \lambda \leq 2
        \label{eq:geom-moment-bound-approx-1}
        \,, \\
    \Exp[X^\lambda] 
    &\leq
        \frac{1+2\lambda}{3(1-\alpha)^\lambda}\,, \qquad\text{for}\quad 1 \leq \lambda \leq 3/2
        \label{eq:geom-moment-bound-approx-2}
        \,.
\end{align}

A critical fact is that although $\Exp[X^\lambda]$ is larger than $1$ for all $\lambda \geq 1$, 
$\Exp[(X-1)^\lambda]$ is in fact less than $1$ for \emph{some} $\lambda \in (1, 2]$. 
Since any positive power of the latter moment is at most one, it will greatly simplify our analysis. 
Let us define
\begin{align}\label{eq:lambda-star}
 \lambda^*(\alpha) &= \begin{cases}
    2\,, & \text{for}\quad 0 < \alpha \leq 1/4 \, ,\\
    3/2\,, & \text{for}\quad 1/4 < \alpha \leq 1/3 \, ,\\
    % (1-\log \alpha)/(1 - \log(1-\alpha)\,)\,, & \text{for}\quad 1/4 \leq \alpha < 1/2 \, .
    (2/5)(6 - 7 \alpha)\,, & \text{for}\quad 1/3 < \alpha < 1/2 \, .
\end{cases}
\end{align}

On the other hand, for any $c > 1$, let us define
\begin{align}\label{eq:lambda-star-c}
    \lambda^*_c(\alpha) &= \begin{cases}
    2\,, & \text{for}\quad 0 < \alpha \leq 1/4 \, ,\\
    3/2\,, & \text{for}\quad 1/4 < \alpha \leq 1/3 \, ,\\
    \frac{b + \ln(3/2\alpha) + \ln(1-1/c)}{a-\ln(1-\alpha)}
        \,, &\text{for}\quad 1/3 < \alpha < \frac{c-1}{2c-1}\,, \quad\text{where}
    \end{cases}\\
        a&= 1 - 2\left( \ln(3/2)-\ln(1+1/2\sqrt{e}) \right)\,,\quad\text{and} \nonumber \\
        b&=1-3\ln(3/2)+2\ln(1+1/2\sqrt{e}) \nonumber \,.
\end{align}
The above $a,b$ above satisfies
\begin{align}\label{eq:lambda-a-b-upperbound}
    1 + 2e^{x-1} \leq 2 e^{a x - b}
    \qquad\text{for}\quad x \in (1, 3/2)
    \, .
\end{align}

\begin{proposition}[Moment convergence]\label{prop:li-convergence}
The geometric random variable $X \sim \GeomAlpha$ satisfies
\begin{equation}\label{eq:li-convergence}
    \Exp[(X-1)^\lambda] 
    = \alpha \Exp[X^\lambda]
    % =(1-\alpha) \Li_{-\lambda}(\alpha) 
    < \begin{cases}
        5/9\,, &\quad\text{if}\quad 0 < \alpha \leq 1/4\,,\\
        5/6\,, &\quad\text{if}\quad 1/4 < \alpha \leq 1/3\,,\\
        1\,, &\quad\text{if}\quad 1/3 < \alpha \leq 1/2\,.
    \end{cases}
\end{equation}
for any $\lambda$ satisfying $1 < \lambda \leq \lambda^*(\alpha)$.
Moreover, 
\[
    \alpha \Exp[X^\lambda] < c\,, 
\]
where $c$ is an absolute positive constant greater than one, 
if $1/3 < \alpha < (c-1)/(2c-1)$ and $1 < \lambda \leq \lambda_c^*(\alpha)$.
% Specifically, if $\alpha < 1/2 - \theta$ for some $\theta \in (0, 1/6)$ then
% \begin{align}
%     \Exp[(X-1)^\lambda] \leq (1 - 2\theta)/6\cdot 2^\lambda (1 + 2\theta)^{-\lambda}(1+2\lambda)
%     \,,
% \end{align}
% which is at most $13/14$ if $\alpha \leq 0.49$.
\end{proposition}
We defer the proof of Propositions~\ref{prop:geom-moment-bound} and~\ref{prop:li-convergence} till Section~\ref{sec:geom-moment-bound-proof}.

% \begin{corollary}[Moment bound for a single slice]\label{coro:geom-moment-slice}
% Let $X \sim \GeomAlpha$ for $\alpha \in (0, 1/2)$. 
% If $\lambda^* = \lambda^*(\alpha)$ from Proposition~\ref{prop:li-convergence} then
% \begin{align}\label{eq:geom-moment-slice}
%     \Exp[X^{\lambda^*}] &< \frac{1}{\alpha}
%     \, .
% \end{align}
% Specifically, for any $\lambda \in (1,2]$,
% \begin{align}\label{eq:geom-moment-slice-absolute}
%     \Exp[X^{\lambda}] 
%     < \begin{cases}
%          32/9\, , & \text{for}\quad 0 < \alpha < 1/4\, , \\
%          4\, ,    & \text{for}\quad 1/4 \leq \alpha < 1/2\, .
%       \end{cases}
% \end{align}
% We also have
% \begin{align}\label{eq:scaled-polylog}
%     (1-\alpha) \Li_{-\lambda}(\alpha) 
%     < \begin{cases}
%          8/9\, ,   & \text{for}\quad 0 < \alpha < 1/4\, , \\
%          1\, ,      & \text{for}\quad 1/4 \leq \alpha < 1/2\, .
%       \end{cases}
% \end{align}
% \end{corollary}
% \begin{proof}
% Equation~\ref{eq:geom-moment-slice} is obtained by
% dividing the both sides of the statement of the proposition by $\alpha$.

% Since the derivative of $\Exp[X^\lambda]$ w.r.t. $\lambda$ is positive for $\lambda > 1$, 
% $\Exp[X^\lambda] \leq \Exp[X^{\lambda^*}]$.
% When $\alpha < 1/4$, we have $\lambda \leq \lambda^* = 2$ and 
% Equation \eqref{eq:geom-moment-bound-relaxed} gives us the bound $2/(1- 1/4)^2 = 32/9$.
% On the other hand, if $\alpha \geq 1/4$, we have $\Exp[X^\lambda] \leq \Exp[X^{\lambda^*}] < 1/\alpha \leq 4$.

% Since \eqref{eq:geom-moment} implies $(1-\alpha) \Li_{-\lambda}(\alpha) = \alpha \Exp[X^\lambda]$, 
% the two cases of \eqref{eq:scaled-polylog} are obtained by using 
% $\alpha < 1/4$ in \eqref{eq:geom-moment-slice}, and $\alpha \geq 1/4, \lambda^* \leq 2$ in \eqref{eq:geom-moment-slice}.
% \end{proof}

%=================================================================





% \begin{claim}
% $\Exp[g^{\lambda^*}] \alpha^{-k}$ where $\lambda^* = \lambda^*(\alpha)$ is defined in~\ref{eq:lambda-star}.
% \end{claim}
% \begin{proof}
% In Game 1, we have $g = \prod_{i=1}^\EpochLength{X}$ so that 
% \[
%     \Exp[g^\lambda] 
%     = \Exp (\prod_{i=1}^\EpochLength{X} )^\lambda 
%     = \Exp[X^\lambda]^\EpochLength 
% \]
% thanks to the independence of the slices.
% Next, using Corollary~\ref{coro:geom-moment-slice}, we conclude that 
% \[
%     \Exp[g^{\lambda^*}] \leq \left(\frac{1}{\alpha}\right)^K
%     \, .
% \]
% \end{proof}

\subsection{Moment bounds for the XOR target games}

\begin{lemma}[Moment bound for the basic XOR target game]\label{lemma:xor-game-basic-moment}
Let $g$ be the grinding power in the basic XOR target game as described in this section.
Let $\lambda = \lambda_c^*(\alpha)$ from \eqref{eq:lambda-star-c} for some large positive constant $c$. 
\begin{align}
    \Exp[g^\lambda]
    &\leq \begin{cases}
        9(1-\alpha)/4\,, & \quad\text{if}\quad \alpha \leq 1/4\,, \\
        6(1-\alpha)\,, & \quad\text{if}\quad 1/4 < \alpha \leq 1/3\,, \\
        O(1)\,, & \quad\text{if}\quad 1/3 < \alpha < 1/2\,.
    \end{cases}
\end{align}
% In addition, the constant in the last expression is at most $14$ if $\alpha \leq 0.49$.
\end{lemma}


\begin{lemma}[Moment bound for the $\ell$-block XOR target game]\label{lemma:xor-game-block-moment-bound}
Let $\alpha \in (0, 1/2)$ and $\lambda = \lambda^*(\alpha)$ from~\eqref{eq:lambda-star}. 
Let $X \sim \GeomAlpha$ and $Y = X - 1$.
% Let $\EpsP < 0$ be arbitrary. 
Consider the $\ell$-block XOR target game $G(T; S_1, \ldots, S_n)$ 
with a uniform initial value where each $|S_i|$ is an independent copy of $Y$. 
Let $g$ be the grinding power of this game. 
If at least one of the sets $S_1, \ldots, S_n$ is empty then
\[
    \Exp[g^{\lambda}] \leq \begin{cases}
        \dfrac{\left(\Exp[X^\lambda]\right)^\ell}{1 - \alpha \Exp[X^\lambda]}
            \,, &\quad\text{if}\quad \alpha \leq 1/3\,, \\
        (n/\ell)\,\left(\Exp[X^\lambda]\right)^\ell
            \,, &\quad\text{if}\quad 1/3 < \alpha < 1/2\,.
    \end{cases}
\]
% Moreover, $\Pr[g > \GrindingMax] \leq \EpsP$ where 
% \begin{align*}
%     \GrindingMax(\alpha) &\leq \begin{cases}
%         \left( 9/2 \right)^{1/2}/(1-\alpha)\cdot \EpsP^{-1/2}
%         \,,&\quad\text{if\quad $\alpha < 1/4$}\,, \\
%         9^{2/3}/(1-\alpha)\cdot \EpsP^{-2/3}
%         \,,&\quad\text{if\quad $1/4 < \alpha \leq 1/3$}\,, \\
%         \dfrac{\left( (n/\ell) (29 - 28\alpha)/15\right)^{5/(12 - 14\alpha)}}{1-\alpha}\cdot \EpsP^{-5/(12 - 14\alpha)}
%         \,,&\quad\text{if\quad $1/4 < \alpha < 1/2$}\,.
%     \end{cases}
% \end{align*}

\end{lemma}


\begin{lemma}[Moment bound for the $\ell$-lookahead XOR target game]\label{lemma:xor-game-lookahead-moment}
Let $\alpha \in (0, 1/2)$ and $\lambda = \lambda^*(\alpha)$ from~\eqref{eq:lambda-star}. 
Let $X \sim \GeomAlpha$ and $Y = X - 1$.
Consider the $\ell$-block XOR target game $G(T\,;\, S_1, \ldots, S_\ell\,;\, P_1, \ldots, P_n)$ 
with a uniform initial value where 
% additionally consider independent and identically distributed random variables $Y_1, \ldots, Y_{\NumSets + \BlockLength}$ 
each $|S_i|$ and $|P_j|$ 
is an independent copy of $Y$ for $i \in [\ell], j \in [n]$.
If at least one of the sets $P_1, \ldots, P_n$ is empty then 
\[
    \Exp[g^\lambda] \leq 
        % (2n)^{1+\lambda} 
        (2n)^{\lambda} 
        \left(\Exp X^\lambda \right)^{\min(n, \ell)}
    \,.
\]
\end{lemma}

\subsection{Proofs for the moment bounds}
\subsubsection*{Proof of Lemma~\ref{lemma:xor-game-basic-moment}}
\begin{proof}
We have a random variable $g = \prod_{i=1}^t x_i$ where $t$ is geometrically distributed with rate $\alpha$, $P(t) = (1-\alpha)\alpha^t$, and each $x_i$ is geometrically distributed with rate $\alpha$ but conditioned on $x_i \ge 1$, so $Q(x_i) = (1-\alpha) \alpha^{x_i-1}$.

Let us fix $t$. Then since the $x_i$ are independent, for any $\lambda$ we have
\[
    \Exp[g^\lambda \mid t] 
    = \Exp[x^\lambda]^t 
    = \left( \frac{1-\alpha}{\alpha} \,\Li_{-\lambda}(\alpha) \right)^t
\]
where $\Li$ is the polylogarithm function defined in \eqref{eq:polylog}.
Summing this over all $t$ gives
\begin{equation}
\label{eq:sumovert}
\Exp[g^\lambda] = \sum_{t} P(t) \Exp[g^\lambda \mid t] 
%\le b^{-\lambda} (1-\alpha) \sum_{t=0}^\infty \left( \frac{\alpha \Gamma(\lambda+1)}{(1-\alpha)^\lambda} \right)^{\!t} 
\le (1-\alpha) \sum_{t=0}^\infty \big( (1-\alpha) \,\Li_{-\lambda}(\alpha) \big)^{t} 
= (1-\alpha)\,\sum_{t=0}^\infty r_\lambda(\alpha)^{t} 
\, ,
\end{equation}
where we define $r_\lambda(\alpha) \defeq (1-\alpha) \,\Li_{-\lambda}(\alpha)$.
By Proposition~\ref{prop:li-convergence}, $r_\lambda(\alpha) = \alpha \Exp[X^\lambda] < 1$ 
and the geometric sum in~\eqref{eq:sumovert} converges to $1/(1-r_\lambda(\alpha) )$. 
Specifically, if $\alpha < 1/4$ the sum is at most $1/(1-5/9) = 9/4$, 
if $\alpha \in (1/4, 1/3]$ the sum is at most $1/(1-5/6) = 6$, and finally, 
the sum is at most $c$ if $1/3 < \alpha < (c-1)/(2c-1)$.


% \par\noindent
% \textbf{Remark. }
% If the reader finds the polylogarithm function unfamiliar, the moments of the geometric distribution can be bounded in terms of the Gamma function as follows. For integer $\lambda$ we have
% \[
% \Exp[x^\lambda] \le \Exp[x(x+1)(x+2)\cdots(x+\lambda-1)] = \frac{\lambda!}{(1-\alpha)^\lambda} \, .
% \]
% There is a nice probabilistic interpretation of this\ldots 
% For noninteger $\lambda$ the rising product becomes the Pochhammer symbol and $\lambda!$ becomes the Gamma function,
% \[
% \Exp[x^\lambda] \le \frac{\Gamma(\lambda+1)}{(1-\alpha)^\lambda} \, .
% \]

\end{proof}

\subsubsection*{Proof of Lemma~\ref{lemma:xor-game-block-moment-bound}}
\begin{proof}
Let $m = n/\ell$ be an integer.
Let $|S_i| = Y_i$ where $Y_i$ is an independent copy of $Y = X - 1$ and $X \sim \GeomAlpha$. 
Using the terminology of Theorem~\ref{thm:xor-game-block}, 
the expression of the grinding power in~\eqref{eq:xor-game-block-gp} simplifies as
\[
    \GrindingPower(h) = \prod_{j=1}^\ell{X} \cdot \prod_{t=m-h+2}^m(Y \Given Y \geq 1)^\ell = X^\ell X^{(h-1)\ell}
\]
due to independence.
Since $\Pr[|S_i| = 0] = 1-\alpha$, it follows that 
$\Pr[h] \leq (1 - \alpha^\ell)\alpha^{(h-1)\ell} \leq \alpha^{(h-1)\ell}$.
Then for $\lambda = \lambda^*$ from~\ref{eq:lambda-star}, we have
\begin{align*}
    \Exp[g^\lambda]
    &= \sum_{b = 1}^{m} {\Pr[h] \, 
        \Exp \left[ g(h)^\lambda \Given h\right]} 
    \leq  \sum_{h = 1}^{m} {  \alpha^{(h-1)\ell} \cdot \Exp \left(X^\ell X^{(h-1)\ell} \right)^\lambda } \\
    &\leq  \sum_{h = 1}^{m} {  \alpha^{(h-1)\ell} \cdot \left(\Exp[X^\lambda]\right)^\ell \left( \Exp[X^\lambda] \right)^{(h-1)\ell} } 
    = \left(\Exp[X^\lambda]\right)^\ell \sum_{h = 1}^{m} { \left(\alpha \Exp[X^\lambda]\right)^{(h-1)\ell} } \,,\\
    % &\leq \left(\Exp[X^\lambda]\right)^\ell \min\left( m, \frac{1}{1 - \left(\alpha \Exp[X^\lambda]\right)^\ell } \right) 
    % \leq \left(\Exp[X^\lambda]\right)^\ell \min\left( m, \frac{1}{1 - \alpha \Exp[X^\lambda]} \right) 
\end{align*}
where we have used the independence of $|S_1|, \ldots, |S_n|$. 
Observe that the sum converges since $\alpha \Exp[X^\lambda] < 1$ by~\eqref{eq:li-convergence}. 
Since $1/(1-x^\ell) \leq 1/(1-x)$ for $0< x < 1$ and $\ell \geq 1$, 
we get the following bounds via~\eqref{eq:li-convergence}:
When $\alpha \leq 1/4, \alpha \Exp[X^\lambda] \leq 5/9$ and consequently, 
$\Exp[g^\lambda] \leq \Exp[X^\lambda]^\ell \cdot 1/(1-5/9) = (9/4) \Exp[X^\lambda]^\ell$. 
Similarly, when $1/4 < \alpha \leq 1/3, \alpha \Exp[X^\lambda] \leq 5/6$ and consequently,
$\Exp[g^\lambda] \leq 6 \Exp[X^\lambda]$. 
When $1/3 < \alpha < 1/2$, $1/(1-\alpha \Exp[X^\lambda])$ grows quickly as $\alpha \rightarrow 1/2$. 
In this case, therefore, we use the upper bound $\Exp[g^\lambda] \leq m\, \Exp[X^\lambda]$.
\end{proof}

\subsubsection*{Proof of Lemma~\ref{lemma:xor-game-lookahead-moment}}
\begin{proof}
Setting $h = \max\{i \in [n] \SuchThat |P_i| = 0\}$, 
consider the expression of the grinding power in~\ref{eq:xor-game-lookahead-gp}. 
Since $|P_h| = 0$ and $|P_i| \geq 0$ for all $i \geq h+1$, it follows that 
the grinding power conditioned on a specific $h$ is
$g_h = A_h + B_h$, where
\begin{align*}
    A_h &= \sum_{k=h}^n 
        \left(\prod_{i=1}^{k-1}(1 + |P_i|)\right) \cdot  
        \prod_{i=k}^\ell |S_i| \cdot 
        \prod_{i=k+1}^\NumSets |P_i| \,,
         \nonumber \\
    B_h &= \sum_{k=h}^n 
        \left(\prod_{i=k - \ell}^{k-1}(1 + |P_i|)\right) \cdot
        \prod_{i=k+1}^\NumSets |P_i|
    \,.
\end{align*}
In addition, $A_h$ vanishes if $h \geq \ell$ and $B_h$ vanishes if $n \leq \ell$. 
Since $|S_i|, |P_i|$ are independent copies of the random variable $Y$, the expressions above can be written as
\[
    A_h = \sum_{k=h}^n 
        X^{k-1} Y^{\ell - k + 1} Y^{n - k} 
        = \sum_{k=h}^n X^{k-1} Y^{n + \ell - 2k + 1}
        \,,\qquad
    B_h = \sum_{k=h}^n 
        X^\ell Y^{n - k}
    \,.
\]
Let $H$ denote the index $h \in [n]$ so that $|P_h| = 0$ and $|P_i| \geq 1$ for all $h < i \leq n$. 
Then $\Pr[H = h]\, = (1-\alpha) \alpha^{n-h}$ since $\Pr[Y = 0] = 1 - \alpha$. 
Now we have all the ingredients to bound $\Exp g^\lambda$ where $\lambda = \lambda^*$ from~\eqref{eq:lambda-star}. 

\paragraph{Case: $n \leq \ell$.}
In this case, $g_h = A_h$. 
In addition, since $X > Y \geq 0$, it follows that $Y^{a+b} X^c \leq Y^a X^{b+c}$ 
for any non-negative $a,b,c$, it follows that
\[
    A_h = \sum_{k=h}^n Y^{n+\ell-2k+1} X^{k-1} \leq (n - h + 1)\, Y^{\ell - n + 1} X^{n-1}\,,
\]
and
\begin{align*}
    \Exp g^\lambda 
        &= \sum_{h=1}^n \Pr[H = h] \Exp A_h^\lambda \\
        &\leq \sum_{h=1}^n (1-\alpha) \alpha^{n-h} (n - h + 1)^\lambda \Exp \left(Y^{\ell - n + 1} X^{n-1}\right)^\lambda \\
        &\leq  (1-\alpha)  \left(\Exp Y^\lambda\right)^{\ell - n + 1} \left(\Exp X^\lambda \right)^{n-1} 
            \sum_{h=1}^n \alpha^{n-h} (n - h + 1)^\lambda \\
        &\leq  (1-\alpha)  \left(\Exp X^\lambda \right)^{n-1} 
            \frac{n^\lambda}{1-\alpha} \\
        &= n^\lambda \left(\Exp X^\lambda \right)^{n-1} \,.
\end{align*}
Here, we have used the independence of the numbers $|S_i|, |P_j|$ for $i \in [\ell], j \in [n]$ 
and the fact that $\Exp Y^\lambda < 1$ by~\eqref{eq:li-convergence}.

\paragraph{Case: $\ell < n$.} 
We consider two sub-cases: $h \leq \ell$ and $h \geq \ell + 1$.
In the former case,
\begin{align*}
    g_h = A_h + B_h 
    &= \sum_{k=h}^\ell Y^{n+\ell - 2k + 1}X^{k-1} + \sum_{k=h}^n Y^{n-k} X^\ell    \\
    &\leq (\ell - h + 1)\, Y^{n-\ell+1} X^{\ell - 1} + (n-h+1)\, Y^{n-h} X^\ell \\
    &\leq 2 (n - h + 1)\, Y^{n-h + 1} X^\ell\,,
\end{align*}
using $h \leq \ell$. 
Hence
\begin{align*}
    % \Pr[h \leq \ell]\cdot \Exp [g^\lambda \Given h \leq \ell]
    %     &= 
        \sum_{h = 1}^\ell \Pr[H = h] \Exp g_h^\lambda 
        &\leq \sum_{h=1}^\ell 
            (1-\alpha)\alpha^{n-h}\cdot 
            2^\lambda 
            (n-h+1)^\lambda 
            \Exp\left( Y^{n-h+1} X^\ell \right)^\lambda \\
        &= 2^\lambda (1-\alpha) \sum_{h=1}^\ell 
            \alpha^{n-h} 
            (n-h+1)^\lambda 
            \left(\Exp Y^\lambda \right)^{n-h+1} 
            \left(\Exp X^\lambda\right)^\ell \\
        &= 2^\lambda (1-\alpha) \left(\Exp X^\lambda\right)^\ell \sum_{h=1}^\ell \alpha^{n-h}\cdot (n-h+1)^\lambda \\
        &\leq 2^\lambda (1-\alpha) \left(\Exp X^\lambda\right)^\ell \frac{n^\lambda}{1-\alpha} \\
        &= (2n)^\lambda \left(\Exp X^\lambda\right)^\ell
        \,.
\end{align*}

In the other sub-case, that is, when $\ell < h \leq n$, 
\[
    g_h = B_h = \sum_{k = h}^n Y^{n-k} X^\ell \leq (n - h + 1) Y^{n-h} X^\ell \,,
\]
which implies
\begin{align*}
    % \Pr[h \geq \ell + 1]\cdot \Exp [g^\lambda \Given h \geq \ell+1]
    %     &= 
        \sum_{h = \ell + 1}^n \Pr[H = h] \Exp g_h^\lambda 
        &\leq \sum_{h=\ell+1}^n 
            (1-\alpha)\alpha^{n-h}\cdot 
            (n-h+1)^\lambda 
            \Exp\left( Y^{n-h} X^\ell \right)^\lambda \\
        &= (1-\alpha) \left(\Exp X^\lambda\right)^\ell \sum_{h=\ell+1}^n 
            \alpha^{n-h} 
            (n-h+1)^\lambda 
            \left(\Exp Y^\lambda \right)^{n-h} \\
        &\leq (1-\alpha) \left(\Exp X^\lambda\right)^\ell 
            % (n - \ell)
            \frac{(n-\ell)^\lambda}{1-\alpha} \\
        % &= (n - \ell)^{1+\lambda} \left(\Exp X^\lambda\right)^\ell 
        &= (n - \ell)^{\lambda} \left(\Exp X^\lambda\right)^\ell 
        \,.
\end{align*}
Combining these two sub-cases for the case $\ell + 1 \leq n$, we have
\begin{align*}
    \Exp g^\lambda 
        &= \sum_{h = 1}^\ell \Pr[H = h] \Exp\left[g_h^\lambda \Given h \leq \ell \right] 
            + \sum_{h = \ell + 1}^n \Pr[H = h] \Exp\left[g_h^\lambda \Given \ell+1 \leq h \right] \\
        &\leq (2n)^\lambda \left(\Exp X^\lambda\right)^\ell +
            (n - \ell)^{\lambda} \left(\Exp X^\lambda\right)^\ell \\
        &\leq (2^{\lambda} + 1) n^\lambda \left(\Exp X^\lambda\right)^\ell \\
        &\leq (2n)^{\lambda} \left(\Exp X^\lambda\right)^\ell
        \,.
\end{align*}
The bound in the claim follows by comparing the bounds from the two cases above. 
\end{proof}



\subsection{Composition of XOR target games}

\begin{theorem}[Composing basic XOR target games]\label{thm:basic-xor-game-geometric}
Fix $\EpsP < 0$.  
Let $\alpha \in (0, 1/2)$, $c > 1$ any large positive constant, 
and $\lambda = \lambda_c^*(\alpha)$ from~\eqref{eq:lambda-star-c}. 
Consider the basic XOR target game $G(T; S_1, \ldots, S_n)$ 
with uniform initial value where $|S_i|$ are i.i.d. random variables with distribution $\GeomAlpha$. 
The grinding power $g$ of this game satisfies $\Pr[g > \gamma] \leq \EpsP$, where
\[
\gamma = 
    \begin{cases}
        O(1)\cdot \EpsP^{-1/2}\, , 
            & \text{for}\quad 0 < \alpha \leq 1/4, \\
        O(1)\cdot \EpsP^{-2/3}\, , 
            & \text{for}\quad 1/4 < \alpha \leq 1/3, \\
        O(1)\cdot \EpsP^{ - 1/\lambda_c^*}\, ,
            &\text{for}\quad 1/4 < \alpha < 1/2 - (c-1)/(2c-1)\, .
    \end{cases}
\]
Let $B$ be the bad event in an $L$-epoch composition of this game (cf. Lemma~\ref{lemma:composition}). 
Then
\[
\Pr[B] \leq 
    \begin{cases}
        O(1)\cdot L\, \EpsP^{1/2}\, , 
            & \text{for}\quad 0 < \alpha \leq 1/4, \\
        O(1)\cdot L\, \EpsP^{1/3}\, , 
            & \text{for}\quad 1/4 < \alpha \leq 1/3, \\
        O(1)\cdot L\, \EpsP^{ 1 - 1/\lambda_c^* }\, ,
            &\text{for}\quad 1/4 < \alpha < 1/2 - (c-1)/(2c-1)\, . 
    \end{cases}
\]
\end{theorem}
\begin{proof}
The expression of $\gamma$ follows directly from Fact~\ref{fact:grinding-max} and Lemma~\ref{lemma:xor-game-basic-moment}, 
and observing that $1- \alpha < 1$.
The rest follows from \eqref{eq:delta-composition-generic}.
\end{proof}

\begin{corollary}[Grinding power bound in the basic XOR target game]
In the context of Lemma~\ref{lemma:xor-game-block-moment-bound}, $\Pr[g > \GrindingMax] \leq \EpsP$ where 
\begin{align*}
    \GrindingMax(\alpha) &\leq \begin{cases}
        \left( 9/2 \right)^{1/2}/(1-\alpha)\cdot \EpsP^{-1/2}
        \,,&\quad\text{if\quad $\alpha < 1/4$}\,, \\
        9^{2/3}/(1-\alpha)\cdot \EpsP^{-2/3}
        \,,&\quad\text{if\quad $1/4 < \alpha \leq 1/3$}\,, \\
        \dfrac{\left( (n/\ell) (29 - 28\alpha)/15\right)^{5/(12 - 14\alpha)}}{1-\alpha}\cdot \EpsP^{-5/(12 - 14\alpha)}
        \,,&\quad\text{if\quad $1/4 < \alpha < 1/2$}\,.
    \end{cases}
\end{align*}
\end{corollary}
\begin{proof}
By substituting these expressions for $\Exp[g^\lambda]$ in Fact~\ref{fact:grinding-max}, we get 
\begin{align*}
    \GrindingMax(\alpha) &\leq \begin{cases}
        \left( 9\lambda/4 \right)^{1/\lambda}/(1-\alpha)^\ell\cdot \EpsP^{-1/\lambda}
        \,,&\quad\text{if\quad $\alpha < 1/4$}\,, \\
        (6\lambda)^{1/\lambda}/(1-\alpha)^\ell\cdot \EpsP^{-1/\lambda}
        \,,&\quad\text{if\quad $1/4 < \alpha \leq 1/3$}\,, \\
        \dfrac{m^{1/\lambda}\, \left( (1 + 2\lambda)/3\right)^{\ell/\lambda}}{(1-\alpha)^\ell}\cdot \EpsP^{-1/\lambda}
        \,,&\quad\text{if\quad $1/4 < \alpha < 1/2$}\,.
    \end{cases}
\end{align*}
The expression of $\GrindingMax$ in the claim follows from using $\lambda = \lambda^*$ from~\eqref{eq:lambda-star}. 
Specifically, in the first case, we use $\lambda = 1/2$ and $\Exp[X^\lambda] \leq \lambda/(1-\alpha)^\lambda$ using~\eqref{eq:geom-moment-bound-approx-1}. 
The second case is similar except for $\lambda = 3/2$. 
In the third case, we use the bound~\eqref{eq:geom-moment-bound-approx-2}, i.e., 
$\Exp[X^\lambda] \leq (1+2\lambda)/3(1-\alpha)^\lambda$ with $\lambda = 12/5 - 14\alpha/5$. 
\end{proof}

\begin{theorem}[Composing $\ell$-block XOR target games]\label{thm:xor-game-block-composition-prob}
In the context of the $\ell$-block XOR target game in Lemma~\ref{lemma:xor-game-block-moment-bound}, 
suppose $0 < \alpha \leq 1/2 - \theta$ for a small positive constant $\theta$ at most $1/6$.
Let $k = \ln(1/\EpsP), m = n/\ell$, and 
suppose $\ell$ and $n$ satisfy 
\[
    \ell \leq 4k/9 - 2\,, \qquad 
    n \geq \lceil k/\ln(2) \rceil \,, \qquad
    \theta > (\ell + \ln m)/2k
    \,.
\] 
In an $L$-epoch composition of this game, 
the bad probability is at most $3 L \EpsP^{\delta}$, for $\delta \in (0, 1)$, 
where
\[
    \delta \geq \begin{cases}
        1/2 - \dfrac{1}{k}\left( \ell  \ln(1/(1- \alpha)) - \ln(3/\sqrt{2}) \right)
            \,, & \quad \text{if}\quad \alpha \leq 1/4\,, \\[1em]
        1/3 - \dfrac{1}{k}\left(\ell \ln(1/(1- \alpha)) - \ln(3) \right)
            \,, & \quad \text{if}\quad 1/4 < \alpha \leq 1/3\,, \\[1em]
        1 - \dfrac{1}{k}\left( \ell \ln(1/(1-\alpha)) + 
            \dfrac{5}{2}\cdot \dfrac{k + \ln m - 2\ell/3}{6-7\alpha} 
            + 2\ell/3 \right)
            \,, & \quad\text{if}\quad 
            % 1/3 < \alpha <  \frac{1}{2} - \frac{5}{14}\cdot \frac{\ell \ln 2 + \ln m}{k - \ell(2/3 + \ln 2)}
            1/3 < \alpha \leq 1/2 - \theta
            \,.
    \end{cases}
\]
\end{theorem}
\begin{proof}
We consider the cases $\alpha \leq 1/4, \alpha \in (1/4, 1/3]$, and $\alpha > 1/3$ separately.

\paragraph{Case: $1/3 < \alpha < 1/2$.}
By~\eqref{eq:delta-composition-generic}, $\delta$ is 
\begin{align*}
    \delta
    &= 1 - \frac{1}{\lambda}\left(1 + \frac{1}{k} \ln \Exp g^\lambda \right) \\
    &= 1 - \frac{1}{\lambda}\left(1 + \frac{1}{k} \ln \left( m (\Exp X^\lambda)^\ell\right) \right) \\
    &= 1 - \frac{1}{\lambda} - \frac{\ln m}{k \lambda} - \frac{\ell}{k \lambda} \ln (\Exp X^\lambda)
    \,.
\end{align*}
Since each term in the above expression is positive (in particular, $\Exp X^\lambda > 1$), 
it easily follows that $\delta < 1$. 
On the other hand, 
\[
    \ln (\Exp X^\lambda) \leq \ln \frac{1+2\lambda}{3(1-\alpha)^\lambda} = \ln ((1+2\lambda)/3) - \lambda \ln(1-\alpha)
    \,.
\]
Note that $\ln((1+2 \lambda)/3) = \ln(1+ 2(\lambda-1)/3) \leq 2(\lambda - 1)/3$ for $\lambda \in (1, 3/2]$. 
In addition, since $\alpha < 1/2$, $-\ln(1-\alpha)$ is at most $\ln 2$. 
This implies 
$
    \ln (\Exp X^\lambda) \leq 2(\lambda-1)/3 + \lambda \ln(2)
$, which gives
\begin{align*}
    \delta 
    &\geq 
        1 
        - \frac{1}{\lambda} 
        - \frac{\ln m}{k \lambda} 
        - \frac{\ell}{k \lambda} \left( 2(\lambda-1)/3 + \lambda \ln(2) \right) \\
   &=   1 
        - \frac{1}{\lambda} 
        - \frac{\ln m}{k \lambda} 
        - \frac{2\ell}{3k} +\frac{2\ell}{3 k\lambda} - \frac{\ell \ln(2)}{k} \\
   &=   \left(1 
        - \frac{1}{\lambda} 
        - \frac{2\ell}{3k} \right)
        - \frac{\ln m}{k \lambda} 
        +\frac{2\ell}{3 k\lambda} 
        \\
        \, .
\end{align*}
The requirement $\delta > 0$ is implied by
\begin{align*}
    &\frac{1}{\lambda} + 
    \frac{1}{k\lambda} \ln m + 
    \frac{\ell}{k\lambda} 2(\lambda - 1)/3 + 
    \frac{\ell}{k} \ln2 
    < 1 \\
    \text{or},\quad& k + \ln m + 2\ell(\lambda - 1)/3 + \lambda \ell \ln 2 < k\lambda \\
    \text{or},\quad& k - 2\ell/3 + \ln m < \lambda( k -2\ell/3 - \ell \ln 2) \\
    \text{or},\quad&\underbrace{k - 2\ell/3 + \ln m}_{A} < (2/5)(6 - 7 \alpha) \underbrace{ ( k -2\ell/3 - \ell \ln 2) }_{B}\\
    \text{or},\quad&6 - 7 \alpha > 5A/2B\\
    \text{or},\quad& \alpha < 6/7 - 5A/14B\,,
\end{align*}
which, upon simplification, becomes
\[
    \alpha < \frac{1}{2} -\frac{5}{14}\cdot \frac{\ell \ln 2 + \ln m}{k - \ell(2/3 + \ln 2)}
    \,.
\]
We want
\begin{align*}
    &\frac{5}{14}\cdot \frac{\ell \ln 2 + \ln m}{k - \ell(2/3 + \ln 2)} < \theta \\
    \text{or,}\quad& \ell \left( (5 \ln 2)/14\theta + 2/3 + \ln 2\right) + (5 \ln m)/14\theta < k 
    \,,
\end{align*}
which is implied by requiring
\begin{align*}
    &\ell \left( 1/4\theta + 3/2 \right) + (\ln m)/2\theta < k \\
    \text{or,}\quad&\ln m < 2k\theta - \ell(1/2 + 3\theta)
    \,,
\end{align*}
which, in turn, is implied by requiring $\ln m < 2k\theta - \ell$ since $\theta < 1/6$. 
This expression is equivalent to requiring $\theta > (\ell + \ln m)/2k$.

\paragraph{Case: $\alpha \leq 1/4$.}
In this case, 
\[
    \Exp g^\lambda 
    = \frac{\left(\Exp[X^\lambda] \right)^\ell}{1 - \alpha \Exp[X^\lambda]} 
    \leq \frac{\lambda^\ell/(1 - \alpha)^{\lambda \ell}}{1 - 5/9}
    = (9/4) \cdot 2^\ell/(1-\alpha)^{2\ell}
    \,,
\]
using~\eqref{eq:geom-moment-bound-approx-1} and $\lambda = 2$. 
Thus
\begin{align*}
    \delta
    &= 1 - \frac{1}{\lambda}\left(1 + \frac{1}{k} \ln \Exp g^\lambda \right) \\
    &\geq 1 - \frac{1}{2}\left(1 + \frac{1}{k} \ln \left( (9/4)  2^\ell/(1-\alpha)^{2\ell} \right) \right) \\
    &= \frac{1}{2} - \frac{1}{2k} \left(\ln(9/4) + \ell \left( \ln 2 - 2 \ln(1-\alpha) \right) \right) \\
    % &= \frac{1}{2} - \frac{\ln(9/4)}{2k} - \frac{\ell}{2k}\left( \ln 2 - 2 \ln(1-\alpha) \right) \\
    &= \frac{1}{2} - \frac{1}{k} \left(\ln(3/2) + \ell \left( \ln \sqrt{2} - \ln(1-\alpha) \right) \right)\\
    &= \frac{1}{2} - \frac{1}{k} \left(\ln(3/2) + \ell \ln \left(  \sqrt{2}/(1-\alpha) \right) \right)
    \,.
\end{align*}
% $\Pr[B]$ is at most 
% \[
%     3 L (9 \lambda/4)^{\ell/\lambda}/(1-\alpha)^\ell \cdot \EpsP^{1-1/\lambda}
%     \,
% \]
% which, using $\lambda = 2$, simplifies to 
% \begin{align*}
%     3 L (9/2)^{1/2}/(1-\alpha)^\ell \cdot \EpsP^{1/2}
%     = 3 L \EpsP^{1/2 - \left( \ell  \ln(1/(1- \alpha)) - \ln(3/\sqrt{2}) \right)/k}
%     \,.
% \end{align*}
It is easy to see that $\delta$ is always less than one.
The requirement that $\delta$ must be positive is 
implied by the (sufficient) condition $\ell \leq 3k/4 -1$.

\paragraph{Case: $1/4 < \alpha \leq 1/3$.}
In this case, 
\begin{align*}
    \ln \Exp g^\lambda 
    &= \ln \frac{\left(\Exp[X^\lambda] \right)^\ell}{1 - \alpha \Exp[X^\lambda]} \\
    &\leq \ln \frac{\lambda^\ell/(1 - \alpha)^{\lambda \ell}}{1 - 5/6}
    = \ln 6 + \ell \ln(3/2) - (3 \ell/2) \ln(1-\alpha) \\
    &\leq \ln 6 +  \ell \ln \left( (3/2)/(2/3)^{(3/2)} \right)
    = \ln 6 + (5\ell/2) \ln(3/2)
    \,,
\end{align*}
since $\lambda = 3/2$ and $\alpha \leq 1/3$.
Using ~\eqref{eq:delta-composition-generic}, let
\begin{align*}
    \delta^*
    &= 1 - \frac{1}{\lambda}\left(1 + \frac{1}{k} \ln \Exp g^\lambda \right) \\
    &\geq 1 - \frac{2}{3}\left(1 + \frac{1}{k} \left(\ln 6 + (5\ell/2) \ln(3/2) \right) \right) \\
    &\geq \frac{1}{3} - \frac{2}{3k} \left( \ln 6 + (5\ell/2)\ln(3/2) \right)
\end{align*}
As before, it is easy to see that this expression is always less than one. 
We require that it should also be positive, which is implied by requiring
\begin{align*}
    &1/3 >\frac{2}{3k} \left( \ln 6 + (5\ell/2)\ln(3/2) \right) \\
    \text{or,}\quad& k > 2\ln 6 + 5\ell\ln(3/2) \\
    \text{or,}\quad& \ell < \frac{k - 2\ln 6}{ 5\ln(3/2)}\,
\end{align*}
which, in turn, is implied by requiring $\ell < 4k/9 - 2$.

The condition $n > k/\ln(2)$ follows from the assumption (in Lemma~\ref{lemma:composition}) that 
the bad event ``each of the sets $S_1, \ldots, S_{n}$ is non-empty''
occurs with probability at most $\EpsP$. 
This is implied by the condition $\alpha^n \leq \EpsP$, or $n \geq k/\ln(1/\alpha) > k/\ln(2)$ since $\alpha < 1/2$. 
\end{proof}



\subsection{A refinement of Proposition~\ref{prop:li-convergence}}

We record a refinement of Proposition~\ref{prop:li-convergence} below.

\begin{proposition}\label{prop:li-convergence-absolute}
% The random variables $X \in \NN, X \sim \GeomAlpha$ and $Y = X-1$ satisfy
Let $c > 1$ and $\alpha \in \big(1/3, (c-1)/(2c-1) \big)$. 
A random variable $Y \sim \GeomAlphaHat$ satisfies
\begin{equation}\label{eq:li-convergence-absolute}
    % \alpha \Exp[X^\lambda]
    \Exp[Y^\lambda] 
    < c
    % \,,    
\end{equation}
where $\lambda \in \big(1, \lambda^*_c(\alpha)\big]$ 
and 
\begin{align}\label{eq:lambda-star-c}
    \lambda^*_c(\alpha) &= \begin{cases}
    \lambda^*(\alpha)\,, &\text{for } 0 < \alpha \leq 1/3\,,\\
    % 2\,, & \text{for}\quad 0 < \alpha \leq 1/4 \, ,\\
    % 3/2\,, & \text{for}\quad 1/4 < \alpha \leq 1/3 \, ,\\
    \frac{b + \ln(3/2\alpha) + \ln(1-1/c)}{a-\ln(1-\alpha)}
        \,, &\text{for}\quad 1/3 < \alpha < \frac{c-1}{2c-1}\,,
         % \quad\text{where}
    \end{cases}\\
        a&= 1 - 2\left( \ln(3/2)-\ln(1+1/2\sqrt{e}) \right)\,,\quad\text{and} \nonumber \\
        b&=1-3\ln(3/2)+2\ln(1+1/2\sqrt{e}) \nonumber \,.
\end{align}
% Specifically, if $\alpha < 1/2 - \theta$ for some $\theta \in (0, 1/6)$ then
% \begin{align}
%     \Exp[(X-1)^\lambda] \leq (1 - 2\theta)/6\cdot 2^\lambda (1 + 2\theta)^{-\lambda}(1+2\lambda)
%     \,,
% \end{align}
% which is at most $13/14$ if $\alpha \leq 0.49$.
\end{proposition}
We remark that the quantities $a,b$ defined in Proposition~\ref{prop:li-convergence} satisfy
\begin{align}\label{eq:lambda-a-b-upperbound}
    1 + 2e^{x-1} \leq 2 e^{a x - b}
    \qquad\text{for}\quad x \in (1, 3/2)
    \, .
\end{align}
