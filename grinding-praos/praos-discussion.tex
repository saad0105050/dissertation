

In this \Section, 
we analyze our findings about the Praos beacon 
using a close-to-the-real-world parameterization.


\section{Praos protocol parameter selection}\label{sec:praos-params}
Suppose that Praos is parameterized according to 
Theorem~\ref{thm:minentropy-loss-praos-multi-epochs} 
and~\cite[Theorem 9]{Praos}, as follows.
An important difference from~\cite[Theorem 9]{Praos} is that 
the current analysis is in the synchronous model 
and that the current analysis does not treat empty slots, a prominent feature of Praos. 
We correct this discrepancy by selecting the message-delivery delay $\Delta = 1$ 
and discarding the slots with no leaders; 
this is in accordance with the distribution $\mathcal{L}$. 
The second discrepancy is not really a problem 
as we can glean important information without using the exact distribution; 
see below in the exposition.

Let the honest bias be $\epsilon \in (0,1)$.
The adversarial stake is $\alpha = (1-\epsilon)/2$. 
Remember that Praos allows empty slots but, 
in our analysis, we consider only non-empty slots. 
In \emph{Cardano}'s~\cite{Cardano} implementation of Praos, 
an epoch is five-day long with slot length one second 
and the active slot coefficient (i.e., average fraction of non-empty slots)  $1/20$. 
This gives us, on average, $5*24*60*60/20 = 21,600$ non-empty slots per epoch.
Let $k$ be the common prefix parameter.  
According to~\cite[Corollary 3]{Praos}, 
The length of an epoch should be $24 k/(1+\epsilon)$. 
Since the worst-case $\epsilon$ is zero, 
a conservative choice would be to take $24 k = 21,600$, yielding $k = 900$. 


In an epoch in Praos, the first two-third of the slots 
are relevant for the beacon protocol; 
thus we set $T = 24k * 2/3 = 16 k$. 
Praos guarantees liveness (equivalent to $\sECQ$) 
with parameter $s = 8 k/(1+\epsilon)$. 

Let $\Blockchain$ be the above incarnation of Praos; 
it satisfies $\kSlotCP$ and $\sECQ$ properties 
except with probability 
$\EpsP = L e^{-k \cdot (\epsilon^3 (1- O(\epsilon) )/2 )}$; 
note that this asymptotic bound is tight for small $\epsilon$ only 
and assumes that no grinding has taken place.

Next, we use these parameters to study our results in the previous \Section. 







\section{Moment bounds}\label{sec:praos-discussion-moment-bounds}
Let $\epsilon, \EpsP \in (0,1)$, $s, k \in \NN, s \geq k$, and set $\alpha = (1-\epsilon)/2$. 
Recall the second moment bound. 
Consider an $(\EpsP, k, s)$-secure execution of Praos.

In Figure~\ref{fig:praos-gp-moments-exact}, 
we have plotted various moment bounds on $g(W)$ 
where $W \sim \DistL(\alpha)$. 
The fact that $V_s(\epsilon) < 1$ for $\epsilon > 0.809$ 
has a profound effect on the grinding power in Praos. 
In particular, this means, for $\epsilon > 0.809$, 
the grinding power is $O(1)$. 
However, for $\epsilon < 0.809$, 
grinding grows exponentially in $s$ 
and the dominant component of the failure probability, 
$\sqrt{V_s(\epsilon) \EpsP}$ grows fast 
(as a function of $1-\epsilon$) 
and renders the failure probability ineffective. 
This was the main motivation for pursuing more secure beacons 
in Part~\ref{part:xorgames} of this dissertation.

Some simple yet slightly weaker moment estimates can be found in \Section~\ref{sec:praos-higher-moments}.


  \begin{figure}[!htb]
    \centering
    \tikzsetnextfilename{gp-first-two-moments-tight}
    \begin{tikzpicture}[trim axis left,
      ]
      \begin{axis}[domain=0:1,
        samples=500,
        enlarge x limits=false,
        grid=both,
        no markers,
        % there is one default value for the `legend pos' that is outside the axis
        % legend pos=outer north east,
        legend pos=north east,
        % (so the legend looks a bit better)
        legend cell align=left,
        x label style={at={(axis description cs:0.5,-0.1)},anchor=north},
        y label style={at={(axis description cs:-0.1,.5)},anchor=south},
        xlabel={$\epsilon$, honest bias},
        title={Base of the exponents in GP moment bounds}
        xmin=0,xmax=1,ymin=0,ymax=2.5
        ]
        \addplot +[ultra thick,black] { PraosMeanBase(\x) } ;  
        \addlegendentry{Base, 1st moment};

        \addplot +[ultra thick,red] { PraosVbase(\x) } ;        
        \addlegendentry{Base, 2nd moment};

        \addplot +[thick,gray] { 1 } ;        
      \end{axis}
    \end{tikzpicture}%    
    \caption[GP moment bounds for Praos]{
      Moment upper bounds for the grinding power $g(W)$ 
      where $W \sim \DistL((1-\epsilon)/2)$.
      Above, we have plotted the functions $\zeta(\epsilon)$ 
      from~\eqref{eq:praos-func-zeta}
      and $V_s(\epsilon)$ from~\eqref{eq:V-eps-praos}. 
      These functions satisfy 
      $\Exp g(W) = \Poly(s) \zeta(\epsilon)^s$ and $\Exp g(W)^2 = \Poly(s) V_s(\epsilon)^s$. 
      When $\epsilon \geq 0.6$, $\zeta(\epsilon)$ is at most $1$, 
      implying that the grinding power is $O(1)$ in expectation.
      Moreover, when $\epsilon \geq 0.809$, $V_s(\epsilon)$ is at most $1$, 
      implying that the second moment of $g(W)$ is $O(1)$ in this region. 
    }
    \label{fig:praos-gp-moments-exact}
  \end{figure}


\section{Praos beacon quality and grinding power}
Consider Theorem~\ref{thm:minentropy-loss-praos-multi-epochs} 
and Figure~\ref{fig:prbad-praos-beacon}.
When $\epsilon > 0.809$, 
the beacon does not incur any min-entropy loss. 
However, when $\epsilon < 0.809$, 
the grinding power $\gamma$ grows exponentially as a function of $s$. 
But that is not all: 
the failure probability in Theorem~\ref{thm:minentropy-loss-praos-multi-epochs} 
becomes meaningless for $\epsilon < 0.8$. 
This happens because the grinding power grows quickly 
as the honest bias $\epsilon$ decreases.

% \iftoggle{drawfigs}{
% 

\begin{figure}[!htb]
    \centering
    \begin{tikzpicture}[trim axis left,
      ]
      \begin{axis}[domain=0.6:1, 
        restrict x to domain=0.6:0.98
        samples=500,
        enlarge x limits=false,
        grid=both,
        no markers,
        % legend pos=outer north east,
        legend pos=north east,
        legend cell align=left,
        x label style={at={(axis description cs:0.5,-0.1)},anchor=north},
        % y label style={at={(axis description cs:-0.1,.5)},anchor=south},
        xlabel={$\epsilon$, honest bias},
        title={$\rho^*$, upper bound on the min-entropy loss},
        xmin=0.6,xmax=1%,%ymin=-100%,ymax=2.5
        ]
        \addplot +[ultra thick,red] { PraosMinEntropyLossMultiEpoch(\x, 1200, 2) };
        \addlegendentry{$k = 1200$};
        \addplot +[ultra thick,black] { PraosMinEntropyLossMultiEpoch(\x, 900, 2) };
        \addlegendentry{$k = 900$};
        \addplot +[ultra thick,blue] { PraosMinEntropyLossMultiEpoch(\x, 600, 2) };
        \addlegendentry{$k = 600$};
      \end{axis}
    \end{tikzpicture}%
    \begin{tikzpicture}[trim axis left,
      ]
      \begin{axis}[domain=0.6:1, 
        restrict x to domain=0.6:0.999,
        % restrict y to domain=0:100,
        samples=500,
        enlarge x limits=false,
        grid=both,
        no markers,
        % legend pos=outer north east,
        legend pos=north west,
        legend cell align=left,
        x label style={at={(axis description cs:0.5,-0.1)},anchor=north},
        y label style={at={(axis description cs:-0.1,.5)},anchor=south},
        xlabel={$\epsilon$, honest bias},
        title={$-\log_2\, \Pr[ \text{min-entropy loss } > \rho^* ]$},
        xmin=0.6,xmax=1%,ymin=0%,ymax=2.5
        ]
        \addplot +[ultra thick,red] { -LnPraosPrBadMultiEpoch(\x, 900, 2)/ln(2) } ;
        \addlegendentry{$L=2, k = 900$};
        \addplot +[ultra thick,black] {  -LnPraosPrBadMultiEpoch(\x, 900, 2000)/ln(2)  };
        \addlegendentry{$L = 2000, k = 900$};
        \addplot +[ultra thick,blue] {  -LnPraosPrBadMultiEpoch(\x, 900, 200000)/ln(2) };
        \addlegendentry{$L = 200000, k = 900$};
      \end{axis}
    \end{tikzpicture}
    \caption{
        (\textbf{Top}) The upper bound $\rho^*$ on $\rho$, 
        Praos beacon's min-entropy loss over a $L$ epochs, 
        expressed in \#bits.  
        Recall that $\rho$ is the logarithm of the grinding power in an epoch.     
        Praos' parameters are selected according to Theorem~\ref{thm:praos-gp-single-epoch}, 
        as follows.
        (Note that the current analysis is in the synchronous model.)
        $k$ is the common prefix parameter. 
        (Cardano, an implementation of Praos, uses $k = 900$.) 
        The adversarial stake is $(1-\epsilon)/2$.
        An epoch contains $R = 16 k/(1+\epsilon)$ slots and the first $2R/3$ slots are used for the beacon. 
        We used the $\ECQ$ parameter $s = 8 k/(1+\epsilon)$; 
        this value is the same as the liveness parameter of Praos. 
        The probability of a consistency violation in an epoch in Praos is 
        $\EpsP = R \exp(1-\epsilon^3 (1-\epsilon)k/2) $. 
        See~\citet[Theorem 9]{Praos} for justification of these parameters. 
        (\textbf{Bottom}) The probabilities that the grinding power is larger than $2^{\rho^*}$.
    }
    \label{fig:praos-beacon-multi-epoch}
\end{figure}

% }
% \UnfinishedWarning{To do: restate figure}


  % \begin{figure}[!htb]
  %   \centering
  %   \tikzsetnextfilename{praos-beacon-prbad}
  %   \begin{tikzpicture}[trim axis left,
  %     ]
  %     \begin{axis}[domain=0:1,
  %       xmin=0.6,xmax=1,%,ymin=0,ymax=2
  %       restrict x to domain=0.6:1,
  %       samples=500,
  %       enlarge x limits=false,
  %       grid=both,
  %       no markers,
  %       unbounded coords=jump,  
  %       % there is one default value for the `legend pos' that is outside the axis
  %       % legend pos=outer north east,
  %       legend pos=north east,
  %       % (so the legend looks a bit better)
  %       legend cell align=left,
  %       x label style={at={(axis description cs:0.5,-0.1)},anchor=north},
  %       y label style={at={(axis description cs:-0.1,.5)},anchor=south},
  %       xlabel={$\epsilon$, honest bias},
  %       title={Logarithm of failure probability in Praos},
  %       ]
  %       \addplot +[ultra thick,red] { LnPraosPrBadMultiEpoch(\x, 900, 1)/ln(2) } ;        
  %       % \addplot +[ultra thick,red] { sqrt(PraosV(\x, 900)) * sqrt(PraosEpsP(\x, 900) ) } ;        
  %       % \addlegendentry{$k=900$};
  %       % % \addplot +[ultra thick,black] { PraosPrBadMultiEpoch(\x, 1200, 1) } ;        
  %       % % \addlegendentry{$k=1200$};
  %       \addplot +[thick,gray] { 1 } ;        
  %     \end{axis}
  %   \end{tikzpicture}%    
  %   \caption[Praos beacon failure probability]{
  %     Plot of the failure probability in 
  %     Theorem~\ref{thm:minentropy-loss-praos-multi-epochs} using $L = 1$.
  %     When $\epsilon \leq 0.8$, the failure probability becomes meaningless. 
  %   }
  %   \label{fig:praos-beacon-prbad}
  % \end{figure}



Here is a back-of-the-envelope calculation to see why. 
We need only consider the dominant term in the failure probability, 
namely $\sqrt{V_s(\epsilon) \EpsP}$ where $V_s$ is defined in~\eqref{eq:V-eps-praos}. 
For $\epsilon \geq 0.6$, it is at least $(5/3)^{s/2} \EpsP^{1/2}$. 
Even if we use the PoW lower bound $\EpsP \geq e^{-\epsilon^2 k}$ 
(which can be inferred from the analysis in\cite{Nakamoto2008}) 
and $s \geq 4 k$ (as described in Section~\ref{sec:praos-params}),
the $k$th root of the above quantity 
is at least $(5/3)^2 e^{-\epsilon^2}$; 
but this is at least one for $\epsilon \in (0,1)$.

Note that we used a special distribution on characteristic strings: 
one that does not produce empty slots. 
Even if we consider empty slots, i.e., 
use the real leader election distribution in Praos, 
the situation would not improve much. 
Since a slot is non-empty with a small constant probability, 
we will essentially dilute $s$ by a constant factor; 
the quantity $\sqrt{V_s(\epsilon) \EpsP}$ will still become meaningless 
when $\epsilon$ is slightly below $0.8$.

The verdict is: the grinding power in Praos is under check for $\epsilon > 0.809$, 
but it renders the security bounds meaningless when $\epsilon$ drops below $0.8$. 
(Note that the actual region of insecurity depends on $k$.)


