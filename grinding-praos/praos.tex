
% \newcommand{\Suffix}[2]{ {#1}^{\overset{#2}{\leftarrow} } }
% \newcommand{\Suffix}[2]{ {#1}[-{#2}:] }
\newcommand{\Suffix}[2]{ \mathsf{suffix}({#1},{#2})}
\newcommand{\CoinTossingLC}{\Pi_\mathsf{lc}^{\Players(\alpha),k,n}}



Let $\EpsP \in (0,1)$ and $s, k \in \NN$. 
Let $\Blockchain$ be an $(\EpsP, k, s)$-secure eventual consensus PoS blockchain protocol 
under the longest-chain rule, such as 
Ouroboros Praos~\cite{Praos} and Snow White~\cite{SnowWhite}. 
Let $\kappa, n$ be two positive integers.\footnote{\color{blue}In this section and the next, 
the symbol $n$ denotes a certain number of rounds. 
However, 
in Section~\ref{sec:intro} and~\ref{sec:model}, 
it denotes the number of participants in a protocol. 
This should not bring confusion since 
the two contexts do not overlap.}
The protocol $\Blockchain$ 
proceeds in epochs $e = 1, 2, \ldots$\,. 
% The length of each epoch is $R \geq n + k$. 
Inside each epoch $e$, 
$\Blockchain$ uses 
an $n$-round 
coin-flipping protocol $\Pi$ 
to generate the \emph{beacon output} $\eta_e$, as follows: 
% which takes a ``seed'' $\eta_{e-1} \UniformIn \{0,1\}^\kappa$ 
% as input and outputs a string $\eta_e \in \{0,1\}^\kappa$, as follows:
\begin{enumerate}
  \item Let $\eta_{e-1}$ be the 
  beacon output from the previous epoch. 
  A uniformly random string $\eta_0 \UniformIn \{0,1\}^\kappa$ is known to all participants at the onset of the first epoch. 

  \item Every new block added during slots $i \in [n]$ contains 
  a uniformly random Boolean string of length $\kappa$; 
  it is called a \emph{nonce}.

  \item Let $u$ be an honest observer at the onset of the next epoch. 
  He computes the beacon output, $\eta_e$, 
  by XORing all nonces recorded in his blockchain 
  along with $\eta_{e-1}$ 
  and then applying a cryptographic hash function to it.
\end{enumerate}
Note that the consistency property of the blockchain 
would ensure that all honestly-held blockchains 
at the onset of epoch $e + 1$ 
will agree about the blocks issued from 
the first $n$ slots of epoch $e$.

Let $L \in \NN$ and suppose that 
the blockchain protocol $\Blockchain$ is $(\EpsP, k, s)$-secure 
inside an epoch. 
Let $R_L$ be the number of ``resets'' 
that the adversary can cause to the beacon, 
i.e., 
the number of candidate values for $\eta_L$ 
is chosen (by the protocol) 
from a set of size $R_L$. 
Then the min-entropy loss 
in the beacon output $\eta_L$ 
is $\log_2(R_L)$. 
We want to show that with high probability, 
this quantity is bounded from above by 
a suitable function. 
The probability in the preceding statement 
depends on 
the randomness in $\eta_0$ and 
the outcomes of the leader election inside the $\Blockchain$, 
i.e., whether a block (and hence its nonce) is emitted by an honest player.



\ignore{

\iftoggle{drawfigs}{

  \begin{figure}
    \centering
    \pgfmathsetmacro{\k}{100}
    \pgfmathsetmacro{\d}{1}
    \pgfmathsetmacro{\T}{24 * \k}
    \pgfmathsetmacro{\tau}{30}
    \pgfmathsetmacro{\n}{(\T - \k)/\d}
    \pgfmathsetmacro{\ell}{\k/\d}
    \pgfmathsetmacro{\ellPlusN}{\ell + \n}
    \begin{tikzpicture}[trim axis left,
      declare function={
        rho_praos_small_alpha(\a) = log2(\n + \ell) + \tau / 2 ; 
        rho_praos_mid_alpha(\a) = log2(\n + \ell) + \tau / 2 + \n * (0.067 + 1.92*\a - 1.44 * \a * \a) ; 
        rho_praos_large_alpha(\a) = log2(\n + \ell) + \tau / 2 + \n * (0.237 + 0.86 * \a); 
        % rho_geom_small_alpha(\a) = 2 * log2(\n) + \tau/2 + \ell * log2( sqrt(1 + \a)/(1 - \a) );
        % rho_geom_large_alpha(\a) = 2 * log2(\n) + (\tau + \ell * log2( 1/\a ) )/(2 - (3 * \a - 1)/(2 * \a * \a));
      }
      %    declare function = { entropy(\x) = 1; },
      ]
      \begin{axis}[domain=0:0.5,
        samples=100,
        enlarge x limits=false,
        grid=both,
        no markers,
        x label style={at={(axis description cs:0.5,-0.1)},anchor=north},
        y label style={at={(axis description cs:-0.1,.5)},anchor=south},
        xlabel={$\alpha$, relative adversarial stake},
        ylabel={$\rho$, loss in min-entropy (bits)},
        % there is one default value for the `legend pos' that is outside the axis
        legend pos=south east,
        % (so the legend looks a bit better)
        legend cell align=left
        ]
        \addplot +[thick,domain=0:(0.0955),red,forget plot] plot (\x, {rho_praos_small_alpha(\x)});
        \addplot +[thick,domain=0.0955:(1/3),red,forget plot] plot (\x, {rho_praos_mid_alpha(\x)});
        \addplot +[thick,domain=(1/3):(0.5),red,legend] plot (\x, {rho_praos_large_alpha(\x)});
        \addlegendentry{Praos}

        % \addplot +[thick,domain=0:(1/3),blue,forget plot] plot (\x, {rho_geom_small_alpha(\x)});
        % \addplot +[thick,domain=(1/3):(1/2),blue,legend] plot (\x, {rho_geom_large_alpha(\x)});
        % \addlegendentry{$(\ell, \n, \alpha)$-geometric game, $d = 1$}
      \end{axis}
    \end{tikzpicture}
    \caption{Min-entropy loss in Praos. 
    Here, we assumed the consistency parameter $k = \k$ and 
    the probability of a consistency violation is $\EpsP = 2^{-\tau}$.
    Nonces from the first $T = 24\k$ slots contribute to the beacon.}
    % \caption{Min-entropy loss in Praos (red) and in the $(\ell, n, \alpha)$-geometric game (blue).}
    \label{fig:rho-praos}
  \end{figure}

} % end iftoggle{drawfigs}


}




\section{The characteristic string}
In Praos and Snow White, 
during each slot, 
each player determine privately and independently 
whether he is a leader for this slot. 
The probability that he succeeds is proportional to his stake. 
Naturally, this means there are empty slots 
which actually help the protocol achieve 
security even with network delay. 
However, since empty slots do not contribute in any blocks, 
we ignore them in our exposition. 
Hence we focus on a leader election process that assigns, 
to each slot, 
a non-empty subset of participants; 
these are the leaders at that slot. 
A leader is eligible to add a block to a 
maximum-length blockchain in his view. 
We can represent the outcome of this leader election scheme 
as a \emph{characteristic string}:


\begin{definition}[Characteristic string]\label{def:char-string}
  A \emph{characteristic string} $w \in \{\A, 1, 2, 3, \ldots \}^*$ 
  corresponds to a particular execution of $\Blockchain$.
  For each slot $t = 1, 2, \ldots$, 
  $w_t = \A$ if $t$ is assigned to an adversarial participant; and 
  $w_t = k \in \{1,2,\ldots\}$ if $t$ is assigned to $k$ honest participants.
  By convention, $w_0 = 1$.
\end{definition}

For two characteristic strings $x$ and $w$ on the same alphabet, 
we write $x \Prefix w$ iff $x$ is a strict prefix of $w$. 
Similarly, 
we write $x \PrefixEq w$ iff either $x = w$ or $x \Prefix w$. 
The empty string $\varepsilon$ is a prefix to all strings. 
If $w_t = \A$, we say that ``$\Slot_t$ is adversarial;'' 
otherwise, we say that ``$\Slot_t$ is honest.'' 
Define $\#_\A(w)$ as the number of occurrences of $\A$ in $w$. 
Define $\#_\h(w) \triangleq |w| - \#_\A(w)$.
A slot $t$ is called \emph{$k$-honest} if $w_t = k \in \NN$.
A characteristic string $w$ is called $\Hheavy$ if 
$\#_\h(w) > \#_\A(w)$; 
otherwise, $w$ is called $\Aheavy$. 
We use $w[i : j]$ to denote the substring $w_i w_{i+1}\ldots w_j$.



% As we discussed after Definition~\ref{def:coin-flipping-security}, 
% the adversarial participants in an $(\EpsP, \rho)$-secure 
% coin-flipping protocol $\Pi$ can choose the output $\eta$ 
% from a set $X$ of size $2^\rho$. 
% As $\Pi$ is conducted on the blockchain protocol $\Blockchain$, 
% the elements of $X$ are determined by 
% the set $C$ of all possible maximally long blockchains 
% at (or after) slot $n + k$. 
% Due the the $\kSlotCP$ property of $\Blockchain$, 
% these chains contain the same blocks corresponding to slots $1, \ldots, n$. 
% It follows that $\rho = \log_2 |X| = \log_2 g$.
% We call $g$ the \emph{grinding power} of the adversary. 





\section{The grinding power (GP)}

% \paragraph{Blockchain execution as a ``fork.''} 
Consider an abstract execution of $\Blockchain$ 
corresponding to the characteristic string $w$ 
where 
all blockchains are arranged in a rooted tree; 
this tree is called a \emph{fork}. 
If a block $B$ is issued from slot $t$, 
we say that $B$ has label $t$.
Two blockchains $\Chain_1, \Chain_2$ in a fork $F$ are considered ``different'' 
if and only if there is a slot from which 
$\Chain_1$ contains a block but $\Chain_2$ does not. 
A blockchain $\Chain \in F$ is called \emph{viable} (or \emph{competitive}) 
if the length of $\Chain$, denoted by $|\Chain|$, 
is at least $\#_\h(w)$. 
The rationale is that an honest blockchain $\Chain$ is broadcast immediately 
and hence any blockchain adopted by a future honest leader
must be at least as long as $\Chain$. 
% In addition, if an honest participant is presented with 
% a blockchain $\Chain$ with length at least $\#_\h(w)$, 
% that participant would possibly prefer $\Chain$ 
% over a blockchain containing only honestly-emitted blocks 
If an honest observer (or leader) has to choose from multiple viable blockchains for adoption, 
we assume that the tie is broken by the adversary. 

Considering any fork $F$ and an honest blockchain $\Chain$ in $F$, 
a \emph{viable adversarial extension of $\Chain$} 
is an adversarial blockchain $\Chain' \in F$ 
such that $\Chain \Prefix \Chain'$, 
$\Chain'$ has the maximum length in $F$, 
and, importantly, $\Chain'$ contains only adversarial blocks after the prefix $\Chain$. 
Note that every adversarial extension 
viable at the onset of slot $n + 1$, 
corresponds to a choice 
for the output of the coin-flipping protocol. 
Each of these extensions contains a last honest block as its prefix. 
Thus we focus on upper bounding, 
over all forks for a given characteristic string, 
the total number of viable adversarial extensions 
of the honest blocks.
Note that a prerequisite to creating an adversarial extension 
of an honest block with label $h$ is that 
the suffix $w[h + 1 : n]$ must be $\Aheavy$.
We formalize this notion in the definition below.

\begin{definition}[Grinding Power]\label{def:option-sequence}
  Let $w$ be a characteristic string of length $n$. 
  We say that a sequence
  $
    0 \leq i_0 < i_1 < \cdots < i_k \leq n
  $
  is an \emph{option sequence} if
  \InlineCases{
    \item $i_0 = 0$ or 
    % $w_{i_0} = \h$, 
    $i_0 \neq \A$, 
    \item $w_{i_j} = \A$ for each $j > 0$, and
    \item $k \geq \#_\h(w[i_0 +1 : n])$.
  }
  We define the \emph{grinding power} of $w$, denoted $g(w)$, 
  as the 
  number of distinct option sequences for $w$.
\end{definition}




\section{The distribution $\mathcal{L}(\alpha)$}
Let $\alpha \in (0, 1/2)$ 
and assume that the adversarial players control an $\alpha$ 
fraction of the stake. 
We consider a leader election mechanism 
which mimics the Praos election except the empty slots. 
Specifically, 
our (idealized) leader election mechanism 
induces 
a distribution $\mathcal{L}_n(\alpha)$ 
on length-$n$ characteristic strings $w = w_1 \ldots w_n$. 
We define $\mathcal{L}_n(\alpha)$ as follows: 
the symbols $w_i$s are independent 
and 
every slot is assigned to at least one slot leader. 
In addition, for all $i$, $\Pr[w_i = \A] = \alpha$.
Furthermore, $\Pr[w_i = j], j \in \NN$ is defined as follows: 
Let $m \in \NN$ be the total number of honest participants. 
Consider the independent Bernoulli random variables $X_i \in \{0, 1\}, i \in [m]$ 
and a tuple $(\sigma_1, \ldots, \sigma_m) \in [0,1]^m$ 
so that $\Pr[X_i = 1] = \sigma_i$ 
and $\sum \sigma_i = 1 - \alpha$. 
Let $H = H_m = \sum_{i =1}^m X_i$. 
Then for $i \in [n]$, 
$$
  \Pr[w_i = j] = (1-\alpha)\cdot
    \frac{\Pr[H = j \mid j \geq 1]}{1 - \Pr[H = 0]}
    \,.
$$

When $n$ is understood from the context, 
we just write $\mathcal{L}(\alpha)$ to denote this distribution. 
Furthermore, let $\epsilon \in (0,1)$ and set 
the adversarial stake $\alpha = (1-\epsilon)/2$. 
We define 
\begin{equation}\label{eq:praos-gneps}
  g(n,\epsilon) = g(W) \qquad \text{where} \quad W \sim \mathcal{L}_n((1-\epsilon)/2)
\end{equation}
as \emph{the grinding power over $n$ slots with adversarial stake $(1-\epsilon)/2$.}
Our goal is to derive tail bounds on $g(n,\epsilon)$.














\section{Praos beacon over multiple epochs}
Define 
\begin{align}
  \phi(\epsilon) &\triangleq \frac{3}{2} \left( (1-\epsilon^2) (1-\epsilon) \right)^{1/3}\label{eq:phi_eps} \\
  V(\epsilon) &= \begin{cases}
      \left( (5-3 \epsilon)/2 \right)^s & \text{if}\quad 0 < \epsilon \leq 1/3\,, \\
      \left( 2^{2/3} \phi(\epsilon) \right)^s & \text{if}\quad 1/3 < \epsilon \leq 3/5\,, \\
      \left( (5/3) \phi(\epsilon) \right)^s & \text{otherwise}\,. 
    \end{cases} 
  \,. \label{eq:V-eps-praos}
\end{align}
\noindent


Let $\epsilon \in (0,1), \alpha = (1-\epsilon)/2, n \in \NN$. 
We can prove that 
conditioned on an $(\EpsP, k, s)$-secure execution of Praos in an epoch of length $n$ 
and characteristic string $W \sim \mathcal{L}_n(\alpha)$, 
$\Exp[g(n,\epsilon)^2] = O(s^2) V(\epsilon)$.
In Figure~\ref{fig:praos-gp-moments}, 
we have plotted various moment bounds on $g(n,\epsilon)$; 
we will prove these bounds next in this exposition.

\iftoggle{drawfigs}{
  

\begin{figure}[!htb]
  \centering
  \begin{minipage}{0.5 \textwidth}
    \centering
    \tikzsetnextfilename{gp-mean}
    \begin{tikzpicture}[trim axis left,
      % declare function={
      %   entropy(\x) = - \x * ln(\x)/ln(2) - (1 - \x) * ln(1 - \x) / ln(2) ; 
      %   gamma(\x) = (1 - \x) / 2 + sqrt( 1 - \x^2 ) ; 
      %   phi(\x) = (3/2) * (1 + \x)^(1/3) * ( 1 - \x )^(2/3) ; 
      %   }
      ]
      \begin{axis}[domain=0:1,
        samples=500,
        enlarge x limits=false,
        grid=both,
        no markers,
        % there is one default value for the `legend pos' that is outside the axis
        % legend pos=outer north east,
        legend pos=south west,
        % (so the legend looks a bit better)
        legend cell align=left,
        x label style={at={(axis description cs:0.5,-0.1)},anchor=north},
        y label style={at={(axis description cs:-0.1,.5)},anchor=south},
        xlabel={$\epsilon$, honest bias},
        % ylabel={$M(\epsilon)$},
        xmin=0,xmax=1,ymin=0,ymax=2.5
        ]
        \addplot +[thick,blue] {(3 - x)/2};
        \addlegendentry{$M(\epsilon)$, Prop.~\ref{prop:praos-moments-simple}};

        \addplot +[thick,domain=(1/3):1,red] {sqrt(2 * (1 - x^2))};      
        % \addplot +[thick,dotted,domain=(1/1.414):1,red] {sqrt(2 * (1 - x^2))};
        \addlegendentry{$M(\epsilon)$, Prop.~\ref{prop:praos-moments-simple-large-eps}};

        % \addplot +[thick,blue] { ((1-x)/2)^(2/3) * (1 + x)^(1/3) * 2^(entropy(x)) };
        % \addlegendentry{$\MeanRoot$, Prop.~\ref{coro:praos-moments-simple-large-eps}}

        \addplot +[ultra thick,black] { PraosGamma(x) } ;  
        \addlegendentry{$M(\epsilon)$, Claim~\ref{claim:t2star-exact}};
        %\addplot +[thick,black, dotted] { sqrt(2) * (1-x)/2+sqrt(1 - x^2)};  
        %\addplot +[thick,dotted,black] {(3/2)(1+x)^(1/3)(1-x)^(2/3)};  
      \end{axis}
    \end{tikzpicture}
  \end{minipage}%
  \begin{minipage}{0.5 \textwidth}
    \centering
    \tikzsetnextfilename{gp-second-moment}
    \begin{tikzpicture}[trim axis left,
      % declare function={
      %   entropy(\x) = - \x * ln(\x)/ln(2) - (1 - \x) * ln(1 - \x) / ln(2) ; 
      %   gamma(\x) = (1 - \x) / 2 + sqrt( 1 - \x^2 ) ; 
      %   phi(\x) = (3/2) * (1 + \x)^(1/3) * ( 1 - \x )^(2/3) ; 
      %   psi(\x) = ( 1 - ln( (1+\x)/(1-\x) ) / ln(2) )^2 * ln(2) / 63; 
      %   }
      ]
      \begin{axis}[domain=0:1,
        samples=500,
        enlarge x limits=false,
        grid=both,
        no markers,
        % there is one default value for the `legend pos' that is outside the axis
        % legend pos=outer north east,
        legend pos=south west,
        % (so the legend looks a bit better)
        legend cell align=left,
        x label style={at={(axis description cs:0.5,-0.1)},anchor=north},
        y label style={at={(axis description cs:-0.1,.5)},anchor=south},
        xlabel={$\epsilon$, honest bias},
        % ylabel={$V(\epsilon)$},
        xmin=0,xmax=1,ymin=0,ymax=2.5
        ]
        \addplot +[thick,blue] {(5 - 3 * x)/2};
        \addlegendentry{$V(\epsilon)$, Prop.~\ref{prop:praos-moments-simple}};

        \addplot +[thick,domain=(3/5):1,red] {2 * sqrt((1 - x^2))};      
        % \addplot +[thick,dotted,domain=(1/1.414):1,red] {sqrt(2 * (1 - x^2))};
        \addlegendentry{$V(\epsilon)$, Prop.~\ref{prop:praos-moments-simple-large-eps}};

        % \addplot +[thick,blue] { ((1-x)/2)^(2/3) * (1 + x)^(1/3) * 2^(entropy(x)) };
        % \addlegendentry{$\MeanRoot$, Prop.~\ref{coro:praos-moments-simple-large-eps}}

        % \addplot +[ultra thick,domain=0:(1/3),black,dashed,forget plot] {(5 - 3 * x) / 2};
        % \addplot +[ultra  thick,domain=(1/3):0.6,black,forget plot] {2^(2/3) * PraosPhi(x)};
        % \addplot +[ultra thick,domain=0.6:1,black] { 5/3 * PraosPhi(x) )};
        % % \addplot +[ultra thick,domain=0.6:1,black] { 2^(2/3 + psi(x) ) * phi(x) )};
        \addplot +[ultra thick,black] { PraosVbase(\x) } ;        
        \addlegendentry{$V(\epsilon)$, Lem.~\ref{lemma:grinding-praos-second-moment}};


      \end{axis}
    \end{tikzpicture}    
  \end{minipage}

  \caption[Grinding power moment bounds for Praos]{
    Moment upper bounds for the grinding power $g(n,\epsilon)$, defined in~\eqref{eq:praos-gneps}.
    Here, $\epsilon \in (0,1)$ is the honest bias, i.e., $\Pr[\text{a slot is adversarial}] = (1-\epsilon)/2$ 
    and $n$ is the number of nonce-generating slots. 
    Above, we have plotted two functions $M(\epsilon)$ (left) and $V(\epsilon)$ (right) so that 
    $\Exp g(n,\epsilon) = \Poly(n) M(\epsilon)^n$ and $\Exp g(n,\epsilon)^2 = \Poly(n) V(\epsilon)^n$. 
    When $\epsilon \geq 0.6$, $M(\epsilon)$ is at most $1$, 
    implying that the grinding power is $\Poly(n)$ in expectation.
    Moreover, when $\epsilon \geq 0.809$, $V(\epsilon)$ is at most $1$, 
    implying that the second moment of $g(n, \epsilon)$ is $\Poly(n)$ as well. 
    In general, we can derive the tail bounds in Theorems~\ref{thm:minentropy-loss-praos-single-epoch} 
    and~\ref{thm:minentropy-loss-praos-muliti-epochs}.
  }
  \label{fig:praos-gp-moments}
\end{figure}

}

We can derive tail bounds on $g(n,\epsilon)$ using the second moment bound.

\begin{theorem}[Praos beacon]\label{thm:minentropy-loss-praos-multi-epochs}  
  Let $\epsilon \in (0, 1)$ and $L,\kappa \in \NN$. 
  Recall $\phi(\epsilon)$ from~\eqref{eq:phi_eps}.
  Suppose Praos 
  remains $(\EpsP, k, s)$-secure 
  inside any epoch 
  with a characteristic string $W \sim \mathcal{L}((1-\epsilon)/2)$. 
  Let 
  \begin{align*}
    \rho &= \begin{cases}
      % Ignoring the additive factor \log_2(0.4 (1+2\epsilon) )/2
      s/2 \cdot \log_2\left( (5-3 \epsilon)/2 \right) + \log_2(s)  & \text{if}\quad 0 < \epsilon \leq 1/3\,, \\
      s/2 \cdot \log_2\left( 2^{2/3} \phi(\epsilon) \right) + \log_2(s) & \text{if}\quad 1/3 < \epsilon \leq 3/5\,, \\
      s/2 \cdot \log_2\left( (5/3) \phi(\epsilon) \right) + \log_2(s) + 0.14 & \text{otherwise}\,. 
    \end{cases}
  \end{align*}

  Let $\eta \in \{0,1\}^\kappa$ be the output of the beacon after $L$ epochs.
  % and $L \in \NN$ so that $L \sqrt{m \EpsP} \leq 1$. 
  Then $g(n,\epsilon) \leq 2^\rho$ except with probability at most $3 L s \EpsP^{1/2} V(\epsilon)^{1/2}$.
  Equivalently,
  $$
    \Pr_{W \sim \mathcal{L}((1-\epsilon)/2)}[H_\infty(\eta) \geq \kappa - \rho] 
      \leq 1 - 3 L s \EpsP^{1/2} V(\epsilon)^{1/2} \,.
  $$
  In particular, for $\epsilon > 0.809$, 
  $\rho = \Poly(\log_2 s)$ and $g(n,\epsilon) = O(s)$ 
  except with probability at most $3Ls \EpsP^{1/2}$.
\end{theorem}
\noindent
The proof of Theorem~\ref{thm:minentropy-loss-praos-multi-epochs} 
is presented in Section~\ref{sec:proof-praos-theorem}. 

\begin{remark*}
  When $\epsilon < 0.809$, 
  the $O(s)$ term in the min-entropy loss $\rho$ 
  is inevitable sine, at this region, 
  $g(n,\epsilon) = e^{\Omega(s)}$.
  The factor $1/2$ in the leading term in $\rho$ 
  is the result of using the second-moment method 
  to derive the tail bound.
\end{remark*}
\iftoggle{drawfigs}{
  

\begin{figure}[!htb]
    \centering
    \begin{tikzpicture}[trim axis left,
      ]
      \begin{axis}[domain=0.6:1, 
        restrict x to domain=0.6:0.98
        samples=500,
        enlarge x limits=false,
        grid=both,
        no markers,
        % legend pos=outer north east,
        legend pos=north east,
        legend cell align=left,
        x label style={at={(axis description cs:0.5,-0.1)},anchor=north},
        % y label style={at={(axis description cs:-0.1,.5)},anchor=south},
        xlabel={$\epsilon$, honest bias},
        title={$\rho^*$, upper bound on the min-entropy loss},
        xmin=0.6,xmax=1%,%ymin=-100%,ymax=2.5
        ]
        \addplot +[ultra thick,red] { PraosMinEntropyLossMultiEpoch(\x, 1200, 2) };
        \addlegendentry{$k = 1200$};
        \addplot +[ultra thick,black] { PraosMinEntropyLossMultiEpoch(\x, 900, 2) };
        \addlegendentry{$k = 900$};
        \addplot +[ultra thick,blue] { PraosMinEntropyLossMultiEpoch(\x, 600, 2) };
        \addlegendentry{$k = 600$};
      \end{axis}
    \end{tikzpicture}%
    \begin{tikzpicture}[trim axis left,
      ]
      \begin{axis}[domain=0.6:1, 
        restrict x to domain=0.6:0.999,
        % restrict y to domain=0:100,
        samples=500,
        enlarge x limits=false,
        grid=both,
        no markers,
        % legend pos=outer north east,
        legend pos=north west,
        legend cell align=left,
        x label style={at={(axis description cs:0.5,-0.1)},anchor=north},
        y label style={at={(axis description cs:-0.1,.5)},anchor=south},
        xlabel={$\epsilon$, honest bias},
        title={$-\log_2\, \Pr[ \text{min-entropy loss } > \rho^* ]$},
        xmin=0.6,xmax=1%,ymin=0%,ymax=2.5
        ]
        \addplot +[ultra thick,red] { -LnPraosPrBadMultiEpoch(\x, 900, 2)/ln(2) } ;
        \addlegendentry{$L=2, k = 900$};
        \addplot +[ultra thick,black] {  -LnPraosPrBadMultiEpoch(\x, 900, 2000)/ln(2)  };
        \addlegendentry{$L = 2000, k = 900$};
        \addplot +[ultra thick,blue] {  -LnPraosPrBadMultiEpoch(\x, 900, 200000)/ln(2) };
        \addlegendentry{$L = 200000, k = 900$};
      \end{axis}
    \end{tikzpicture}
    \caption{
        (\textbf{Top}) The upper bound $\rho^*$ on $\rho$, 
        Praos beacon's min-entropy loss over a $L$ epochs, 
        expressed in \#bits.  
        Recall that $\rho$ is the logarithm of the grinding power in an epoch.     
        Praos' parameters are selected according to Theorem~\ref{thm:praos-gp-single-epoch}, 
        as follows.
        (Note that the current analysis is in the synchronous model.)
        $k$ is the common prefix parameter. 
        (Cardano, an implementation of Praos, uses $k = 900$.) 
        The adversarial stake is $(1-\epsilon)/2$.
        An epoch contains $R = 16 k/(1+\epsilon)$ slots and the first $2R/3$ slots are used for the beacon. 
        We used the $\ECQ$ parameter $s = 8 k/(1+\epsilon)$; 
        this value is the same as the liveness parameter of Praos. 
        The probability of a consistency violation in an epoch in Praos is 
        $\EpsP = R \exp(1-\epsilon^3 (1-\epsilon)k/2) $. 
        See~\citet[Theorem 9]{Praos} for justification of these parameters. 
        (\textbf{Bottom}) The probabilities that the grinding power is larger than $2^{\rho^*}$.
    }
    \label{fig:praos-beacon-multi-epoch}
\end{figure}

}




% Let $\Suffix{w}{d}$ denote the length-$d$ suffix of a string $w$.

