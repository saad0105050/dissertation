

Let $\kappa, n \in \NN$. 
Recall the Praos beacon with parameters $\kappa$ and $n$ from Definition~\ref{def:praos-beacon}. 
The goal of this \Section is to state our main results 
about the quality of the Praos beacon 
in terms of its min-entropy after multiple epochs (Definition~\ref{def:beacon-security}).

As we have done in the modeling of the consistency problem 
(\Section~\ref{sec:model-multihonest}), 
we start by defining the so-called ``characteristic string'' $w$. 
It captures the outcomes of the private leader election process $\PrivateLeaderElection$ 
in these slots. 
Next, we use the definition of the Praos beacon protocol to 
derive an expression for its grinding power on $w$.
Then we define a distribution on characteristic strings of length $n$; 
it resembles the outcome of $\PrivateLeaderElection((1-\epsilon)/2)$ 
modulo the empty slots. 
We develop the expression of the moment generating function of the 
grinding power $g(W)$ from~\eqref{eq:g_praos} 
where $W$ has this distribution. 
% where the characteristic string $W \sim \mathcal{L}_n((1-\epsilon)/2)$. 
Then, we derive an upper bound on the second moment of $g(W)$;
this leads to a  tail bound on $g(W)$.
% Next, we show how to lift these results 
% for characteristic strings with empty slots.
We finish this \Section by stating a theorem about the min-entropy loss in the Praos beacon. 
\UnfinishedWarning{Bad writing}





% \newcommand{\Suffix}[2]{ {#1}^{\overset{#2}{\leftarrow} } }
% \newcommand{\Suffix}[2]{ {#1}[-{#2}:] }
\newcommand{\Suffix}[2]{ \mathsf{suffix}({#1},{#2})}
\newcommand{\CoinTossingLC}{\Pi_\mathsf{lc}^{\Players(\alpha),k,n}}



% Let $\EpsP \in (0,1)$ and $s, k \in \NN$. 
% Let $\Blockchain$ be an $(\EpsP, k, s)$-secure eventual consensus PoS blockchain protocol 
% under the longest-chain rule, such as 
% Ouroboros Praos~\cite{Praos} and Snow White~\cite{SnowWhite}. 
% Let $\kappa, n$ be two positive integers. 
% Recall the beacon protocol from Section~\ref{sec:praos-model-grindingpower-minentropy}.
% Let $L \in \NN$ and suppose that 
% the blockchain protocol $\Blockchain$ is $(\EpsP, k, s)$-secure 
% inside an epoch. 
% Let $R_L$ be the number of ``resets'' 
% that the adversary can cause to the beacon, 
% i.e., 
% the number of candidate values for $\eta_L$ 
% is chosen (by the protocol) 
% from a set of size $R_L$. 
% Then the min-entropy loss 
% in the beacon output $\eta_L$ 
% is $\log_2(R_L)$. 
% We want to show that with high probability, 
% this quantity is bounded from above by 
% a suitable function. 
% The probability in the preceding statement 
% depends on 
% the randomness in $\eta_0$ and 
% the outcomes of the leader election inside the $\Blockchain$, 
% i.e., whether a block (and hence its nonce) is emitted by an honest player.



\ignore{

\iftoggle{drawfigs}{

  \begin{figure}
    \centering
    \pgfmathsetmacro{\k}{100}
    \pgfmathsetmacro{\d}{1}
    \pgfmathsetmacro{\T}{24 * \k}
    \pgfmathsetmacro{\tau}{30}
    \pgfmathsetmacro{\n}{(\T - \k)/\d}
    \pgfmathsetmacro{\ell}{\k/\d}
    \pgfmathsetmacro{\ellPlusN}{\ell + \n}
    \begin{tikzpicture}[trim axis left,
      declare function={
        rho_praos_small_alpha(\a) = log2(\n + \ell) + \tau / 2 ; 
        rho_praos_mid_alpha(\a) = log2(\n + \ell) + \tau / 2 + \n * (0.067 + 1.92*\a - 1.44 * \a * \a) ; 
        rho_praos_large_alpha(\a) = log2(\n + \ell) + \tau / 2 + \n * (0.237 + 0.86 * \a); 
        % rho_geom_small_alpha(\a) = 2 * log2(\n) + \tau/2 + \ell * log2( sqrt(1 + \a)/(1 - \a) );
        % rho_geom_large_alpha(\a) = 2 * log2(\n) + (\tau + \ell * log2( 1/\a ) )/(2 - (3 * \a - 1)/(2 * \a * \a));
      }
      %    declare function = { entropy(\x) = 1; },
      ]
      \begin{axis}[domain=0:0.5,
        samples=100,
        enlarge x limits=false,
        grid=both,
        no markers,
        x label style={at={(axis description cs:0.5,-0.1)},anchor=north},
        y label style={at={(axis description cs:-0.1,.5)},anchor=south},
        xlabel={$\alpha$, relative adversarial stake},
        ylabel={$\rho$, loss in min-entropy (bits)},
        % there is one default value for the `legend pos' that is outside the axis
        legend pos=south east,
        % (so the legend looks a bit better)
        legend cell align=left
        ]
        \addplot +[thick,domain=0:(0.0955),red,forget plot] plot (\x, {rho_praos_small_alpha(\x)});
        \addplot +[thick,domain=0.0955:(1/3),red,forget plot] plot (\x, {rho_praos_mid_alpha(\x)});
        \addplot +[thick,domain=(1/3):(0.5),red,legend] plot (\x, {rho_praos_large_alpha(\x)});
        \addlegendentry{Praos}

        % \addplot +[thick,domain=0:(1/3),blue,forget plot] plot (\x, {rho_geom_small_alpha(\x)});
        % \addplot +[thick,domain=(1/3):(1/2),blue,legend] plot (\x, {rho_geom_large_alpha(\x)});
        % \addlegendentry{$(\ell, \n, \alpha)$-geometric game, $d = 1$}
      \end{axis}
    \end{tikzpicture}
    \caption{Min-entropy loss in Praos. 
    Here, we assumed the consistency parameter $k = \k$ and 
    the probability of a consistency violation is $\EpsP = 2^{-\tau}$.
    Nonces from the first $T = 24\k$ slots contribute to the beacon.}
    % \caption{Min-entropy loss in Praos (red) and in the $(\ell, n, \alpha)$-geometric game (blue).}
    \label{fig:rho-praos}
  \end{figure}

} % end iftoggle{drawfigs}


}






\section{The ``hash the chain'' beacon protocol in Praos}\label{sec:praos-beacon-def}
% \label{sec:praos-model-grindingpower-minentropy}
The Praos blockchain protocol $\Blockchain$ 
proceeds in epochs $t = 1, 2, \ldots$. 
An epoch is long enough 
so that after it is complete, 
the first $n$ slots become settled 
(in the sense of $\kSlotCP$).


\begin{definition}[Praos beacon with parameter $\kappa, n \in \NN$]\label{def:praos-beacon}
  Let $\eta_0 \in \{0,1\}^\kappa$ be a uniformly random string, known to all participants 
  at the outset of the protocol.
  After each epoch $t$, 
  a \emph{beacon output} $\eta_t \in \{0,1\}^\kappa$ is computed, as follows: 
  % which takes a ``seed'' $\eta_{e-1} \UniformIn \{0,1\}^\kappa$ 
  % as input and outputs a string $\eta_e \in \{0,1\}^\kappa$, as follows:
  \begin{enumerate}
    \item $\eta_{t-1}$ is called the \emph{initial value} for epoch $t$.

    \item Every new block added during slots $i \in [n]$ in an epoch contains 
    a uniformly random Boolean string of length $\kappa$; 
    it is called a \emph{nonce}.

    \item Let $u$ be an honest observer at the onset of the next epoch. 
    He computes the beacon output for this epoch, $\eta_t$, 
    by XORing all nonces recorded in his blockchain 
    along with $\eta_{t-1}$ 
    and then applying to it a cryptographic hash function.
  \end{enumerate}
  Note that the consistency property of the blockchain 
  would ensure that all honestly-held blockchains 
  at the onset of epoch $e + 1$ 
  will agree about the blocks issued from 
  the first $n$ slots of epoch $e$.  
\end{definition}




\section{Current grinding analysis in Praos}
\UnfinishedWarning{Existing grinding analysis in Praos}


\section{The characteristic string}
In Praos and Snow White, 
during each slot, 
each player determines privately and independently 
whether he is a leader for this slot. 
The probability that he succeeds is proportional to his stake. 
Naturally, this means there are empty slots 
which actually help the protocol achieve 
security even with network delay. 
However, since empty slots do not contribute in any blocks, 
we ignore them in our exposition. 
Hence we focus on a leader election process that assigns, 
to each slot, 
a non-empty subset of participants; 
these are the leaders at that slot. 
A leader is eligible to add a block to a 
maximum-length blockchain in his view. 
We can represent the outcome of this leader election scheme 
as a \emph{characteristic string}:


\begin{definition}[Characteristic string]\label{def:char-string-praos}
  A \emph{characteristic string} $w \in \{\A, 
  % \perp, 
  1, 2, 3, \ldots \}^*$ 
  corresponds to a particular execution of $\Blockchain$.
  For each slot $t = 1, 2, \ldots$, 
  $w_t = \A$ if $t$ is assigned to an adversarial participant; 
  % $w_t = \perp$ if $t$ is assigned to no participants; 
  and 
  $w_t = k \in \{1,2,\ldots\}$ if $t$ is assigned to $k$ honest participants.
  By convention, $w_0 = 1$.
\end{definition}

Note that the above definition of characteristic strings 
is a generalization of characteristic strings 
from Part~\ref{part:multihonest} 
(Definition~\ref{def:trivalent-char-string} 
and Definition~\ref{def:semisync-char-string}, especially)
in that it counts the number of honest leaders. 


For two characteristic strings $x$ and $w$ on the same alphabet, 
we write $x \Prefix w$ iff $x$ is a strict prefix of $w$. 
Similarly, 
we write $x \PrefixEq w$ iff either $x = w$ or $x \Prefix w$. 
The empty string $\varepsilon$ is a prefix to all strings. 
% If $w_t = \perp$ then we say that ``slot $t$ is empty.'' 
If $w_t = \A$, we say that ``slot $t$ is adversarial;'' 
otherwise, we say that ``slot $t$ is honest.'' 
A slot $t$ is called \emph{$k$-honest} if $w_t = k \in \NN$.
Define $\#_\A(w)$ as the number of occurrences of $\A$ in $w$. 
Define $\#_\h(w) \triangleq |w| - \#_\A(w) 
% - \#_\perp(w)
$.
A characteristic string $w$ is called $\Hheavy$ (or \emph{honest-heavy}) if 
$\#_\h(w) > \#_\A(w)$; 
otherwise, $w$ is called $\Aheavy$. 
We use $w[i : j]$ to denote the substring $w_i w_{i+1}\ldots w_j$.

% Let $w$ be a characteristic string.
% If we obtain a string $x$ by deleting all $\perp$ symbols in $w$, 
% then $x$ is called a \emph{non-empty restriction of $w$}.


% As we discussed after Definition~\ref{def:coin-flipping-security}, 
% the adversarial participants in an $(\EpsP, \rho)$-secure 
% coin-flipping protocol $\Pi$ can choose the output $\eta$ 
% from a set $X$ of size $2^\rho$. 
% As $\Pi$ is conducted on the blockchain protocol $\Blockchain$, 
% the elements of $X$ are determined by 
% the set $C$ of all possible maximally long blockchains 
% at (or after) slot $n + k$. 
% Due the the $\kSlotCP$ property of $\Blockchain$, 
% these chains contain the same blocks corresponding to slots $1, \ldots, n$. 
% It follows that $\rho = \log_2 |X| = \log_2 g$.
% We call $g$ the \emph{grinding power} of the adversary. 





\section{Understanding the grinding power (GP)}

We briefly recall some fork-theoretic concepts
developed in Section~\ref{sec:viable-blockchains-adv-extensions}. 
Consider an abstract execution of $\Blockchain$ 
corresponding to the characteristic string $w$
where 
all blockchains are arranged in a rooted tree; 
this tree is called a \emph{fork}. 
If a block $B$ is issued from slot $t$, 
we say that $B$ has label $t$.
Two blockchains $\Chain_1, \Chain_2$ in a fork $F$ are considered ``different'' 
if and only if there is a slot from which 
$\Chain_1$ contains a block but $\Chain_2$ does not. 
A blockchain $\Chain \in F$ is called \emph{viable} (or \emph{competitive}) 
if the length of $\Chain$, denoted by $|\Chain|$, 
is at least $\#_\h(w)$. 
The rationale is that an honest blockchain $\Chain$ is broadcast immediately 
and hence any blockchain adopted by a future honest leader
must be at least as long as $\Chain$. 
% In addition, if an honest participant is presented with 
% a blockchain $\Chain$ with length at least $\#_\h(w)$, 
% that participant would possibly prefer $\Chain$ 
% over a blockchain containing only honestly-emitted blocks 
If an honest observer (or leader) has to choose from multiple viable blockchains for adoption, 
we assume that the tie is broken by the adversary. 

Assume that the blockchain protocol 
has the $\sECQ$ property (Definition~\ref{def:ECQ}) 
and the $\kSlotCP$ property (Definition~\ref{def:cp-slot-mh} 
with $s \geq k$. 
Let $n = |w|$.
It follows that 
every blockchain held by an honest observer must contain 
at least one honest block 
issued from the most recent $s$ slots in $w$. 
As $s \geq k$, 
\begin{equation}
  \parbox{0.8\textwidth}{the blockchains held by the honest players 
    are consistent 
    over the prefix $w_1 w_2 \ldots w_{n-s}$.
  }
  \label{eq:last-honest-block-praos}
\end{equation}
(Note that in fact, they could be consistent for some prefix of $w_{n-s+1} w_{n-s+2} \ldots w_n$.)
This honest block commits to a single blockchain. 

Considering any fork $F$ and an honest blockchain $\Chain$ in $F$, 
a \emph{viable adversarial extension of $\Chain$} 
is an adversarial blockchain $\Chain' \in F$ 
such that $\Chain \Prefix \Chain'$, 
$\Chain'$ has the maximum length in $F$, 
and, importantly, $\Chain'$ contains only adversarial blocks after the prefix $\Chain$. 
Note that every adversarial extension 
viable at the onset of slot $n + 1$, 
corresponds to a choice 
for the output of the coin-flipping protocol. 
As we have argued before, 
each of these extensions contains a last honest block as its prefix 
and this block must come from the most recent $s$ slots.

Thus, we focus on upper bounding, 
over all forks for a given characteristic string, 
the total number of viable adversarial extensions 
of the honest blocks from the most recent $s$ slots.
Naturally, we refer to the last $s$ slots in an epoch as \emph{the grinding window}.
Note that a prerequisite to creating an adversarial extension 
of an honest block with label $h$ is that 
the suffix $w[h + 1 : n]$ must be $\Aheavy$.
We formalize this notion in the definition below.

\begin{definition}[Grinding Power with parameter $s$]\label{def:option-sequence}
  Let $w$ be a characteristic string of length $n$ 
  and let $s, k \in [n], s \geq k$. 
  Suppose that the blockchain satisfies $\kSlotCP$ and $\sECQ$ properties.
  We say that a sequence
  $
    \max(n-s+1,0) \leq i_0 < i_1 < \cdots < i_k \leq n
  $
  is an \emph{option sequence} if
  \InlineCases{
    \item $i_0 = 0$ or 
    % $w_{i_0} \neq \A$, 
    $w_{i_0} 
    \neq \A
    % \not \in \{\A, \perp\}
    $, 
    \item $w_{i_j} = \A$ for each $j > 0$, and
    \item $k \geq \#_\h(w[i_0 +1 : n])$.
  }
  We define the \emph{grinding power of $w$ with parameter $s$}, denoted $g_s(w)$, 
  as the 
  number of distinct option sequences for $w$. 
  When $s$ is understood from the context, 
  we write $g(w)$ to denote the grinding power.
\end{definition}
\noindent


% \Paragraph{Grinding power of a characteristic string.}
Let $w = xy$ be a characteristic string. 
Define
\begin{equation}\label{eq:S_y}
    % S(d, z) 
    % \defeq \sum_{i \geq z}^{d - z}{ \binom{d - z}{i} } 
    S(y) 
    \defeq \sum_{i \geq \#_\h(y)}^{\#_\A(y)}{ \binom{\#_\A(y)}{i} } 
    % \,.
\end{equation}
and notice that it is the number of viable blockchains $\Chain$
in an execution on $w$ 
so that the last honest block in $\Chain$ occurs in slot $|x|$. 
By summing over all honest slots, 
the total number of viable blockchains for $w = w_1 w_2 \cdots w_n$,
i.e., the grinding power for $w$, is 
\begin{equation}\label{eq:g_praos}
    g(w_1 w_2 \cdots w_n) 
    = \sum_{ \text{$t \in [n] : t$ is an honest slot} }
      w_t \cdot S(w_{t+1} \cdots w_n)
    \,.  
\end{equation}

% In this exposition, 
% we first handle a distribution 
% on characteristic strings 
% which assigns zero probability on empty slots. 
















% Let $\Suffix{w}{d}$ denote the length-$d$ suffix of a string $w$.

