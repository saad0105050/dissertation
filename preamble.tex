%There are portions of this file which are editable, and portions which should not be adjusted.  I've clearly labeled the section where your customizations should be written.

%%%%%%%%%%%%%%%%%%%%
%%                   Do not edit!!!                     %%
%%%%%%%%%%%%%%%%%%%%
\usepackage{
	amsmath, 
	amsthm, 
	amssymb,
	fancyhdr, 
	setspace, 
}
\usepackage[
	paper=letterpaper,
	left=1.5in,
	top=1in,
	bottom=1in,
	right=1in,
	includehead,
	includefoot
]
{geometry}
\usepackage[titles]{tocloft}
\setlength{\cftbeforechapskip}{2ex}
\setlength{\cftbeforesecskip}{0.5ex}
\renewcommand{\cftchapfont}{\bfseries}
\renewcommand{\cftchapaftersnum}{.}
\renewcommand{\cftsecfont}{}
\renewcommand{\cftchappagefont}{\mdseries}
\renewcommand\cftchappresnum{Ch.\,}
\setlength{\cftchapnumwidth}{3.75em}
\usepackage{sectsty}
\allsectionsfont{\singlespacing}
\pagestyle{fancy} 
\rhead{\thepage}
\cfoot{}
\lhead{}
\chead{}
\renewcommand{\headrulewidth}{0pt} 
\renewcommand{\footrulewidth}{0pt}

%%%%%%%%%%%%%%%%%%%%
%%%%%            Editable            %%%%%
%%%%%%%%%%%%%%%%%%%%

%\usepackage{<add your packages>}
\usepackage[titletoc]{appendix}
\usepackage{hyperref}
\usepackage[center,small,sc]{caption,subcaption}
\usepackage{array}
\usepackage{subfiles}
\usepackage{accents}
\usepackage{multirow}
\usepackage{pgfgantt}
\usepackage{algorithm}
\usepackage{algpseudocode}
\usepackage[inline]{enumitem}
\usepackage{blindtext}
\usepackage{hhline}
\usepackage{multirow}
\usepackage{etoolbox}
\usepackage{mathtools}
\usepackage{lscape}
\usepackage{color}
\usepackage{framed}

% \renewcommand{\paragraph}{\noindent\paragraph}


\usepackage[numbers,sectionbib]{natbib}
\usepackage{bibentry}
\nobibliography*


% Requires package etoolbox
\newtoggle{Dissertation}
\toggletrue{Dissertation}

\iftoggle{Dissertation} {
	\newcommand{\Section}{Chapter}
}{
	\newcommand{\Section}{Section}
}

\newtoggle{drawfigs}
\toggletrue{drawfigs}
% \togglefalse{drawfigs}


%%%%%%%%%%%%%%%%%%
% \usepackage{stix2}




%%%%%%%%%%%%%%%%%%%%
% Common macros
\newtheorem{theorem}{Theorem}[chapter]
\newtheorem*{theorem*}{Theorem}
\newtheorem{proposition}[theorem]{Proposition}
\newtheorem{definition}[theorem]{Definition}
\newtheorem{claim}[theorem]{Claim}
\newtheorem{lemma}[theorem]{Lemma}
\newtheorem{corollary}[theorem]{Corollary}
\newtheorem{observation}[theorem]{Observation}
\newtheorem{bound}[theorem]{Bound}
\newtheorem{fact}[theorem]{Fact}

\theoremstyle{definition}
\newtheorem{remark}[theorem]{Remark}
\newtheorem*{remark*}{Remark}


\newcommand{\ignore}[1]{}


%%%%%%%%%%%%%%%%%%%%%%%
% Consistency macros
\newcommand\gmargin{\mathbf{m}}
\DeclareMathOperator{\Exp}{\mathbb{E}}
\DeclareMathOperator{\argmax}{arg\,max}
\DeclareMathOperator{\argmin}{arg\,min}
\newcommand{\gf}[1]{\mathsf{#1}}
\newcommand{\Z}{\mathbb{Z}}
\newcommand{\R}{\mathbb{R}}
\newcommand{\ZZ}{\mathbb{Z}}
\newcommand{\CC}{\mathbb{C}}
\newcommand{\RR}{\mathbb{R}}
\newcommand{\NN}{\mathbb{N}}
\newcommand{\hdepth}{\mathbf{d}}
\DeclareMathOperator{\length}{length}
\DeclareMathOperator{\depth}{depth}
\DeclareMathOperator{\height}{height}
\DeclareMathOperator{\gap}{gap}
\DeclareMathOperator{\reach}{reach}
\DeclareMathOperator{\reserve}{reserve}
\DeclareMathOperator{\divergence}{div}
\DeclareMathOperator{\tdivergence}{tdiv}
\newcommand{\slot}{\textsl{sl}}
\newcommand{\Slot}{\textsl{sl}}
\newcommand{\fprefix}{\sqsubseteq}
\newcommand{\Adversary}{\mathcal{A}}
\newcommand{\Challenger}{\mathcal{C}}
\newcommand{\Distribution}{\mathcal{D}}
\newcommand{\dominatedby}{\preceq}
\newcommand{\StationaryRho}{\mathcal{R}_\infty}
\newcommand{\DistRho}{\mathcal{R}}
\newcommand{\Poly}{\mathrm{poly}}
\newcommand{\SuchThat}{:}
\newcommand{\Given}{\mid}
\newcommand{\Union}{\cup}
\newcommand{\BigUnion}{\bigcup}
\newcommand{\PrefixEq}{\preceq}
\newcommand{\Prefix}{\prec}
\newcommand{\ForkPrefix}{\fprefix}
\newcommand{\Trim}[1]{^{\lceil {#1}}}
% \newcommand{\TrimSlot}[1]{\left[1:-{#1}\right]}
\newcommand{\TrimSlot}[1]{^{\lceil {#1}}}
\newcommand{\Fork}{\vdash}
\newcommand{\pinch}[2]{{#2}^{\vartriangleright {#1} \vartriangleleft} }
\newcommand{\cut}[2]{{#2}^{{#1} \vartriangleleft} }
\newcommand{\Chain}{\mathcal{C}}
\newcommand{\Intersect}{\cap}
\newcommand{\SlotCP}{\mathrm{CP}^{\mathsf{slot}}}
\newcommand{\kSlotCP}[1][k]{{#1}\text{-}\SlotCP}
\newcommand{\CP}{\mathrm{CP}}
\newcommand{\kCP}[1][k]{{#1}\text{-}\CP}
\newcommand{\defeq}{\triangleq}
\newcommand{\SlotDivergence}{\mathrm{div}_{\mathsf{slot}}}



%%%%%%%%%%%%%%%%%%%%%%%
% Multihonest macros
\newcommand{\h}{\mathtt{h}}
\renewcommand{\H}{\mathtt{H}}
% \newcommand{\HH}{\mathsf{H}}
\newcommand{\A}{\mathtt{A}}
\newcommand{\Hheavy}{\h\H\text{-heavy}}
\newcommand{\Aheavy}{\A\text{-heavy}}
\newcommand{\DominatedBy}{\preceq}
\newcommand{\Reduce}{\rho_\Delta}
\newcommand{\DeltaFork}{\Fork_\Delta}



%%%%%%%%%%%%%%%%%%%%%%%
% Grinding macros
\newcommand{\EpsG}{\mathsf{\varepsilon_g}}
\newcommand{\EpsA}{\mathsf{\varepsilon}_a}
\newcommand{\EpsP}{\mathsf{\varepsilon}_p}
\newcommand{\EpsTotal}{\mathsf{\varepsilon}_{total}}
\newcommand{\gSlice}{\gamma}
\newcommand{\Slice}{\mathsf{slice}}
\newcommand{\Epoch}{\mathsf{epoch}}
\newcommand{\Block}{\mathsf{block}}
\newcommand{\GameXor}{\mathcal{G}}
\newcommand{\GameNoSynch}{\GameXor_\mathsf{nosynch}}
\newcommand{\GameAllSynch}{\GameXor_\mathsf{allsynch}}
\newcommand{\BlockLength}{\ell}
\newcommand{\EpochLength}{k}
\newcommand{\NumEpochs}{L}
\newcommand{\NonceDim}{\kappa}
\newcommand{\Binomial}{\mathsf{Bin}}
\newcommand{\Poisson}{\mathsf{Poi}}
\newcommand{\Bernoulli}{\mathsf{Ber}}
\newcommand{\Geometric}{\mathsf{Geo}}
\newcommand{\Beacon}{\mathsf{B}}
\newcommand{\Protocol}{\mathsf{P}}
\newcommand{\DoverThree}{d/3}
\newcommand{\DoverTwo}{d/2}

\DeclareMathOperator*{\Var}{\mathbf{Var}}
\DeclareMathOperator{\wt}{wt}


\DeclareRobustCommand{\Eulerian}{\genfrac<>{0pt}{}}
\DeclareMathOperator{\Li}{Li}
\newcommand{\Lifunc}{\Li_{-\lambda}(\alpha)}
\newcommand{\Geom}{\mathcal{G}}
\newcommand{\GeomAlpha}{\Geom_\alpha^+}
\newcommand{\GeomOneMinusAlpha}{\Geom_{1 - \alpha}^+}
% \newcommand{\GeomAlphaHat}{\hat{\Geometric}_\alpha}
\newcommand{\GeomAlphaHat}{\Geom_\alpha}
\newcommand{\EmptyString}{\varepsilon}
% \newcommand{\EmptyString}[1][\kappa]{0^{#1}}
\newcommand{\ZeroString}[1][\kappa]{0^{#1}}
\newcommand{\MinEntropy}{H_\infty}
\newcommand{\Indicator}{\mathbf{1}}


\newcommand{\Player}{u}
\newcommand{\Players}{\mathcal{U}}
\newcommand{\Leaders}{\mathcal{L}}
\newcommand{\NoncePlayers}{\mathcal{N}}
\newcommand{\Endorser}{E}
\newcommand{\Nonce}{\mathsf{nonce}}
\newcommand{\BeaconProtocol}{\pi_{\mathsf{beacon}}}
\newcommand{\ECQ}{\exists\mathrm{CQ}}
\newcommand{\sECQ}[1][s]{{#1}\text{-}\ECQ}
\newcommand{\Hrule}{\noindent\rule{\textwidth}{1pt}}
\newcommand{\RandOracle}{\mathsf{H}}
\newcommand{\VRF}{\mathsf{VRF}}
\newcommand{\pk}{\mathsf{pk}}
\newcommand{\sk}{\mathsf{sk}}
\newcommand{\Or}{\text{or,}\quad}
\newcommand{\Blockchain}{\Pi_\mathsf{bc}}
\newcommand{\Astar}{\mathtt{A\star}}
\newcommand{\UniformIn}{\sim_{\mathrm{u}}}

\newcommand{\InlineCases}[1]{
  \begin{enumerate*}[label=(\textit{\roman*})]
    #1
  \end{enumerate*}}


%%%%%%%%%%%%%%%%%%%%%%%%%%%%%%%%%%%%%%%
%%% Plotting/drawing-related functions
%%%%%%%%%%%%%%%%%%%%%%%%%%%%%%%%%%%%%%%
\usepackage{tikz}
\usetikzlibrary{automata,arrows,positioning,calc,math}
\usepackage{pgfplots}
\pgfplotsset{compat=1.14}

% https://tex.stackexchange.com/questions/349766/pgfplots-on-tikzmath-function-with-conditionals-returns-an-error
\makeatletter
\def\tikz@math@if@@doif{%
    \pgfmathparse{\tikz@math@if@condition}%
    \ifpgfmathfloatparseactive%                 <--- Notice this
        \pgfmathfloattofixed{\pgfmathresult}%   <--- Notice this
    \fi%                                        <--- Notice this
    \ifdim\pgfmathresult pt=0pt\relax%
    \else%
        \expandafter\tikz@math\expandafter{\tikz@math@if@trueaction}%
    \fi%
    \tikz@math@parse%
}
\def\tikz@math@if@@doifelse{%
    \pgfmathparse{\tikz@math@if@condition}%
    \let\tikz@math@if@falseaction=\tikz@math@collected%
    \message{^^JCheck this: \pgfmathresult^^J}% <--- Notice this
    \ifpgfmathfloatparseactive%                 <--- Notice this
        \pgfmathfloattofixed{\pgfmathresult}%   <--- Notice this
    \fi%                                        <--- Notice this
    \message{^^JCheck again: \pgfmathresult^^J}%<--- Notice this
    \ifdim\pgfmathresult pt=0pt\relax%
        \expandafter\tikz@math\expandafter{\tikz@math@if@falseaction}%
    \else%
        \expandafter\tikz@math\expandafter{\tikz@math@if@trueaction}%
    \fi%
    \tikz@math@parse%
}

\tikzmath{
		\TooSmall = 0.0000001 ; % For avoiding division by zero
    function entropy (\x) {
        return \x * ln(\x)/ln(2) - (1 - \x) * ln(1 - \x) / ln(2) ; 
    };
    % Function gamma() from praos analysis <-- not to confuse with grinding power gamma
    function PraosGamma (\e) {
        return  (1 - \e) / 2 + sqrt( 1 - \e^2 ) ;
    };
    % Function phi() from praos analysis
    function PraosPhi (\e) {
        return (3/2) * (1 + \e)^(1/3) * ( 1 - \e )^(2/3) ; 
    };
    % Function phi() from praos analysis
    function PraosPsi (\e) {
    		\lnr = ln(1+\e) - ln(1-\e) ;
        return ( 1 - \lnr / ln(2) )^2 * ln(2) / 63; 
    };
    % Adversarial stake
    function Alpha(\e) {
        return (1 - \e) / 2 ; 
    };
    % Honest bias
    function Eps(\a) { 
        return 1 - 2 * \a ;
    };
    % Epoch length
    function R(\e, \k) {
        return 16 * \k  / (1 + \e) ; 
    };
    % Number of nonce-generating slots
    function PraosN(\e, \k) {
        return R(\e, \k) * 2 / 3 ; 
    };
    % Consistency error prob
    function PraosEpsP(\e, \k) {
        return R(\e, \k) * exp( 1 - \e^3 * (1-\e) * \k / 2) ;
    };
    function LnPraosEpsP(\e, \k) {
        return ln(R(\e, \k)) + 1 - \e^3 * (1-\e) * \k / 2 ;
    };
    % ECQ parameter
    function PraosECQs(\e, \k) {
        return 8 * \k / (1 + \e) ;
    };
    % Praos GP second moment base, without the polynomial part
    function PraosVbase(\e) {
    	if (\e < 0 ) then {
    		return 1; 
    	} else {
	    	if (\e <= 1/3) then {
	    		return (5 - 3 * \e) / 2 ;
	    	} else { 
	    		if (\e <= 3/5) then {
	    			return 2^(2/3) * PraosPhi(\e) ;
	    		} else { 
	    			return (5/3) * PraosPhi(\e) ;
	  			};%
	  		};
    	}; 
    };%
    % Praos GP second moment, full
    function PraosV(\e, \k) {
    	\s = PraosECQs(\e, \k) ;
    	\b = PraosVbase(\e) ;
    	return 0.4 * \s^2 * (1 + 2 * \e) * (\b)^(\s) ; 
    };
    function LnPraosV(\e, \k) {
    	\s = PraosECQs(\e, \k) ;
    	\b = PraosVbase(\e) ;
    	return ln(0.4 * (1 + 2 * \e) ) + 2 * ln(\s) + \s * ln(\b) ; 
    };
    % Praos min-entropy loss in a single epoch
    function PraosMinEntropyLossSingleEpoch (\e, \k) {
    	% \g = sqrt( PraosV(\e, \k) / PraosEpsP(\e, \k) ) ;% Grinding power
    	% return ln(\g)/ln(2) ;
    	\lng = 0.5 * ( LnPraosV(\e, \k) - LnPraosEpsP(\e, \k) ) ;% Logarithm of Grinding power
    	return \lng / ln(2) ;% Base-2 logarithm
    };
    function PraosMinEntropyLossMultiEpoch (\e, \k) {
    	\Tau = - LnPraosEpsP(\e, \k) / ln(2) ;
    	\s = PraosECQs(\e, \k) ;
    	\MinLoss = \Tau/2 + ln(\s) / ln(2) + 0.14 - Alpha(\e) ;
    	% return 1 ;
    	if (\e > 0.809) then {
	    	% return 1/2 \cdot \log_2(1/\EpsP) + \log_2(s) + 0.14 - \alpha
	    	return \MinLoss ;
	    } else {
	    	% return 1/2 \cdot \log_2(1/\EpsP) + \log_2(s) + (0.14 - \alpha) + (s/2) \log_2(1+3\alpha) 
	    	% return \Tau/2 + \s * (3/2) * Alpha(\e) / ln(2) ;
	    	return \MinLoss + (\s/2) * ln(1 + 3*Alpha(\e)) / ln(2) ;
	    };
    };
    function LnPraosPrBadMultiEpoch (\e, \k, \L) {
    	\s = PraosECQs(\e, \k) ;
    	\sqepsp = sqrt(PraosEpsP(\e, \k)) ;
    	\p = ln(\L) + LnPraosEpsP(\e, \k)/2 ;
    	if (\e > 0.809) then {
	    	% return L \cdot (2 \EpsP + s \sqrt{\EpsP} )
	    	% return L \cdot \sqrt{\EpsP} (2 \sqrt{\EpsP} + s )
    		return \p + ln(2 * \sqepsp + \s) ;
    	} else {
				% return L \cdot (2 \EpsP + (1.1 - 0.8 \alpha) s (1+3\alpha)^s \sqrt{\EpsP} )
	    	% return L \cdot \sqrt{\EpsP} (2 \sqrt{\EpsP} + (1.1 - 0.8 \alpha) s (1+3\alpha)^s )
	    	\q = ln(2*\sqepsp + \s * (1.1 - 0.8 * Alpha(\e)) * (1 + 3*Alpha(\e) )^(\s) ) ;
				return \p + \q ;
    	};
    };
}%



%%%%%%%%%%%%%%%%%%%%
%%                   Do not edit!!!                     %%
%%%%%%%%%%%%%%%%%%%%
%%%% Add any personal macros here.
\newcommand{\ubar}[1]{\underaccent{\bar}{#1}}
\newcommand*\rot{\rotatebox{90}}
\newcommand{\pluseq}{\mathrel{+}=}
\newcommand{\minuseq}{\mathrel{-}=}
\renewcommand*{\bmod}{\mathbin{\%}}
% tru\DeclareMathOperator*{\argmin}{arg\,min}
\DeclareMathOperator{\asin}{arcsin}

\makeatletter
\newcommand*{\emptystyles}{%
\let\oldplain\ps@plain
\let\ps@plain\ps@empty
\pagestyle{empty}
\pagenumbering{gobble}
\headsep = 45pt}

\newcommand*{\restorestyles}{%
\Gm@restore@org
\clearpage\thispagestyle{empty}
\let\ps@plain\oldplain
\pagestyle{headings}
 \pagenumbering{roman}}
\makeatother

\makeatletter
\def\setChapterprefix#1{\gdef\@Prefix{#1}}
\setChapterprefix{}
\renewcommand*\l@chapter[2]{%
  \ifnum \c@tocdepth >\m@ne
    \addpenalty{-\@highpenalty}%
    \vskip 1.0em \@plus\p@
    \setlength\@tempdima{1.5em}%
    \begingroup
      \parindent \z@ \rightskip \@pnumwidth
      \parfillskip -\@pnumwidth
      \leavevmode \bfseries
      \advance\leftskip\@tempdima
      \hskip -\leftskip
      \@Prefix 
      #1\nobreak\hfil \nobreak\hb@xt@\@pnumwidth{\hss #2}\par
      \penalty\@highpenalty
    \endgroup
  \fi}
    \makeatother
    
\let\oldbar\bar
\let\oldunderline\underline
\let\oldoverline\overline


\newdimen\slantmathcorr
\def\oversl#1{%assuming that mathslant=0.25
\setbox0=\hbox{$#1$}
\slantmathcorr=\wd0
%\hskip 0.2\slantmathcorr \overline{\hbox to 0.8\wd0{%
\hskip 0.2\slantmathcorr \oldoverline{\hbox to 0.8\wd0{%
\vphantom{\hbox{$#1$}}}}
\hskip-\wd0\hbox{$#1$}
}

\def\undersl#1{%assuming that mathslant=0.25
\setbox0=\hbox{$#1$}
\slantmathcorr=\wd0
%\underline{\hbox to 0.8\wd0{%
\oldunderline{\hbox to 0.8\wd0{%
\vphantom{\hbox{$#1$}}}}
\hskip-0.8\wd0\hbox{$#1$}
}

\pgfplotsset{
    discard if/.style 2 args={
        x filter/.code={
            \ifdim\thisrow{#1} pt=#2pt
                \def\pgfmathresult{inf}
            \fi
        }
    },
    discard if not/.style 2 args={
        x filter/.code={
            \ifdim\thisrow{#1} pt=#2pt
            \else
                \def\pgfmathresult{inf}
            \fi
        }
    }
}

\let\oldbar\bar
\let\oldunderline\underline
\let\oldoverline\overline
\algnewcommand{\Inputs}[1]{%
  \State \textbf{Inputs:}
  \Statex \hspace*{\algorithmicindent}\parbox[t]{.8\linewidth}{\raggedright #1}
}
\algnewcommand{\Initialize}[1]{%
  \State \textbf{Initialize:}
  \Statex \hspace*{\algorithmicindent}\parbox[t]{.8\linewidth}{\raggedright #1}
}
\doublespacing
