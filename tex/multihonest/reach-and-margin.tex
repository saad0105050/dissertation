
\subsection{Closed forks, reach, and extensions}
\begin{definition}[Closed fork]
  A fork $F$ is \emph{closed} if every leaf is honest. For convenience, we say the trivial fork is closed.
\end{definition}
Closed forks have two nice properties that make them especially useful in reasoning about the view of honest parties.
First, 
all honest observers will select a unique longest tine from this fork 
(since all longest tines in a closed fork are honest, 
honest parties are aware of all previous honest blocks, 
they observe the longest chain rule, and they employ the same consistent tie-breaking rule).  
Second, 
% recalling our description of the
% settlement game, 
closed forks intuitively capture decision points for the adversary.
The adversary can potentially show many tines to many honest parties, 
but once an honest node has been placed on top of 
a tine, any adversarial blocks beneath it are part of the public record and are visible to all honest parties. 
For these
reasons, we will often find it easier to reason about closed forks than arbitrary forks. % (without loss of generality).

The next few definitions are the start of a general toolkit for reasoning about an adversary's capacity to build highly diverging paths in forks, based on the underlying characteristic string.
%current state of a fork.

%%%%Reach and margin
\begin{definition}[Gap, reserve, and reach]\label{def:gap-reserve-reach}
For a closed fork $F \vdash w$ and its unique longest tine $\hat{t}$, we define the \emph{gap} of a tine $t$ to be $\gap(t)=\length(\hat{t})-\length(t)$.
Furthermore, we define the \emph{reserve} of $t$, denoted $\reserve(t)$, to be the number of adversarial indices in $w$ that appear after the terminating vertex of $t$. More precisely, if $v$ is the last vertex of $t$, then
\[
  \reserve(t)=|\{\ i \mid w_i=1 \ and \ i > \ell(v)\}|\,.
  \]
These quantities together define the \emph{reach} of a tine: $
\reach(t)=\reserve(t)-\gap(t)$.
\end{definition}

The notion of reach can be intuitively understood as a measure of
the resources available to our adversary in the settlement
game. Reserve tracks the number of slots in which the adversary has
the right to issue new blocks.  When reserve exceeds gap (or
equivalently, when reach is nonnegative), such a tine could be
extended---using a sequence of dishonest blocks---until it is as long
as the longest tine. Such a tine could be offered to an honest player
who would prefer it over, e.g., the current longest tine in the
fork. In contrast, a tine with negative reach is too far behind to be
directly useful to the adversary at that time.

\begin{definition}[Maximum reach]
For a closed fork $F\vdash w$, we define $\rho(F)$ to be the largest reach attained by any tine of $F$, i.e., 
\[
\rho(F)=\underset{t}\max \ \reach(t)\,.
\]
Note that $\rho(F)$ is never negative (as the longest tine of any fork always has reach at least 0). We overload this notation to denote the maximum reach over all forks for a given characteristic string: 
\[
\rho(w)=\underset{\substack{F\vdash w\\\text{$F$ closed}}}\max\big[\underset{t}\max \ \reach(t)\big]\,.
\]
\end{definition}

Reach of vertices is always non-increasing as we move down a tine. 
That is, if $B_1, B_2, \ldots$ are vertices on the same tine in the root-to-leaf order, then 
$\reach(B_i) \leq \reach(B_{i+1})$. 
The inequality is strict if $B_{i + 1}$ is honest. 
Consequently, the reach of an adversarial tine is no more than 
the reach of the last honest vertex in that tine. 
In any fork, the reach of a maximum-length tine is always non-negative. 
Hence, an honest tine with the maximum length over all honest tines 
will always have a non-negative reach. 
Thanks to the monotonicity of the honest-depth function $\hdepth(\cdot)$, 
if there are multiple honest tines 
having the (same) maximum length among all honest tines, 
they must have the same label. 
Therefore, if $h$ is the last honest slot in $w$ and 
$t$ a maximum-length honest tine with label $h$,  
then $\reach(t) \geq 0$. 


% {\color{red} Fix this.}
% \begin{fact}\label{fact:fork-structure-reach}
%   Let $w \in \{\h, \H, \A\}^T$ be a characteristic string, 
%   $s \in [T + 1]$ be an integer, 
%   $x \PrefixEq w, |x| = s - 1$. 
%   Let 
%   $F$ be a fork for $w$, 
%   $B$ an honest vertex in $F$, 
%   $h = \ell(B)$, and 
%   $I = [h + 1, s - 1]$.
%   Let $F_x \Fork x$ be a fork prefix of $F$ so that 
%   $F_x$ contains all honest tines from $F$ with labels at most $s - 1$. 
%   The following statements are equivalent: 
%   \begin{enumerate*}[label=(\roman*)]
%     \item \label{fact-reach-part:Aheavy} $I$ is $\Aheavy$; 
%     \item \label{fact-reach-part:viable-adv-ext} 
%       $B$ has an adversarial extension $t, \ell(t) \in I$ so that $t$ is 
%       viable at the onset of slot $s$; and
%     \item \label{fact-reach-part:nonneg-reach} $\reach_{F_x}(B) \geq 0$;
%   \end{enumerate*}      
% \end{fact}
% \begin{proof}
%   The equivalence between items~\ref{fact-reach-part:Aheavy} and~\ref{fact-reach-part:viable-adv-ext} has already been shown in Fact~\ref{fact:fork-structure}. 
%   \begin{description}[font=\normalfont\itshape\space]

%     \item[\ref{fact-reach-part:nonneg-reach} implies~\ref{fact-reach-part:Aheavy}.]
%       By assumption, $\reach_{F_x}(B)  = \reserve_{F_x}(B) - \gap_{F_x}(B) \geq 0$.               
%       Since $\reserve_{F_x}(B) = \#_\A(I)$ and $\gap_{F_x}(B) \geq \#_\h(I) + \#_\H(I)$, 
%       it follows that $\#_\A(I) \geq \#_\h(I) + \#_\H(I)$. 

%     \item[\ref{fact-reach-part:viable-adv-ext} implies~\ref{fact-reach-part:nonneg-reach}.]
%       Since $t$ is an adversarial extension of $B$, 
%       it contains only adversarial vertices from $I$. 
%       By assumption, $t$ is viable at the onset of slot $s$.
%       It follows that $\#_\A(I) \geq \gap_{F_x}(B)$.
%       Since $\reserve_{F_x}(B) = \#_\A(I)$, we have 
%       $\reach_{F_x}(B) = \reserve_{F_x}(B) - \gap_{F_x}(B) \geq 0$.  
      
%   \end{description}            
% \end{proof}






\paragraph{Non-negative reach, $\A$-heaviness, and viable adversarial extensions.}
Let $w \in \{\h, \H, \A\}^T$,   
$s \in [T + 1]$, and 
$F \Fork w_1 \ldots w_{s - 1}$ an arbitrary fork. 
Let $B \in F$ be an honest vertex 
and $t$ a maximum-length tine in $F$.
Consider the following statements: 
\begin{enumerate}[label=(\alph*)]
  \item \label{fact-reach-part:viable-adv-ext} $B$ has an adversarial extension viable at the onset of slot $s$.
  \item \label{fact-reach-part:nonneg-reach} $\reach_{F}(B)$ is non-negative.
  \item \label{fact-reach-part:Aheavy} The interval $I = [\ell(B) + 1, s - 1]$ is $\Aheavy$. 
  \item \label{fact-reach-part:conservative} $\length(t) = \#_\h(I) + \#_\H(I) + \length(B)$.     
\end{enumerate}

\begin{fact}\label{fact:fork-structure-reach}
    ~\ref{fact-reach-part:viable-adv-ext} $\Longrightarrow$
    \ref{fact-reach-part:nonneg-reach} $\Longrightarrow$
    \ref{fact-reach-part:Aheavy}.
    In addition, if we assume~\ref{fact-reach-part:conservative}, then 
    ~\ref{fact-reach-part:Aheavy} $\Longrightarrow$ 
    ~\ref{fact-reach-part:nonneg-reach} $\Longrightarrow$
    ~\ref{fact-reach-part:viable-adv-ext}.
\end{fact}
Fact~\ref{fact:fork-structure-reach} can be seen as 
a refinement of Fact~\ref{fact:fork-structure} 
when $F$ is a closed fork. 
\begin{proof}~
  \begin{description}[font=\normalfont\itshape\space]
    \item[\ref{fact-reach-part:viable-adv-ext} implies~\ref{fact-reach-part:nonneg-reach}.]
      An adversarial extension of $B$ 
      contains only adversarial vertices from $I$. 
      If this extension is viable at the onset of slot $s$, 
      $\#_\A(I)$ must be at least $\gap_{F}(B)$.
      Since $\reserve_{F}(B) = \#_\A(I)$, we have 
      $\reach_{F}(B) = \reserve_{F}(B) - \gap_{F}(B) \geq 0$.  
      
    \item[\ref{fact-reach-part:nonneg-reach} implies~\ref{fact-reach-part:Aheavy}.]
      By assumption, $\reach_{F}(B)  = \reserve_{F}(B) - \gap_{F}(B) \geq 0$.               
      $t$ contains at least $\#_\h(I) + \#_\H(I)$ vertices from 
      the interval $I$; 
      hence, $\gap_{F}(B) \geq \#_\h(I) + \#_\H(I)$. 
      Since $\reserve_{F}(B) = \#_\A(I)$, 
      it follows that $\#_\A(I) \geq \#_\h(I) + \#_\H(I)$. 

    \item[\ref{fact-reach-part:conservative} and~\ref{fact-reach-part:Aheavy} implies~\ref{fact-reach-part:nonneg-reach}.]
      Since $I$ is $\Aheavy$, 
      $\reserve_F(B) = \#_\A(I) \geq \#_\h(I) + \#_\H(I)$. 
      However, since~\ref{fact-reach-part:conservative} holds, 
      the latter quantity equals $\length(t) - \length(B) = \gap_F(B)$. 
      It follows that $\reach_F(B) = \reserve_F(B) - \gap_F(B) \geq 0$. 

    \item[\ref{fact-reach-part:conservative} and~\ref{fact-reach-part:nonneg-reach} implies~\ref{fact-reach-part:viable-adv-ext}.]
      $I$ contains at least $\gap_F(B)$ adversarial slots. 
      We can use these slots augment $B$ 
      into an adversarial tine $t'$ 
      of length at least $\length(t)$. 
      Thus $t'$ will be viable at the onset of slot $s$.
  \end{description}  
\end{proof}


Observe that for any characteristic string $x$, 
one can \emph{extend} (i.e., augment) a closed fork prefix $F \Fork x$ 
into a larger closed fork $F' \Fork x0$ so that $F \ForkPrefix F'$. 
A \emph{conservative extension} is a minimal extension in that 
it consumes the least amount of reserve (cf. Definition~\ref{def:gap-reserve-reach}), 
leaving the remaining reserve to be used in future.
Extensions and, in particular, conservative extensions 
play a critical role in the exposition that follows. 

\begin{definition}[Extensions]\label{def:extension}  
  Let $w \in \{\h, \H, \A\}^*$ be a characteristic string 
  and $F$ a closed fork for $w$. 
  Let $F'$ be a closed fork for $wb, b \in \{\h, \H\}$ 
  so that $F \ForkPrefix F'$. 
  We say that \emph{$F'$ is an extension of $F$} if 
  every honest vertex in $F'$ either belongs to $F$ or has label $|w| + 1$. 
  Let $\sigma \in F'$ be an honest vertex with $\ell(\sigma) = |w| + 1$ 
  and let $s$ be the longest honest prefix of $\sigma$. 
  (Necessarily, $s \in F$.)
  We say that \emph{$\sigma$ is an extension of $s$}. 
  The new tine $\sigma$ is a \emph{conservative extension} if 
  % $\height(F) + 1 = \max_{t \in S} \length(t)$.  
  $\height(F') = \height(F) + 1$.  
\end{definition} 
Since $F'$ is closed, all longest tines in $F'$ are honest and they have label $|w| + 1$.
% , i.e., they belong to $S$ in the above definition.
Let $\hat{t}$ be the unique longest honest tine in $F'$ 
under the consistent longest-chain selection rule 
in Axiom~\ref{axiom:tie-breaking}.
Now consider a tine $\sigma \in S$. 
Since $\sigma$ is honest, 
it follows that 
$\length(\sigma) 
\geq 1 + \height(F) 
= 1 + \length(s) + \gap_F(s)$ 
where $s \in F$ is the longest honst prefix of $\sigma$.
The root-to-leaf path in $F^\prime$ 
that ends at $\sigma$ 
contains at least $\gap_F(s)$ adversarial vertices $u \in F'$ 
so that $\ell(u) \in [\ell(s) + 1, |w|]$ and 
$u \not \in F$. 
If $\sigma$ is a conservative extension, 
the number of such vertices is exactly $\gap_F(s)$. 
% Finally, if $F'$ is a conservative extension, 
% the height of $F'$ is exactly one more than the height of $F$.


\begin{fact}[Extensions and reach]\label{fact:reach-fork-ext}
  Let $b \in \{\h, \H\}$. 
  Let $F \Fork w$ and $F^\prime \Fork wb$ be closed forks so that 
  $F \ForkPrefix F^\prime$ and 
  $F^\prime$ is obtained from $F$ via one or more extensions 
  $\sigma \in F^\prime, \ell(\sigma) = |w| + 1$.
  Then $\reach_{F^\prime}(t) \leq \reach_F(t) - 1$ for every $t \in F$. 
  If all these extensions are conservative, then 
  $\reach_{F^\prime}(t) = \reach_F(t) - 1$ for every $t \in F$. 
  Furthermore, a conservative extension $\sigma$ satisfies 
  $\reach_{F^\prime}(\sigma) = 0$.
\end{fact}
The above fact follows from the claims below.
\begin{claim}\label{claim:nex}
  Let $b \in \{\h, \H\}$. 
  Consider a closed fork $F\vdash w$ and some closed fork $F'\vdash wb$ such that $F\fprefix F'$. 
  If $t \in F$ then 
  $\reach_{F'}(t)\leq \reach_{F}(t) - 1$. 
  The inequality becomes and equality 
  if $F'$ is obtained via 
  conservative extensions from $F$.
\end{claim}
\begin{proof}
  We know that $\reach_{F'}(t)=\reserve_{F'}(t)-\gap_{F'}(t).$ From $F$ to $F'$, the length of the longest tine increases by at least one, and the length of $t$ does not change. 
  It follows that $\gap_{F'}(t) \geq \gap_{F}(t) + 1$. 
  The inequality becomes an equality 
  if $F'$ is obtained from $F$ via only conservative extensions. 
  The reserve of $t$ does not change, because there are no new $\A$s in the characteristic string. Therefore, 
  $
    \reach_{F'}(t)
    =\reserve_{F'}(t)-\gap_{F'}(t)
    \leq \reserve_{F}(t)-\gap_{F}(t) - 1
    =\reach_{F}(t) - 1
  $. 
\end{proof}
\begin{claim}\label{claim:ex}
  Conservative extensions have reach zero.
  % Let $b \in \{\h, \H\}$. 
  % Consider closed forks $F\vdash w, F'\vdash wb$ 
  % such that $F\fprefix F'$. 
  % If a tine $t$ of $F'$ is a conservative extension 
  % then $\reach_{F'}(t)=0$.
\end{claim}
\begin{proof}
  Let $b \in \{\h, \H\}$. 
  Consider closed forks $F\vdash w, F'\vdash wb$ 
  such that $F\fprefix F'$. 
  Let $t \in F'$ be a conservative extension. 
  This means $t$ is honest, $\ell(t) = |w| + 1$, 
  and 
  $t$ is a longest tine in $F'$. 
  The last statement implies $\gap_{F'}(t)=0$. 
  Since $\reserve_{F'}(t)=0$, it follows that 
  $\reach_{F'}(t)=\reserve_{F'}(t)-\gap_{F'}(t) = 0$. 
\end{proof}


\subsection{Relative margin}
\begin{definition}[The $\sim_x$ relations]
  For two tines $t_1$ and $t_2$ of a fork $F$, we write $t_1 \sim t_2$
  when $t_1$ and $t_2$ share an edge; otherwise we write
  $t_1 \nsim t_2$. We generalize this equivalence relation to reflect
  whether tines share an edge over a particular suffix of $w$: for
  $w = xy$ we define $t_1 \sim_x t_2$ if $t_1$ and $t_2$ share an edge
  that terminates at some node labeled with an index in $y$;
  otherwise, we write $t_1 \nsim_x t_2$ (observe that in this case the
  paths share no vertex labeled by a slot associated with $y$).  We
  sometimes call such pairs of tines \emph{disjoint} (or, if
  $t_1 \nsim_x t_2$ for a string $w = xy$, \emph{disjoint over
    $y$}). Note that $\sim$ and $\sim_\varepsilon$ are the same
  relation.
\end{definition}

\begin{definition}[Margin]\label{def:margin}
The \emph{margin} of a fork $F\vdash w$, denoted $\mu(F)$, is defined as 
\begin{equation}\label{eq:margin-absolute}
\mu(F)=\underset{t_1\nsim t_2}\max \bigl(\min\{\reach(t_1),\reach(t_2)\}\bigr)\,,
\end{equation}
where this maximum is extended over all pairs of disjoint tines of
$F$; thus margin reflects the ``second best'' reach obtained over all
disjoint tines. In order to study splits in the chain over particular portions of a
string, we generalize this to define a ``relative'' notion of margin:
If $w = xy$ for two strings $x$ and $y$ and, as above, $F \vdash w$,
we define
\[
  \mu_x(F)=\underset{t_1\nsim_x t_2}\max \bigl(\min\{\reach(t_1),\reach(t_2)\}\bigr)\,.
\]
Note that $\mu_\varepsilon(F) = \mu(F)$.

For convenience, we once again overload this notation to denote the
margin of a string. $\mu(w)$ refers to the maximum value of $\mu(F)$
over all possible closed forks $F$ for a characteristic string $w$:
\[
\mu(w)=\underset{\substack{F\vdash w,\\ \text{$F$ closed}}}\max \, \mu(F)\,.
\]
Likewise, if $w = xy$ for two strings $x$ and $y$ we define
\[
\mu_x(y)=\underset{\substack{F\vdash w,\\ \text{$F$ closed}}} \max \, \mu_x(F)\,.
\]
%(Cf.~\cite{KRDO17}, which defined and studied the ``absolute'' version
%$\mu(\cdot)$ of this quantity of~\eqref{eq:margin-absolute}.)
\end{definition}
Note that, at least informally, 
disjoint tines with large reach are of natural
interest to an adversary who wants to build an $x$-balanced fork, 
since such a fork contains two (partially disjoint) long tines.
It is easy to see that 
if $w = xx'y$ and 
$\mu_{xx'}(y)$ is negative then $\mu_x(x'y)$ is negative as well.

% \begin{fact}\label{fact:neg-margin}
%   Let $w$ be a characteristic string 
%   and let $w = xx'y$ be an arbitrary decomposition. 
%   If $\mu_{xx'}(y) < 0$ then $\mu_x(x'y) < 0$.
% \end{fact}

The theorem below shows how to recursively compute $\mu_x(y)$ 
for a given decomposition $w = xy$.


% \subsection{The relative margin recurrence}
\begin{theorem}\label{thm:relative-margin}
  Let $\varepsilon$ be the empty string 
  and $b \in \{\h, \H\}$. 
  Then $\rho(\varepsilon) = 0$ 
  and, for all nonempty strings $w \in \{\h, \H, \A\}^*$ 
  \begin{equation}
    \rho(w\A) = \rho(w) + 1\,, \qquad\text{and}\qquad
    \rho(wb) = \begin{cases} 0 & \text{if $\rho(w) = 0$,}\\
      \rho(w)-1 & \text{otherwise.}
    \end{cases}
    \label{eq:rho-recursive}
  \end{equation}


  Furthermore, for any strings $x, y \in\{\h, \H, \A\}\text{\emph{*}}$,
  $\mu_x(\varepsilon) =\rho(x)$, 
  \begin{equation}
    \mu_x(y\A)= \mu_x(y)+1\,,\qquad\text{and}\qquad
    \mu_x(yb)= \begin{cases}
      0 & \text{if $\rho(xy) > \mu_x(y)=0$}\,, \\
      0 & \text{if $\rho(xy) = \mu_x(y) = 0$ and $b = \H$}\,, \\
      \mu_x(y)-1 & \text{otherwise.}
    \end{cases}
    \label{eq:mu-relative-recursive}
  \end{equation}

  % Additionally, for every $w \in \{\h, \H, \A\}^*$ 
  % there exists a canonical fork for $w$. 
\end{theorem}
The proof of Theorem~\ref{thm:relative-margin} is given in Section~\ref{sec:margin-proof}. 
Let $w$ be a characteristic string and 
let $m, k \in \NN$ so that $m + k \leq |w|$. 
Let $x \Prefix w, |x| = m-1$ and $xy \PrefixEq w, |xy| \geq m + k$.
If the symbols in $w$ are independent and identically distributed, 
the recursive formulation in~\eqref{eq:mu-relative-recursive} implies an algorithm --- which takes time and space $O(|w|^3)$ --- 
for computing the probability that $\mu_x(y) \geq 0$. 
But this is exactly the probability that slot $m$ is not $k$-settled, 
according to~\eqref{eq:settlement-uvp} 
and Lemma~\ref{lemma:uvp-margin} below. 
In Section~\ref{sec:exact-prob}, 
we describe this algorithm in more detail and 
compile some explicit values for this probability.




