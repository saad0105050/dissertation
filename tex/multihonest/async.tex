We set the stage by stating the $\Delta$-synchronous model.

% \subsection{Semisynchronous executions}
  \begin{definition}[Semi-synchronous characteristic string]\label{def:semisync-char-string}
    Let $\slot_1, \ldots, \slot_{n}$ be a sequence of slots. 
    A \emph{semi-synchronous characteristic string} $w$ 
    is an element of $\{\h,\H,\A, \perp\}^n$ 
    defined for a particular execution of a blockchain protocol 
    on these slots so that 
    for $t \in [n]$, 
    $w_t = \perp$ if $\slot_{t}$ was assigned to no participants; otherwise, 
    $w_t = \A$ if $\slot_{t}$ was assigned to an adversarial participant; otherwise, 
    $w_t = \h$ if $\slot_{t}$ was assigned to a single honest participant; otherwise 
    $w_t = \H$.
  \end{definition}


  In the $\Delta$-synchronous setting, axiom~\ref{axiom:honest-depth} 
  is replaced by 
  \begin{enumerate}[label={\textbf{A\arabic*}\textsubscript{$\Delta$}}., ref={\textbf{A\arabic*}\textsubscript{$\Delta$}}, start=4]
  \item\label{axiom:honest-depth-delta} 
    In a $\Delta$-synchronous execution, 
    if two honestly generated blocks $B_1$ and $B_2$ are labeled
    with slots $\slot_1$ and $\slot_2$ for which $\slot_1 + \Delta < \slot_2$,
    the length of the unique blockchain terminating at $B_1$ is
    strictly less than the length of the unique blockchain terminating at $B_2$.
  \end{enumerate}

  \begin{definition}[$\Delta$-Fork]\label{def:delta-fork}
    Let $w\in \{\h, \H, \A, \perp\}^n, \Delta \in \{0, 1, 2, \ldots \}$, 
    $P = \{ i : w_i = \h\}$, and $Q = \{ j : w_j = \H\}$. 
    A \emph{$\Delta$-fork} for the semi-synchronous string $w$ consists of a directed and rooted
    tree $F=(V,E)$ with a labeling $\ell:V \to \{0,1,\ldots,n\}$. We insist
    that each edge of $F$ is directed away from the root vertex. 
    We require conditions~\ref{fork:root-mh}--\ref{fork:unique-honest} 
    from Definition~\ref{def:fork} and 
    % further require that
    \begin{enumerate}[label=(F{\arabic*}\textsubscript{$\Delta$}), start=4]
      % \item\label{fork:root-mh-delta} the root vertex $r$ has label $\ell(r)=0$;

      % \item\label{fork:monotone-mh-delta} the labels of vertices along any directed path are strictly increasing;

      % \item\label{fork:unique-honest-delta}\label{fork:multiply-honest-delta} 
      % each index $i\in P$ 
      % is the label of exactly one vertex of $F$ and 
      % each index $j\in Q$ 
      % is the label of at least one vertices of $F$; and

      \item\label{fork:honest-depth-delta} 
      for any indices $i,j\in P \Union Q$, 
      if $i + \Delta < j$ then 
      the depth of a vertex with label $i$ 
      is strictly less than 
      the depth of a vertex with label $j$.
    \end{enumerate}
  \end{definition}
  If $F$ is a $\Delta$-fork for the semi-synchronous characteristic string $w$, we write
  $F\vdash_\Delta w$.  
  A $\Delta$-fork generalizes a synchronous fork in Definition~\ref{def:fork} 
  since the latter is a $\Delta$-fork with $\Delta = 0$. 
  We sometimes emphasize this fact by writing $F' \Fork_0 w'$ 
  where $w'$ is a synchronous characteristic string and $F'$ is a synchronous fork.
  Note that 
  condition~\ref{fork:honest-depth-delta} 
  is a direct analogue of axiom~\ref{axiom:honest-depth-delta}.
  % conditions~\ref{fork:root-mh-delta}--\ref{fork:honest-depth-delta} 
  (We already know that conditions~\ref{fork:root-mh}--\ref{fork:unique-honest-mh} 
  are direct
  analogues of axioms~\ref{axiom:root}--~\ref{axiom:honest}.) 


  \begin{definition}[Reduction map]\label{def:reduction-map}
    For $\Delta \in \NN$, 
    we define the function $\Reduce : \{\perp, \h, \H, \A\}^* \rightarrow \{\h, \H, \A\}^*$ 
    inductively as follows: $\Reduce(\varepsilon) = \varepsilon$ and 
    for $w \in \{\perp, \h, \H, \A\}^*$, 
    \begin{align}
      \Reduce(b w) &= 
      \begin{cases}
        \Reduce(w) & \quad\text{if $b = \perp$}\,, \\
        b \Reduce(w) & \quad\text{if $b \in \{\h, \H\}$ and $\{\perp, \A\}^{\Delta} \PrefixEq w$}\,, \\
        \A \Reduce(w) & \quad\text{otherwise}
        \,.
      \end{cases}
    \end{align}    
  \end{definition}
  \noindent
  Note that in the above definition, 
  if $w' = \Reduce(w)$ and 
  $A = \{i : w_i \neq \perp\}$ then $|A| = |w'|$. 
  Also note that 
  the reduction $\Reduce$ implicitly defines, for each $w$, 
  a 
  bijective, increasing function $\pi : A \rightarrow [|w'|]$. 
  Note that 
  $\Reduce$ turns an $\h$ or $\H$ symbol in $w$ 
  into an $\A$ symbol in $w'$ with a constant probability. 
  Therefore, for any slot $t$ in $w$, 
  the reduction map $\Reduce$ 
  amplifies the probability that the slot $\pi(t)$ 
  in $w' = \Reduce(w)$ is adversarial. 



  \begin{definition}[$\Delta$-settlement with parameters $s,k \in \NN$]\label{def:settlement-mh-delta}
    Let $n \in \NN$ and let $w \in \{\perp, \h, \H, \A\}^n$. 
    Let $t \in [s + k, n]$ be an integer, $\hat{w} \PrefixEq w, |\hat{w}| = t$, and 
    let $F$ be any $\Delta$-fork for $\hat{w}$. 
    We say that a slot $s$ is \emph{not $(k, \Delta)$-settled in $F$} 
    if $F$ contains 
    two maximum-length tines $\Chain_1, \Chain_2$ so that 
    at least one of these tines contains a vertex with label $s$,
    both tines contain at least $k$ vertices after slot $s$, and 
    the label of their last common vertex is at most $s - 1$. 
    % $\LengthBeyond{\Chain_i}{s} \geq 1 + k$ for $i = 1, 2$. 
    % that ``diverge prior to $s$,'' i.e., they either
    % contain different vertices labeled with $s$, or one contains a
    % vertex labeled with $s$ while the other does not. 
    Otherwise, we say that \emph{slot $s$ is $(k, \Delta)$-settled in $F$}. 
    We say that \emph{slot $s$ is $(k, \Delta)$-settled in $w$} if, 
    for each $t \geq s+k$, 
    it is $(k, \Delta)$-settled in every $\Delta$-fork $F \Fork \hat{w}$ 
    where $\hat{w} \PrefixEq w, |\hat{w}| = t$.
  \end{definition}
  Note that in the above definition, 
  we truncated $k$ trailing blocks from a tine 
  whereas in Definition~\ref{def:settlement-mh}, 
  we truncated from a tine 
  all trailing blocks corresponding to the last $k$ slots. 
  Note that this change of perspective is necessary since 
  $w$ may contain $\perp$ symbols, i.e., empty slots.




    % We define the partial order $\leq$ on $\{\h, \H, \A, \perp\}^T$ 
    % as follows: 
    % $x \leq y$ if and only if $\Reduce(x) \leq \Reduce(y)$


    \begin{theorem}[Main theorem; $\Delta$-synchronous setting]\label{thm:main-mh-async}
      Let $f, \epsilon \in (0,1)$ and $\Delta \in \{0, 1, 2, \ldots\}$. 
      Let $s, k, T \in \NN$ so that $T \geq s + k + \Delta$. 
      Write $p_\perp = 1 - f$ and $\beta = (1 - f)^\Delta$. 
      Let $p_\A \in [0, f)$ so that $p_\A, f, \epsilon$, and $\beta$ 
      satisfy 
      \begin{equation}\label{eq:condition-prob-delta}
        p_\A  \beta/f + (1 - \beta) \leq (1 - \epsilon)/2 
        \,.
      \end{equation}      
      Let $p_\h \in (0, f - p_\A]$ and write $p_\H = f - p_\A - p_\h$. 
      Let $w \in \{\perp, \h, \H, \A\}^T$ be a random variable 
      so that each 
      $w_i, i \in [T]$, is independent and identically distributed as
      follows: $\Pr[w_i = \sigma] = p_\sigma$ for
      $\sigma \in \{\h, \H, \A, \perp\}$. 
      Let $\mathcal{B}$ be the distribution of $w$. 
      % Let $A$ be the event that 
      % slot $s$ is not $(k, \Delta)$-settled in $w$.
      Then 
      $$
        % \Pr_{w \sim \mathcal{B}}[A]  
        \Pr_w[\text{slot $s$ is not $(k, \Delta)$-settled in $w$}]  
          \leq
        % \exp\bigl(-k\cdot \Theta(\min(\epsilon^3, \epsilon^2 p_\h)) + \Delta \bigr)
        \exp\left( 
          % -k\cdot \Omega(\epsilon^2)
          -k\cdot \Omega(\min(\epsilon^3, \epsilon^2 p_\h \beta/f ) ) 
          + 
          \frac{\epsilon(1+\Delta)}{1 - \epsilon} 
        \right)
        \,.
      $$
      (Here, the asymptotic notation hides constants that do not depend on $\epsilon$ or $k$.)
      % Let $\mathcal{W}$ be a distribution on $\{\perp, \h, \H, \A\}^T$ 
      % so that 
      % $\mathcal{W} \DominatedBy \mathcal{B}$. 
      % Then $\Pr_{w \sim \mathcal{W}}[A] \leq \Pr_{w \sim \mathcal{B}}[A]$.
    \end{theorem}
    
  
    The main observation for proving the theorem above is that 
    a $\Delta$-settlement violation in $w$, 
    implies a certain combinatorial event (parameterized by $\Delta$) 
    in a prefix of $\Reduce(w)$. 
    Specifically, we can analyze the latter event 
    using techniques developed in proving Theorem~\ref{thm:main-mh}. 

    \paragraph{A comment on consistent chain selection.} 
    Assuming axiom~\ref{axiom:tie-breaking} is satisfied, 
    it is easy to prove an analogue of Theorem~\ref{thm:main-mh-bivalent} 
    in the $\Delta$-synchronous setting; 
    we need only use Bound~\ref{bound:two-catalans} 
    in lieu of Bound~\ref{bound:unique-honest-catalan}. 
    The resulting bound on the probability of a $(k, \Delta)$-settlement violation would be 
    $$
      \exp\left( 
        % -k\cdot \Omega(\epsilon^2)
        -k\cdot \Omega(\epsilon^3) 
        + 
        \frac{\epsilon(1+\Delta)}{1 - \epsilon} 
      \right)
      \,.
    $$
    We omit further details.
    % The proof is presented in \Section~\ref{sec:async-multihonest}.

    % Notice that $\Delta$ can only weakly impact the bound when $k$ is large. 
    % On the other hand, 
    % the bound ceases to be useful 
    % when $k \epsilon^3$ is comparable to $\Delta \ln k$ 
    % or equivalently, 
    % when $\Delta \gtrapprox \epsilon^3 k/\ln k$. 



\paragraph{Road-map for the proof.}
Let $w \in \{\perp, \h, \H, \A\}^*$, 
$w' = \Reduce(w), n = |w|$, and $m = |\Reduce|$. 
Our roadmap forward is as follows:
\begin{enumerate}
  \item 
  Show that there is a bijection between 
  $\Delta$-forks for $w$ and 
  synchronous forks for $w'$. 
  In particular, for each $\Delta$-fork $F \DeltaFork w$ 
  there is an isomorphic synchronous fork $F' \Fork_0 w'$ 
  and a bijective map $\{i \in [n] : w_i \neq \perp\} \rightarrow [m]$. 
  This is shown in Proposition~\ref{prop:reduction-bijection}.

  \item Show that if $w$ violates $\Delta$-settlement 
  then some prefix $b \Prefix \Reduce(w)$ violates 
  a suitably-defined combinatorial event $B_\Delta$.   
  It is important that we can analyze this event 
  using the techniques and results we have already established.
  This is done in Lemma~\ref{lemma:async-catalan-uvp}.

  \item Since the decisions made by $\Reduce$ at each slot 
  depends on the $\Delta$ future slots, 
  the distribution of the last few symbols of $\Reduce(w)$ 
  will be ``distorted'' no matter how $w$ is distributed. 
  Assuming $w$ has i.i.d.\ symbols, we need to 
  show that the symbols 
  in the aforementioned prefix $b \Prefix \Reduce(w)$ 
  are i.i.d.\ as well. 
  This is done in Lemma~\ref{prop:reduction-indep}.

  \item Obtain a bound on $Pr[B_\Delta]$ in Bound~\ref{bound:unique-honest-catalan-Delta} and 
  proceed to prove Theorem~\ref{thm:main-async}.
\end{enumerate}



\subsection{Structural properties of the reduction map}
An important property of the reduction $w' = \Reduce(w)$ is that 
it readily provides a bijection between $\Delta$-forks for $w$ 
and synchronous forks for $w'$.

\begin{proposition}\label{prop:reduction-bijection}
  Let $w \in \{\perp, \h, \H, \A\}^*$ 
  and $w' = \Reduce(w)$. 
  Then, for every $\Delta$-fork $F \Fork w$ there is 
  a synchronous fork $F' \Fork_0 w'$ 
  which is isomorphic to $F$. 
  $F'$ is called the \emph{image of $F$ under $\Reduce$}.
  % Let $m : V(F) \rightarrow V(F')$ be the said isomorphism. 
  % Furthermore, writing $w' = xy$ and taking 
  % any $t \in F$, 
  % let $u \in F'$ be the longest prefix of $m(t), \ell(u) \leq |x|$. 
  % Then $\length(u) \geq \length(t) - |y|$.
\end{proposition}
\begin{proof}[Proof sketch.]
  Let $F'$ be a copy of $F$. 
  % Let $m : V(F) \rightarrow V(F')$ be the said isomorphism. 
  Establish the natural bijection $m: V(F) \rightarrow V(F')$ 
  given by the copying proess, 
  i.e., $u \mapsto m(u)$, and 
  relabel the vertices as 
  \begin{equation}\label{eq:pi-reduction}
    \text{$\ell(m(u)) = \pi(\ell(u))$ for each vertex $u \in F$}
    \,.
  \end{equation}
  Set $r(F') = m(r(F))$ and $\ell(r(F')) = 0$.
  It suffices to check that $F' \Fork_0 w'$, 
  i.e., $F'$ is a valid (synchronous) fork for $w'$. 
  Specifically, 
  if there are two honest slots $h_1, h_2$ in $w$ 
  within a distance $\Delta$ of each other, 
  then the former honest slot is mapped to 
  an adversarial slot in $w'$. 
  Therefore, in $F'$, 
  an honest vertex is aware of 
  all honest vertices with smaller labels.
% 
  % The previous discussion implies that $t \in F$ implies  $\length(t) = \length(m(t))$. 
  % The equation in the claim follows since 
  % the trailing $|y|$ slots in $w'$ can add at most $|y|$ vertices to 
  % any tine $m(t) \in F', \ell(m(t)) \leq |x|$.
\end{proof}


Next, we show that 
a $\Delta$-settlement violation in $w$ implies 
a combinatorial event in $\Reduce(w)\TrimSlot{\Delta} \in \{\h, \H, \A\}^*$. 
It follows that we can use our existing stochastic techniques 
to bound $\Delta$-settlement violations on $w$. 

Let $w' \in \{\h, \H, \A\}^*$ be a characteristic string. 
Define $b_i \in \{\pm 1\}$ as $b_i = 1$ iff $w'_i = \A$. 
Let $S = (S_i)_{i =0}^{|w'|}$ be a simple biased walk on $\Z$ 
defined as $S_0 = 0, S_i = S_{i-1} + b_i$.



\begin{lemma}\label{lemma:async-catalan-uvp}
  Let $w \in \{\perp, \h, \H, \A\}^*, \Delta, s, k \in \NN$ 
  so that $|x| = s$ and $x_{s} \neq \perp$. 
  % and $a' = \Reduce(a), |a'| = \Delta$.     
  Let $w' = \Reduce(w)$ and 
  write $w' = x'y'z'a'$ so that $|a'| = \Delta$ and $|y'| \geq 2k$. 
  Recall the simple biased walk $S = (S_i)$ on $w'$ defined above.
  Let $E$ denote the event that 
  a slot $c'$ in $y'$ is Catalan in $x'y'z'$ 
  \emph{and} $S_{c' + k + i} \leq S_{c'} - \Delta$ for all $i \geq 0$.
  % Let $G_y$ be the event that 
  % there is a uniquely honest slot $c$ in $y'_1 \ldots y'_k$ 
  % that is Catalan in $w'$. 
  % Let $B_\Delta$ be the event that 
  % $S_{c + k + i} \geq S_{c} - \Delta$ for some $i \geq 0$. 
  % % Let $E_y$ be the event that 
  % there are at least $1 + \Delta$ 
  % slots in $y'$ that are 
  % Catalan in $x'y'z'$ and, in addition, 
  % that the earliest one of these slots is uniquely honest. 
  If $E$ occurs then 
  $s$ is $(|y'|,\Delta)$-settled in $w$. 
  % Here, 
  % settlement is defined in~\ref{def:settlement-delta} and 
\end{lemma}
% We insist that the event $E_y$ in the lemma is sufficient, but not necessary, 
% for the conclusion to hold.\footnote{ 
%   In fact, there are other combinatorial events that lead to tighter tail bounds; 
%   we defer the relevant analysis for a future version of this manuscript.
% }
\begin{proof}
  % \subsection{Proof of Lemma~\ref{lemma:async-catalan-uvp}}
  % Let $k' = k$ 
  % and observe that any sequence of $k'$ vertices on a tine 
  % must span at least $k'$ contiguous slots. 
  % Let $r' = \pi(s) + k$ 
  % and let $I' = [\pi(s), r']$.
  Let $\pi$ be the bijection described after Definition~\ref{def:reduction-map}.
  Note that $|x'| = \pi(s)$. 
  Assume that $E$ occurs. 
  % Let the slots $c'_i, i \in [1+\Delta]$ be Catalan in $x'y'z'$ 
  % so that 
  %  $|x'| < c_1 < \cdots < c_{1 + \Delta} \leq |x'y'|$.
  Thus $y'$ contains a uniquely honest slot $c'$ which is Catalan in $x'y'z'$. 
  Note that $S_{|w'|} \leq S_{|x'y'z'|} + \Delta \leq (S_{c} - \Delta) + \Delta \leq S_{c'}$ 
  where the second inequality follows from the assumption that $E$ occurs. 
  It follows that $c'$ is Catalan in $w'$ as well. 
  Therefore, 
  by Theorem~\ref{thm:unique-honest}, 
  $c'$ has the UVP in $w'$.     
  Let $c$ be the integer satisfying $c' = \pi(c)$. 

  Let $b \PrefixEq xyz, |b| \geq |xy|$ and $b' = \Reduce(b) \PrefixEq x'y'z'$. 
  (Necessarily, $|b'| \geq |x'y'|$.)
  Since the reduction map gives an isomorphism between every 
  $\Delta$-fork for $b$ and 
  its unique image (which is a synchronous fork for $b'$) 
  under the reduction $\Reduce$, 
  it follows that $c$ has the UVP in $w$. 

  For any $\Delta$-fork $F \DeltaFork b$, 
  let $u \in F, \ell(u) = c$ be 
  the unique vertex contained by every tine 
  $t \in F$ viable at the onset of any slot after $c$. 
  Consider all tines $\tau \in F$ so that 
  % $\LengthBeyond{\tau}{s} \geq 1 + |y'|$ 
  $\tau$ has at least $|y'|$ vertices with label at least $s + 1$.
  and $\tau$ is viable at the onset of slot $\ell(\tau) + 1$. 
  Since $\ell(\tau) \geq |xy| \geq c$, 
  it follows that $u \PrefixEq \tau$. 
  Thus all these tines $\tau$ 
  agree about slot $s$ since $s < c = \ell(u)$. 
  In particular, if $F$ contains two maximum-length tines $\tau_1, \tau_2$, 
  each with at least $|y'|$ vertices after slot $s$, 
  then they would agree about slot $s$. 
  In fact, $\ell(\tau_1 \Intersect \tau_2) \geq c > s$. 
  Hence $s$ must be $(|y'|, \Delta)$-settled in $F$ and, 
  since $F$ was arbitrary, 
  $s$ must be $(|y'|, \Delta)$-settled in $w$. 
  % \hfill$\qed$
  % 
\end{proof}




\subsection{Stochastic properties of the reduction map}
It turns out that 
if the bits in $w$ are i.i.d.\ then 
so are the bits in a suitable prefix of $\Reduce(w)$ 
albeit with a slightly different distribution 
(which accounts for the absence of the empty slots). 
Specifically, for any string $x = x_1 x_2 \ldots $ on any alphabet and any $k \in \NN$, 
define $x\TrimSlot{k} \triangleq x_1 \ldots x_{|x| - k}$.

\begin{proposition}\label{prop:reduction-indep}
  Let $T \in \NN, w = w_1 \ldots w_T \in \{\perp, \h, \H, \A\}^T$ 
  be a sequence of i.i.d.\ symbols, 
  and define $p_\sigma \triangleq \Pr[w_1 = \sigma]$ for each $\sigma \in \{\perp, \h, \H, \A\}$.
  Let $x = \Reduce(w)$ and let $\ell = |x|$. 
  Write $f = 1 - p_\perp$ and $\alpha = (1 - f)^\Delta$. 
  Then the symbols in the string $x\TrimSlot{\Delta}$ are i.i.d.\ 
  with 
  \begin{align}\label{eq:reduction-dist}
    \begin{matrix*}[l]
      \Pr[x_i = \h] &=& p_\h \cdot \alpha / f\,, \\
      \Pr[x_i = \H] &=& p_\H \cdot \alpha / f\,, \quad \text{and}\\
      \Pr[x_i = \A] &=& 1 - \alpha + p_\A \cdot \alpha / f\,
    \end{matrix*}
  \end{align}
  for each $i \in [\ell - \Delta]$.
\end{proposition}
\begin{proof}
  % \subsection{Proof of Proposition~\ref{prop:reduction-indep}}
  First let us pretend for a moment that $T = \infty$; 
  then $\ell = \infty$ as well. 
  Let us write the infinite sequence $w$ as a concatenation of segments 
  of $\perp$s punctuated by a single non-$\perp$ symbol. 
  That is, write $w = b_0 e_1 b_1 e_2 b_2 \ldots$ 
  where, for $i = 0, 1, \ldots$, $b_i = \perp^*$ and $e_i \in \{\h, \H, \A\}$. 
  The reduction map $\Reduce$ translates a segment $e_i b_i$ into a symbol $z_i$ 
  as follows:
  \begin{align*}
    z_i &= \begin{cases}
      \A &\quad \text{if $e_i = \A$ or $|b_i| \leq \Delta - 1$}\, \\
      e_i &\quad \text{if $e_i \in \{\h, \H\}$ and $|b_i| \geq \Delta$}\,.
    \end{cases}
  \end{align*}
  In particular, the segments $e_i b_i$ as well as 
  the events that determine the value of an $z_i$ are disjoint. 
  Therefore, the symbols in the infinite sequence 
  $z_1 z_2 \ldots = \Reduce(w_1 w_w \ldots)$ are 
  independent and identically distributed.

  If $T$ is finite, however, the last $\Delta$ symbols 
  of $x = \Reduce(w)$ are ``distorted'' 
  in that the translated symbols in this region will be more favored to be $\A$s. 
  However, since the last $\Delta$ symbols of $x$ must correspond to 
  at least $\Delta$ trailing symbols of $w$, 
  it follows that $x_1 \ldots x_{\ell - \Delta}$ 
  is a prefix of $z_1 z_2 \ldots\ $. 

 %  The probabilities in~\eqref{eq:reduction-dist} 
  % are a straightforward interpretation of the translation rule above.
  It remains to compute the probabilities. 
  Let $q_\sigma = \Pr[z_i = \sigma]$ for any $i$ and $\sigma \in \{\h, \H, \A\}$. 
  Then 
  $
    q_\h = p_\h/(1 - p_\perp) p_\perp^{\Delta} = p_\h \alpha/f,
    q_\H = p_\H \alpha/f
  $, 
  and 
  $
  q_\A 
    = 1 - (q_\h + q_\H) 
    = 1 - (p_\h + p_\H)\alpha/f 
    = 1 - (f - p_\A)\alpha/f 
    = 1 - \alpha + p_\A \alpha/f
  $.
  % \hfill$\qed$    
\end{proof}

The final ingredient to proving Theorem~\ref{thm:main-async} 
is a tail bound for (the complement of) the event $E$ 
in Lemma~\ref{lemma:async-catalan-uvp}. 


\begin{bound}\label{bound:unique-honest-catalan-Delta}
  % Assume the premise of Bound~\ref{bound:unique-honest-catalan}. 
  Let $T, s, k \in \NN, T \geq s + 2k + \Delta$ and 
  $\epsilon, q_\h \in (0, 1)$ so that 
  the characteristic string $w' \in \{\h, \H, \A\}^T$ 
  satisfies the $(\epsilon, q_\h)$-Bernoulli condition. 
  Write $w' = x'y'z'$ so that $y' = w_s \ldots w_{s + 2k - 1}$. 
  Let $G$ denote the event that 
  $w'$ has a Catalan slot $c$ 
  which belongs to $y'_1 \ldots y'_k$. 
  Condition on $G$. 
  Let $\Delta \in \NN$ and 
  recall the simple biased random walk $S = (S_i)$ on $w'$ 
  defined above Lemma~\ref{lemma:async-catalan-uvp}. 
  Let $B_\Delta$ be the event that 
  $S_{c + k + i} \geq S_{c} - \Delta$ for some $i \geq 0$. 
  Then for large $k$, 
  \begin{align}\label{eq:prob-Delta-after-catalan}
    \Pr_w[B_\Delta \mid G] \leq 
      \exp\left( 
        -k\cdot \Omega(\epsilon^2)
        % -k \left(\Omega(\min(\epsilon^3, \epsilon^2 q_\h)) + \epsilon^2 /2 \right) 
        + 
        \frac{\epsilon(1+\Delta)}{1 - \epsilon} 
    \right)
    \,.
  \end{align}
  % where $b(k, \epsilon)$ is the right-hand side 
  % in the probability in Bound~\ref{bound:unique-honest-catalan}.
\end{bound}
% \subsection{Proof sketch of Bound~\ref{bound:unique-honest-catalan-Delta}}\label{sec:catalan-Delta-estimates}
\begin{proof}~
  For simplicity, write $p = q_\A$, and $q = q_\h + q_\H$. 
  Conditioned on $G$, $S_c \geq S_{c + i}$ for all $i \geq 1$. 
  Let $y = y'[c + 1 : c + k]$ so that $|y| = k$.
  Moreover, $\#_\A(y) \leq \#_\h(y) + \#_\H(y)$. 
  Let $f_i(k), i = 0, 1, \ldots$ be the probability that 
  $S_{c + k} = S_c - i$. 
  Thus we wish to upper-bound $f(\Delta, k) \triangleq \sum_{i = 0}^\Delta f_j(k)$.

% \sum_{a = 0}^{\lceil k/2 \rceil - 1} 
  Write $a = E_\A(y)$ and $h = k - a$ 
  and suppose $h - a = j$ for some $j = 0, 1, 2, \ldots$\ .
  Hence, for a fixed $j$, we have $h = (k+j)/2$ and $a = (k-j)/2$. 
  In addition, $k$ and $j$ has the same parity.
  Thus, 
  \begin{align*}
    f_j(k) 
    &= {k\choose (k+j)/2} p^{(k-j)/2} q^{(k+j)/2} 
    = {k\choose (k+j)/2} (pq)^{k/2} (q/p)^{j/2} 
    \leq {k\choose k/2} (pq)^{k/2} (q/p)^{j/2} \\
    &= O(1)\cdot \frac{2^k}{\sqrt{\pi k}} \cdot (1 - \epsilon^2)^k 2^{-k} \cdot (q/p)^{j/2} 
    = O(1)\cdot  \frac{(1 - \epsilon^2)^{k/2}}{\sqrt{k}} \cdot (q/p)^{j/2}
    % \,.
  \end{align*}
  since $p = (1 - \epsilon)/2$ and $q = (1+\epsilon)/2$. 
  It follows that 
  \begin{align*}
    f(\Delta, k) &=\sum_{j = 0}^\Delta f_j(k) 
    \leq \frac{O(1)}{\sqrt{k}}\cdot  (1 - \epsilon^2)^{k/2} \sum_{j = 0}^\Delta (q/p)^{j/2}
    \leq \frac{O(1)}{\sqrt{k}}\cdot  \exp(- k \epsilon^2/2) \cdot (1 + \Delta) (q/p)^{\Delta/2}
    \,.
  \end{align*}
  Since 
  \begin{align*}
    (q/p)^{1/2}
    &= \left(\frac{1+\epsilon}{1 - \epsilon}\right)^{1/2}
    = \left(1 + \frac{2\epsilon}{1 - \epsilon}\right)^{1/2} 
    \leq \exp(\epsilon/(1 - \epsilon)) 
    \,,
  \end{align*}
  we have 
  \begin{align*}
    % \Pr[B_\Delta \mid G_c]
    f(\Delta, k)
    &\leq \frac{O(1 + \Delta)}{\sqrt{k}}\cdot \exp\bigl(-k \epsilon^2/2 + (1 + \Delta) \epsilon/(1 - \epsilon) \bigr)
    \,.
  \end{align*}
  Note that for fixed $\epsilon$ and $\Delta$, $f(\Delta, k)$ decreases geometrically in $k$. 
  Thus $\Pr[B_\Delta \mid G_c] = \sum_{t \geq k} f(\Delta, t)$ 
  is no more than the quantity in~\eqref{eq:prob-Delta-after-catalan}.
\end{proof}


\subsection{Proof of Theorem~\ref{thm:main-async}}
  The symbols in $w$ are independent and identically distributed.
  Write 
  % $w = xa$ where $|x| \geq s, |a| = \Delta$ and 
  $w' = \Reduce(w), w' = x'y'z'a', |a'| = \Delta$ and $|y'| \geq 1+\Delta$. 
  Let $k$ be an integer so that $|y'| = 2k$. 
  Recall the random walk $S = (S_i)$ on $w'$ 
  defined above Lemma~\ref{lemma:async-catalan-uvp}. 
  Let $G_1$ denote the (good) event that 
  a slot $c'$ in $y'$ is Catalan in $x'y'z'$. 
  Let $G_2$ denote the (good) event that 
  $S_{c' + k + i} \leq S_{c'} - \Delta$ for all $i \geq 0$. 
  By Lemma~\ref{lemma:async-catalan-uvp}, 
  $G_1 \Intersect G_2$ implies $\overline{A}$. 
  (Here, $\overline{\cdot}$ denotes the complement.) 
  The contrapositive of the above statement gives us 
  \begin{equation}\label{eq:G1G2}
    \Pr[A] \leq \Pr[\overline{G_1}] + \Pr[\overline{G_2} \mid G_1]
    \,.
  \end{equation}

  The terms on the right-hand side can be bounded from above 
  using Bounds~\ref{bound:unique-honest-catalan} 
  and~\ref{bound:unique-honest-catalan-Delta}, respectively, 
  provided the symbols in $x'y'z'$ are i.i.d.\ with 
  $\Pr[x'_1 = \A] = (1 - \epsilon)/2$. 
  Let us check whether this condition holds. 
  % It remains to check whether the symbols in $x'y'z'$ are i.i.d.\ with 
  % $\Pr[x'_i = \A] = (1 - \epsilon)/2$. 
  We have $f = 1 - p_\perp$ and $\alpha = (1-f)^\Delta$. 
  Proposition~\ref{prop:reduction-indep} 
  states that 
  the symbols of $x'y'z'$ are i.i.d.\ 
  with distribution given by~\eqref{eq:reduction-dist}. 
  For each $\sigma \in \{\h, \H, \A\}$ we write 
  $p'_\sigma = \Pr[x'_1 = \sigma]$. 

  The condition~\eqref{eq:condition-prob-delta} 
  can be equivalently stated as 
  $1 - (1 - p_\A/f)\alpha = (1 - \epsilon)/2$. 
  We check that 
  $p'_\A 
  = 1 - (p'_\h + p'_\H) 
  = 1 - (p_\h + p_\H)\alpha/f
  = 1 - (f - p_\A)\alpha/f
  = 1 - (1 - p_\A/f)\alpha
  = (1 - \epsilon)/2
  $ 
  and, consequently, $p'_\h + p'_\H = (1 + \epsilon)/2$. 

  Hence we can directly apply 
  Bound~\ref{bound:unique-honest-catalan-Delta} 
  on the terms in the right-hand side of~\eqref{eq:G1G2} 
  to conclude that 
  \[
    \Pr[A] \leq \exp\left( 
      % -k\cdot \Omega(\epsilon^2)
      -k \left(\Omega(\min(\epsilon^3, \epsilon^2 q_\h)) \right) 
      + 
      \frac{\epsilon(1+\Delta)}{1 - \epsilon} 
    \right)
    \,.
  \]

  The claim involving the distribution $\mathcal{W}$ 
  follows from the analogous claim in Theorem~\ref{thm:main}. 
  % This completes the proof of Theorem~\ref{thm:main-async}. 
  \hfill $\qed$  



