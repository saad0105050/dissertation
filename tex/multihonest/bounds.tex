
In the previous section, 
we explored the structural connection 
between the UVP and Catalan slots. 
In this section, we present two bounds 
on the stochastic event ``Catalan slots are rare.'' 
Specifically,
Bound~\ref{bound:unique-honest-catalan} 
concerns uniquely honest Catalan slots and complements Theorem~\ref{thm:unique-honest}; 
Bound~\ref{bound:two-catalans} concerns 
two consecutive Catalan slots and complements Theorem~\ref{thm:multiple-honest}. 
We defer the proofs till the next section 
and prove the main theorems below.

% As we shall see, 
% the main theorems follow easily from these connections. 


Recall the $(\epsilon, p_\h)$-Bernoulli condition 
from~\ref{def:bernoulli-cond}.
\begin{bound}\label{bound:unique-honest-catalan}
  Let $T, s, k \in \NN, T \geq s + k$ and  $\epsilon, q_\h \in (0, 1)$. 
  Let $w$ be a characteristic string satisfying 
  the $(\epsilon, q_\h)$-Bernoulli condition 
  and let $y = w_s \ldots w_{s+k-1}$.
  Then 
  \[
    \Pr_w[\text{$w$ does not contain a uniquely honest Catalan slot in $y$}]  
      \leq 
      \exp\left(
        -k\cdot \Omega(\min(\epsilon^3, \epsilon^2 q_\h)) 
      \right)
      \,.
  \]
\end{bound}
In particular, 
% when $q_\H = 0$, 
when $q_\h = (1+\epsilon)/2$, 
the bound above coincides with the 
bound in~\cite{LinearConsistency}; 
it follows that the current analysis 
subsumes their result.




\begin{bound}\label{bound:two-catalans}
  Let $T, s, k \in \NN, T \geq s + k$ and  $\epsilon \in (0, 1)$. 
  Let $w$ be a bivalent characteristic string satisfying 
  the $(\epsilon, 0)$-Bernoulli condition 
  and let $y = w_s \ldots w_{s+k-1}$.
  Then 
  \[
    \Pr_w[\text{$w$ does not contain two consecutive Catalan slots in $y$}]  
      \leq 
      \exp\left(
        - k\cdot \Omega(\epsilon^3(1 + O(\epsilon))) 
      \right)
      \,.
  \]
\end{bound}

% \noindent 
% For now, let us defer the proofs of these bounds in Section~\ref{sec:estimates} 
% and prove the main theorems below. 




\paragraph{Proof of Theorem~\ref{thm:main}.}

  We consider the distribution $\mathcal{B}$ first. 
  Write $w = xyz, |x| = s - 1$.
  Recall that  
  $
    \mathbf{S}^{s,k}[\mathcal{B}] 
    = \Pr_{w \sim \mathcal{B}}[\text{$s$ is not $k$-settled in $w$}]
  $. 
  Theorem~\ref{thm:unique-honest} and Equation~\eqref{eq:settlement-uvp} implies that 
  if $w$ contains a uniquely honest Catalan slot $c \in [s, s + k]$ 
  then slot $s$ must be $k$-settled in $w$. 
  In fact, by virtue of Fact~\ref{fact:catalan-unique-longest}, 
  it suffices to take $c \in [s, s + k - 1]$, 
  i.e., $|x| \leq c \leq |xy|$. 
  Thus the probability above is bounded by 
  Bound~\ref{bound:unique-honest-catalan} 
  which renames $p_\h = q_\h$.
   % and $p_\H = q_\H$. 
  This proves the first inequality. 
  
  Now let us prove the second inequality. 
  % Let $a, b \in \{\h,\H,\A\}^*, |a| = |b|$.
  % Define the partial order $\leq$ on equal-length characteristic strings as follows: 
  % $a \leq b$ if and only if
  % for all $i = 1, \ldots, |a|$, $a_i = 1$ implies $b_i = 1$. 
  For any player playing the settlement game, 
  let $C$ be the set of strings on which the player wins. 
  Clearly, $C$ is monotone 
  with respect to the partial order $\leq$ 
  defined on $\{\h, \H, \A\}^T$ 
  (see below Definition~\ref{def:dominance}).  
  To see why, note that if the player wins 
  on a specific string $w$, 
  he can certainly win on any string $w^\prime$ so that $w \leq w^\prime$. 
  By assumption, 
  $\mathcal{W} \DominatedBy \mathcal{B}$. 
  % by the definition of $\epsilon$-martingale condition, 
  It follows from Definition~\ref{def:dominance} that 
  $\Pr_{\mathcal{W}}[w] \leq \Pr_{\mathcal{B}}[w]$ 
  for any $w$ in the monotone set $C$. 
  By referring to the definition of settlement insecurity 
  (see Definition~\ref{def:settlement-insecurity}), 
  we conclude that 
  $
    \mathbf{S}^{s,k}[\mathcal{W}] \leq \mathbf{S}^{s,k}[\mathcal{B}]
  $.
  \hfill $\qed$  

\paragraph{Proof of Theorem~\ref{thm:main-bivalent}.}
  This proof is identical to the proof of Theorem~\ref{thm:main} 
  except that 
  we need to refer to Theorem~\ref{thm:multiple-honest} in lieu of Theorem~\ref{thm:unique-honest}
  and Bound~\ref{bound:two-catalans} in lieu of Bound~\ref{bound:unique-honest-catalan}.
  \hfill $\qed$  



