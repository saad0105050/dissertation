


\newcommand{\gfA}{\gf{A}}
\newcommand{\gfD}{\gf{D}}
\newcommand{\gfL}{\gf{L}}
\newcommand{\gfLhat}{\gf{\hat{L}}}
\newcommand{\gfLtilde}{\gf{\tilde{L}}}
\newcommand{\gfE}{\gf{E}}
\newcommand{\gfEhat}{\gf{\hat{E}}}
\newcommand{\gfEcheck}{\gf{\check{E}}}
\newcommand{\gfF}{\gf{F}}
\newcommand{\gfFhat}{\gf{\hat{F}}}
\newcommand{\gfFcheck}{\gf{\check{F}}}
\newcommand{\gfEtilde}{\gf{\tilde{E}}}
\newcommand{\gfR}{\gf{R}}
\newcommand{\gfRhat}{\gf{\hat{R}}}
\newcommand{\gfRtilde}{\gf{\tilde{R}}}

\newcommand{\gfC}{\gf{C}}
\newcommand{\gfChat}{\gf{\hat{C}}}
\newcommand{\gfCcheck}{\gf{\check{C}}}
\newcommand{\gfCtilde}{\gf{\tilde{C}}}

\newcommand{\gfM}{\gf{M}}
\newcommand{\gfMhat}{\gf{\hat{M}}}
\newcommand{\gfMtilde}{\gf{\tilde{M}}}


\newcommand{\gfX}{\gf{X}}
\newcommand{\gfXinf}{\gf{X}_\infty}

\newcommand{\EventRCat}{E_\mathsf{right-cat}}
\newcommand{\EventReset}{E_\mathsf{reset}}

% \newcommand{\SeqGF}{\overset{\mathrm{gf}}{\longleftrightarrow}}
\newcommand{\SeqGF}{\longleftrightarrow}



% In this section, 
% we present the proofs of Bounds~\ref{bound:unique-honest-catalan} 
% and~\ref{bound:two-catalans} 
% stated in Section~\ref{sec:bounds-main-proofs}. 
% Let us recall stochastic dominance from Definition~\ref{def:dominance} 
% and that,
As a rule, we denote the
probability distribution associated with a random variable using
uppercase script letters. 
Observe that if $Y \dominatedby X$ and 
$Z$ is independent of both $X$ and $Y$, 
then $Z + Y \dominatedby Z + X$. 
In addition, for any non-decreasing function $u$ defined on $\Omega$, 
$Y \dominatedby X$ implies $u(Y) \leq u(X)$.



  \paragraph{Generating functions.}
    We reserve the term
    \emph{generating function} to refer to an ``ordinary'' generating
    function which represents a sequence $a_0, a_1, \ldots$ of
    non-negative real numbers by the formal power series
    $\gf{A}(Z) = \sum_{t = 0}^\infty a_t Z^t$. 
    We denote the above correspondence as $\{a_t\} \SeqGF \gfA(Z)$. 
    When
    $\gf{A}(1) = \sum_t a_t = 1$ we say that the generating function is
    a \emph{probability generating function}; in this case, the
    generating function $\gf{A}$ can naturally be associated with the
    integer-valued random variable $A$ for which $\Pr[A = k] = a_k$. If
    the probability generating functions $\gf{A}$ and $\gf{B}$ are
    associated with the random variables $A$ and $B$, it is easy to
    check that $\gf{A} \cdot \gf{B}$ is the generating function
    associated with the convolution $A + B$ (where $A$ and $B$ are
    assumed to be independent).  Translating the notion of stochastic
    dominance to the setting with generating functions, we say that the
    generating function $\gf{A}$ \emph{stochastically dominates}
    $\gf{B}$ if $\sum_{t \leq T} a_t \leq \sum_{t \leq T} b_t$ for all
    $T \geq 0$; we write $\gf{B} \dominatedby \gf{A}$ to denote this state of
    affairs. If $\gf{B}_1 \dominatedby \gf{A}_1$ and
    $\gf{B}_2 \dominatedby \gf{A}_2$ then
    $\gf{B}_1 \cdot \gf{B}_2 \dominatedby \gf{A}_1 \cdot \gf{A}_2$ and
    $\alpha \gf{B}_1 + \beta \gf{B}_2 \dominatedby \alpha \gf{A}_1 + \beta
    \gf{A}_2$ (for any $\alpha, \beta \geq 0$).  Moreover, if
    $\gf{B} \dominatedby \gf{A}$ then it can be checked that
    $\gf{B}(\gf{C}) \dominatedby \gf{A}(\gf{C})$ for any probability
    generating function $\gf{C}(Z)$, where we write $\gf{A}(\gf{C})$ to
    denote the composition $\gf{A}(\gf{C}(Z))$.


    Finally, we remark that
    if $\gf{A}(Z)$ is a generating function which converges as a
    function of a complex $Z$ for $|Z| < R$ for some non-negative $R$, 
    $R$ is called the \emph{radius of convergence} of $\gf{A}$.  
    It follows from Theorem 2.19 in~\cite{WilfGF} that 
    $\lim_{k \rightarrow \infty} {|a_k|}R^k = 0$ and that $|a_k| = O(R^{-k})$. 
    In addition, if $\gf{A}$ is a probability generating function associated with the
    random variable $A$ then it follows that
    $\Pr[A \geq T] = O(R^{-T})$. 



\subsection{Proof of Bound~\ref{bound:unique-honest-catalan}}\label{sec:catalan-estimates}
  % \begin{proof}
  Let $p = (1 - \epsilon)/2$ and $q = (1 + \epsilon)/2$ 
  so that $q - p = \epsilon$. 
  Let $q_\H = q - q_\h$. 
  Let $B$ denote the event that 
  $w$ does not contain a uniquely honest Catalan slot in $y$. 
  We would like to bound $\Pr_w[B]$ from above.


  Define the process $W = (W_t : t \in \NN), W_t \in \{\pm 1\}$ as $W_t = 1$ if and only if $w_t = \A$. 
  Let $S = (S_t : t \in \NN), S_t = \sum_{i \leq t} W_i$ be the position of the particle at time $t$. 
  Thus $S$ is a random walk on $\ZZ$ with $\epsilon$ negative (i.e., downward) bias. 
  By convention, set $W_0 = S_0 = 0$. 

  \paragraph{Case 1: $x$ is an empty string.} 
  % \paragraph{Case: $x = \varepsilon$.} 
  % \subsubsection*{If $x = \varepsilon$} 
  % First, let us consider the case when $x = \varepsilon$, i.e., $|x| = 0$. 
  In this case, we write $w = yz$ so that $|y| = k$. 
  Let $c_t$ be the probability that $t$ is the first uniquely honest Catalan slot in $w$ 
  with $c_0 = 0$, and consider the probability generating function 
  $\{c_t\} \SeqGF \gfC(Z) = \sum_{t = 0}^\infty c_t Z^t$. 
  Controlling the decay of the coefficients $c_t$ suffices
  to give a bound on $\Pr[B]$, i.e., 
  the probability that 
  $y$ \emph{does not} contain a Catalan slot, 
  because this probability is at most 
  $
    1 - \sum_{t =0}^{k-1} c_t 
      = \sum_{t = k}^\infty c_t
      % \,.
  $. 
  % It seems challenging to give a closed-form algebraic expression for
  % the generating function $\gfC$; 
  To this end, we develop a
  closed-form expression for a related probability generating function
  $\gfChat(Z) = \sum_t \hat{c}_t Z^t$ which stochastically
  dominates $\gfC(Z)$. 
  Recall that this means that for any $k, \sum_{t \geq k} c_k \leq \sum_{t \geq k} \hat{c}_k$. 
  Finally, bound the latter sum  
  by using the analytic properties of $\gfChat(Z)$. 

  % \paragraph{Generating functions for the first ascent and descent.}
  % Before proceeding further, we set down two
  % elementary generating functions for the ``first descent'' and the ``first ascent''
  % stopping times. 
  Treating the random variables $W_1, \ldots$ as
  defining a (negatively) biased random walk, define $\gfD$ (resp. $\gfA$) to be
  the generating function for the \emph{descent stopping time} 
  (resp. the \emph{ascent stopping time}) 
  of the walk; this is the first time the random walk, starting at 0, visits
  $-1$ (resp. $+1$). 
  The natural recursive formulation of these descent time yield 
  simple algebraic equations for the descent generating function,
  $\gfD(Z) = qZ + pZ \gfD(Z)^2$ and $\gfA(Z) = pZ + qZ \gfA(Z)^2$, 
  and from this we may conclude
  \begin{align*}
    \gfD(Z) &= (1 - \sqrt{1 - 4pqZ^2})/2pZ\,, \\
    \gfA(Z) &= (1 - \sqrt{1 - 4pqZ^2})/2qZ\,.
  \end{align*}
  % Here, we discard the other candidate solution 
  % $(1 + \sqrt{1 - 4pqZ^2})/(2pZ)$ since in that case, for any real $\delta \in (0,1)$ we would get 
  % $\gfD(1 - \delta) > \gfD(1)$; 
  % but this conflicts with the fact that the coefficients of $\gfD(Z)$ are non-negative. 
  Note that while $\gfD$ is a probability generating function, 
  $\gfA$ is not: according to the classical
  ``gambler's ruin'' analysis,
  % ~\cite{Grinstead:1997ng}, 
  the probability
  that a negatively-biased random walk starting at 0 ever rises to 1
  is exactly $p/q$; thus $\gfA(1) = p/q$.

  % \paragraph{Generating function for the first uniquely honest Catalan slot.}
  Recall that a slot is Catalan in $w$ if and only if 
  it is both left-Catalan and right-Catalan. 
  % Thus we wish to devise the generating function $\gfC(Z)$ for 
  % the first occurrence of a left-Catalan slot 
  % which is a right-Catalan slot as well. 
  A slot is left-Catalan if the walk $S$ descends to a new low at that slot. 
  In addition, the same slot (say $s$) is right-Catalan 
  if the walk never reaches to that level in future, 
  i.e., $S_s \geq S_{i}, i \geq s + 1$. 
  The probability of this event is $1 - \gfA(1) = 1 - p/q = \epsilon/q$, 
  conditioned on the fact that $W_s = -1$.
  
  Assume that the walk is now at its historical minimum. 
  (It may or may not be a new minimum.)
  We can think of the generating function $\gfC(Z)$ as a search procedure 
  for finding the first uniquely honest Catalan slot. 
  Let $v$ be the first symbol we observe. 
  Let $\gfE(Z)$ be the generating function for a walk which 
  makes an ascent with certainty 
  and then descends again to its historical minimum.
  We claim that 
  % % $\gfC(Z) \DominatedBy \gfChat(Z)$
  % \[
  %    \gfC(Z) = pZ \gfD(Z) \gfC(Z) + q_\h Z\cdot \epsilon/q + q_\h Z \cdot p/q \cdot \gfE(Z) \gfC(Z) + q_\H Z \gfC(Z)
  %     \,.
  % \]
  \begin{align}
    \gfC(Z) 
      &= pZ \gfD(Z) \gfC(Z) + q_\h Z\cdot \epsilon/q + q_\h Z \cdot p/q \cdot \gfE(Z) \gfC(Z) + q_\H Z \gfC(Z) \nonumber
      \\
      % \,.
      % &= (q_\h \epsilon/q) Z/\bigl(1 - \left( pZ \gfD(Z) + (q_\h p/q) Z \gfE(Z) + q_\H Z\right) \bigr)      % \\
      &= \frac{(q_\h \epsilon/q) Z}{1 - \bigl( pZ \gfD(Z) + (q_\h p/q) Z \gfE(Z) + q_\H Z\bigr)}      % \\
    \label{eq:gfC}
    \,.
  \end{align}
  Here is the explanation. 
  Regarding the value of $v$, there can be four alternatives 
  for the walk which is currently at its historical minimum: 
  \begin{enumerate}[label=(\textit{\roman*})]
    \item With probability $p$, we have $v = \A$ and the walk moves up. 
    Then we wait till the walk makes a first descent and restart.

    \item With probability $q_\h \cdot  \epsilon/q$, we have 
    $v = \h$ and the walk diverges below. 
    Hence our search has succeeded and we stop.

    \item With probability $q_\h \cdot  (1 - \epsilon/q) = q_\h p/q$, we have 
    $v = \h$ and the walk returns to the origin from below. 
    Then we wait for the walk to match its minimum again 
    before we can restart. 
    Note that $\gfE(Z)$ is 
    the generating function for this ``guaranteed ascent then match minimum'' walk.

    \item With probability $q_\H$, we have $v = \H$ and the walk moves down. 
    Since we will reach a new minimum, we restart.
  \end{enumerate}
  % After rearranging, we get 
  % \begin{align}
  %     \gfC(Z) 
  %     % &= (q_\h \epsilon/q) Z/\bigl(1 - \left( pZ \gfD(Z) + (q_\h p/q) Z \gfE(Z) + q_\H Z\right) \bigr)      % \\
  %     &= \frac{(q_\h \epsilon/q) Z}{1 - \bigl( pZ \gfD(Z) + (q_\h p/q) Z \gfE(Z) + q_\H Z\bigr)}      % \\
  %     \label{eq:gfC}
  %     \,.
  % \end{align}
  Since $\gfE(1) = 1$ by assumption, 
  $p  + (q_\h p/q) + q_\H = 1 - q_\h(1 - p/q) = 1 - q_\h\epsilon/q$. 
  It follows that 
  $\gfC(1) = (q_\h \epsilon/q) / (1 -(1 - q_\h\epsilon/q) ) = 1$; 
  hence $\gfC(Z)$ is a probability generating function.    

  % A generating function of a stopping time of a random walk 
  % is ill suited to ``remember'' its historical minimum/maximum. 
  % However, it can remember the length of the walk for free. 
  Instead of working directly with $\gfE(Z)$, 
  we can work with a generating function $\gfEhat(Z)$ 
  which is identical to $\gfE(Z)$ for the initial ascending part 
  but differs in the descending part. 
  Specifically, in the descending part, 
  the walk represented by $\gfEhat(Z)$ descends as many levels 
  as the number of steps it took to return to the origin. 
  Clearly, $
      \gfE(Z) \DominatedBy \gfEhat(Z) \triangleq \gfA(Z \gfD(Z) )/\gfA(1)
      % \,.
  $. 
  Here, an individual term in $\gfA(Z \gfD(Z)) = \sum_i a_i Z^i \gfD(Z)^i$ 
  has the interpretation 
  ``if the first ascent took $i$ steps then immediately descend $i$ levels.''
  Since $\gfA(Z)$ is not a probability generating function, 
  we have to normalize it by $\gfA(1)$ to make sure that 
  the ascent happens with certainty. 
  Writing 
  \[
    \gfF(Z) \triangleq pZ \gfD(Z) + q_\h  Z \gfA(Z \gfD(Z) ) + q_\H Z
    \,,
  \]
  note that 
  \begin{align}
    % &\gfF(Z) \triangleq pZ \gfD(Z) + q_\h  Z \gfA(Z \gfD(Z) ) + q_\H Z
    % \,,\quad\text{note that}\label{eq:gfF}\\
    &\gfC(Z) 
        \DominatedBy \gfChat(Z) 
        \triangleq 
        % \frac{(q_\h \epsilon/q) Z}{1 - \gfF(Z) }
        (q_\h \epsilon/q) Z/(1 - \gfF(Z) )
    \,. \label{eq:gfChat}
  \end{align}
  Since $\gfF(1) = p + q_\h p/q + q_\H = 1 - q_\h(1 - p/q) = 1 - q_\h \epsilon/q$, 
  we have $\gfChat(1) = 1$, i.e.,  
  $\gfChat(Z)$ is a probability generating function. 
  % \paragraph{Condition for the convergence of $\gfC(Z)$.}
  It remains to establish a bound on the radius of convergence of
  $\gfChat$. 
  A sufficient condition for the convergence of
  $\gfChat(z)$ for some $z \in \RR$ is 
  that all generating functions appearing in the definition of
  $\gfChat(z)$ converge at $z$ and 
  that $\gfF(z) \neq 1$. 
  % The bulk of the remaining exposition revolves around 
  % proving this inequality.


  % \paragraph{Convergence of $\gfA(Z), \gfD(Z)$, and $\gfA(Z \gfD(Z))$.}
  The generating functions $\gfD(z)$ and $\gfA(z)$ converge when
  the discriminant $1 - 4pqz^2$ is positive; equivalently
  $|z| < 1/\sqrt{1 - \epsilon^2}
  = 1 + \epsilon^2/2 + O(\epsilon^4)$. 
  In addition, conditioned on the convergence of $\gfA(z)$ and $\gfD(z)$, 
  we can check that 
  \begin{equation}\label{eq:conv-value-A-and-D}
    \gfA(z) < 1/2qz\, \quad \text{and}\quad \gfD(z) < 1/2pz
    \,.
  \end{equation}
  On the other hand, the convergence of $\gfF(z)$ 
  depends on the convergence of $\gfD(z)$ and $\gfA(z \gfD(z))$. 
  The convergence of $\gfA(z \gfD(z))$ is likewise determined by the 
  positivity of its discriminant, i.e.,
  $$
    1 - (1 - \epsilon^2)\, \left(z \cdot \frac{1 - \sqrt{1 - (1 - \epsilon^2) z^2}}{(1 - \epsilon) z}\right)^2  > 0
    % 1 - (1 - \epsilon^2)\, \left(z \cdot (1 - \sqrt{1 - (1 - \epsilon^2) z^2})/(1 - \epsilon) z\right)^2  > 0
    \,.
  $$
  The inequality above implies that 
  if $\gfA(z \gfD(z))$ converges when 
  $$
    |z| 
    < R_1 
    \triangleq \left(\left(2/\sqrt{1 - \epsilon^2} - 1/(1+\epsilon)\right)/(1 + \epsilon)\right)^{1/2} 
    \,,
  $$ where 
  \begin{align}\label{eq:roc-AZDZ}
    R_1 = 1 + \epsilon^3/2 + O(\epsilon^4) 
    \approx \exp\left(\epsilon^3 (1 + O(\epsilon))/2 \right)
    \,.
  \end{align}
  \noindent
  Note that the radius of convergence of $\gfA(Z \gfD(Z))$ 
  is smaller than that of $\gfA(Z)$ or $\gfD(Z)$.

  % \paragraph{An upper bound on $\gfF(z)$.}
  We can check that when $\gfF(z)$ converges, 
  it satisfies $$\gfF(z) \leq \gfF(|z|)\,.$$ 
  The claim is trivial for $z = 0$. 
  Otherwise,
  note that $\gfD(z)$ is an odd function and hence, 
  $z \gfD(z) = |z|\, \gfD(|z|)$. 
  Thus, for the claim to hold, we need only show that 
  $
    z\,(q_\h \gfA(z\gfD(z)) + q_\H) 
    \leq |z|\,(q_\h \gfA(|z|\gfD(|z|)) + q_\H) 
  $. 
  But the right-hand side equals $|z|\,(q_\h \gfA(z\gfD(z)) + q_\H)$ 
  and $\gfA(x) > 0$ for real $x > 0$, 
  we can divide both sides by $q_\h \gfA(z\gfD(z)) + q_\H$. 
  The reduced inequality becomes 
  $z/|z| \leq 1$. 
  However, $z/|z| = \pm 1$ for any non-zero real $z$.
  Therefore, it suffices for us to require that $F(z) \neq 1$ for $z > 0$. 
  
  % \paragraph{$\gfF(z)$ is convex and increasing for $z \in [0, R_1]$.}
  We can also check that 
  \begin{equation}\label{eq:Fz-convex-increasing}
    \text{$F(z)$ is convex and increasing for $z \in [0, R_1)$}
    \,.
  \end{equation} 
  To see why, note that since $z^2$ is convex in $z$, 
  $(1 - 4pq z^2)$ is concave. 
  Since square root is non-decreasing and convex for positive $z$, 
  $\sqrt{1 - 4pqz^2}$ is concave and consequently, 
  $-\sqrt{1 - 4pqz^2}$ is convex. 
  Since $1/z^2$ is convex, 
  it follows that $\gfD(z)$ and, by a similar reasoning, $\gfA(z)$ are convex.
  Next, observe that $\gfA(z \gfD(z))$ converges for $z \in [0, R_1)$ 
  and hence it is also convex in $z$. 
  Thus $\gfF(z)$ turns out to be a convex combination of convex functions; 
  it follows that $F(z)$ is convex for $z \in (0, R_1)$. 
  In addition, 
  since $\gfF(0) = 0$ and $\gfF(1) > 0$, 
  $\gfF(z)$ must be increasing as well. 


  % \paragraph{A linear lower bound $f(z) \leq \gfF(z), z \geq 1$.}
  Let $$\text{ $R_2$ be the solution to the equation $\gfF(z) = 1, z > 0$}\,. $$ 
  Then $\gfChat(z)$ would converge for $|z| < R \triangleq \min(R_1, R_2)$. 
  It remains to characterize $R_2$ in terms of $\epsilon$ and $q_\h$. 
  Note that $R_1 < 2$ as long as $\epsilon \leq 0.97$.
  Since the final bounds will be only asymptotic in $\epsilon$, 
  it suffices for us to consider small $\epsilon$. 
  That is to say, we consider the case where $0 < z < R_1 < 2$, 
  i.e., $z - 1 < 1$.

  If we express $\gfF(z)$ as its power series around $z = 1$, we can check that 
  % $\gfF(1) = 1 - \epsilon q_\h/q$, 
  % $\gfF''(1) = \frac{1-\epsilon}{\epsilon^5}\left( q_\h (1+3\epsilon) + q_\H \epsilon^2 \right)$, and 
  % $\gfF'(1) = p(1+1/\epsilon) + q_\h(p/q)\bigl( 1+(1+1/\epsilon)/\epsilon \bigr) + q_\H
  % $.
  \begin{align*}
    \gfF(1) &= 1 - \epsilon q_\h/q\,,\\
    \gfF''(1) &= 
    % \frac{2p}{\epsilon^5}\left( 2q_\h (q+\epsilon) + q_\H \epsilon^2 \right) 
      \frac{1-\epsilon}{\epsilon^5}\left( q_\h (1+3\epsilon) + q_\H \epsilon^2 \right)\,, \quad\text{and}\\
    \gfF'(1) &= p(1+1/\epsilon) + q_\h(p/q)\bigl( 1+(1+1/\epsilon)/\epsilon \bigr) + q_\H
    \,.
  \end{align*}
  Since $\gfF''(1) > 0$ and $\gfF(z)$ is convex and increasing, 
  the first-order approximation 
  \begin{equation}
    f(z) = (1 - \epsilon q_\h/q) + \gfF'(1)(z-1) 
  \end{equation}
  is a lower bound for $\gfF(z)$ when $1 \leq z < R_1$. 
  The approximation error at any $z \in (1, 2)$ is 
  $\gfF(z) - f(z) = O(h(z))$ 
  where we define $$h(z) \triangleq \gfF''(1) (z-1)^2\,.$$
  Since the bounds we develop will have 
  either $O(\cdot)$ or $\Omega(\cdot)$ in the exponent, 
  it suffices to ensure that $R_2 = \Theta(R_2^*)$.
  In the exposition below, 
  we will only develop approximations $R_2^*$ satisfying 
  $R_2 = (1 - \theta) R_2^*$ 
  for a small positive constant $\theta \in (0, 1)$. 

  % \paragraph{The special case $q_\h = q$.}
  In the special case $q_\H = 0$, 
  $\gfF(Z)$ simplifies as $\gfF(Z) = pZ \gfD(Z) + q  Z \gfA(Z \gfD(Z) )$. 
  Note that $\gfF(Z)$ converges when $\gfA(Z \gfD(Z) )$ does 
  and it is not hard to check that $\gfF(z) < 1$. 
  Specifically, 
  we know that $\gfF(z)$ converges when $z \in[0, R_1)$ 
  and when it does, we claim that $\gfF(z) < 1$. 
  Specifically, when $z \in [0, 1]$, 
  $\gfF(z) \leq \gfF(1) = 1 - \epsilon q_\h/q = 1 - \epsilon < 1$ since $\epsilon < 1$. 
  On the other hand, 
  % when $z \in [1, R_1)$, 
  % we know that $\gfF(Z)$ is dominated by $pZ \gfD(Z) + q  Z \gfA(Z \gfD(Z) ) \cdot \gfD(Z)$ 
  % and, therefore, 
  we can check that $\gfD(z)$ is convex for $z \geq 0$ and, 
  in particular, the first order approximation $1 + (z-1)/\epsilon$ around $z = 1$ 
  is a lower bound for $\gfD(z), z \geq 1$.
  It follows that $\gfD(z) \geq 1$ for $z \in [1, R_1)$. 
  Consequently, 
  $\gfF(z) 
  \leq pZ \gfD(Z) + q  z \gfA(z \gfD(z) )\cdot \gfD(z) 
  = pz \gfD(z) + q  x \gfA(x) 
  < 1/2 + 1/2 = 1$ 
  where we write $x = z \gfD(z)$ and use~\eqref{eq:conv-value-A-and-D}. 
  Thus 
  the radius of convergence of $\gfChat$ is $R_1$ if $q_\H = 0$.

  The remainder of the exposition considers 
  the general case $0 < q_\h < q$. 
  % \paragraph{Approximating the $z$ that satisfies $f(z) = 1$.}
  Let the solution to the equation $f(z) = 1$ be denoted by 
  $$
    R_2^* \triangleq 1 + \epsilon (q_\h/q)/\gfF'(1)
    \,.
  $$ 
  % {\color{red}Therefore, as $\epsilon \rightarrow 0$, $R_2^* \rightarrow R_2$.} 
  % Let us focus on $h(R_2^*)$, i.e., 
  % the approximation error at $z = R_2^*$. 
  % Writing $b_\epsilon = 1 + 1/\epsilon$, 
  % this quantity, denoted by $\delta$, is  
  % \[
  %     \frac{
  %       (1-\epsilon)\left( q_\h (1+3\epsilon) + q_\H \epsilon^2 \right)\,(\epsilon q_\h/q)^2
  %     }{
  %       \epsilon^5\, (p b_\epsilon + q_\h(p/q)\bigl( 1+b_\epsilon/\epsilon \bigr) + q_\H)^2
  %     }
  %     % \frac{1-\epsilon}{\epsilon^5}\left( q_\h (1+3\epsilon) + q_\H \epsilon^2 \right)
  %     % \cdot
  %     % \left( \frac{\epsilon q_\h/q}{p(1+1/\epsilon) + q_\h(p/q)\bigl( 1+(1+1/\epsilon)/\epsilon \bigr) + q_\H} \right)^2
  %     \,.
  % % \end{align*}
  % \]
  If $q_\h$ is small, $q = (1+\epsilon)/2, p+\epsilon = q$ and $p/q^3 \in [1,4]$, 
  we can check that 
  % $\delta = O(\epsilon q_\h^2)$. 
  \[
    h(R_2^*)
    = O\left(\frac{pq}{\epsilon^3} \cdot \left( \frac{\epsilon^2 q_h/q}{p(1+\epsilon) + \epsilon q} \right)^2 \right) 
    =O\left(\frac{ \epsilon q_\h^2 \cdot pq}{q^2\,(p + \epsilon )^2}\right)
    =O\left(\frac{ \epsilon q_\h^2 \cdot p}{q^3}\right)
    = O(\epsilon q_\h^2)
    \,,
  \]
  i.e., it vanishes.
  % Therefore, $\lim_{q_\h \rightarrow 0} \delta = 0$ 
  Thus $f(z)$ is a good approximation for $\gfF(z)$.
  It follows that 
  $\gfF'(1) \approx p(1+1/\epsilon) + q = q/\epsilon$ 
  and, therefore, 
  $$R_2^* 
    \approx 1 + (\epsilon q_\h/q)/(q/\epsilon) 
    = 1 + q_\h(\epsilon/q)^2 
    \approx \exp(\epsilon^2 q_\h/q^2)
    = e^{O(\epsilon^2 q_\h)}
  $$ 
  since $q \in (1/2, 1)$. 
  (Although we have an asymptotic notation, 
  it is important that we have the right exponent on $q_\h$.)
  % As any lower bound on th right-hand side suffices, 
  % we take $R_2^* \approx \exp(\epsilon^2 q_\h)$.
  % However, $\epsilon^2/q^2 = 4 \epsilon^2(1 + \epsilon)^{-2} = 4 \epsilon^2(1 - 2 \epsilon + O(\epsilon^2))$. 
  % Hence $R_2^* \approx \exp(4\epsilon^2 q_\h(1 - 2\epsilon))=\Theta(\epsilon^2 q_\h)$.    
  
  If, on the contrary, $q_\h = O(1)$ but $\epsilon$ vanishes then 
  $\gfF'(1)$ will be dominated by its second term; 
  that is to say, 
  $\gfF'(1) 
  \approx q_\h(p/q) \left(1+(1+1/\epsilon)/\epsilon \right) 
  = O(q_\h/\epsilon^2)
  $
  and, therefore, 
  $$R_2^* 
  \approx 1 + O\left( (\epsilon q_\h/q)/( q_\h/\epsilon^2) \right) 
  = 1 + O(\epsilon^3) 
  = e^{O(\epsilon^3)}
  $$ since $q \approx 1/2$.
  % On the other hand, 
  % if $q_\H \rightarrow q$, 
  % i.e., if $q_\h$ is smaller than $\epsilon$, 
  % then 
  % $\gfF'(1) = \Theta(q/\epsilon^2)$ 
  % and consequently, 
  % $R_2^* 
  % \approx 1 + \epsilon/\gfF'(1) 
  % = 1 + \Theta(\epsilon^3/q) 
  % \approx \exp(\Theta(\epsilon^3))$ since $q \in (1/2, 1]$.
 

  % \paragraph{Putting it together for the case $|x| = 0$.}
  Recall that $R_1 = \exp\left(O(\epsilon^3 (1 + O(\epsilon)))\right)$. 
  It follows that $\gfChat(z)$ converges for 
  $|z|$ less than 
  \begin{align}\label{eq:RoC-unique-honest}
    R &= \exp\left(O(\min(\epsilon^3, \epsilon^2 q_\h))\right)
    \,.
  \end{align}

  % \begin{enumerate*}[label=(\textit{\roman*})]
  %   \item $R= e^{\epsilon^3 (1 + O(\epsilon))/2}$ if $q_\h = q$; 
  %   \item $R= e^{\Theta(\epsilon^3)}$ if $q_\h \rightarrow q$; and 
  %   \item $R= e^{\Theta( \min(\epsilon^3, \epsilon q_\h/q_\H)}$ otherwise. 
  % \end{enumerate*}
  % \begin{align}\label{eq:RoC-unique-honest}
  %   \ln R &= 
  %     \begin{cases}
  %       \epsilon^3 (1 + O(\epsilon))/2 & \text{if $q_\h = q$}\,,\\
  %       \Theta(\epsilon^3) & \text{if $q_\h \rightarrow q$}\,,\\
  %       % \Theta(\min(\epsilon^3, \epsilon^2 q_\h/q_\H^2)) & \text{if $q_\h \rightarrow 0$}\,,\\
  %       \Theta(\min(\epsilon^3, \epsilon^2 q_\h))  & \text{if $q_\h \rightarrow 0$}
  %       % \,\\
  %       % \Theta( \min(\epsilon^3, \epsilon q_\h/q_\H) & \text{otherwise}
  %       \,.
  %     \end{cases}
  % \end{align}
  Recall that if the radius of convergence of
  $\gfChat$ is $\exp(\delta)$ then 
  $\hat{c}_k = O(e^{-\delta k})$. 
  Hence, $\Pr[B]$ is a geometric sum and it is 
  at most $O(e^{-\delta k})$ as well. 
  We conclude that 
  % for large $k$,
  $$
    \Pr_w[B] 
      \leq O\left(e^{-k \ln R }\right)
      = \exp\left(-k\cdot \Omega(\min(\epsilon^3, \epsilon^2 q_\h))\right)
      \,.
  $$
  % \begin{align}
  %   \Pr[H] &\leq O(1)\cdot e^{-k \ln R }
  %     \,.
  % \label{eq:prob_catalan_gf}
  % \end{align}


  \paragraph{Case 2: $x$ is non-empty.}
  % \paragraph{Case: $|x| \geq 1$.}
  % \subsubsection*{If $|x| \geq 1$}
  Next, let us consider the case when $x \neq \varepsilon$, i.e., $|x| \geq 1$. 
  Let $m = |x|$ and write $w = xyz$ where $|y| = k$. 
  Recall the processes $(W_t)$ and $(S_t)$ defined on $w$
  and, in addition, define $M = (M_t : t \in \NN), M_t = \min_{0 \leq i \leq t } S_i$ 
  and $X = (X_t : t \in \NN), X_t = S_t - M_t$. 
  By convention, set $M_0 = X_0 = 0$. 
  Thus $X_t$ denotes the height of the walk $S$, at time $t$, 
  with respect to its minimum $M_t$.

  % For any slot $m + s$ of $w$, it is left-Catalan 
  % if and only if the walk $S$, 
  % starting at level $S_m = M_m + X_m$, 
  % descends to level $M_m - 1 = S_m - (X_m + 1)$ 
  % for the first time at slot $m + s$. 
  % In other words, 
  For a fixed value $h = X_m$, the relevant generating function 
  would be $\gfD(Z)^{h}\gfChat$. 
  Hence the final generating function we seek is
  $$
    \gfCtilde(Z) \defeq \sum_{h = 0}^\infty \Pr[X_m = h] \cdot \gfD(Z)^h  \gfChat(Z)
    % \,.
  $$
  whose $t$th coefficient is the probability that 
  $t$ is a Catalan slot in $y$.

  Note that $X = (X_t)$ is an $\epsilon$-downward biased random walk 
  on non-negative integers with a reflective barrier at $-1$. 
  Specifically, 
  for any $h \in \NN, \Pr[X_t = h \Given X_{t-1} = h -1] = p$ and 
  $\Pr[X_t = h - 1 \Given X_{t-1} = h] = \Pr[X_t = 0 \Given X_{t-1} = 0] = q$. 
  In~\cite[Lemma 6.1]{LinearConsistencySODA}, it is proved that 
  the distribution of $X_m$ is stochastically dominated by 
  the distribution of $X_\infty$, written $\mathcal{X}_\infty$ and 
  defined, for $k = 0, 1, 2, \ldots$, as 
  % $
  %     \mathcal{X}_\infty(k) 
  %     = \Pr[X_\infty = k] 
  %     = (1 - \beta) \beta^k
  % $
  % where $\beta \defeq (1-\epsilon)/(1+\epsilon)$. 
  \begin{equation}
    \label{eq:stationary}
      \mathcal{X}_\infty(k) = \Pr[X_\infty = k] 
      \defeq 
        \left(\frac{2\epsilon}{1+\epsilon}\right)\cdot \left(\frac{1-\epsilon}{1 + \epsilon}\right)^k
        = 
      (1 - \beta) \beta^k
  \end{equation}
  where $\beta \defeq (1-\epsilon)/(1+\epsilon)$. 
  Let 
  $$
    \{\mathcal{X}_\infty(k)\} \SeqGF \gfXinf(Z) = 
    % (1 - \beta)/(1 - \beta Z)
    \frac{1 - \beta}{1 - \beta Z}
    \,.
  $$ 
  It follows that $\gfCtilde(Z)$ is dominated by 
  $$
      \sum_{h = 0}^\infty \mathcal{X}_\infty(h) \gfD(Z)^h \gfChat(Z)
    = \gfXinf(\gfD(Z)) \gfChat(Z)
    = \frac{(1 - \beta)\gfChat(Z)}{1 - \beta \gfD(Z)}
    % = (1 - \beta)\gfChat(Z)/(1 - \beta \gfD(Z))
    \,.
  $$
  % \end{align*}

  Let $\star$ denote the quantity above. 
  For it to converge, 
  we need to check that $\gf{D}(Z)$
  should never converge to $1/\beta$.  
  Since the radius of convergence of $\gf{D}(Z)$---which is
  $(1-\epsilon^2)^{-1/2}$---is strictly less than 
  $(1+\epsilon)/(1-\epsilon)$ for $\epsilon > 0$, 
  we conclude that $\star$ converges if
  both $\gf{D}(Z)$ and $\gfChat(Z)$ converge.  The radius of
  convergence of $\star$ would be the smaller of the radii
  of convergence of $\gfD(Z)$ and $\gfChat(Z)$.  We already
  know from the previous analysis that $\gfChat(Z)$ has the
  smaller radius of convergence of these two; 
  therefore, the bound
  % in~\eqref{eq:prob_catalan_gf} 
  on $\Pr_w[B]$ from the previous case holds for $|x| \geq 0$. 
  % \end{proof}
  \hfill$\qed$




\subsection{Proof of Bound~\ref{bound:two-catalans}}
  % (The full proof is presented in 
  % Section~\ref{sec:two-catalan-estimates}.
  % % ~\cite{MultiHonestFullVersion}.
  % )
  % The random walk of interest is the same as in the previous proof. 
  % However, we are interested in a slightly different stopping time. 
  % Its generating function is  
  % $\gfM(Z) = \frac{\epsilon \gfD(Z)}{1 - (1 - \epsilon) \gfE(Z) }$ 
  % where $\gfE(Z)$, the ``epoch generating function,'' is dominated by 
  % $\gfEhat(Z) = p Z \gfD(Z) + q Z \gfA(Z \gfD(Z) )/\gfA(1)$. 
  % % That is, the epochs in this case can have two shapes: 
  % % either the walk goes up and then comes back down, 
  % % or it goes down, comes back up with certainty (say, after $t$ steps), 
  % % and then descends $t$ levels.
  % It can be easily checked that $\gfM(Z)$ converges as long as 
  % $\gfA(Z), \gfD(Z)$, and $\gfA(Z \gfD(Z))$ converge 
  % and $(1 - \epsilon)\gfEhat(Z) < 1$.
  % Thus the radius of convergence of $\gfM(Z)$ is determined by~\eqref{eq:roc-AZDZ}. 
  % \hfill$\qed$
    % \input{estimates-two}
  % \begin{proof}
  Let $p = (1 - \epsilon)/2$ and $q = 1 - p$; 
  thus $q - p = \epsilon$. 
  Let $B$ denote the event that 
  $w$ does not contain two consecutive Catalan slots in $y$. 
  We would like to bound $\Pr_w[B]$ from above.

  Define the process $W = (W_t : t \in \NN), W_t \in \{\pm 1\}$ as $W_t = 1$ if and only if $w_t = \A$. 
  Let $S = (S_t : t \in \NN), S_t = \sum_{i \leq t} W_i$ be the position of the particle at time $t$. 
  Thus $S$ is a random walk on $\ZZ$ with $\epsilon$ negative (i.e., downward) bias. 
  By convention, set $W_0 = S_0 = 0$. 

  \paragraph{Case 1: $x$ is an empty string.} 
  In this case, we write $w = yz$ so that $|y| = k$. 
  Let $m_t$ denote the probability that 
  $t$ is the first index so that both $t$ and $t+1$ are Catalan slots in $w$, 
  with $m_0 = 0$, and consider the probability generating function 
  $\{m_t\} \SeqGF \gfM(Z) = \sum_{t = 0}^\infty m_t Z^t$. 
  Controlling the decay of the coefficients $m_t$ suffices
  to give a bound on $\Pr[B]$, i.e., 
  the probability that 
  $y$ \emph{does not} contain two consecutive Catalan slots, 
  because this probability is at most 
  $
    1 - \sum_{t =0}^{k-1} m_t 
      = \sum_{t = k}^\infty m_t
      % \,.
  $. 
  % It seems challenging to give a closed-form algebraic expression for
  % the generating function $\gfM$; 
  To this end, we develop a
  closed-form expression for a related probability generating function
  $\gfMhat(Z) = \sum_t \hat{m}_t Z^t$ which stochastically
  dominates $\gfM(Z)$. 
  Recall that this means that for any $k, \sum_{t \geq k} m_k \leq \sum_{t \geq k} \hat{m}_k$. 
  Finally, bound the latter sum  
  by using the analytic properties of $\gfMhat(Z)$. 


  % \paragraph{Generating function for the first instance of consecutive Catalan slots.}
  Recall the ``first ascent'' and ``first descent'' 
  generating functions $\gfA(Z)$ and $\gfD(Z)$ 
  from the proof of Bound~\ref{bound:unique-honest-catalan}. 
  We wish to devise the generating function for 
  the first occurrence of a left-Catalan slot 
  immediately followed by a right-Catalan slot. 
  To that end, 
  note that $\gfD(Z)$ is the generating function for 
  the first left-Catalan slot. 
  The generating function for the first right-Catalan slot 
  can be devised as follows. 
  Consider the walk $S$ starting at the origin. 
  With probability $q(1 - p/q) = \epsilon$, the walk will 
  immediately descend a step and never return to the origin. 
  But this means $S_1 \leq S_t, t \geq 2$ and hence 
  the first slot is a right-Catalan slot and we are done. 
  Otherwise, i.e., with probability $1 - \epsilon$, 
  the walk makes a (guaranteed) return to the origin in future. 
  In this case, we will have to restart our search 
  for the next consecutive Catalan slots but, 
  before that, 
  we will have to ensure that we are in a ``safe position.'' 
  In particular, we can safely restart our search if 
  Specifically, if the current position (i.e., level) of the walk is at its historical minimum, 
  we can restart our search by applying $\gfD(Z)$ to find the next left-Catalan slot.
  Thus an ``epoch'' begins with a guaranteed return and 
  ends when the walk descends to a new level for the first time. 
  Let $\gfE(Z)$ be the generating function of an epoch. 
  Thus we can write 
  \begin{align}\label{eq:gfM}
    \gfM(Z) 
    &= \gfD(Z) \cdot \{\epsilon + (1-\epsilon)\gfE(Z)\gfM(Z) \} \nonumber \\
    &= \frac{\epsilon \gfD(Z)}{1 - (1 - \epsilon) \gfE(Z) }
    \,.
  \end{align}

  
  An epoch can have two shapes. 
  If an epoch starts with an up-step (i.e., an ``up'' shape), 
  it is easy to see that the epoch ends as soon as the walk 
  returns to the origin from above and, importantly, 
  that the walk will (eventually) return to the origin with probability one. 
  However, if the epoch starts with a down-step (i.e., a ``down'' shape), 
  we have to ``remember'' the lowest level $\ell$ touched 
  by the walk in its way to its (sure) ascent to the origin 
  and then descend $\ell$ levels to end the epoch. 
  In particular, we have to ensure that we return to the origin with probability one. 
  
  A generating function of a stopping time of a random walk 
  is ill suited to ``remember'' its historical minimum/maximum. 
  However, it can remember the length of the walk for free. 
  Thus, instead of working directly with $\gfE(Z)$, 
  we work with a generating function $\gfEhat(Z)$ 
  which is identical to $\gfE(Z)$ for the up shape 
  but differs in the down shape. 
  Specifically, in the down shape, 
  the walk represented by $\gfEhat(Z)$ descends as many levels 
  as the number of steps it took to return to the origin. 
  Clearly, $\gfE \DominatedBy \gfEhat$ where 
  \[
      \gfEhat(Z) \triangleq p Z \gfD(Z) + q Z \gfA(Z \gfD(Z) )/\gfA(1)
      \,.
  \] 
  Here, the first term denotes the ``return to origin from above'' shape. 
  An individual term in $\gfA(Z \gfD(Z)) = \sum_t a_t Z^t \gfD(Z)^t$ 
  has the interpretation 
  ``if the first ascent took $t$ steps then follow it by descending $t$ levels.''
  Since $\gfA(Z)$ is not a probability generating function, 
  we have to normalize it by $\gfA(1)$ to denote that 
  the ascent happens with certainty. 
  This implies, 
  \[
      \gfM(Z) 
          \DominatedBy \gfMhat(Z) 
          \triangleq \frac{\epsilon \gfD(Z)}{1 - (1 - \epsilon) \gfEhat(Z) }
  \]

  % \paragraph{Convergence of $\gfM(Z)$.}
  It remains to establish a bound on the radius of convergence of
  $\gfMhat$. 
  A sufficient condition for the convergence of
  $\gfMhat(z)$ for some $z \in \RR$ is 
  that all generating functions appearing in the definition of
  $\gfMhat$ converge at $z$ and 
  that $(1-\epsilon) \gfEhat(Z) \neq 1$. 

  By retracing our footsteps as in the proof of Bound~\ref{bound:unique-honest-catalan}, 
  we can see that $\gfD(z), \gfA(z)$, and $\gfA(z \gfD(z))$ converge 
  when $|z|$ satisfies~\eqref{eq:roc-AZDZ}. 
  Moreover, since $\gfD(Z)$ is a probability generating function, 
  it follows that $\gfEhat(Z)$ is stochastically dominated by 
  $p Z \gfD(Z) + q Z \gfA(Z \gfD(Z) )/\gfA(1) \cdot \gfD(Z)$.
  Therefore, when $\gfEhat(z)$ converges for some $z$, it satisfies 
  \begin{align*}
      \gfEhat(z)
      &\leq pz\gfD(z) + (q/p) (q z\gfD(z))\gfA(z\gfD(z)) \\
      &< 1/2 + (q/p)/2
  \end{align*}
  since $\gfA(1) = p/q, pz\gfD(z) < 1/2$, 
  and $qx\gfA(x) < 1/2$ for any $z, x$ so that $\gfA(x)$ and $\gfD(z)$ converge, respectively. 
  Therefore, $(1-\epsilon)\gfEhat(z) = 2p \gfEhat(z) < p + q = 1$. 
  It follows that 
  $\gfMhat(z)$ converges for
  $|z| < 1 + \epsilon^3/2 + O(\epsilon^4) \leq \exp(\epsilon^3/2 + O(\epsilon^4))$. 
  Recall that if the radius of convergence of
  $\gfMhat$ is $\exp(\delta)$ then 
  $\Pr[B]$ is 
  $O(1) \cdot e^{-\delta k}$. 
  We conclude that
  \begin{align}
    \Pr_w[B] 
      &\leq O(1) \cdot e^{-\epsilon^3(1 + O(\epsilon))k/2} \,.
  \label{eq:prob_two_catalan_gf}
  \end{align}


  \paragraph{Case 2: $x$ is non-empty.}
  This part of the proof is the same as the $|x| \geq 1$ case 
  in the proof of Bound~\ref{bound:unique-honest-catalan}. 
  The only difference is that 
  $\gfChat(Z)$ and $\gfCtilde(Z)$ would be replaced by 
  $\gfMhat(Z)$ and $\gfMtilde(Z)$, respectively, where 
  \[
    \gfMtilde(Z)\DominatedBy \sum_{h = 0}^\infty \mathcal{X}_\infty(h) \gfD(Z)^h \gfMhat(Z)
    \,.
  \]
  We conclude that the bound
  in~\eqref{eq:prob_two_catalan_gf} holds when $|x| \geq 0$. 
  \hfill$\qed$
 % \end{proof}


  



