
Recall that in Part~\ref{part:praos}, 
we analyzed the randomness beacon in Ouroboros Praos\cite{Praos} and Snow White~\cite{SnowWhite} 
in the synchronous setting. 
We showed 
that the min-entropy loss in the beacon output 
is logarithmic in the security parameter if the adversarial stake is below $9\%$; 
otherwise, it grows linearly in the security parameter.
As Praos is deployed in practice (since August 2020), 
it remains an open question to design simple PoS beacons for Praos 
which can withstand large adversarial stakes.

In this part of the dissertation, 
we present PoS beacons 
whose two salient features are simplicity and high quality.
In the private leader election setting (e.g., Praos and Snow White), 
our beacon has a superior output quality 
(compared to Praos) 
as long as the adversarial bias is at most $41\%$. 
Although this beacon does not cover adversarial stakes $\alpha$ above $41\%$, 
the regime $\alpha \in (0, 0.41)$ is nonetheless practical: 
for practical values for the consistency parameter $k$, 
Praos' consistency error bound 
starts to degrade when the adversarial stake goes above $40\%$.) 
As an immediate application, it would improve the grinding tolerance in Praos; 
see Figure~\ref{fig:rho-poisson-beacon} for a comparison.




We also design a PoS beacon in the public leader election setting 
which can work with Ouroboros classic~\cite{Ouroboros}. 
Recall that Ouroboros classic uses an MPC-based beacon protocol 
which precludes any grinding. 
Implementing this protocol in practice, however, is not palatable. 
We believe that this beacon will be valuable for 
PoS protocols 
that do not have the means to use/implement the MPC-based beacon 
and, in exchange, are willing to tolerate a very low min-entropy loss 
in the beacon. 
See Figure~\ref{fig:rho-bernoulli-beacon}.


\section{Related work}
Our new beacons are closely related to the Algorand beacon~\cite{Algorand}. 
In both protocols, 
the players are equipped with verifiable
random functions (VRF, see Definition~\ref{def:VRF}) 
From each round, multiple inputs are submitted 
but the lexicographically smallest one is considered as the winning input for that round.
Algorand obtains the beacon output by applying an XOR to a number of winning inputs. 
Both protocols guarantee that the beacon output (which is based on past $R$ rounds, for some integer $R$) 
is uniform with high probability, i.e., 
the beacon output incurs only a small loss in min-entropy. 

But the two analyses differ in whether the consensus is eventual or instant.
In Algorand, each winning input is confirmed via 
an expected constant-time Byzantine Agreement. 
Therefore, 
a player cannot delay his input and neither does he have any direct knowledge 
about future inputs. 
This is the dynamic of an ``instant consensus'' setting.

In contrast, our analysis is tailored toward the the eventual consensus setting.
The model allows a player at the current round to 
submit his input after observing a number of future inputs 
(called a \emph{lookahead}). 
This lookahead captures the eventual consensus dynamic 
where the blockchain takes some time to ``settle.'' 

See Section~\ref{sec:praos-related-work} for 
a more detailed survey of randomness beacon schemes.



\section{A technical overview}
To do



\section{Outline of the exposition}
To do




