
In a grinding attack, 
the adversary tries to reduce the output quality 
of a randomness beacon protocol.
One way to characterize the severity of a grinding attack 
is to analyze the so-called ``grinding power;'' 
this is the number of \emph{potential} outputs 
from which the adversary effectively selects the beacon output.

The goal of Part~\ref{part:praos} of this dissertation was to understand the grinding power in 
eventual consensus PoS blockchains such as 
Ouroboros Praos~\cite{Praos} and SnowWhite~\cite{SnowWhite} 
in the synchronous setting. 
In Theorem~\ref{thm:minentropy-loss-praos-multi-epochs}, 
we showed that the min-entropy loss in these beacons  
is logarithmic in the security parameter if the adversarial stake is below $9.55\%$; 
otherwise, it grows linearly in the security parameter. 
As of now, there does not exist a simple eventual consensus PoS beacon (i.e., without using MPC or special hardware) 
which can withstand an adversarial stake above $10\%$. 
In addition, as Praos is deployed in the real-world cryptocurrency Cardano~\cite{Cardano,CardanoPraos} 
designing such a secure beacon is a pressing issue.

Our goal in Part~\ref{part:xorgames} of this dissertation is 
to design beacons that assume 
only the eventual consensus mechanism. 
In particular, they will be agnostic about whether the underlying consensus (or leader election) mechanism is based on PoW or PoS. 
As long as we can plug in an appropriate leader election distribution to our beacon, 
we can get it off the ground.







\section{Drawbacks of existing eventual consensus PoS beacons}  
    Below, we glean our insights into why the beacon protocol in Praos 
    is prone to a high grinding power. 
    In Praos, 
    nonce inputs are contained within the blocks in the blockchain 
    and, after an epoch is over, 
    the beacon output is obtained by XORing the nonce inputs from a consistent chain prefix 
    and then applying a hash.
    While this is simple, 
    it actually gives the adversary a formidable power. 
    Specifically, he wields so large a grinding power in Praos 
    because 
    he can completely suppress the nonce inputs from an honest slot 
    by creating a viable alternative blockchain (see Section~\ref{def:extension-mh}) 
    that does not contain a block from this honest slot. 

    Having the nonce inputs \emph{on-chain} (i.e., inside the blocks)
    actually couples the grinding problem with the consistency problem. 
    In the Praos beacon, a grinding attack is maximized in a time window 
    if the adversary can maximize the number of viable blockchains emanating out of this window. 
    However, this also maximizes a consistency attack. 
    Thus the Praos beacon allows a consistency attack to 
    become a grinding attack as well.

    One might think that we could just ``slow down the beacon'' in Praos 
    by taking, for example, nonce inputs from every other slot. 
    While this will reduce (by half) the number of nonce-emitting slots in a given window, 
    it will \emph{not reduce the number of competing blockchains} emanating out of this window. 
    Each of these blockchains will yield a different beacon value. 


    See Section~\ref{sec:praos-related-work} for 
    a more detailed survey of randomness beacon schemes.



\section{Our contributions}
    We design eventual consensus PoS beacons 
    (Definition~\ref{def:new-beacon}) 
    with two salient features: simplicity and high quality. 
    The heart of our beacon algorithm has a one-line description: 
    \quotebox{\singlespacing
        Take the lexicographically smallest nonce 
        submitted by the nonce-leader(s) 
        at every nonce-emitting slot, 
        XOR them along with the initial value, 
        and hash the result.
    }


    In the private leader election setting (e.g., as in Praos and Snow White), 
    our beacon has a superior min-entropy 
    as long as the adversarial bias is at most $41\%$ (Theorem~\ref{thm:beacon-poisson-multi-epoch}). 
    This regime---albeit short of the ``adversarial sub-majority'' model of Praos---is practical nonetheless: 
    for practical values for the consistency parameter $k$, 
    Praos' consistency error bound 
    starts to degrade when the adversarial stake goes above $40\%$.) 
    As an immediate application, it would improve the grinding tolerance in Praos; 
    see Figure~\ref{fig:poisson-beacon} for comparison.


    We also design a PoS beacon in the public leader election setting (Theorem~\ref{thm:beacon-bernoulli-multi-epoch})
    which can work with Ouroboros classic~\cite{Ouroboros}. 
    Recall that Ouroboros classic uses an MPC-based beacon protocol 
    which precludes any grinding. 
    Implementing this protocol in practice, however, is not palatable. 
    We believe that this beacon will be valuable for 
    PoS protocols 
    that do not have the means to use/implement the MPC-based beacon 
    and, in exchange, are willing to tolerate a very low min-entropy loss 
    in the beacon. 
    See Figure~\ref{fig:rho-bernoulli-beacon}.


    Finally, for eventual consensus PoS blockchains, 
    our analysis highlights 
    how a grinding attack can weaken a tight guarantee on the consistency error. 
    Such a cohesive treatment of both attacks is a first in the literature.
    In addition, our technique for analyzing 
    the foreknowledge afforded to the adversary in the eventual consensus setting 
    may be of independent interest.




\section{A technical overview}~
    These insights will lead us to new, simple beacons. 
    Some ideas in our beacon (e.g., min-then-XOR) can be dated back to Algorand (see below) 
    whose beacon is in a different setting than ours.

    \Paragraph{Main ideas for mitigating the shortcomings of the Praos beacon.}
    It is clear that to design a better beacon, 
    we must make it difficult for the adversary to 
    bypass honest nonce submissions, 
    and conduct both a consistency attack and a grinding attack using the same strategy. 
    Above all, we should decouple the eventual consensus mechanism 
    from the beacon mechanism.

    In fact, it \emph{is} possible to decouple the nonce submission process from the eventual consensus process. 
    As long as the blockchain protocol satisfies the consistency property (e.g., $\kSlotCP$, Definition~\ref{def:cp-slot-mh}), 
    we can use its ledger functionality as a black-box for recording the nonce inputs. 
    Moreover, we can use the liveness property (e.g., $\sECQ$, Definition~\ref{def:ECQ}) 
    to make sure that \emph{every honest nonce input is recorded in the ledger.}\footnote{ 
        An example of doing the latter is the ``input endorsing'' mechanism in Ouroboros~\cite{Ouroboros} 
        or the ``fruitchains'' mechanism in Snow White~\cite{SnowWhite,Fruitchains}.
    }
    Furthermore, once the blockchain execution does not control the grinding power, 
    we are able to ``slow down the beacon'' by reducing the number of nonce emitting slots 
    and, thereby, reduce the (still) exponential dependence on $k$.



    \Paragraph{Beacon abstraction via XOR target games.}
    In XOR target games (\Section~\ref{sec:xor-games}), 
    we deconstruct the eventual consensus blockchain dynamic 
    to highlight a simple yet important phenomena: 
    \begin{enumerate}
        \item The adversary can observe the honest nonce inputs up to $k$ slots in the future 
        before announcing his nonce inputs from a given slot.
        Here, $k$ is given by the blockchain protocols consistency property.

        \item If he delays his announcement any further, the current slot will be ``settled'' 
        and the ledger will never accept the yet-unannounced nonce.

        \item If a slot has no adversarial winners but one or more honest winners, 
        it effectively re-randomizes the future output distribution.
    \end{enumerate}
    In Theorem~\ref{thm:xor-game-lookahead}, 
    we prove an upper bound on the grinding power of an XOR target game 
    by harnessing these observations. 

    \Paragraph{Distribution of the sizes of the option sets.}
    Note that the sizes of the option sets come from 
    the leader election process $\LeaderElection$
    and let $\Distribution = \Distribution(\alpha)$ be the distribution 
    of these sizes, induced by $\LeaderElection$, 
    where $\alpha$ is the adversarial stake ratio.
    Therefore, the grinding power would be small if 
    $\Distribution$ assigns large probabilities to small non-negative integers.
    If $\LeaderElection = \PublicLeaderElection(\alpha)$ then $\Distribution$ is, in fact, 
    a Bernoulli distribution. 
    On the other hand, if $\LeaderElection = \PrivateLeaderElection(\alpha)$ then 
    $\Distribution$ has a richer structure.

    \Paragraph{Moment bounds on $\Distribution$.}
    The distribution $\Distribution$ corresponds to the random variable 
    ``number of adversarial winners whose nonces are lexicographically smaller 
    than the lexicographically smallest nonce from the honest winners.'' 
    Our wish is to show that 
    $\Distribution$ has an exponentially decaying tail 
    and, to that end, 
    we need moment bounds on $\Distribution$. 
    Instead of directly analyzing the random variable above, 
    we make the observation that it is (clearly) 
    stochastically dominated by 
    the number of adversarial winners.
    This latter distribution has the Poisson binomial distribution 
    $\PoissonBinomial(\alpha)$ 
    (Definition~\ref{def:poisson-binomial-distribution}).
    
    LeCam's theorem states that $\PoissonBinomial(\alpha)$ is 
    statistically close (in the total variation distance) 
    to the Poisson distribution $\Poisson(\alpha)$. 
    If $\PoissonBinomial(\alpha) \DominatedBy \Poisson(\alpha)$ were true, 
    we could have used Poisson moment bounds to 
    derive a tail bound on $\Distribution$; 
    but this premise is incorrect. 
    Fortunately, however, 
    the second moment of $\Poisson(\alpha)$ 
    is larger than that of $\PoissonBinomial(\alpha)$.
    The monotonicity of non-negative real moments 
    of non-negative random variables
    (a special case of H\:{o}lder's inequality) 
    suggests that 
    we can upper bound the $\lambda$-th moment of $\Distribution(\alpha)$, 
    $\lambda \in (1,2]$, via the second moment of $\Poisson(\alpha)$. 
    This moment bound holds for $\alpha \in (0, 0.41)$, 
    resulting in Lemma~\ref{thm:xor-game-private-election} 
    which implies our main theorem, 
    Theorem~\ref{thm:minentropy-loss-praos-multi-epochs}.


    We remark that an exact (or stricter) analysis of 
    the moments of $\Distribution$ 
    would result in improved bounds on the grinding power. 
    We leave this as future work.



    \Paragraph{Our new beacon and Algorand.}
        Our new beacons are closely related to the Algorand beacon~\cite{Algorand}. 
        In both protocols, 
        the players are equipped with verifiable
        random functions (VRF, see Definition~\ref{def:VRF}). 
        From each round, multiple nonce inputs are submitted 
        but the lexicographically smallest one is considered as the winning input for that round.
        The beacon output is obtained by XORing all winning nonces and then applying a hash. 
        Both protocols guarantee that the beacon output 
        is uniform with high probability, i.e., 
        it incurs only a small loss in min-entropy. 

        But the two protocols (and respective analyses) differ in important ways. 
        In Algorand, each winning input is confirmed via 
        an expected constant-time Byzantine Agreement. 
        Therefore, 
        a player cannot delay his input and neither does he have any direct knowledge 
        about future inputs. 
        This is the dynamic of an ``instant consensus'' setting. 
        This also means that honest inputs (contained within respective blocks) 
        cannot be suppressed or bypassed.

        In contrast, our analysis is tailored toward the eventual consensus setting.
        The model allows a player at the current round to 
        submit his input after observing a number of future inputs 
        (called a \emph{lookahead}). 
        This lookahead captures the eventual consensus dynamic 
        where the blockchain takes some time to ``settle.'' 






Next, we describe a ``nonce channel'' which guarantees that 
every honest nonce inputs will contribute to the beacon output, 
i.e., they cannot be bypassed. 


\section{Nonce leaders and the nonce channel}
Recall the private and public lottery-based leader election schemes 
from Section~\ref{sec:static-dynamic-adversary}. 
Consider a beacon protocol of dimension $\kappa$, 
i.e., one where the initial values and the outputs are $\kappa$-bit Boolean strings. 
Inside an epoch, 
the participants $\Players$ take part in the \emph{nonce leader election}. 
If a player is elected a nonce leader, 
he emits a uniformly random string from $\{0,1\}^\kappa$  
in the \emph{nonce channel}. 

\begin{definition}[Nonce channel with parameter $k \in \NN$]\label{def:nonce-channel}
  Messages announced in the nonce channel is a tuple $m = (u, x, i, j)$ 
  which reads ``player $u$ is submitting nonce $x$ for slot $i$, 
  and the announcement is made at slot $j$.''
  A nonce message, received at slot $t$, becomes \emph{invalid} 
  if $j > i + k$.
  The nonce channel is a public, immutable ledger: 
  once a valid nonce is announced in this channel, 
  it remains there forever. 
\end{definition}
\noindent
The basic property that we seek from the nonce channel is that 
honest nonces are never lost. 


Notice that the nonce channel can be implemented 
on top of a blockchain protocol $\Blockchain$ 
and mechanisms such as ``input endorsers'' in Ouroboros~\cite{Ouroboros} 
or ``fruits'' in Fruitchain or SnowWhite~\cite{Fruitchains,SnowWhite}.
Here is an example: 
Suppose $\Blockchain$ has the $\sECQ$ property and $\kSlotCP$ property. 
(The validity parameter $k$ in the nonce channel of $\Pi_\Beacon$ 
must match the consistency parameter $k$ in $\Blockchain$.)
Then
\begin{itemize}
  \item A nonce leader in $\Pi_\Beacon$ asks to record his nonce message on $\Blockchain$.
  This announcement is seen (and remembered) by all honest observers in $\Pi_\Beacon$.
  
  \item Suppose an honest observer finds that a valid nonce message he had previously received, 
  does not appear in the ledger implemented by $\Blockchain$. 
  Then he re-announces this message with a proof of this re-announcement.

  \item Eventually, by the persistence guarantee of $\Blockchain$, 
  this message gets recorded in the ledger.
\end{itemize}
\noindent




\newcommand{\PublicElectionDistribution}{\hat{\mathcal{L}}}
\newcommand{\PrivateElectionDistribution}{\mathcal{L}}






\section{The new beacon}

Next, we formally describe our beacon protocol $\Pi_\Beacon$. 
Since the leader election process 
is decoupled from the beacon output calculation, 
it can leverage the lottery schemes from Section~\ref{sec:leader-election-public-private}
and deal with both an adaptive adversary 
and a static adversary. 
The first setting is relevant to Praos and SnowWhite (private leader election) ~\cite{Praos,SnowWhite} 
and the second setting 
is relevant to Ouroboros classic (public leader election) ~\cite{Ouroboros}. 




\begin{definition}[$(T, k, s, d, \LeaderElection,\EpsP)$-Beacon]\label{def:new-beacon}
  Let $T, k, s,d \in \NN$ and $\EpsP \in (0,1)$ 
  so that $k$ divides $T$.
  Let $\LeaderElection$ be a given leader election mechanism 
  (see Section~\ref{sec:leader-election-public-private}).
  Let $\Blockchain$ be an $(\EpsP, k, s)$-secure blockchain protocol 
  with epoch length at least $T + s + k$. 
  Consider the following simple beacon protocol on $\{0,1\}^\kappa$ 
  that takes place inside every epoch of $\Blockchain$. 
  $\Pi_\Beacon$ operates inside an epoch of $\Blockchain$ 
  and it computes a function 
  $\Pi_\Beacon : \{0,1\}^\kappa \rightarrow \{0,1\}^\kappa$. 
  Given an initial value $x$ before an epoch, 
  it outputs $\Pi_\Beacon(x)$ 
  after the epoch, as follows:
  \begin{itemize}
    \item The first $T$ slots of the epoch participate in $\Pi_\Beacon$. 
    Of these, only one in every $d$ slots is a \emph{nonce-emitting slot}. 

    \item In a nonce-emitting slot, one or more players are elected as a \emph{nonce leader} 
    according to the leader election mechanism $\LeaderElection$.
    A nonce leader announces a uniformly random $\kappa$-bit Boolean string, 
    called a \emph{nonce},  
    in the nonce channel (which has parameter $k$). 

    \item 
    Let $S$ be the set of all nonce-emitting slots in epoch $t$.
    After th epoch is finished, the output $\eta$ is calculated as follows:
    Let $C_i$ be the set of valid nonces recorded from slot $i$ in this epoch. 
    Let $y_i$ be the lexicographically smallest item in $C_i$.
    Finally, we set $\Pi_\Beacon(x) = x \oplus_{i \in S} y_s$.
  \end{itemize}
  Let $\eta_0$ be a uniformly random string in $\{0,1\}^\kappa$. 
  When the beacon is executed for $L$ epochs, 
  the output is a sequential composition of 
  $L$ independent instances of $\Pi_\Beacon$; 
  the output of after epoch $t$ is the initial value used in epoch $(t+1)$. 
  Specifically, 
  the beacon output for epoch $t \in [L]$ is 
  \begin{align}%\label{eq:beacon-iterated}
    \eta_t = \Pi_\Beacon(\eta_{t-1}) = \Pi_\Beacon^t(\eta_0)\,.
  \end{align}
\end{definition}











\section{Outline of the remaining exposition}
In \Section~\ref{sec:xor-games}, we develop an abstraction of this beacon 
via a single-player multi-round game, called the XOR target game, 
and derive moment bounds of its grinding power. 
We can plug in the distributions induced by different leader election mechanisms 
and develop fine-grained 
tail bounds for the grinding power 
in each scenario. 
Guarantees about the iterated beacon 
(Theorems~\ref{thm:beacon-poisson-multi-epoch} 
and~\ref{thm:beacon-bernoulli-multi-epoch}) 
appear in \Section~\ref{sec:main-thm-proofs}.




