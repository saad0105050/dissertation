


Let $k,s,d,T, R, \kappa \in \NN$. 
Assume, for simplicity, that $d$ divides both $k$ and $T$. 
Let $\EpsP \in (0,1)$ and 
let $\Blockchain$ be an $(\EpsP, k, s)$-secure eventual consensus blockchain protocol, 
such as Praos~\cite{Praos} or Snow White~\cite{SnowWhite}, 
where an epoch comprises of $R$ slots, $R \geq T + k$. 
% Let $\Chain$ denote the abstract ledger implemented by the protocol $\Blockchain$. 
Of these, if a slot $t \in [R]$ is congruent to  $1 \bmod d$ 
then slot it is called a \emph{nonce-emitting slot.} 
% At any given moment, let $\Chain_u$ 
% denote the state of the ledger in the view of a nonce player $u \in \Players$. 

Let $\alpha \in (0, 1/2)$. 
Let $\Players = \Players(\alpha)$ be the set of participants 
in the protocol $\Blockchain$ so that: 
\begin{itemize}
	\item With each player $u \in \Players$ 
	is associated a non-negative real number $\sigma_u \in (0, 1)$ 
	satisfying $\sum_u \sigma_u = 1$. 

	\item There is a designated subset $A \subset \Players$ of players 
	so that 
	$\sum_{u \in A} \sigma_u = \alpha$. 
	If $u \in A$ then $u$ is called an \emph{adversarial player}; 
	otherwise, $u$ is called an \emph{honest player}.
\end{itemize}
The players have access to 
% a VRF and 
a common random string $\eta_0 \in \{0,1\}^\kappa$.

At every slot $t \in [R]$, 
every player $u \in \Players$ 
privately elects himself as the slot leader for slot $t$ 
with probability $\sigma_u$. 
If $u$ is elected a leader, 
he is entitled to issue a new block 
along with the proof of this election. 

Below, we describe a \emph{coin-flipping protocol with dimension $\kappa$} $\CoinTossingPrivate$. 


\paragraph{Nonce announcement and recording in $\CoinTossingPrivate$.}~ 
\begin{itemize}
	\item As soon as a player $u$ becomes a leader of some slot $t$, 
	he immediately \emph{announces} his \emph{nonce message} $\mu_t$. 
	That is, he broadcasts $\mu_t$ to all honest players 
	and includes $\mu_t$ in his newly minted block. 
	This nonce message $\mu_t$ is a tuple $(y_t, t, u, \pi)$ 
	which means ``player $u$ is a legitimate leader for slot $t$ and 
	he is contributing a legitimate nonce $y_t \in \{0,1\}^\kappa$'' and, furthermore, 
	$\mu_t$ is \emph{valid} if anyone can verify this statement using $\pi$. 

	\item Suppose a player $v$ receives the nonce message $\mu_t$ at some slot $r$. 
	If $r< t + k$, he stores $\mu_t$ permanently in a set $Q_v$ in his local (off-chain) memory.

	\item Suppose a slot leader $w$ extends a chain (say, $\Chain$). 
	If $Q_w$ contains a nonce message $\mu$ which 
	does not appear in $\Chain$, 
	$w$ includes $\mu$ in the new block he creates.
\end{itemize}
Note that an adversarial slot leader 
may choose to delay the announcement of his nonce message, 
or not announce it at all.
However, an announcement can be delayed by at most $k - 1$ slots since 
otherwise, it will never be recorded in the blockchain.


\paragraph{Output of $\CoinTossingPrivate$.}
% (Nonce curation.)
Let $\Chain_u$ be a blockchain 
adopted by an honest player 
at the end of slot {\color{red}$\tau \geq T + s + 1$}. 
Let $t \in [T], t \equiv 1 (\bmod d)$. 
Let $Y_t$ be the set of nonces $y_t$
contained in valid nonce messages 
$\mu_t = (y_t, t, \cdot, \cdot)$ 
in $\Chain$. 
Let $m_t$ be the lexicographically smallest element of $Y_t$. 
The \emph{output} of the coin-flipping protocol is a $\kappa$-bit Boolean string 
$$
	% \eta = \eta_0 \oplus \bigoplus_{\substack{t \in [R] \\ t \equiv 1 (\bmod d)}} m_t
	\eta = \eta_0 \oplus m_1 \oplus m_{1 + d} \oplus m_{1 + 2d} \oplus \cdots \oplus m_{1 + \lfloor T/d \rfloor \cdot d}
$$ 


% In the above discussion, 
% $k$ is the length of the \emph{confirmation window} of the protocol $\Blockchain$. 
% Suppose a slot leader at slot $t$ 
% possesses a yet-unannounced nonce $y_t$. 
% If he does not announce it before slot $t + k$, 
% the views of all honest observers at any slot $r \geq t + k$ 
% would agree that $y_t$ was never announced. 
% Thus announcing $y_t$ later than $t + k$ 
% is equivalent to not announcing at all.

% The nonce messages announced by the nonce players 
% are put in a publicly-visible queue $Q$; 
% messages in $Q$ gets included in $\Chain$ via protocol $\Blockchain$. 
% Importantly, $Q$ has a \emph{$k$-recency constraint}: 
% at any slot $r$, only messages originating from slots $t \geq r - k$ 
% can be inserted into $Q$. 
% The recency constraint mimics the 
% \emph{confirmation window} of the protocol $\Blockchain$. 
% In particular, 
% suppose a nonce leader at slot $t$ 
% possesses a yet-unannounced message $y_t$ tagged with slot $t$. 
% If he does not announce it by slot $t + k$, 
% the views of all honest observers at any slot $r \geq t + k + 1$ 
% would agree that $y_t$ was \emph{never announced}. 
% Thus announcing $y_t$ later than $t + k$ 
% is equivalent to not announcing at all.



%========================================


% \subsection{When a slot has zero or more nonce leaders}\label{sec:coin-tossing-multi-leader}


% Recall the coin-tossing scheme from Figure~\ref{alg:coin-tossing-single-leader} 
% and modify it as follows: 
% \InlineCases{
% 	\item A nonce-emitting slot may have zero or more nonce leaders.
% 	\item Since there can be multiple nonces emitted from a slot, 	
% 	only the nonce with the minimum value is considered in the output. 
% }

%============ Praos-style election =============
	% Specifically, let $f \in (0,1)$ and define, for $\sigma \in [0,1]$, 
	% \begin{equation}\label{eq:phi-testing}
	% 	\phi_f(\sigma) = 1 - (1 - f)^\sigma
	% 	\,.
	% \end{equation}
	% The function $\phi_f$ has the ``independent aggregation proeprty.'' 
	% Specifically, for any $\alpha \in (0,1]$ and 
	% any collection of positive reals $\sigma_1, \ldots, \sigma_m$ 
	% so that $\sum_{i=1}^m \sigma_i = \alpha$, 
	% it holds that 
	% \begin{equation}\label{eq:phi-indep-agg}
	% 	1 - \phi_f\left(\sum_i \sigma_i \right) 
	% 	= (1-f)^{\sum_i \sigma_i} 
	% 	= \prod_i (1-f)^{\sigma_i} 
	% 	= \prod_i (1 - \phi_f(\sigma_i))
	% 	\,.	
	% \end{equation}
	% Moreover, we can check that $\phi_f$ is concave and subadditive:
	% \begin{equation}\label{eq:phi-subadditive}
	% 	f \alpha \leq \phi_f(\alpha) \leq \sum_i \phi_f(\sigma_i) 
	% 	\,.
	% \end{equation}
	% Importantly, the right-hand side of~\eqref{eq:phi-subadditive} 
	% has an upper bound 
	% which is independent of $m$:
	% \begin{equation}\label{eq:phi-upperbound}
	% 	\sum_i \phi_f(\sigma_i) \leq -\alpha\ln(1-f)
	% 	\,.
	% \end{equation}
	% {\color{red}We defer the proof of this last inequality to the appendix.}

% The modified scheme is presented in Figure~\ref{alg:coin-tossing-multi-leader} 
% as protocol $\CoinTossingPrivate$. 



% \begin{figure}
% 	\begin{framed}
% 		\begin{center}
% 				\textbf{Coin-flipping protocol $\CoinTossingPrivate$
% 					% with multiple nonce leaders per slot
% 					}
% 		\end{center}
% 		\begin{enumerate}
% 			\item 
% 			(Nonce leaders.) 
% 			Independently for each nonce-emitting slot $t$, 
% 			every player $u \in \Players$ independently 
% 			attempts to elect himself a \emph{nonce leader at slot $t$} 
% 			with probability $\sigma_u$.
% 			% $\phi_f(\sigma_u)$. The function $\phi_f$ is defined in~\eqref{eq:phi-testing}. 
			
% 			\item 
% 			(Nonce announcement.)
% 			At a nonce-emitting slot $t$, the designated nonce leader 
% 			broadcasts a random string $y_t \in \{0,1\}^\kappa$ 
% 			to all nonce players and inserts $y_t$ in $Q$. 
			
% 			\item 
% 			(Nonce relay.) 
% 			At slot $r$, suppose there is a nonce player $u$ who 
% 			has previously heard about a nonce $y$, 
% 			emitted from some slot $t \geq r - k$, 
% 			which appears neither in $Q$ nor in his view of $\Chain$. 
% 			Then $u$ re-inserts $y$ in $Q$ with timestamp $r$.
			
% 			\item 
% 			% (Nonce curation.)
% 			At any slot {\color{red}$\tau \geq T +k + s + 1$,} 
% 			every user $u \in \Players$ examines his 
% 			adopted blockchain $\Chain_u$. 
% 			Let $Y_t$ be the set of (valid) nonces emitted from slot $t$ 
% 			and recorded in $\Chain_u$ 
% 			and let $m_t$ be the lexicographically smallest element of $Y_t$. 
% 			Then $u$ computes $\eta = \eta_0 \oplus \bigoplus_t m_t$ 
% 			where the exclusive-OR runs over 
% 			all nonce-emitting slots $t$. 
% 			The string $\eta$ is called the \emph{output} nonce.
% 		\end{enumerate}
% 	\end{framed}
% 	\caption{Coin tossing with zero or more nonce leaders per slot}
% 	\label{alg:coin-tossing-multi-leader}
% \end{figure}



% Let $\ell, n, m \in \NN, \alpha \in (0, 1/2)$ and 
% $\sigma \in \RR^m$ so that $\sigma_i \in (0, 1)$ and $\sum_{i=1}^m \sigma_i = \alpha$. 
% Let $X_1, \ldots, X_m \in \{0,1\}$ be independent Bernoulli random variables with 
% $\Pr[X_i = 1] = \sigma_i$. 
% Define $S = \sum_{i = 1}^m X_i$ 
% and let $\Distribution$ be the distribution of $S$.

\begin{claim}\label{claim:correspondence-poissongame}
	The output distribution of the coin-flipping protocol $\Pi = \CoinTossingPrivate$ 
	is 
	stochastically dominated by the 
	output distribution of 
	an $(\ell, n, \alpha)$-Poisson game $U$ 
	from Definition~\ref{def:xor-game-poisson}.
\end{claim}
We postpone a proof of the claim for the moment.

\begin{proposition}\label{prop:coin-tossing-security-private}
	Let $\alpha \in (0, 1/2), \EpsP \in (0, 1)$ and $d, k, T \in \NN$ 
	so that $d$ divides both $k$ and $T$. 
	Let $n = (T - k)/d$ and $\ell = k/d$ 
	and, furthermore, assume that $n \geq \ln(1/\alpha)/\ln(1/\EpsP)$.
	Let $\Blockchain$ be $(\EpsP, k)$-secure. 
	Let $\eta \in \{0,1\}^\kappa$ be the output of 
	the coin-flipping protocol $\Pi = \CoinTossingPrivate$. 
	% Let $\gamma$ be defined in~\eqref{eq:xor-geom-tail-gamma}.
	Then 
	% $\Pi$ is 
	% $(\alpha^n + 2^{-\tau + 1}, \rho)$-secure 
	% where
	\begin{equation}
		\Pr\left[\MinEntropy(\eta) \geq \kappa - \log_2(\gamma) \right] \geq 1 - 3 \EpsP
	\end{equation}
	where $\gamma$ is defined in~\eqref{eq:xor-poisson-tail-gamma}.


  % \begin{equation}
  %     \label{eq:minentropy-loss-poisson}
  %     \rho = (\ell/2) \log_2 (1 + 3 \alpha + \alpha^2) + \begin{dcases}
  %     	\tau/2 + (3/2) \log_2 n 	
  %         	&\quad\text{if}\quad\alpha \in (0, 0.41]\,, \\
  %       0.73 \tau + 1.73 \log_2 n 	
  %         	&\quad\text{if}\quad\alpha \in (0.41, 0.5)\,.
  %     \end{dcases}
  % \end{equation}
\end{proposition}


\begin{proof}
	% Let $\EpsP = 2^{-\tau}$. 
	It follows from Claim~\ref{claim:correspondence-poissongame} that the 
	min-entropy of the output distribution of 
	protocol $\Pi$ 
	is the same as 
	the min-entropy of the output distribution of $U$. 
	However, 
	Lemma~\ref{lemma:xor-game-poisson-gamma} and 
	Lemma~\ref{lemma:xor-game-minentropy} together 
	states that 
	except with probability $\alpha^n + \EpsP$, 
	the latter quantity is at least $\kappa - \log_2(\gamma)$. 
	Note that our assumption on $n$ implies that $\alpha^n \leq \EpsP$.
	% where $\gamma$ is defined in~\eqref{eq:xor-poisson-tail-gamma}. 
	Since the game $U$ is played on top of 
	the ledger protocol $\Blockchain$, 
	any guarantee provided by $U$ is 
	further conditioned on the (external) event that 
	$\Blockchain$ remains $(\EpsP, k)$-secure through the execution; 
	thus the total failure probability is $3 \EpsP$.
	% Using Definition~\ref{def:coin-flipping-security}, 
	% It follows that the protocol $\Pi$ is $(3 \EpsP, \rho)$-secure 
	% where $\rho = \log_2(\gamma)$; 
	% this $\rho$ is given in~\eqref{eq:minentropy-loss-poisson}.
	It remains to prove Claim~\ref{claim:correspondence-poissongame}.


	\paragraph{Proof of Claim~\ref{claim:correspondence-poissongame}.}
		% {\color{red} As in the case of the protocol $\CoinTossingPublic$}, 
		Note that all nonces emitted by honest nonce leaders appear in the output calculation. 
		Specifically, for all blockchains $\Chain$ 
		held by an honest player at slot $\tau \geq T +  s + 1$, 
		the segment $\Chain[T:T+s]$ must contain an honest block 
		which must contain all honest (i.e., immediately announced) nonces. 

		First, note the {\color{red}obvious} correspondence 
		between the lookahead region and the confirmation window.


		Every nonce-emitting slot is associated with an option set. 
		Specifically, fix a nonce-emitting slot $t$. 
		The option set associated with $t$ 
		contains all possible values for $m_t$; let $P$ be this option set. 
		Note that the sizes of the option sets (for different slots) 
		are identically distributed.
		Let $\Distribution$ be the distribution of $|P|$. 
		% Adopt the convention that the lexicographically smallest honest nonce 
		% for a nonce-emitting slot 
		% is $\infty$ if that slot does not have an honest leader. 
		% It follows that 
		
		Note that 
		$|P| = 1 + D$ where $D$ is the number of adversarial nonces that are 
		lexicographically smaller than the smallest honest nonce. 
		The additive term $1$ corresponds to the case that 
		no adversarial leaders in this slot announces a nonce message. 
		% Note that in Definition~\ref{def:xor-game-explicit}, 
		If an honest leader is associated with that slot 
		then it means $\star \in P$; this $\star$ later becomes a random nonce. 
		Otherwise, this slot will be associated with no nonces; 
		this means $0^\kappa \in P$.

		Let $S$ is the total number of adversarial leaders at this slot.
		Then the obvious upper bound $|P| \leq 1 + S$ holds with probability $1$; 
		hence $|P| \DominatedBy 1 + S$. 
		Observe that $S = \sum_{u \in A} X_u$ where $X_u \in \{0,1\}$ and $\Pr[X_u = 1] = \sigma_u$. 
		This $S$ is the same as the $S$ appearing in the definition  
		of an $(\ell, n, \alpha)$-Poisson game; see Definition~\ref{def:xor-game-poisson}.
		It follows that the 
		output distribution of the protocol $\Pi$
		is 
		stochastically dominated by the 
		output distribution of 
		an $(\ell, n, \alpha)$-Poisson game of dimension $\kappa$.
\end{proof}












