

In this \Section, 
we establish the vocabulary for properly discussing the present research.


\Paragraph{Eventual consensus.}
In the decentralized consensus protocol, 
a group of participants exchange messages with each other 
to agree about a ``common view'' 
(e.g., about a linear ordering of events)
without needing to trust anyone. 
Recall the concept of \emph{eventual consistency} in database systems: 
in such a database system, 
the views of the nodes may become inconsistent after a new update is made to one node, 
but the views become consistent once the change is propagated to all nodes. 
A similar concept applies to some decentralized consensus applications as well: 
at round $t$, 
the nodes can only agree about the ordering of the events from rounds $1, \ldots, t-k$ 
but may disagree about the events from rounds $t-k+1, \ldots, t$. 
This consensus paradigm is called \emph{eventual consensus}.
This can be contrasted with distributed systems that offer ``strong consistency'' 
or ``instant confirmation:'' 
here, the protocol progresses to the next round 
only after all nodes have agreed about their view of the events.

Example of blockchains in the eventual consensus setting 
is Bitcoin~\cite{Nakamoto2008}, Ouroboros~\cite{Ouroboros}, Praos~\cite{Praos}, and Snow White~\cite{SnowWhite}. 
In contrast, Algorand~\cite{Algorand} operates on a 
Byzantine Agreement-based protocol where, 
once the protocol progresses past a round, 
everyone agrees about the events up to that round.

The focus of this dissertation is on the eventual consensus model.
One way to achieve this is via the longest-chain rule.


\Paragraph{The longest-chain rule.}
We study the behavior of the elementary \emph{longest-chain rule}
algorithm, carried out by a collection of participants. 
\begin{itemize}
  \item In each round,
  each participant collects all valid blockchains from the network; if a
  participant is a leader in the round, he adds a block to the longest
  chain and broadcasts the result.
\end{itemize}
See below for how leaders are elected.
Here, ``valid'' indicates that any block appearing in the
chain was indeed issued by a leader from the associated slot; in the
PoS setting, this property is guaranteed with digital
signatures.
Any active participant who is not a leader at a given slot, 
is called an ``observer'' at that slot.

It is a remarkable fact that this simple chain-selection rule 
can lead to eventual consensus in both PoW and PoS blockchains. 
The rationale behind choosing the longest chain is that 
it contains more ``effort by the participants'' than any other chains. 
Since the longest-chain rule was pioneered by \citet{Nakamoto2008} in the PoW setting, 
any PoS protocol implementing this rule is called the \emph{Nakamoto-style PoS}. 
Note that there are other chain selection rules: in GHOST~\cite{Ghost},
the ``heaviest subtree'' is prioritized instead of the longest; 
in \cite{Tangle}, a new block builds upon multiple chains instead of a single longest chain.

The focus of this dissertation is on Nakamoto-style PoS. 
These protocols advance in discrete rounds, called ``slots.''



\Paragraph{Slots and epochs.}
The execution of the protocol is divided into discrete time periods called \emph{slots}. 
In practice, slots have to long enough so that 
messages are actually delivered (via, e.g., gossip) 
but they have to be short enough so that 
not too many blocks are issued from the same slot.
In there are multiple blocks in the same slot, 
at most one of these blocks will appear in the settled blockchain 
and, importantly, 
the multiplicity of these blocks can be utilized by the adversary 
in constructing two competing longest blockchains in the future.

A predefined number of slots are grouped into an \emph{epoch}. 
Some information, such as the stake ratio or the computational power, 
is assumed to remain constant in an epoch. 
Moreover, some computations taking place in an epoch 
share some common information, such a common random string. 
These common information are collectively recomputed 
at the onset of every epoch. 
The first epoch is bootstrapped with a special block of information, 
called the ``genesis block'' which contains, e.g., a fresh random string to 
initialize the computations in the first epoch.

Next, let us focus on the participants in a Nakamoto-style PoS protocol.


\Paragraph{Players and relative stakes in Nakamoto-style PoS.}
Let $\Players$ be the set of participants 
where each player $i \in \Players$ is associated with a positive real number 
$\sigma_i \in [0,1]$, called his \emph{relative stake ratio}, or just \emph{stake} in short. 
The stake ratios satisfy $\sum_{i \in \Players} \sigma_i = 1$ 
and, for this analysis, remains constant over the course of the execution of the protocol. 
(As is common in the literature (e.g., in \cite{Ouroboros,Praos,SnowWhite}), 
we can handle a change in stake separately from this analysis and then hook these two together.)

Let $\HonestPlayers, \DishonestPlayers \subseteq \Players$ 
denote the set of honest and dishonest players, respectively;
these sets are disjoint and $\Players = \HonestPlayers \Union \DishonestPlayers$.
Given a real $\alpha \in (0,1/2)$, 
we use the notation $\Players(\alpha)$ 
to denote the fact that the total dishonest stake bounded by $\alpha$: i.e., 
$\sum_{i \in \DishonestPlayers} \sigma_i \leq \alpha$
$\sum_{i \in \HonestPlayers} \sigma_i \geq 1 - \alpha$. 
We do not, however, make any assumptions on the number of participants.

The upper bound $\alpha < 1/2$ can be thought of a variant of the ``adversarial minority'' condition 
found in many places in the literature. 
Such a condition stems from the nature of the security analysis of the protocol, 
in particular, if it uses a biased random walk then the condition $\alpha < 1/2$ denotes the bias of the walk.


A critical part of the model is how long it takes for the players to exchange messages. 



\Paragraph{Synchronous vs. semi-synchronous communication model.}
In the \emph{synchronous communication setting}, 
all messages issued from a slot reach their destinations 
before the onset of the next slot.
In the \emph{$\Delta$-synchronous communication setting}, 
these messages are delivered with at most a $\Delta$ delay. 
While the $\Delta$-synchronous setting is closer to the practical applications 
(e.g., communications over the Internet), 
analyses of distributed systems in the semi-synchronous model 
are more difficult than the synchronous setting. 
While some analyses (e.g., \cite {SnowWhite,Sleepy,PSS}) 
handles the semi-synchronous setting directly,
other analyses (e.g., \cite{Praos}) handle the easier synchronous setting first, 
before lifting the results into the semi-synchronous setting 
by incurring some slack in the security guarantee. 


\Paragraph{The adversary $\Adversary$; Static vs. adaptive adversaries}\label{sec:static-dynamic-adversary}
We model the network behavior in terms of 
a \emph{rushing adversary} $\Adversary$: 
he can reorder the messages, but he must deliver the message before the maximum allowed delay. 
Plus, he cannot modify or drop the messages.

Besides message delivery, 
$\Adversary$ is assumed to make the selection 
on behalf of an honest observer 
when, in the observer's view, there are multiple longest chains. 
This is called \emph{longest-chain tie-breaking} or \emph{LCR tie-breaking}. 
This is the primary way through which $\Adversary$ can hinder eventual consensus among honest players 
in both Nakamoto-style PoS and PoW.


\Paragraph{Static vs. adaptive adversaries.}
One can consider two types of adversaries in terms of 
how fast they can corrupt an honest player. the so-called \emph{static adversary} 
Suppose that an epoch is $R$ slots long. 
and the \emph{dynamic (or adaptive) adversary}. 
A \emph{static adversary} takes at least $R$ slots to corrupt an honest player 
but an \emph{adaptive adversary} can instantly corrupt an honest player. 
Ouroboros~\cite {Ouroboros} is secure against a static adversary 
while Praos~\cite {Praos} and SnowWhite~\cite{SnowWhite} 
are secure against an adaptive adversary.
The adversarial model critically depends on how a slot-leader is elected. 



\Paragraph{Verifiable Random Functions (VRF).}
VRFs are an important ingredient in many distributed algorithms. 
Informally speaking, they act as a hash function $F$ which, when evaluated at a certain input $x$, 
\InlineCases{
  \item outputs a string $y = F(x)$ which is uniformly random with high probability, and
  \item supplies a proof $\pi$ certifying that $y = F(x)$.
}
For completeness, we include a definition of VRFs in
Definition~\ref{def:VRF}.




\Paragraph{Leader election outcomes; the probabilities $p_\h, p_\H,$ and $p_\A$.}\label{sec:leader-election-probs}
Leader election~\cite{RussellZuckerman} is an important problem in distributed computing.
In every slot, 
a number of participants are elected as \emph{leaders} who can add block to a chain. 
The claim ``player $u$ won the leader election for slot $i$ at epoch $e$'' 
can be proved using digital signatures and VRFs (Definition~\ref{def:VRF}).
Based on the outcome of the leader election in a slot, 
we can annotate each slot using three qualitative tags: 
\begin{itemize}
\item \emph{uniquely honest}, having a single honest leader;
\item \emph{multihonest}, having multiple, but honest,
  leaders; and
\item \emph{adversarial}, having at least one adversarial leader.
\end{itemize}
The leader election process is usually designed in a way so that 
the elections across different slots are \emph{independent}, 
meaning the outcomes of one slot does not impact the outcomes of any other slot.\footnote{
  In some leader election schemes (e.g., Ouroboros Genesis~\cite{Genesis}), 
  the model specifically allows 
  the outcome distribution of a slot to depend on the outcomes of previous slots. 
  In other cases, 
  such a dependence on the history can be analyzed 
  by instead analyzing a ``more adversarial'' outcome distribution 
  coming from independent leader elections in different slots. 
}
Thus, for each slot, there are three probabilities of interest:
\begin{itemize}
\item $p_\h$, the probability that a slot is uniquely honest;
\item $p_\H$, the probability that a slot is multihonest; and
\item $p_\A$, the probability that a slot is adversarial.
\end{itemize}

The above probabilities play a critical part 
in the analysis of blockchain protocols. 
Specifically, 
their security guarantee hinges upon conditions such as 
$p_\h < p_\A$~\cite{SnowWhite}, 
$p_\h - p_\H < p_\A$~\cite{Ouroboros,Praos,Genesis}, 
or $p_\h + p_\H < p_\A$~\cite{Nakamoto2008,GKL,PSS}.
These thresholds are analogs of the well known inequality 
$n \geq 3f + 1$ in the Byzantine Fault Tolerance (BFT) protocols~\cite{BFT} 
where $n$ is the total number of players and $f$ is the number of dishonest/faulty players.


\Paragraph{Common Prefix (CP), Chain Quality (CQ), and Chain Growth(CG) properties.}
An eventual consensus blockchain protocol 
must satisfy a set of key properties. Informally speaking, these are as follows:
\begin{description}[font=\normalfont\itshape\space]
  \item[Common Prefix Property (CP) with parameter $k$:]
  Consider two blockchains held by two honest observers and 
  truncate, from these chains, any block from the last $k$ slots. 
  Then one of the chains (after truncation) must be a prefix of the other. 
  This property is called $\kSlotCP$ in short and is defined in Definition~\ref{def:cp-slot-mh}. 

  \item[Chain Quality Property (CQ) with parameter $s$:]
  Consider a blockchain held by an honest observer. 
  Any segment of this chain 
  corresponding to a time window of length $s$ slots or more, 
  must contain at least one honest block.
  This property is called $\sECQ$ 
  and we define it rigorously in Definition~\ref{def:cp-slot-mh}. 

  \item[Chain Growth Property (CG) with parameter $r$:]
  Consider a blockchain held by an honest observer at some slot. 
  Any past time window of length $r$ slots or more, 
  contributes at least one block to this chain.
\end{description}
Please refer to~\cite{GKL,Ouroboros,Praos,Genesis,SnowWhite} for a precise description of these properties.


The most important of these properties is the first one; 
it determines how fast the blockchain achieves eventual consensus. 
As shown in ~\cite{Ouroboros}, 
the other two properties can be derived from (some version of) $\kSlotCP$.
In some cases, such as in~\cite{GKL,Ouroboros,SnowWhite}, 
the CP property is defined in terms of truncating $k$ blocks 
(and the corresponding property is named $\kCP$); 
but these differences are only superficial. 
We prefer the slot truncation-based definition 
as it is amenable to a better modeling and analysis.


The entire Part~\ref{part:multihonest} of this dissertation, 
is devoted to prove strong security bounds for the $\kSlotCP$ property. 
The $\sECQ$ property is used in the analysis in 
Part~\ref{part:praos} and Part~\ref{part:xorgames}.







\Paragraph{Nakamoto-style PoS blockchain axioms in the synchronous setting.}
To model the dynamic of a Nakamoto-style PoS blockchain, 
we adopt a set of axioms; see below for an informal description:

\begin{description}[font=\normalfont\itshape\space]
  \item[Message delivery.] 
    Any message broadcast by an honest participant at the beginning of a
    particular slot is received by the adversary first, who may decide
    strategically and individually for each recipient in the network
    whether to inject additional messages and in which order all messages
    are to be delivered prior to the conclusion of the slot. 

  \item[Genesis block.]
  The blockchain begins with a fixed ``genesis'' block, assigned to slot $0$.
  
  \item[Block monotonicity] 
  Each block is labeled by the index of the slot it was issued from. 
  In a blockchain, the (slot) labels of the blocks are in strictly increasing order.

  \item[Honest blocks.] 
  An honest leader in a slot creates a single block.
\end{description}
These axioms are made formal in Section~\ref{sec:pos-axioms}.






\Paragraph{A (more) precise definition of VRFs.}
Below, we present a definition of VRFs that is good enough for our exposition. 
For a more complete definition, see~\cite{VRF}.

\newcommand{\Gen}{\mathsf{gen}}
\newcommand{\Prove}{\mathsf{prove}}
\newcommand{\Verify}{\mathsf{verify}}
% \newcommand{\pk}{\mathsf{pk}}
% \newcommand{\sk}{\mathsf{sk}}
\begin{definition}[Verifiable Random Function (VRF)]\label{def:VRF}
  A family $\mathcal{F}$ 
  of functions $F : \{0,1\}^\ell \rightarrow \{0,\}^k$ 
  is a family of VRFs if there exist algorithms 
  $(\Gen, \Prove, \Verify)$ 
  so that the following holds: 
  $\Gen(1^k)$ outputs a pair of keys $(\pk, \sk)$; 
  $\Prove_\sk(x)$ outputs a pair $(F_\sk(x), \pi_\sk(x))$ 
  where 
  $F_\sk \in \mathcal{F}$, $F_\sk(x)$ is the function value, and 
  $\pi_\sk(x)$ is the proof of correctness; and 
  $\Verify_\pk(x,y,\pi_\sk(x))$ efficiently verifies 
  that $y = F_\sk(x)$ using the proof $\pi_\sk(x)$, 
  outputting 1 if $y$ is valid and 0 otherwise. 
  Additionally, we require the following properties:

  \begin{enumerate}
    \item \emph{Uniqueness:} 
    No values $(\pk, x, y, y', \pi, \pi')$  can satisfy both 
    $\Verify_\pk(x,y,\pi) = 1$ and $\Verify_\pk(x,y',\pi') = 1$ 
    unless $y = y'$.

    \item \emph{Provability:} 
    If $(y, \pi) = \Prove_\sk(x)$ then $\Verify_\pk(x,y, \pi) = 1$.

    \item \emph{Pseudorandomness:} 
    To all probabilistic polynomial-time (PPT) algorithm 
    which runs $\Poly(k)$ steps when its first input is $1^k$ 
    and does not query the $\Prove$ oracle on $x$, 
    the distribution of $F_\sk(x)$ appears uniform in $\{0,1\}^k$, 
    except with a probability negligible in $k$.
  \end{enumerate}

\end{definition}




\section{Lottery-based leader election schemes}\label{sec:leader-election-public-private}
Let $\alpha \in (0,1)$ and consider the players $\Players = \Players(\alpha)$
Let $C_i$ be the set of leaders for slot $i$. 
We consider two types of leader election schemes for an epoch: 
% \begin{enumerate}[label=\textbf{Scheme \Alph*:},ref=\Alph*,leftmargin=3em]
\begin{description}
    \item[\textbf{$\PublicLeaderElection(\alpha)$: Publicly elect a single leader per slot.}] \label{lottery:public}
    At the onset of an epoch, 
    all players use a common random string to 
    publicly sample, 
    independently for each round $i \in [\ell + n]$, 
    a single player $\ell_i \in \Players$ and  
    set $C_i = \{\ell_i\}$. 
    For all players $u \in \Players$, 
    the probability that $u = \ell_i$ is $\sigma_u$. 
    Thus, the leader schedule for an epoch 
    is public knowledge before an epoch commences. 

    \item[\textbf{$\PrivateLeaderElection(\alpha)$: Privately elect zero or more leaders per slot.}] \label{lottery:private}
    Independently for each round $i \in [\ell + n]$, 
    player $u \in \Players$ 
    privately and independently 
    evaluates a Boolean random variable $D$
    which has expectation $\sigma_u$; 
    if $D = 1$, $u$ 
    inserts himself into the set $C_i$ 
    and announces this fact during slot $i$. 
\end{description}
\noindent
Ouroboros uses a public leader election 
at the outset of an epoch 
and, therefore, it is secure only against a static adversary. 
SnowWhtie and Praos (and Bitcoin as well) uses a private leader election 
and therefore, must prove security against an adaptive adversary. 

