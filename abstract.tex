% \blindtext

In the past decade, blockchain technology has emerged as a foundation for a diverse range of distributed applications, notably cryptocurrencies and decentralized ledgers. 
These protocols allow the users---across the globe and not 
needing to trust one another---exchange messages and eventually reach an agreement. 

 

Currently, the most popular blockchain technology is the Proof-of-Work (PoW) paradigm, popularized by the cryptocurrency Bitcoin. Alternative paradigms, such as Proof-of-Stake (PoS), aim to provide the same functionalities and security as PoW while requiring only a fraction of a PoW network's enormous energy cost. This, however, is non-trivial.

 

In this dissertation, we study the security of PoS blockchains in the eventual consensus model under the so-called longest-chain rule; 
these are also called the ``Nakamoto-style PoS.''
\begin{itemize}
  \item We show that the security guarantees 
  for the consistency property of these blockchains are asymptotically similar 
  to those of the Nakamoto-style PoW blockchains (ACM SODA 2020). 

  \item We present a security analysis under a simple honest-majority assumption; this is a sweeping improvement over all existing analyses as they needed a stricter honest-majority assumption (IEEE ICDCS 2020).

  \item We analyze the security of the simple `hash the chain'' randomness beacon 
  in PoS; two important protocols, Ouroboros Praos and SnowWhite, use this beacon. 
  Our results demarcate the parameter settings for which these protocols remain secure. 

  \item We also present and analyze new, more secure randomness beacons for eventual consensus.
\end{itemize}

Our results have already made an impact in the industry: \emph{Cardano}---a leading PoS cryptocurrency--is using our results and recommendations for designing their protocol.

% \begin{flushleft}
% \textbf{Thesis Supervisor:} \majoradvisor \\
% \textbf{Title:} Professor, Department of Computer Science \& Engineering, University of Connecticut  
% \end{flushleft}
