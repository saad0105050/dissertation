% \blindtext

In the past decade, blockchain technology has emerged as a foundation for a diverse range of distributed applications, notably cryptocurrencies and decentralized ledger. These protocols allow the users -- across the globe and not needing to trust one another -- exchange messages and eventually reach an agreement. 

 

Currently, the most popular blockchain technology is the Proof-of-Work (PoW) paradigm, popularized by the cryptocurrency \emph{Bitcoin}. Alternative paradigms, such as Proof-of-Stake (PoS), aim to provide the same functionalities and security as PoW while requiring only a fraction of a PoW network's enormous energy cost. This, however, is non-trivial.

 

In this dissertation, we study the security of PoS blockchains in the eventual consensus model under the so-called ``longest-chain rule.'' 
\begin{itemize}
  \item First, we show that these blockchains have the same asymptotic security guarantee as those of the PoW blockchains (published at ACM SODA 2020). 

  \item Next, we present a security analysis under the simple honest-majority assumption; this is a sweeping improvement over all existing analyses -- including the one above -- as they need a stricter majority assumption (accepted at IEEE ICDCS 2020).

  \item Finally, we analyze the security of the ``randomness beacon'' within two important PoS blockchain protocols: Ouroboros Praos and Snow White. Our results demarcate the parameter settings for which these protocols remain secure. We also analyze some new, simple, and more secure randomness beacons for PoS blockchains.


\end{itemize}

Our results have already made an impact in the industry: \emph{Cardano} -- a leading PoS cryptocurrency -- is using our results and recommendations for designing their protocol.

% \begin{flushleft}
% \textbf{Thesis Supervisor:} \majoradvisor \\
% \textbf{Title:} Professor, Department of Computer Science \& Engineering, University of Connecticut  
% \end{flushleft}
