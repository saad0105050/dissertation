




Consensus means agreement.

In distributed computing, 
consensus algorithms provide strategies 
so that different people (or computers) 
may come to agreement. 
These algorithms are often called \emph{protocols} as they 
specify for the participants 
the actions they should take, 
e.g., when to send a certain message to others, etc. 
Sometimes, the participants cannot trust each other; 
this is the \emph{decentralized consensus problem}. 
The consensus has to be reached against threats and issues such as 
network delays, network failure, crashing computers, and even 
dishonest (or malicious) participants. 
Often, we use the term \emph{adversary} to broadly refer to these threats. 
A \emph{Byzantine adversary} can deviate arbitrarily from the protocol. 


The decentralized consensus problem in the Byzantine threat model 
is called the \emph{Byzantine Fault Tolerance} (or BFT in short). 
This problem has been studied extensively over the past few decades, 
starting from the ground-breaking paper~\citet{BFT}. 
See~\citet{GK} and ~\citet{ConsensusBlockchains} for two recent surveys.
The consensus problem can have a rich structure and varying difficulty 
depending on the model e.g. network delays or Byzantine adversaries. 
(In this exposition, 
we use the word ``slot'' interchangeably with the word ``round.'')

Like many protocols, consensus protocols 
progress in logical time segments called \emph{rounds} 
and, at each round, the participants execute relevant instructions of the protocol. 
The protocol guarantees that after a number of rounds, 
a consensus would be reached. 

Consensus can mean different things. 
Typically, we expect \emph{instant settlement}, 
i.e., that once a consensus is reached, 
it is final. 
However, if we allow the consensus to happen gradually, 
things suddenly become more interesting.
In \emph{eventual consensus}, 
we ask for an \emph{iterated} consensus protocol 
where the consensus solidifies as the rounds passes. 
Suppose that we want to reach consensus about a series of events 
(called a \emph{ledger}). 
The participants may disagree about what transpired in recent rounds, 
but the protocol guarantees that they would all agree about 
what happened, say, one hundred rounds ago. 
This is called the \emph{consistency property} of the protocol. 
Note that in this setting, 
the current round cannot be considered ``settled'' 
before a number of round passes by; 
this ``wait time, '' or delay, is called the \emph{settlement delay}. 
It is essential that the protocol gives us a guarantee 
about the settlement delay;
otherwise, we wouldn't know if (or when) we have reached consensus.

But why should we even consider the eventual consensus model in the first place? 
The answer is simplicity: 
While consensus algorithms in the instant settlement model 
are typically complex, 
they can be surprisingly simple in the eventual consensus model. 
The famous \emph{longest-chain rule}, put forth in the 
celebrated work of~\citet{Nakamoto2008}, states that:
\begin{enumerate}
  \item Each participant maintains a list of events; 
  this list is called a \emph{chain}.

  \item At the beginning of every round, 
  each participant adopts a longest chain in his view.

  \item A participant can test whether he is eligible to add a new record to his chain. 
  If he is, he adds this record (called a \emph{block}) on top of his chain 
  and shows the new chain to everyone.
\end{enumerate}

An eventual consensus protocol that uses the longest-chain rule is dubbed a \emph{Nakamoto-style} protocol.
It is a remarkable fact that this simple rule leads to the eventual consensus. 
The notion of chains and blocks above has led to the term ``blockchain'' 
which has come to define a class of consensus algorithms 
in both eventual consensus model as well as in the traditional 
instant confirmation model~\cite{ConsensusBlockchain,GK18}. 
Note that there are other chain selection rules for eventual consensus: 
for example, in GHOST~\cite{Ghost},
the ``heaviest subtree'' is prioritized instead of the longest chain. 
% in \cite{Tangle}, a new block builds upon multiple chains instead of a single longest chain.
% We also use the term ``blockchain'' to implicitly refer to the 
% ledger maintained by the participants.

A participant who is eligible to add a new record is called a \emph{leader}. 
The word ``leader'' is a reference to the \emph{leader election problem}, 
a consensus problem where the participants want to 
reach agreement about who, among them, would get 
the special designation of being a leader.
Note how the longest-chain rule allows us to plug in 
different leader election mechanisms. 
This flexibility gives a rich structure to the algorithm.

The leader election process is designed to be random. 
If most of the participants are honest, 
a randomly picked participant is more likely to be honest as well. 
In the long term, the accumulated effect of these good events 
should lead to good outcomes, i.e., reaching consensus. 
However, having this randomness built into the system means 
that the consensus is a random event as well; 
therefore, the consistency is achieved only with a high probability. 
Proving that an eventual consensus protocol is secure, 
therefore, is to prove a 
\emph{consistency error bound} 
% associated with a settlement delay parameter
on the probability (of the unlikely event) that a consensus could not be reached 
under the stipulated settlement delay. 
We desire that the consistency error should decrease exponentially 
in the settlement delay parameter. 
If it indeed happens, 
we say that the protocol has achieved the \emph{linear consistency}.

Bitcoin---the poster boy of the longest-chain rule---achieves 
linear consistency; it does so by employing a clever leader election mechanism 
called the \emph{Proof-of-Work lottery} (PoW). 
The proof part is easy: we know how to prove claims using 
digital signatures and zero-knowledge proofs~\cite{GoldreichFoC}. 
The ``work'' part addresses the rarity of leadership. 
Specifically, 
the probability of being elected a leader is kept small by design 
so that we do not have a deluge of competing chains. 
In PoW, participants have to solve a computationally difficult 
cryptographic puzzle~\cite{Nakamoto2008,GKL,GKL17,Ghost} 
to be a leader. 
The difficulty of the puzzle ensures that at any given round, 
it is unlikely to have many lottery winners. 
However, since being elected a leader has its rewards (e.g., commissions and fees), 
rational participants invest in computational power 
and keep solving their puzzles. 
As result, a PoW network consumes a lot of energy: 
Bitcoin's energy consumption is 
on par with a medium-sized country~\cite{BitcoinEnergy}.

One way to avoid such a waste of energy 
is to rely on a proof mechanism which provides 
a useful service, such as Proof-of-Storage~\cite{ProofOfSpace,Spacemint}. 
Yet another alternative is to bypass the necessity of a physical resource 
(such as computational power or storage space)
altogether and instead, perform the lottery based on an abstract resource. 
An abstract resource could be, for example, 
a participant's \emph{stake} in the system, 
a numeric quantity which can be inferred 
from the settled part of the blockchain; 
thus the \emph{Proof-of-Stake} (PoS) paradigm is born.

A difficulty in the PoS analysis 
comes from its adversarial model. 
Under the longest-chain rule (for both PoS and PoW), 
if an honest participant 
is presented with multiple longest chains, 
he has no way to differentiate between these chains. 
In this scenario, the model gives the adversary the power to 
select an arbitrary longest chain; 
this is called the \emph{longest-chain tie-breaking}. 
Since the PoS model puts no inherent cost 
(e.g., spending money to buy computational resource) 
for a leader to produce new blocks, 
adversarial leaders may 
freely produce new blocks. 
Furthermore, they may strategically 
extend existing chains 
so that they can 
present two longest chains to an honest observer. 
Keep in mind that only one of these chains 
is going to belong in the consensus. 
Conceivably, these two chains may diverge 
long ago -- even before the boundary set by the settlement parameter -- and 
contain conflicting records issued from a round $s$; 
this particular consistency violation is called a \emph{settlement violation for round $s$}.  
It follows that in PoS, 
the adversary is given more power to 
cause a settlement violation (or delay the consensus) 
compared to PoW. 


A more subtle difficulty with PoS is that 
we have to maintain the stake information on the ledger itself. 
Since the leader election is based on stakes indicated by the 
consistent part of the blockchain, 
the integrity of the leader election mechanism in the PoS setting 
is tightly intertwined 
with its consistency property. 
We did not have this issue in PoW: 
the model did not require the participants to prove 
how much computational resource they own.
% Thus, proving the security of a PoS system is 
% trickier than that of a PoW system.

There is yet another way for the adversary to boost his attack on the consistency. 
In this attack, called \emph{the grinding attack}, 
the adversary introduces bias in the leader election process. 
The feasibility of this attack depends on the randomness beacon of the protocol. 
Specifically, when the protocol execution starts, 
the first block---a special block called the genesis block---contains 
a public random string which is used in the leader election. 
However, as more and more elections are made, 
this string becomes less and less random. 
A weak random string can introduce bias in the election. 
To mitigate this behavior, 
the protocol leans on the elected leaders to 
submit random inputs (called the \emph{nonce inputs}) in the ledger. 
Periodically, the participants combine these nonce inputs in a specified manner 
and compute a new, public random string. 
This sub-protocol is called the \emph{randomness beacon} protocol 
and these periods are called \emph{epochs}. 
An epoch is long enough so that the settled part of the blockchains contain enough nonce inputs. 
We do not need to worry about the randomness in the nonce inputs: 
these are produced by cryptographically secure Verifiable Random Functions (VRFs)~\cite{VRFMicali,VRFDodis} 
that are tied to the identity of the participants (implemented via a public key infrastructure). 
The method of combining the nonces, however, is critical. 

The eventual consensus allows the adversary 
the full knowledge of the unsettled part of the ledger. 
If the adversary can strategically vary his nonce inputs 
(i.e., whether to submit a nonce input or not)
\emph{after} knowing the potential beacon output, 
he can effectively choose the beacon output from a number of possible candidate values. 
Each of these candidate values will lead to a different sequence of leaders 
(and a different execution) 
in the next epoch. 
If any one of those execution leads to a consistency violation, the adversary wins. 
Therefore, the probability of a consistency violation 
is multiplied by the number of candidate beacon values 
that the adversary can choose from. 
This number is called his \emph{grinding power}.

In some blockchain protocols, 
the beacon output is immune to the dynamic of the eventual consensus. 
This is achieved by opting for a separate, heavy-duty cryptographic multi-party computation (MPC) protocol 
to ensure that the adversary cannot precompute the beacon output.
% such as Publicly Verifiable Secret Sharing (PVSS)~\cite{Ouroboros}, 
% Threshold signatures~\cite{DFINITY}, etc. 
Consider a simple alternative beacon protocol 
where, before the onset of an epoch, 
every participant computes the beacon output by simply 
concatenating all nonces on the settled part of his blockchain and hash the result. 
We call this the \emph{hash-the-blockchain beacon}.
Since these MPC protocols are cumbersome, 
several blockchain protocols instead opt for simple mechanisms such as these. 
As discussed above, this yields the adversary the opportunity to 
launch a grinding attack.



\section{Research focus and objectives}
In this dissertation, 
we focus on the consistency property and the grinding attack in Nakamoto-style PoS blockchains.

As we mentioned before, Nakamoto-style PoW (such as Bitcoin) 
is known to achieve linear consistency. 
What is means is, if we wait $k$ rounds before declaring the current round ``settled,'' 
the risk of a settlement violation drops exponentially in $k$. 
However, existing analyses of the Nakamoto-style PoS has a much weaker guarantee. 
Specifically, they can prove that in the same situation as above, 
the settlement error probability drops exponentially in $\sqrt{k}$. 
Although we know that a PoS adversary has more leeway than a PoW adversary, 
it is not certain whether it should necessarily translate into such 
an order-of-magnitude difference in the consistency bounds. 
Moreover, this weak consistency error bound would lead to large security parameters in PoS protocol. 
Transactions recorded on a Nakamoto-style PoS ledger, therefore, 
would take longer to be considered settled, 
sending a cascaded slow-down on the entire system built on top of these ledgers.
Settling this open question, therefore, is urgent and practical.


The security guarantees of any consensus protocol 
critically rely on the structure 
of the outcomes of its leader election process. 
Depending on whether a dishonest leader has been elected in a round, 
we can speak of ``honest rounds'' and ``dishonest rounds.'' 
In an honest round, the leaders adhere to the protocol whereas 
in a dishonest round, the leaders may arbitrarily deviate from the protocol. 
Furthermore, an honest round can be \emph{uniquely honest} if it has a single leader, 
or \emph{multihonest} if it features multiple leaders. 

Multihonest slots are a natural (and frequent) outcome 
in a broad swath of practical leader election schemes, 
spanning both PoW, PoS, proof-of-space, and other paradigms. 
However, the existing PoS security analyses put a penalty on multihonest slots. 
This seems counterintuitive: multihonest slots feature multiple honest blocks 
and as such, they should be treated as good events for the analysis. 
However, when a round contains multiple blocks---no matter they are adversarial or honest---the adversary can, 
in principle, use these blocks to build two competing longest chains. 

Instead of performing a delicate modeling, 
existing analyses treat multihonest slots as either a neutral event~\cite{SnowWhite,Sleepy} 
or worse, as an adversarial event~\cite{Ouroboros,Praos,Genesis}. 
Such a conservative modeling approach weakens the consistency error bounds 
by discounting the contribution from the honest leaders. 
It also prevents Nakamoto-style PoS protocols from 
utilizing many attractive leader election schemes 
where multihonest rounds are frequent. 
In particular, existing analyses imply 
that we cannot use a leader election scheme 
where a uniquely honest slot is less likely than an adversarial slot.



An important class of Nakamoto-style PoS protocols use the hash-the-blockchain beacon~\cite{SnowWhite,Sleepy,Praos,Genesis}.
Of these, Praos~\cite{Praos} has been deployed in practice 
as the foundation of the cryptocurrency Cardano~\cite{Cardano}. 
But there is a problem: 
although the analysis of Praos recognizes the grinding power, 
its security guarantee is valid as long as it is ``bounded'' 
in terms of the number of adversarial participants and 
the total number of hashing queries they can make. 
This is problematic in several ways.

First, both a PoW-style hashing-power bound 
and a BFT-style number-of-users bound are thrust into 
a PoS security guarantee; 
this goes against the PoS philosophy which 
refused to recognize the computing power as a resource. 
Second, 
the said bound does not depend on the stake of the adversarial players 
and, in particular, 
the literature is yet to characterize the grinding power grows as a function 
of the security parameter and the adversarial stake. 
This is a glaring issue since 
a large grinding power can seriously weaken the consistency error bound and 
send repercussions in every aspect of the protocol design. 


Furthermore, 
current Nakamoto-style PoS analyses focus on the consistency guarantee in isolation. 
But we need an analysis of the grinding power that underscores 
how a seemingly strong consistency bound 
can be weakened via a grinding attack. 




\Paragraph{Objectives.}
The observations above can be congealed into the following research objectives:

\newcounter{ConsistencyObjectives}
\begin{description}[font=\normalfont\itshape\space]
  \item[Consistency Problem:]~
  \begin{enumerate}
    \item Derive tight asymptotic consistency bounds for Nakamoto-style PoS blockchains, 
    harnessing the positive effect of having multiple honest leaders in a round.

    \setcounter{ConsistencyObjectives}{\value{enumi}}
  \end{enumerate}

  \item[Grinding Problem:]~
  \begin{enumerate}
    \setcounter{enumi}{\value{ConsistencyObjectives}}

    \item Quantify the grinding power in Praos and SnowWhite 
    as a function of the protocol's security parameter and the 
    stake held by dishonest players.
    
    \item Use the dynamic of Nakamoto-style eventual consensus to design 
    beacons that does not rely on MPC but still has superior resistance 
    against a grinding attack.     

    \item In both cases, 
    highlight how a grinding attack affects the overall security guarantee.
  \end{enumerate}
\end{description}





\section{Significance}
We contend that the research in this dissertation is important, in the following ways:

  Our research proves that, asymptotically speaking, the consistency error 
  in the Nakamoto-style PoS has the same dependence on the security parameter 
  (i.e., settlement delay) 
  as in the Nakamoto-style PoW. 
  This is an important finding: 
  it proves that although the PoS setting 
  admits stronger adversarial behavior, 
  it is not inherently weaker than the PoW. 
  It also shows that Nakamoto-style PoS blockchains settle much faster than what we used to think. 
  Thus, deployed PoS protocols can safely select stronger security parameters. 

  We also provide algorithms and software tools 
  for practitioners to compute (by simulation) 
  explicit consistency error probabilities. 
  Unlike an asymptotic bound mentioned above, 
  these numbers can directly inform the practitioners and users 
  about selecting protocol parameters. 
  In addition, these numbers are closer to the truth 
  than the asymptotic bounds 
  since the latter bounds are tailored with 
  the worst-case adversarial stake in mind. 
  A \texttt{C++} implementation of this algorithm can be found in 
  \url{https://github.com/saad0105050/multihonest-code}.

  See Figure~\ref{fig:prbad-multihonest} 
  to appreciate these results.

\iftoggle{drawfigs}{%
  \begin{figure}[!htb]
    \centering
    \tikzsetnextfilename{prbad}
    \begin{tikzpicture}[trim axis left,
      ]
      \begin{axis}[domain=0:1, 
        xmin=0,xmax=1,%
        % ymax=0,%,ymax=2.5
        % restrict x to domain=0.6:0.999,
        % restrict y to domain=0:100,
        unbounded coords=jump, 
        samples=500,
        enlarge x limits=false,
        grid=both,
        no markers,
        % yticklabel pos=right,
        % yticklabel style={xshift=2em},
        legend pos=outer north east,
        % legend pos=south west,
        legend cell align=left, 
        x label style={at={(axis description cs:0.5,-0.1)},anchor=north},
        y label style={at={(axis description cs:-0.1,.5)},anchor=south},
        xlabel={$\Pr[\text{honest slot}] - \Pr[\text{dishonest slot}]$},
        % xlabel={Honest stake ratio minus Adversarial stake ratio},
        %
        title={Base-2 logarithm of $\Pr[\text{consistency failure}]$}, 
        ]
        \addplot +[ultra thick,red] {  LnPraosEpsPRootKbound(\x, 900)/ln(2)  };
        \addlegendentry{Asymptotic bound (existing)};
        \addplot +[ultra thick,black] {  LnPraosEpsP(\x, 900)/ln(2)  };
        \addlegendentry{Asymptotic bound (new)};
        % \addplot +[ultra thick,red] { LnPraosEpsP(\x, 1200)/ln(2) } ;  
        % \addlegendentry{$k = 1200$, asymptotic bound};
        % \addplot +[ultra thick,blue] {  LnPraosEpsP(\x, 600)/ln(2) };
        % \addlegendentry{$k = 600$}; 
        %
        \addplot +[ultra thick,blue,mark=triangle] table [x=eps, y=k900, col sep=space] {Log2PraosEpsPoints.csv};
        \addlegendentry{Simulation};
        % \addplot +[ultra thick,blue] table [x=eps, y=k1200, col sep=space] {Log2PraosEpsPoints.csv};
        % \addlegendentry{$k = 1200$, simulation};
        % \addplot +[ultra thick,red] { LogTwoEpsPKnineActual(\x)  };   
        % \addlegendentry{$k = 900$ Actual}; 
      \end{axis}
    \end{tikzpicture}
    \caption[Consistency error bounds]{        
        {\color{red} To do: Describe}
        Security param $k = 900$
    }
    \label{fig:prbad-multihonest}
\end{figure}
}

  A leading PoS cryptocurrency technology company, \emph{Cardano}~\cite{Cardano}, 
  is actively using the results and recommendation from this dissertation.
  In doing so, it paves the way for a widespread practical adoption of these protocols.


  Unlike existing analyses, 
  our security guarantees hold as long as on average,
  an honest slot is more likely than an adversarial slot. 
  Not only does it improve the security guarantees of several existing analyses, 
  but it also opens the door for a wide range of leader election schemes to be used 
  with the Nakamoto-style PoS.


  For the first time in the literature, 
  our research quantifies the grinding power in 
  the beacon used in SnowWhite and Praos.  
  Critically, we pinpoint the parameter regime 
  where the protocol can operate securely. 
  (This is when the adversarial stake ratio remains below $9.55\%$.)
  In doing so, 
  we remove the problematic PoW-like assumption that the grinding power be bounded 
  in terms of the adversarial hashing power. 
  We also highlight how a grinding attack can cripple the overall security 
  by amplifying the consistency error. 
  Praos (and SnowWhite) cease to be secure 
  soon after the adversarial stake increases beyond the limit mentioned above. 
  This dynamic is depicted in Figure~\ref{fig:prbad-praos-beacon}.


\iftoggle{drawfigs}{%
  

\begin{figure}[!htb]
    \centering
    \begin{tikzpicture}[trim axis left,
      ]
      \begin{axis}[domain=0.6:1, 
        restrict x to domain=0.6:0.98
        samples=500,
        enlarge x limits=false,
        grid=both,
        no markers,
        % legend pos=outer north east,
        legend pos=north east,
        legend cell align=left,
        x label style={at={(axis description cs:0.5,-0.1)},anchor=north},
        % y label style={at={(axis description cs:-0.1,.5)},anchor=south},
        xlabel={$\epsilon$, honest bias},
        title={$\rho^*$, upper bound on the min-entropy loss},
        xmin=0.6,xmax=1%,%ymin=-100%,ymax=2.5
        ]
        \addplot +[ultra thick,red] { PraosMinEntropyLossMultiEpoch(\x, 1200, 2) };
        \addlegendentry{$k = 1200$};
        \addplot +[ultra thick,black] { PraosMinEntropyLossMultiEpoch(\x, 900, 2) };
        \addlegendentry{$k = 900$};
        \addplot +[ultra thick,blue] { PraosMinEntropyLossMultiEpoch(\x, 600, 2) };
        \addlegendentry{$k = 600$};
      \end{axis}
    \end{tikzpicture}%
    \begin{tikzpicture}[trim axis left,
      ]
      \begin{axis}[domain=0.6:1, 
        restrict x to domain=0.6:0.999,
        % restrict y to domain=0:100,
        samples=500,
        enlarge x limits=false,
        grid=both,
        no markers,
        % legend pos=outer north east,
        legend pos=north west,
        legend cell align=left,
        x label style={at={(axis description cs:0.5,-0.1)},anchor=north},
        y label style={at={(axis description cs:-0.1,.5)},anchor=south},
        xlabel={$\epsilon$, honest bias},
        title={$-\log_2\, \Pr[ \text{min-entropy loss } > \rho^* ]$},
        xmin=0.6,xmax=1%,ymin=0%,ymax=2.5
        ]
        \addplot +[ultra thick,red] { -LnPraosPrBadMultiEpoch(\x, 900, 2)/ln(2) } ;
        \addlegendentry{$L=2, k = 900$};
        \addplot +[ultra thick,black] {  -LnPraosPrBadMultiEpoch(\x, 900, 2000)/ln(2)  };
        \addlegendentry{$L = 2000, k = 900$};
        \addplot +[ultra thick,blue] {  -LnPraosPrBadMultiEpoch(\x, 900, 200000)/ln(2) };
        \addlegendentry{$L = 200000, k = 900$};
      \end{axis}
    \end{tikzpicture}
    \caption{
        (\textbf{Top}) The upper bound $\rho^*$ on $\rho$, 
        Praos beacon's min-entropy loss over a $L$ epochs, 
        expressed in \#bits.  
        Recall that $\rho$ is the logarithm of the grinding power in an epoch.     
        Praos' parameters are selected according to Theorem~\ref{thm:praos-gp-single-epoch}, 
        as follows.
        (Note that the current analysis is in the synchronous model.)
        $k$ is the common prefix parameter. 
        (Cardano, an implementation of Praos, uses $k = 900$.) 
        The adversarial stake is $(1-\epsilon)/2$.
        An epoch contains $R = 16 k/(1+\epsilon)$ slots and the first $2R/3$ slots are used for the beacon. 
        We used the $\ECQ$ parameter $s = 8 k/(1+\epsilon)$; 
        this value is the same as the liveness parameter of Praos. 
        The probability of a consistency violation in an epoch in Praos is 
        $\EpsP = R \exp(1-\epsilon^3 (1-\epsilon)k/2) $. 
        See~\citet[Theorem 9]{Praos} for justification of these parameters. 
        (\textbf{Bottom}) The probabilities that the grinding power is larger than $2^{\rho^*}$.
    }
    \label{fig:praos-beacon-multi-epoch}
\end{figure}

}



  Realizing that the beacon in Praos has its limitations,
  our research focuses on new beacons for Nakamoto-style PoS. 
  The new beacons constructed in our research 
  are simple (i.e., no cumbersome MPC is used) 
  and have superior bias-resistance 
  i.e., the grinding power, as a function of the security parameter and the adversarial stake, 
  grows much slowly compared to Praos. 
  We achieve this by 
  recording the input nonces outside the blockchain 
  (in a separate ledger)
  and by controlling the number of slots 
  that contribute in the computation of the beacon output.
  Note that sometimes, 
  the protocol designers are willing to tolerate some 
  a limited grinding attack 
  in exchange of a simple-to-implement beacon 
  with a well-understood grinding analysis. 
  Our new beacons fill this void. 
  See Figure~\ref{fig:poisson-beacon}.

\iftoggle{drawfigs}{%
  
\begin{figure}[!htb]
    \centering
    \tikzsetnextfilename{rho-poisson-praos}
    \begin{tikzpicture}[trim axis left,
      ]
      \begin{axis}[domain=0:1, 
        xmin=0,xmax=1,ymin=0,%,ymax=2.5
        unbounded coords=jump, 
        samples=500,
        enlarge x limits=false,
        grid=both,
        no markers, 
        % legend pos=outer north east,
        legend pos=south west,
        legend cell align=left,
        x label style={at={(axis description cs:0.5,-0.1)},anchor=north},
        y label style={at={(axis description cs:-0.1,.5)},anchor=south},
        xlabel={$\epsilon$, honest bias},
        title={Min-entropy loss, Praos (upper bound)}
        ]
        \addplot +[ultra thick,red] { PraosMinEntropyLossMultiEpoch(\x, 900 ) };
        \addlegendentry{Praos, $k = 900$};
        \addplot +[ultra thick,blue] { PraosMinEntropyLossMultiEpoch(\x, 1200) };
        \addlegendentry{Praos, $k = 1200$};
        % \addplot +[ultra thick,black,domain=0.18:1] { PoissonBeaconMinEntropyLoss(\x, 900) };
        % \addlegendentry{New, $k = 900$};
        % \addplot +[ultra thick,orange,mark=*,domain=0.18:1] { PoissonBeaconMinEntropyLoss(\x, 1200) };
        % \addlegendentry{New, $k = 1200$};
      \end{axis}
    \end{tikzpicture}%
    \tikzsetnextfilename{rho-poisson}
    \begin{tikzpicture}[trim axis left,
      ]
      \begin{axis}[domain=0:1, 
        xmin=0,xmax=1,ymin=0,%,ymax=2.5
        yticklabel pos=right,
        yticklabel style={xshift=2em},
        restrict x to domain=0:1,
        unbounded coords=jump, 
        samples=500,
        enlarge x limits=false,
        grid=both,
        no markers, 
        % legend pos=outer north east,
        legend pos=north west,
        legend cell align=left,
        x label style={at={(axis description cs:0.5,-0.1)},anchor=north},
        y label style={at={(axis description cs:-0.1,.5)},anchor=south},
        y tick label style={anchor=east},
        xlabel={$\epsilon$, honest bias}, 
        title={Min-entropy loss, New beacon (upper bound)}
        ]
        \addplot +[ultra thick,black,domain=0.18:1] { PoissonBeaconMinEntropyLoss(\x, 900) };
        \addlegendentry{New, $k = 900$};
        \addplot +[ultra thick,orange,mark=*,domain=0.18:1] { PoissonBeaconMinEntropyLoss(\x, 1200) };
        \addlegendentry{New, $k = 1200$};
      \end{axis}
    \end{tikzpicture}%
    \caption{Min-entropy loss in Poisson beacon}
    \label{fig:rho-poisson-beacon}
\end{figure}
\begin{figure}[!htb]
  \centering
    \tikzsetnextfilename{prbad-poisson}
    \begin{tikzpicture}[trim axis left,
      ]
      \begin{axis}[domain=0:1, 
        xmin=0,xmax=1,%,ymax=0,
        restrict x to domain=0:1,
        unbounded coords=jump, 
        samples=500,
        enlarge x limits=false,
        grid=both,
        no markers,
        % legend pos=outer north east,
        legend pos=north west,
        legend cell align=left,
        x label style={at={(axis description cs:0.5,-0.1)},anchor=north},
        y label style={at={(axis description cs:-0.1,.5)},anchor=south},
        % yticklabel pos=right,
        xlabel={$\epsilon$, honest bias},
        title={$\log_2 \left(\Pr[\mathrm{Bad}_{\mathrm{New} }]/\Pr[\mathrm{Bad}_{\mathrm{Praos} }] \right)$},
        ]
        \addplot +[ultra thick,red,domain=0.18:1] { \log_2(3) + PoissonBeaconMinEntropyLoss(\x, 900) - PraosMinEntropyLossMultiEpoch(\x, 900 ) } ;
        \addlegendentry{$k = 900$};
        \addplot +[ultra thick,black,domain=0.18:1] { \log_2(3) + PoissonBeaconMinEntropyLoss(\x, 1200) - PraosMinEntropyLossMultiEpoch(\x, 1200 ) } ;
        \addlegendentry{$k = 1200$};
        % \addplot +[ultra thick,black,domain=0.18:1] { LnPoissonBeaconPrBadMultiEpoch(\x, 2000, 1 ) } ;
        % \addlegendentry{Praos, $k = 2000$};
        % \addplot +[ultra thick,black,dashed] { LnPoissonBeaconPrBadMultiEpoch(\x, 900, 1, PraosEpsP(\x, 900) )/ln(2) } ;
        % \addlegendentry{New, $k = 900$};
        % \addplot +[ultra thick,blue] { LnPraosPrBadMultiEpoch(\x, 1200, 1 )/ln(2) } ;
        % \addlegendentry{Praos, $k = 1200$};
        % \addplot +[ultra thick,orange,dashed] { LnPoissonBeaconPrBadMultiEpoch(\x, 1200, 1, PraosEpsP(\x, 1200) )/ln(2) } ;
        % \addlegendentry{New, $k = 1200$};
      \end{axis}
    \end{tikzpicture}
    \caption{Failure probability for the Poisson beacon}
    \label{fig:prbad-poisson-beacon}
\end{figure}

  
}

  The cryptocurrency Cardano is built on top of the Nakamoto-style PoS protocol, Praos~\cite{Praos}.
  As of \today, 
  the market capitalization of Cardano is US\$ 3.5 billion~\cite{CardanoMarketcap}. 
  The protocol parameters in Praos is 
  informed by the consistency bounds developed in this dissertation. 
  In particular, its settlement parameter can be set 
  so that the best compromise can be achieved between 
  reducing the settlement delay, 
  reducing the consistency error, 
  and reducing the risk due to the grinding power.
  Specifically, our research 
  can inform the practitioners and the users 
  about the amplified risk of a consistency error due to grinding.





\section{Outline of this dissertation}
In \Section~\ref{sec:pos-primer}, 
we collect some important concepts and background information about Nakamoto-style PoS algorithms. 
The main content of this dissertation are spread over three parts. 
Instead of an exhaustive list of the Chapters within these parts, 
we mention the ones that are central to the presentation.

\begin{description}[font=\normalfont\ ]
  \item[In Part~\ref{part:multihonest},]
    we analyze the consistency problem in Nakamoto-style PoS.
    \Section~\ref{sec:intro-multihonest} introduces the problem in more detail, 
    and \Section~\ref{sec:model-multihonest} formally lays down the model and 
    states the main theorem (Theorem~\ref{thm:main-mh}) in the synchronous setting.
    \Section~\ref{sec:async-multihonest} lifts these results in the semi-synchronous setting 
    via a standard reduction introduced in~\cite{Praos}, 
    but incurring some slack in the consistency guarantee. 
  
  \item[In Part~\ref{part:praos},]
    we analyze the grinding problem in Ouroboros Praos~\cite{Praos} and SnowWhite~\cite{SnowWhite} 
    in the synchronous communication setting.
    \Section~\ref{sec:grinding-intro} introduces the problem in more detail, 
    and \Section~\ref{sec:model-grinding} formally lays down the model and 
    states the main theorem (Theorem~\ref{thm:minentropy-loss-praos-multi-epochs}). 

  \item[In Part~\ref{part:xorgames},]
    we analyze a new beacon for eventual consensus blockchains. 
    \Section~\ref{sec:xorgames-intro} introduces the problem in more detail. 
    \Section~\ref{sec:xorgames-model} formally lays down the model, 
    specifies the new beacon in Definition~\ref{def:new-beacon} and 
    states the main theorem (Theorem~\ref{thm:beacon-poisson-multi-epoch}) 
    for private leader elections.
    These beacons have superior output quality compared to the beacon in Praos or SnowWhite. 
    In addition, (Theorem~\ref{thm:beacon-bernoulli-multi-epoch}) 
    analyzes the beacon for public leader elections; 
    this beacon gives an interesting alternative to the multi-party computation-based 
    beacon in Ouroboros~\cite{Ouroboros}.
\end{description}




\section{Limitations and mitigations}
Below, we discuss some limitations in our analysis 
and how one could mitigate them.


\Paragraph{Simple honest majority and beyond.}
Our analysis in this dissertation assumes that the leader election outcomes obey 
the ``simple honest majority'' rule, on average: 
that is, an honest slot is more likely than an adversarial slot. 
However, it is conceivable that the adversarial slots may become more likely, even if only for a limited time. 
This may stem from, e.g., a temporary spike in the adversarial stake in the system 
which gets dissipated over the long term. 
The recent paper by~\citet{AdversarialSupremacy} deals with this scenario in both Nakamoto-style PoS and PoW. 
However, like Praos, their model awards a multihonest slot to the adversary; 
but the raison d'\^{e}tre of Part~\ref{part:multihonest} of this dissertation 
is to alleviate this difficulty.
It would be an interesting future work to combine these ideas with those in~\cite{AdversarialSupremacy}.

\Paragraph{Synchronous grinding analysis.}
Our analysis of the grinding problem in Praos (and SnowWhite) 
is in the synchronous setting. 
Since the security guarantees in Praos are in the semi-synchronous setting, 
it would be fitting if we could either lift this analysis to the semi-synchronous setting 
or, better yet, did it directly in the semi-synchronous setting. 
One way to accomplish this would be to 
count the number of adversarial blockchains in a semi-synchronous fork (Definition~\ref{def:delta-fork}). 
Another way would be to apply a synchronous-to-semi-synchronous reduction (see \Section~\ref{sec:async-multihonest})
to the synchronous grinding power.


\Paragraph{Slack in estimating the grinding power in the new beacon.}
Our analysis of the new beacon in Section~\ref{sec:poisson}, 
in the private leader election 
has some slack. 
Instead of exactly analyzing 
the distribution of the option sets 
induced by the nonce leader election $\PrivateLeaderElection(\alpha)$, 
we derived an upper bound by analyzing a distribution 
that was easier to analyze. 
Specifically, 
instead of counting ``the number of nonces submitted by adversarial nonce-leaders, 
that are lexicographically smaller than 
the lexicographically smallest nonce submitted by the honest nonce-leaders,'' 
we counted ``the number of adversarial nonce-leaders.'' 
Thus, our bound could be tightened via an exact (or stricter) analysis.
















\pagebreak












