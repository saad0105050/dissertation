




Consensus means agreement.

In distributed computing, 
consensus algorithms provide strategies 
so that different people (or computers) 
may come to agreement. 
These algorithms are often called \emph{protocols} as they 
tell the participants which specific actions they should take 
e.g., when to send a certain message to others, etc. 
Sometimes, the participants cannot trust each other; 
this is the \emph{decentralized consensus problem}. 
The consensus has to be reached against threats and issues such as 
network delays, network failure, crashing computers, and even 
dishonest (or malicious) participants. 
Often, we use the term \emph{adversary} to broadly refer to these threats. 
A \emph{Byzantine adversary} can deviate arbitrarily from the protocol. 


The decentralized consensus problem in the Byzantine threat model 
is called the \emph{Byzantine Fault Tolerance} (or BFT in short). 
This problem has been studied extensively over the past few decades, 
starting from the ground-breaking paper~\citet{BFT}. 
The consensus problem can have a rich structure and varying difficulty 
depending on the model e.g. network delays or Byzantine adversaries. 

Like many protocols, consensus protocols 
progress in \emph{rounds} 
(i.e., a logical time segment) 
and, at each round, the participants execute relevant instructions of the protocol. 
The protocol guarantees that after a number of rounds, 
a consensus would be reached. 

Consensus can mean different things. 
Typically, we expect \emph{instant settlement} 
i.e., that once a consensus is reached, 
it is final. 
However, if we allow the consensus to happen gradually, 
things suddenly become more interesting.
In \emph{eventual consensus}, 
we ask for an \emph{iterated} consensus protocol 
where the consensus solidifies as the rounds passes. 
Suppose that we want to reach consensus about a series of events. 
The participants may disagree about what transpired in recent rounds, 
but the protocol guarantees that they would all agree about 
what happened, say, one hundred rounds ago. 
This is called the \emph{consistency property} of the protocol. 
Note that in this setting, 
the current round cannot be considered ``settled'' 
before a number of round passes by; 
this delay is called the \emph{settlement delay}. 
It is essential that the protocol gives us a guarantee 
about the settlement delay;
otherwise, we wouldn't know if (or when) we have reached consensus.

But why should we even consider the eventual consensus model in the first place? 
The answer is simplicity: 
While consensus algorithms in the instant settlement model 
are typically complex, 
they can be surprisingly simple in the eventual consensus model. 
The famous \emph{longest-chain rule}, put forth in the 
celebrated work of~\citet{Nakamoto2008}, states that:
\begin{enumerate}
  \item Each participant maintains a list of events; 
  this list is called a \emph{chain}.

  \item At the beginning of every round, 
  each participant adopts a longest chain in his view.

  \item A participant can test whether he is eligible to add a new record to his chain. 
  If he is, he adds this record (called a \emph{block}) on top of his chain 
  and shows the new chain to everyone.
\end{enumerate}
\noindent
It is remarkable that this simple rule actually allows eventual consensus. 
% A guarantee on the settlement delay can be derived by analyzing this algorithm.

A participant who is eligible to add a new record is called a \emph{leader}. 
The word is a reference to the \emph{leader election problem}. 
Leader election is a consensus problem itself: 
in this problem, the participants 
reach agreement about who among them would get 
the special designation of their ``leader.''
Note how the longest-chain rule does not say anything about 
how to elect a leader: 
we can plug in, to the main protocol, 
different leader election mechanisms as we choose. 
This flexibility gives a rich structure to the problem 
which we can exploit to achieve eventual consensus 
in different settings.





\pagebreak











\iftoggle{drawfigs}{%
  \begin{figure}[!htb]
    \centering
    \tikzsetnextfilename{prbad}
    \begin{tikzpicture}[trim axis left,
      ]
      \begin{axis}[domain=0:1, 
        xmin=0,xmax=1,%
        % ymax=0,%,ymax=2.5
        % restrict x to domain=0.6:0.999,
        % restrict y to domain=0:100,
        unbounded coords=jump, 
        samples=500,
        enlarge x limits=false,
        grid=both,
        no markers,
        % yticklabel pos=right,
        % yticklabel style={xshift=2em},
        legend pos=outer north east,
        % legend pos=south west,
        legend cell align=left, 
        x label style={at={(axis description cs:0.5,-0.1)},anchor=north},
        y label style={at={(axis description cs:-0.1,.5)},anchor=south},
        xlabel={$\Pr[\text{honest slot}] - \Pr[\text{dishonest slot}]$},
        % xlabel={Honest stake ratio minus Adversarial stake ratio},
        %
        title={Base-2 logarithm of $\Pr[\text{consistency failure}]$}, 
        ]
        \addplot +[ultra thick,red] {  LnPraosEpsPRootKbound(\x, 900)/ln(2)  };
        \addlegendentry{Asymptotic bound (existing)};
        \addplot +[ultra thick,black] {  LnPraosEpsP(\x, 900)/ln(2)  };
        \addlegendentry{Asymptotic bound (new)};
        % \addplot +[ultra thick,red] { LnPraosEpsP(\x, 1200)/ln(2) } ;  
        % \addlegendentry{$k = 1200$, asymptotic bound};
        % \addplot +[ultra thick,blue] {  LnPraosEpsP(\x, 600)/ln(2) };
        % \addlegendentry{$k = 600$}; 
        %
        \addplot +[ultra thick,blue,mark=triangle] table [x=eps, y=k900, col sep=space] {Log2PraosEpsPoints.csv};
        \addlegendentry{Simulation};
        % \addplot +[ultra thick,blue] table [x=eps, y=k1200, col sep=space] {Log2PraosEpsPoints.csv};
        % \addlegendentry{$k = 1200$, simulation};
        % \addplot +[ultra thick,red] { LogTwoEpsPKnineActual(\x)  };   
        % \addlegendentry{$k = 900$ Actual}; 
      \end{axis}
    \end{tikzpicture}
    \caption[Consistency error bounds]{        
        {\color{red} To do: Describe}
        Security param $k = 900$
    }
    \label{fig:prbad-multihonest}
\end{figure}
}

