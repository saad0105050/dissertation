




Consensus means agreement.

In distributed computing, 
consensus algorithms provide strategies 
so that different people (or computers) 
may come to agreement. 
These algorithms are often called \emph{protocols} as they 
tell the participants which specific actions they should take 
e.g., when to send a certain message to others, etc. 
Sometimes, the participants cannot trust each other; 
this is the \emph{decentralized consensus problem}. 
The consensus has to be reached against threats and issues such as 
network delays, network failure, crashing computers, and even 
dishonest (or malicious) participants. 
Often, we use the term \emph{adversary} to broadly refer to these threats. 
A \emph{Byzantine adversary} can deviate arbitrarily from the protocol. 


The decentralized consensus problem in the Byzantine threat model 
is called the \emph{Byzantine Fault Tolerance} (or BFT in short). 
This problem has been studied extensively over the past few decades, 
starting from the ground-breaking paper~\citet{BFT}. 
The consensus problem can have a rich structure and varying difficulty 
depending on the model e.g. network delays or Byzantine adversaries. 

Like many protocols, consensus protocols 
progress in logical time segments called \emph{rounds} 
and, at each round, the participants execute relevant instructions of the protocol. 
The protocol guarantees that after a number of rounds, 
a consensus would be reached. 

Consensus can mean different things. 
Typically, we expect \emph{instant settlement} 
i.e., that once a consensus is reached, 
it is final. 
However, if we allow the consensus to happen gradually, 
things suddenly become more interesting.
In \emph{eventual consensus}, 
we ask for an \emph{iterated} consensus protocol 
where the consensus solidifies as the rounds passes. 
Suppose that we want to reach consensus about a series of events 
(called a \emph{ledger}). 
The participants may disagree about what transpired in recent rounds, 
but the protocol guarantees that they would all agree about 
what happened, say, one hundred rounds ago. 
This is called the \emph{consistency property} of the protocol. 
Note that in this setting, 
the current round cannot be considered ``settled'' 
before a number of round passes by; 
this ``wait time, '' or delay, is called the \emph{settlement delay}. 
It is essential that the protocol gives us a guarantee 
about the settlement delay;
otherwise, we wouldn't know if (or when) we have reached consensus.

But why should we even consider the eventual consensus model in the first place? 
The answer is simplicity: 
While consensus algorithms in the instant settlement model 
are typically complex, 
they can be surprisingly simple in the eventual consensus model. 
The famous \emph{longest-chain rule}, put forth in the 
celebrated work of~\citet{Nakamoto2008}, states that:
\begin{enumerate}
  \item Each participant maintains a list of events; 
  this list is called a \emph{chain}.

  \item At the beginning of every round, 
  each participant adopts a longest chain in his view.

  \item A participant can test whether he is eligible to add a new record to his chain. 
  If he is, he adds this record (called a \emph{block}) on top of his chain 
  and shows the new chain to everyone.
\end{enumerate}

It is a remarkable fact that this simple rule leads to the eventual consensus. 
The notion of chains and blocks above has led to the term ``blockchain'' 
which has come to define a class of consensus algorithms 
in both eventual consensus model as well as in the traditional, 
instant confirmation model~\cite{BlockchainAlgorithms}. 
% We also use the term ``blockchain'' to implicitly refer to the 
% ledger maintained by the participants.

A participant who is eligible to add a new record is called a \emph{leader}. 
The word ``leader'' is a reference to the \emph{leader election problem}, 
a consensus problem where the participants want to 
reach agreement about who, among them, would get 
the special designation of being a leader.
Note how the longest-chain rule allows us to plug in 
different leader election mechanisms. 
This flexibility gives a rich structure to the algorithm.

The leader election process is designed to be random. 
If most of the participants are honest, 
a randomly picked participant is more likely to be honest as well. 
In the long term, the accumulated effect of these good events 
should lead to good outcomes, i.e., reaching consensus. 
However, having this randomness built into the system means 
that the consensus is a random event as well; 
therefore, the consistency is achieved only with a high probability. 
A \emph{consistency error bound} 
% associated with a settlement delay parameter
estimates the probability 
(of the unlikely event) that we did not reach consensus 
even after waiting for a stipulated amount of rounds. 
We desire that the consistency error should decrease exponentially 
in the settlement delay parameter. 
If it indeed happens, 
we say that the protocol has achieved the \emph{linear consistency}.

Bitcoin---the poster boy of the longest-chain rule---achieves 
linear consistency; it does so by employing a clever leader election mechanism 
called the \emph{Proof-of-Work lottery} (PoW). 
The proof part is easy: we know how to prove claims using 
digital signatures and zero-knowledge proofs~\cite {DigitalSignatures,ZK}. 
The ``work'' part addresses the rarity of leadership. 
Specifically, 
the probability of being elected a leader is kept small by design 
so that we do not have a deluge of competing chains. 
In PoW, participants have to solve a computationally difficult 
cryptographic puzzle~\cite{PoW} 
to be a leader. 
The difficulty of the puzzle ensures that at any given round, 
it is unlikely to have many lottery winners. 
However, since being elected a leader has its rewards (e.g., commissions and fees), 
rational participants invest in computational power 
and keep solving their puzzles. 
As result, a PoW network consumes a lot of energy: 
Bitcoin's energy consumption is 
on par with a medium-sized country~\cite{BitcoinEnergy}.

One way to avoid such a waste of energy 
is to rely on a proof mechanism which provides 
a useful service, such as Proof-of-Storage~\cite{ProofOfStorage}. 
Yet another alternative is to bypass the necessity of a physical resource 
(such as computational power or storage space)
altogether and instead, perform the lottery based on an abstract resource. 
An abstract resource could be, for example, 
a participant's \emph{stake} in the system, 
a numeric quantity which can be inferred 
from the settled part of the blockchain; 
thus the \emph{Proof-of-Stake} (PoS) paradigm is born.

A difficulty in the PoS analysis 
comes from its adversarial model. 
Under the longest-chain rule (for both PoS and PoW), 
if an honest participant 
is presented with multiple longest chains, 
he has no way to differentiate between these chains. 
In this scenario, the model gives the adversary the power to 
select an arbitrary longest chain; 
this is called the \emph{longest-chain tie-breaking}. 
Since the PoS model puts no inherent cost 
(e.g., spending money to buy computational resource) 
for a leader to produce new blocks, 
adversarial leaders may 
freely produce new blocks. 
Furthermore, they may strategically 
extend existing chains 
so that they can 
present two longest chains to an honest observer. 
Keep in mind that only one of these chains 
is going to belong in the consensus. 
Conceivably, these two chains may diverge 
long ago -- even before the boundary set by the settlement parameter -- and 
contain conflicting records regarding a round, 
in violation of the consistency property.  
It follows that in PoS, 
the adversary is given more power to 
break (or delay) the consensus, 
compared to PoW. 


A more subtle difficulty with PoS is that 
we have to maintain the stake information on the ledger itself. 
Since the leader election is based on stakes indicated by the 
consistent part of the blockchain, 
the integrity of the leader election mechanism in the PoS setting 
is tightly intertwined 
with its consistency property. 
We did not have this issue in PoW: 
the model did not require the participants to prove 
how much computational resource they own.
Thus, proving the security of a PoS system is 
trickier than that of a PoW system.



In a \emph{double spending attack}, 







\pagebreak











\iftoggle{drawfigs}{%
  \begin{figure}[!htb]
    \centering
    \tikzsetnextfilename{multihonest-prbad}
    \begin{tikzpicture}[trim axis left,
      ]
      \begin{axis}[domain=0:1, 
        xmin=0,xmax=1,%
        % ymax=0,%,ymax=2.5
        % restrict x to domain=0.6:0.999,
        % restrict y to domain=0:100,
        unbounded coords=jump, 
        samples=500,
        enlarge x limits=false,
        grid=both,
        no markers,
        % yticklabel pos=right,
        % yticklabel style={xshift=2em},
        legend pos=south west,
        % legend pos=south west,
        legend cell align=left, 
        x label style={at={(axis description cs:0.5,-0.1)},anchor=north},
        y label style={at={(axis description cs:-0.1,.5)},anchor=south},
        xlabel={$\epsilon$, Honest bias},
        % xlabel={Honest stake ratio minus Adversarial stake ratio},
        %
        title={Base-2 logarithm of $\Pr[\text{failure}]$}, 
        ]
        \addplot +[ultra thick,red] {  LnPraosEpsPRootKbound(\x, 900)/ln(2)  };
        \addlegendentry{Asymp. bound (existing)};
        \addplot +[ultra thick,black] {  LnPraosEpsP(\x, 900)/ln(2)  };
        \addlegendentry{Asymp. bound (new)};
        % \addplot +[ultra thick,red] { LnPraosEpsP(\x, 1200)/ln(2) } ;  
        % \addlegendentry{$k = 1200$, asymptotic bound};
        % \addplot +[ultra thick,blue] {  LnPraosEpsP(\x, 600)/ln(2) };
        % \addlegendentry{$k = 600$}; 
        %
        \addplot +[ultra thick,blue,mark=triangle] table [x=eps, y=k900, col sep=space] {Log2PraosEpsPoints.csv};
        \addlegendentry{Simulation};
        % \addplot +[ultra thick,blue] table [x=eps, y=k1200, col sep=space] {Log2PraosEpsPoints.csv};
        % \addlegendentry{$k = 1200$, simulation};
        % \addplot +[ultra thick,red] { LogTwoEpsPKnineActual(\x)  };   
        % \addlegendentry{$k = 900$ Actual}; 
      \end{axis}
    \end{tikzpicture}
    \caption[Consistency error bounds]{  
        Estimates of the consistency violation probability $\EpsP$ using       
        security parameter $k = 900$. 
        The x-axis corresponds to the honest bias $\epsilon \in (0,1)$. orking
        
        Leader elections in slots are i.i.d. 
        with $\Pr[\text{slot honest}]=(1+\epsilon)/2$ and 
        $\Pr[\text{slot dishonest}]=(1-\epsilon)/2$.
        We plotted the existing $e^{-\Omega(\sqrt{k})}$ asymptotic bound~\cite{Ouroboros,SnowWhite,Sleepy} (red) and the new $e^{-\Omega(k)}$ bound proved in this dissertation (Theorem~\ref{thm:main-mh}).
        The blue curve corresponds to the explicit probabilities computed using 
        the Markov chain simulation algorithm described in \Section~\ref{sec:exact-prob-multihonest}. 
        The asymptotic bounds are more accurate for small $\epsilon$.
    }
    \label{fig:prbad-multihonest}
\end{figure}
}

