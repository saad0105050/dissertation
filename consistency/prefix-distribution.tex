Stochastic dominance plays an important role in the arguments
below. First of all, we observe that the distribution
$\mathcal{B}_\epsilon$ stochastically dominates any distribution
satisfying the $\epsilon$-martingale condition; this yields the first
inequality in Theorem~\ref{thm:main}. A more delicate application of
stochastic dominance is used in order to achieving bounds, such as
those of Section~\ref{sec:bounds}, that are independent of the length of
$x$. This follows from the fact that $\reach(B_{\epsilon})$ converges to a
particular, dominant distribution as its argument increases in length.

For notational convenience, we denote
the probability distribution associated with a random variable using
uppercase script letters; for example, the distribution of a random
variable $R$ is denoted by $\mathcal{R}$.  This usage should be clear
from the context.


\begin{definition}[Monotonicity and stochastic dominance]\label{def:dominance}
  Let $\Omega$ be a set endowed with a partial order $\leq$. A subset
  $A \subset \Omega$ is monotone if for all $x \leq y$, $x \in A$
  implies $y \in A$.  Let $X$ and $Y$ be random variables taking
  values in $\Omega$.
  % Let $\mathcal{X}$ and $\mathcal{Y}$ be distributions associated with 
  % $X$ and $Y$, respectively. 
  We say that $X$ \emph{stochastically dominates} $Y$, 
  written $Y \dominatedby X$, if 
  $
    \mathcal{X}(A) \geq \mathcal{Y}(A)
    % \,.
    $ for all monotone $A \subseteq \Omega$.  As a special case, when
    $\Omega = \R$, $Y \dominatedby X$ if
    $\Pr[X \geq \Lambda] \geq \Pr[Y \geq \Lambda]$ for every
    $\Lambda \in \R$.  We extend this notion to probability
    distributions in the natural way.
\end{definition}
% \begin{definition}[Stochastic dominance]\label{def:dominance} 
% Let $X$ and $Y$ be random
%   variables taking values in $\R$. We say that $X$ \emph{stochastically
%   dominates} $Y$, written $Y \dominatedby X$ if
%   \[
%     \Pr[X \geq \Lambda] \geq \Pr[Y \geq \Lambda]
%   \]
%   for every $\Lambda \in \R$.  We extend this notion to probability
%   distributions in the natural way.
% %  In addition, for two distributions $\mathcal{X}, \mathcal{Y}$,
% %  we say that $\mathcal{X}$ stochastically dominates $\mathcal{Y}$ 
% %  if and only if there are random variables $X \sim \mathcal{X}, Y \sim %\mathcal{Y}$ such that $Y \dominatedby X$;
% %  this is written as $\mathcal{Y} \dominatedby \mathcal{X}$.
% \end{definition}
Observe that for any non-decreasing function $u$ defined on $\Omega$,
$Y \dominatedby X$ implies $u(Y) \leq u(X)$. Finally, we note that for
real-valued random variables $X$, $Y$, and $Z$, if $Y \dominatedby X$
and $Z$ is independent of both $X$ and $Y$, then
$Z + Y \dominatedby Z + X$.

% Let $m \in \NN$ and suppose 
% $W = (W_1, \ldots, W_m) \in \{0,1\}^m$ satisfies the 
% $\epsilon$-martingale condition. 
% It turns out that $\rho(W)$ 
% is stochastically dominated by 
% the distribution of $\rho(B_1, \ldots, B_m)$, 
% where each $B_i \in \{0, 1\}$ is 
% an independent Bernoulli random variable with parameter $(1 - \epsilon)/2$.
% In addition, $\rho(B_1, \ldots, B_m)$ is stochastically dominated by 
% its limiting (stationary) distribution where we take $m \rightarrow \infty$.

%=======================================================
\begin{lemma}\label{lemma:rho-stationary}
  % Let $n \in \NN$ and consider a sequence of random variables
  % $W = (W_1, \ldots, W_n) \in \{0,1\}^n$ satisfying the
  % $\epsilon$-martingale condition. 
  Suppose $W = (W_1, \ldots, W_n) \in \{0,1\}^n$ satisfies the 
  $\epsilon$-martingale condition. 
  Let $\epsilon \in (0, 1)$ and $B = (B_1, \ldots, B_n) \in \{0,1\}^n$ 
  where each $B_i$ is independent with expectation $(1- \epsilon)/2$.
  Let $R_\infty \in \{0, 1, \ldots\}$ be a random variable 
  whose distribution $\StationaryRho$ is defined as 
    \begin{equation}
      \label{eq:stationary}
      \StationaryRho(k) 
        = \Pr[R_\infty = k] 
        \defeq \left(\frac{2\epsilon}{1+\epsilon}\right)\cdot \left(\frac{1-\epsilon}{1 + \epsilon}\right)^k
        \qquad \text{for $k = 0, 1, 2, \ldots$}\ 
      \,.
    \end{equation}
  Then $\rho(W) \dominatedby \rho(B) \dominatedby R_\infty$.
\end{lemma}

\begin{proof}
  We begin by observing that $B$ stochastically dominates $W$. As a
  matter of notation, for any fixed values
  $w_1, \ldots, w_k \in \{0,1\}^k$, let
  \[
    \theta[w_1, \ldots, w_k] = \Pr[ W_{k+1} = 1 \mid
    \text{$W_i = w_i$, for $i \leq k$}] \leq (1 - \epsilon)/2
  \]
  and $\theta[\varepsilon] = \Pr[W_1 = 1]$ 
  where $\varepsilon$ is the empty string. Then consider $n$ uniform and
  independent real numbers $(A_1, \ldots, A_n)$, each taking a value
  in the unit interval $[0,1]$; we use these random variables to construct a monotone
  coupling between $W$ and $B$. 
  Specifically, define $\beta: [0,1]^n \rightarrow \{0,1\}^n$
  by the rule $\beta(\alpha_1, \ldots, \alpha_n) = (b_1, \ldots, b_n)$
  where
  \[
    b_t = \begin{cases} 1 & \text{if $\alpha_t \leq (1-\epsilon)/2$},\\
      0 & \text{if $\alpha_t > (1 - \epsilon)/2$},
    \end{cases}
  \]
  and define
  $B = (B_1, \ldots, B_n) = \beta(A_1, \ldots, A_n)$; these
  $B_i$s are independent zero-one Bernoulli random variables with expectation
  $(1-\epsilon)/2$. Likewise define the function
  $\omega:[0,1]^n \rightarrow \{0,1\}^n$ so that
  $\omega(\alpha_1, \ldots, \alpha_n) = (w_1, \ldots, w_n)$
  where each $w_t$ is assigned by the iterative rule
  \[
    w_{t+1} = \begin{cases} 1 & \text{if $\alpha \leq \theta[w_1, \ldots, w_t]$},\\
      0 & \text{if $\alpha > \theta[w_1, \ldots, w_t]$},
    \end{cases}
  \]
  and observe that the probability law of
  $\omega(A_1, \ldots, A_n)$ is precisely that of
  $W = (W_1, \ldots, W_n)$. For convenience, we simply identify the
  random variable $W$ with $\omega(A_1, \ldots, A_n)$. Note
  that for any $\alpha = (\alpha_1, \ldots, \alpha_n)$ and for each
  $i$, the $i$th coordinates of $\beta(\alpha)$ and $\omega(\alpha)$ satisfy
  $\omega(\alpha)_i \leq \beta(\alpha)_i$ 
  % (which is to say that $W_i \leq B_i$). 
  % It follows immediately that
  % $\rho(\omega(\alpha)) \leq \rho(\beta(\alpha))$ with probability 1 and
  % hence $\rho(W) \dominatedby \rho(B)$. 
  % See~\cite[Lemma 22.5]{LevinPeres}. 
  (which is to say that $W_i \leq B_i$ with probability 1). 
  But this is equivalent to saying $W \dominatedby B$. 
  (See~\cite[Lemma 22.5]{LevinPeres}.) 
  Now consider the following partial order $\leq$ on the $n$-bit Boolean strings: 
  for $x,y \in \{0,1\}^n$, 
  we write $x \leq y$ if and only if $x_i = 1$ implies $y_i = 1, i \in [n]$.
  Since $\rho$ is non-decreasing with respect to this partial order, 
  we have  
  $\rho(\omega(\alpha)) \leq \rho(\beta(\alpha))$ with probability 1 and
  hence $\rho(W) \dominatedby \rho(B)$ as well. 
  

  %  \paragraph{$R_\infty$ stochastically dominates $\rho(B)$.}
  To complete the proof, we now establish that
  $\rho(B) \dominatedby R_\infty$.  We remark that the random variables
  $\rho(B)$ (and $R_\infty$) have an immediate interpretation in terms
  of the Markov chain corresponding to a biased random walk on $\Z$
  with a ``reflecting boundary'' at -1. Specifically, consider the
  Markov chain on $\{0, 1, \ldots\}$ given by the transition diagram
  \begin{center}
    \begin{tikzpicture}[scale=1,>=stealth', auto, semithick,
      flat/.style={circle,draw=black,thick,text=black,font=\small}]
      \node[flat] (n0) at (0,0)  {$0$};
      \node[flat] (n1) at (1,0)  {$1$};
      \node[flat] (n2) at (2,0)  {$2$};
      \node[flat,white] (n3) at (3,0) {$ \ \ \ $};
      \node[] at (3,0) {$\ldots$};
      \draw[thick,->,bend left] (n0) to (n1);
      \draw[thick,->,bend left] (n1) to (n2);
      \draw[thick,->,bend left] (n2) to (n3);
      \draw[thick,->,loop left] (n0) to (n0);
      \draw[thick,->,bend left] (n1) to (n0);
      \draw[thick,->,bend left] (n2) to (n1);
      \draw[thick,->,bend left] (n3) to (n2);
    \end{tikzpicture}
  \end{center}
  where edges pointing right have probability $(1-\epsilon)/2$ and edges pointing left---including the loop at 0---have probability $(1+\epsilon)/2$. Examining the recursive description of $\rho(w)$, it is easy to confirm that the random variable $\rho(B_1, \ldots, B_n)$ is precisely given by the result of evolving the Markov chain above for $n$ steps with all probability initially placed at 0. It is further easy to confirm that the distribution given by~\eqref{eq:stationary} above is stationary for this chain.

  % The above Markov chain is \emph{irreducible} since every state is
  % reachable from every state.  In addition, it is easy to see that this
  % chain is \emph{aperiodic}---at time $n$, the walk visits every state
  % $s \in \{0, 1, \cdots, n\}$ with a strictly positive probability.
  % According to~\citep[Theorem 21.12]{LevinPeres}, an irreducible Markov
  % chain is positive recurrent if and only if there exists a distribution
  % $\pi$ on the state space satisfying $\pi P = \pi$, where $P$ is the
  % transition matrix of the chain.  We can take $\pi$ to be the
  % distribution given by~\eqref{eq:stationary}, implying that the chain
  % is positive recurrent.  Consequently, by ~\citep[Theorem
  % 21.14]{LevinPeres}, this irreducible, aperiodic, positive recurrent
  % Markov chain converges to the unique stationary distribution $\pi$.
  % It follows that the distributions $R_n$ limit to $R_\infty$.

  To establish stochastic dominance, it is convenient to work with the
  underlying distributions and consider walks of varying lengths: let
  $\DistRho_n: \Z \rightarrow \R$ denote the probability distribution given by
  $\rho(B_1, \ldots, B_n)$; likewise
  define $\DistRho_\infty$. For a distribution $\DistRho$ on $\Z$, we define $[\DistRho]_0$
  to denote the probability distribution obtained by shifting all
  probability mass on negative numbers to zero; that is, for $x \in \Z$,
  \[
    [\DistRho]_0(x) = \begin{cases} \DistRho(x) & \text{if $x > 0$},\\
      \sum_{t \leq 0} \DistRho(t) & \text{if $x = 0$},\\
      0 & \text{if $x < 0$.}
    \end{cases}
  \]
  We observe that if $A \dominatedby C$ then $[A]_0 \dominatedby [C]_0$ for any
  distributions $A$ and $C$ on $\Z$. It will also be convenient to
  introduce the shift operators: for a distribution
  $\DistRho: \Z \rightarrow \R$ and an integer $k$, we define $S^k\DistRho$ to be the
  distribution given by the rule $S^k\DistRho(x) = \DistRho(x-k)$. With these
  operators in place, we may write
  \[
    \DistRho_t = \left(\frac{1 - \epsilon}{2}\right) S^1 \DistRho_{t-1} +
      \left(\frac{1 + \epsilon}{2}\right) \left[S^{-1}\DistRho_{t-1} \right]_0\,,
  \]
  with the understanding that $\DistRho_0$ is the distribution placing unit probability at $0$. The proof now proceeds by induction. It is clear that $\DistRho_0 \dominatedby \DistRho_\infty$. Assuming that $\DistRho_n \dominatedby \DistRho_\infty$, we note that for any $k$
  \[
    S^k \DistRho_n \dominatedby S^k \DistRho_\infty \qquad \text{and, additionally, that
    }\qquad [S^{-1}\DistRho_n]_0 \dominatedby [S^{-1}\DistRho_{\infty}]_0\,.
  \]
  Finally, it is clear that stochastic dominance respects convex combinations, 
  in the sense that if $A_1 \dominatedby C_1$ and $A_2 \dominatedby C_2$ then 
  $\lambda A_1 + (1-\lambda) A_2 \dominatedby \lambda C_1 + (1-\lambda) C_2$ (for $0 \leq \lambda \leq 1$). We conclude that
  \[
    \DistRho_{t+1} = \left(\frac{1 - \epsilon}{2}\right) S^1 \DistRho_{t} +
      \left(\frac{1 + \epsilon}{2}\right) \left[S^{-1}\DistRho_{t} \right]_0 \dominatedby \left(\frac{1 - \epsilon}{2}\right) S^1 \DistRho_{\infty} +
      \left(\frac{1 + \epsilon}{2}\right) \left[S^{-1}\DistRho_{\infty} \right]_0 
      \,.
  \]
  By inspection, the right-hand side equals $\DistRho_{\infty}$, as desired. 
  Hence $\rho(B) \dominatedby R_\infty$.
  % $\qedhere$
\end{proof}

  \paragraph{Remark.} 
  In fact, the random variable $\rho(B)$
  actually converges to $R_\infty$ as $n \rightarrow \infty$. 
  This can be seen, for example, 
  by solving for the stationary distribution of the Markov chain in the proof above. 
  However, we will only require the dominance for our exposition. 
  Importantly, since $\mu_x(\varepsilon) = \rho(x)$, and 
  $\Pr[\mu_x(y) \geq 0]$ increases monotonically 
  with an increase in $\Pr[\mu_x(\varepsilon) \geq r]$ for any $r \geq 0$, 
  it suffices to take $|x| \rightarrow \infty$ 
  when reasoning about an upper bound on $\Pr[\mu_x(y) \geq 0]$.