%=======================================================
% \subsection{Proof of Bound~\ref{bound:geometric}}\label{sec:martingale-proof}
Below we present a bound (Bound~\ref{bound:geometric-original}) on the probability that 
a characteristic string satisfying the $\epsilon$-martingale condition has a non-negative relative margin. 
We remark that the bound below is weaker than Bound~\ref{bound:geometric}. 
Before we proceed, recall 
the following standard large deviation bound for supermartingales.
\begin{theorem}[Azuma's inequality (Azuma; Hoeffding). See {\cite[4.16]{Motwani:1995:RA:211390}} for a discussion]\label{thm:azuma}
  Let $X_0, \ldots, X_n$ be a sequence of real-valued random variables
  so that, for all $t$,
  $\Exp[X_{t+1} \mid X_0, \ldots, X_{t}] \leq X_t$ and
  $|X_{t+1} - X_t| \leq c$ for some constant $c$. Then  
  $
    \Pr[X_n - X_0 \geq \Lambda] \leq
    \exp\left(-{\Lambda^2}/{2nc^2}\right)
    %\,.
  $ 
  for every $\Lambda \geq 0$.
\end{theorem}

\begin{bound}\label{bound:geometric-original}
  Let $x \in \{0,1\}^m$ and $y \in \{0,1\}^k$ be random variables,
  satisfying the $\epsilon$-martingale condition (with respect to the ordering $x_1, \ldots, x_m, y_1, \ldots, y_k$). Then
  \[
    \Pr[\mu_x(y) \geq 0] \leq
    % \exp({-2\epsilon^4 (1 - O(\epsilon))n})
    3 \exp\left( -\epsilon^4 (1 - O(\epsilon) ) k/64 \right)  
    \, .
  \]
\end{bound}

%Now we are ready to prove Bound~\ref{bound:geometric-original}.

%\begin{proof}[of Bound~\ref{bound:geometric}]
\begin{proof}
  Let $w_1, w_2, \ldots$ be random variables obeying the
  $\epsilon$-martingale condition.  Specifically,
  $\Pr[w_t = 1 \mid E] \leq (1 - \epsilon)/2$ conditioned on any event
  $E$ expressed in the variables $w_1, \ldots, w_{t-1}$.  For
  convenience, define the associated $\{\pm1\}$-valued random
  variables $W_t = (-1)^{1+w_t}$ and observe that
  $\Exp[W_t] \leq -\epsilon$.

%\vspace{-2ex}
\Paragraph{If $x$ is empty.}
Observe that in this case, the relative margin $\mu_x(y)$ reduces to 
the non-relative margin $\mu(y)$ from Lemma~\ref{lem:margin}. 
Since the sequence $y_1, y_2, \ldots$ in the statement of the claim 
is identical to the sequence $w_1, w_2, \ldots$ defined above, 
we focus on the reach and margin of the latter sequence. 
Specifically, define $\rho_t = \rho(w_1 \ldots w_t)$ and
$\mu_t = \mu(w_1 \ldots w_t)$ to be the two random variables from
Lemma~\ref{lem:margin} acting on the string $w=w_1 \ldots w_t$. The
analysis will rely on the ancillary random variables
$\overline{\mu}_t = \min(0,\mu_t)$.  Observe that $\Pr[\text{$w$
  forkable}] = \Pr[\mu(w) \geq 0] = \Pr[\overline{\mu}_k = 0]$, so we
may focus on the event that $\overline{\mu}_k = 0$. As an additional
preparatory step, define the constant
$\alpha = (1+\epsilon)/(2\epsilon) \geq 1$ and define the random
variables $\Phi_t \in \mathbb{R}$ by the inner product
  \[
    \Phi_t = (\rho_t, \overline{\mu}_t) \cdot
    \left(\begin{array}{c} 1\\ \alpha\end{array}\right) = \rho_t +
    \alpha \overline{\mu}_t\,.
  \]
  The $\Phi_t$ will act as a ``potential function'' in the analysis:
  we will establish that $\Phi_k < 0$ with high probability and,
  considering that
  $\alpha\overline{\mu}_k \leq \rho_k + \alpha \overline{\mu}_k =
  \Phi_k$, this implies $\overline{\mu}_k < 0$, as desired.
  
  Let $\Delta_t = \Phi_t - \Phi_{t-1}$; we claim that---conditioned
  on any fixed value $(\rho, \mu)$ for $(\rho_t, \mu_t)$---the
  random variable $\Delta_{t+1} \in [-(1 + \alpha),1+ \alpha]$ has
  expectation no more than $-\epsilon$. The analysis has four cases,
  depending on the various regimes of $\rho$ and $\mu$ from Lemma~\ref{lem:margin}. 
  When $\rho > 0$ and $\mu < 0$,
  $\rho_{t+1} = \rho + W_{t+1}$ and
  $\overline{\mu}_{t+1} = \overline{\mu} + W_{t+1}$, where
  $\overline{\mu} = \max(0,\mu)$; then
  $\Delta_{t+1} = (1 + \alpha)W_{t+1}$ and
  $\Exp[\Delta_{t+1} ] \leq -(1 + \alpha)\epsilon \leq -\epsilon$. When
  $\rho > 0$ and $\mu \geq 0$, $\rho_{t+1} = \rho + W_{t+1}$
  but $\overline{\mu}_{t+1} = \overline{\mu}$ so that
  $\Delta_{t+1} = W_{t+1}$ and $\Exp[\Delta_{t+1} ] \leq -\epsilon$. Similarly, when
  $\rho = 0$ and $\mu < 0$,
  $\overline{\mu}_{t+1} = \overline{\mu} + W_{t+1}$ while
  $\rho_{t+1} = \rho + \max(0, W_{t+1})$; we may compute
  \[
    \Exp[\Delta_{t+1} ] \leq \frac{1 - \epsilon}{2}(1 + \alpha) - \frac{1 +
      \epsilon}{2}\alpha = \frac{1 - \epsilon}{2} - \epsilon\alpha =
    \frac{1 - \epsilon}{2} - \epsilon\left(\frac{1}{\epsilon} \cdot
      \frac{1 + \epsilon}{2}\right) = -\epsilon\,.
  \]
  Finally, when $\rho = \mu = 0$ exactly one of the two random
  variables $\rho_{t+1}$ and $\overline{\mu}_{t+1}$ differs from
  zero: if $W_{t+1} = 1$ then
  $(\rho_{t+1}, \overline{\mu}_{t+1}) = (1,0)$; likewise, if
  $W_{t+1} = -1$ then
  $(\rho_{t+1}, \overline{\mu}_{t+1}) = (0,-1)$. It follows that
  \[
    \Exp[\Delta_{t+1} ] \leq \frac{1 - \epsilon}{2} - \frac{1 +
      \epsilon}{2}\alpha \leq -\epsilon\,.
  \]

  \noindent
  Thus $
  \Exp[\Phi_k] = \Exp \sum_{t=1}^k \Delta_t  
  \leq -\epsilon k
  $. 
  We wish to apply Azuma's inequality to conclude that
  $\Pr[\Phi_k \geq 0]$ is exponentially small. For this purpose, we
  transform the random variables $\Phi_t$ to a related supermartingale by
  shifting them: specifically, define
  $\tilde{\Phi}_t = \Phi_t + \epsilon t$ and
  $\tilde{\Delta}_t = \Delta_t + \epsilon$ so that
  $\tilde{\Phi}_t = \sum_i^t \tilde{\Delta}_t$. Then
  \[
    \Exp[\tilde{\Phi}_{t+1} \mid \tilde{\Phi}_1, \ldots,
    \tilde{\Phi}_{t}] = \Exp[\tilde{\Phi}_{t+1} \mid W_1, \ldots,
    W_{t}]\leq \tilde{\Phi}_t\,,
    \qquad
    \tilde{\Delta}_t \in [-(1 + \alpha) + \epsilon, 1+ \alpha +
    \epsilon]\,,
  \]
  and $\tilde{\Phi}_k = \Phi_k + \epsilon k$. It follows
  from Azuma's inequality that
  \begin{align}\label{eq:azuma-bound}
    \Pr[\text{$w$ forkable}] 
    &= \Pr[\overline{\mu}_k = 0] \leq \Pr[\Phi_k \geq 0] = \Pr[\tilde{\Phi}_k \geq \epsilon k] 
    \nonumber \\ 
    &\leq \exp\left(-\frac{\epsilon^2 k^2}{2k (1 + \alpha + \epsilon)^2}\right)
       = \exp\left(-\left(\frac{2 \epsilon^2}{1 + 3 \epsilon + 2\epsilon^2}\right)^2 \cdot \frac{k}{2}\right) \nonumber \\
    &\leq \exp\left(-\frac{2\epsilon^4}{1 + 35\epsilon} \cdot k\right)
    \,.
    %                  \qedhere
  \end{align}


\newcommand{\muxr}{\mu_x^{(r)}}
\newcommand{\Snr}{S_k^{(r)}}
\newcommand{\Sr}{S^{(r)}}
\newcommand{\Srstar}{S^{(r^*)}}
\newcommand{\event}[1]{\mathsf{#1}}
\newcommand{\notevent}[1]{\overline{\event{#1}}}

%\vspace{-2ex}
\Paragraph{If $x$ is not empty.} 
In this case, we go back to study the sequences $x$ and $y$ as in the statement of the claim.
Recall the reach distribution (i.e., the distribution of the random variable $\rho(x)$) 
$\DistRho_m : \Z \rightarrow [0,1]$ from~\eqref{eq:dist-rho}. 
Since $x = (x_1, \ldots, x_m)$ satisfies the $\epsilon$-martingale condition, 
Lemma~\ref{lemma:rho-stationary} states that $\DistRho_m \dominatedby \StationaryRho$.
We reserve the symbol $\muxr$ for the relative margin 
random walk $\mu_x$ which starts at a non-negative initial position $r$. 
Thus $\rho(x) = \mu_x(\epsilon) = r$, and
\begin{align}\label{eq:azuma-generic}
\Pr[\mu_x(y) \geq 0] 
&= \sum_{r \geq 0}{\DistRho_m(r) \Pr[\muxr(y) \geq 0]} 
\leq \sum_{r \geq 0}{\StationaryRho(r) \Pr[\muxr(y) \geq 0]} 
\, 
\end{align}
since the sequence $( \, \Pr[\muxr(y) \geq 0] \, )_{r=0}^\infty$ is non-decreasing and $\DistRho_m \dominatedby \StationaryRho$. Fix a ``large enough'' positive integer $r^*$ whose value will be assigned later in the analysis. 
Let us define the following events:
 \begin{itemize}
  %\item Event $\event{A}_r$:~when $r > r^*$. 
  \item Event $\event{B}_r$:~it occurs when $r \in [0, r^*]$ and the $\muxr$ walk is strictly positive on every prefix of $y$ with length at most $k/2$; and 
  \item Event $\event{C}_{r,s}$:~it occurs when $r \in [0, r^*]$ and 
  $\hat{y}$ is the smallest prefix of $y$ of length $s \in [r, k/2]$ 
  such that $\muxr(\hat{y}) = 0$. 
  We say that $\hat{y}$ is a witnesses to the event $\event{C}_{r, s}$.
\end{itemize}
%Note that these two events cannot happen simultaneously. 
The right-hand side of~\eqref{eq:azuma-generic} can be written as
\begin{align*}
     &\quad \sum_{r>r^*}{\StationaryRho(r) \Pr[\muxr(y) \geq 0]} 
		+ \sum_{r \leq r^*}{\StationaryRho(r) \Pr[\event{B}_r] \cdot \Pr\left[\muxr(y) \geq 0 \mid \event{B}_r\right]} \\
    &\quad+ \sum_{r \leq r^*}{\StationaryRho(r) \sum_{s = r}^{k/2}{\Pr[\event{C}_{r,s}] \cdot \Pr[\muxr(y) \geq 0 \mid \event{C}_{r,s}]} }
    \, .
\end{align*} 
We observe that the probabilities $\Pr[\muxr(y) \geq 0]$ and $\Pr[\muxr(y) \geq 0 \mid \event{B}_r]$ are at most one. 
In addition, recall that for two non-negative sequences $(a_i), (b_i)$ of equal lengths, 
we have $\sum{a_i b_i} \leq \max b_i$ if $\sum{a_i} \leq 1$. 
Thus~\eqref{eq:azuma-generic} can be simplified as
\begin{align}\label{eq:three-terms}
\Pr[\mu_x(y) \geq 0] 
 &\leq 
    \sum_{r > r^*}{\StationaryRho(r)} 
  + \sum_{r \leq r^*}{\StationaryRho(r) \Pr[\event{B}_r]} \nonumber \\
  &\quad+ \sum_{r \leq r^*}{\StationaryRho(r)\, \max_{r \leq s \leq k/2}{\Pr[\muxr(y) \geq 0 \mid \event{C}_{r,s}]} }
  \nonumber \\
 &\leq    
      \sum_{r > r^*}{\StationaryRho(r)}            
  + 
      \max_{r \leq r^*}{\Pr[\event{B}_r]}          
  + 
      \max_{\substack{r \leq r^* \\ r \leq s \leq k/2}}{\Pr[\muxr(y) \geq 0 \mid \event{C}_{r,s}]}   
\, .
\end{align}


\emph{The first term in~\eqref{eq:three-terms} } is the right-tail of the distribution $\StationaryRho$. 
Using Lemma~\ref{lemma:rho-stationary}, 
this quantity is at most $\beta^{r^*}$ where $\beta := (1-\epsilon)/(1+\epsilon)$. 
Furthermore, it can be easily checked that the above quantity is at most $\exp(-5 \epsilon/3)$.

\emph{The second term in~\eqref{eq:three-terms} } concerns the event
$\event{B}_r$ and calls for more care.  Define
\[
  \Snr := \sum_{t=0}^k {W_t}
\]
where $W_0 = r$ and the random variables $W_t$ are defined at the
outset of this proof for $t \geq 1$.  We know that the $\muxr$ walk
starts with $\rho(x) = \mu(x) = r \geq 0$.  Since $\event{B}_r$ holds,
both the margin $\mu_x(\hat{y})$ and the reach $\rho(x\hat{y})$ remain
non-negative for all prefixes $\hat{y}$ of length
$t = 1, 2, \cdots, k/2$.  These two facts imply that the random
variable $\muxr(\hat{y})$ is identical to the sum $\Sr_t$ for all
prefixes $\hat{y}$ of length $t = 1, 2, \cdots, k/2$.

To be precise,
\[
  \Pr[\event{B}_r] = \Pr[\Sr_t \geq 0 \quad \text{for all } t \leq k/2]\,.
\]
The latter probability is at most $\Pr[\Sr_{k/2} \geq 0]$ because the
event $\Sr_{k/2} \geq 0$ does not constrain the intermediate sums
$\Sr_t$ for $t < k/2$.  Since $\Pr[\Sr_{k/2} \geq 0]$ increases
monotonically in $r$, we conclude that the second term
in~\eqref{eq:three-terms} is at most $\Pr[\Srstar_{k/2} \geq 0]$.  Now
we are free to shift our focus from the relative margin walk to the
sum of a martingale sequence.

For notational clarity, let us write $S := \Srstar_{k/2}$. 
Since the sequence $(w_t)$ obeys the $\epsilon$-martingale condition, 
$\Exp S$ is at most $M := r^* - k\epsilon/2$. 
Let us set $r^* = W_0 = k\epsilon/4$. Then $\Exp S$ is at most $-k\epsilon/4$ and Azuma's inequality gives us
\[
\Pr[S \geq 0] 
= \Pr[(S - \Exp S) \geq k\epsilon/4] 
\leq \exp\left( - \frac{(k\epsilon/4)^2}{2(k/2)\cdot 2^2}\right) 
= \exp\left( -\frac{k \epsilon^2}{64} \right)
\, .
\]
This is an upper bound on the second term in~\eqref{eq:three-terms}.

\emph{The third term in~\eqref{eq:three-terms}} concerns the event $\event{C}_{r,s}$ and it can be bounded using 
our existing analysis of the $|x|=0$ case. 
Specifically, suppose $y = \hat{y} z$ where
$\hat{y}$ is a witness to the event $\event{C}_{r,s}$. 
Since the $\muxr$ walk remains non-negative over the entire string $\hat{y}$, 
it follows that $\rho(x\hat{y}) = \mu(x\hat{y}) = 0$ 
and as a consequence, the $\mu_{x\hat{y}}$ walk on $z$ is identical to 
the $\mu$ walk on $z$. 
Our analysis in the $|x| = 0$ case suggests that 
$\Pr[\mu(z) \geq 0]$ is at most $A(k-s, \epsilon)$ 
where $|z| = k - s$ and $A(k, \epsilon)$ is the bound in~\eqref{eq:azuma-bound}. 
Since $A(\cdot,\epsilon)$ decreases monotonically in the first argument, 
$A(k-s, \epsilon)$ is at most $A(k/2, \epsilon)$. 
However, since the last quantity is independent of $r$, 
the third term in~\eqref{eq:three-terms} is at most 
$A(k/2, \epsilon) = \exp\left( -k \epsilon^4/(1+35\epsilon) \right)$. 


Returning to~\eqref{eq:three-terms} and using $r^* = k\epsilon/4$, we get
\begin{align*}
\Pr[\mu_x(y) \geq 0] 
 &\leq    \exp\left(-\frac{5 \epsilon}{3} \cdot \frac{k\epsilon}{4} \right)  
        + \exp\left(-\frac{2\epsilon^4}{1 + 35\epsilon} \cdot \frac{n}{2}\right)
        + \exp\left( -\frac{k \epsilon^2}{64} \right)
\, .
\end{align*}
It is easy to check that the above quantity is at most
$$
  3 \exp\left( - k \epsilon^4/(64 + 35 \epsilon) \right) 
= 3 \exp\left( - \epsilon^4 (1 - O(\epsilon) ) k/64 \right)
\,.
$$

% $\qed$
\end{proof}
