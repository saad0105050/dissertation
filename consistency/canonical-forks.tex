
% \newcommand{\EarliestDiverging}{\mathrm{EarliestDivergingTine}}
% \newcommand{\Disjoint}{\mathrm{DisjointWith}}
% \newcommand{\MaxReachTines}{\mathrm{MaxReachTines}}
% \newcommand{\Tines}{\mathrm{Tines}}





Let $w$ be a characteristic string, written $w = xy$, 
and recall the online fork-building strategy from Section~\ref{sec:strategy-x}. 
In Proposition~\ref{prop:muxy0-lowerbound-adv}, 
we showed that the fork produced by this strategy (for the string $w$) 
always contains a tine-pair $(t_\rho, t_x)$ that witnesses $\mu_x(y)$. 
In this section, we present an online fork-building strategy 
which produces a fork that \emph{simultaneously} contains, 
for every prefix $x \PrefixEq w$, 
a tine-pair that witnesses $\mu_x(y)$. 
These forks are called \emph{canonical forks}, defined below.
\begin{definition}[Canonical forks]
	Let $w_1 \ldots w_T \in \{0,1\}^T$. 
	For $n = 0, 1, \ldots, T$, a \emph{canonical fork $F_n$ for $w = w_1\ldots w_n$} 
	is inductively defined as follows. 
	If $n = 0$ then $F_0$ is the trivial fork for the empty string; 
	it consists of a single (honest) vertex and no edge. 
	If $n \geq 1$, the following holds: 
	$F_n$ is a closed fork so that $F_{n-1} \ForkPrefix F_n$. 
	$F_n$ contains an honest tine $\tau_\rho$ so that 
	$\reach(\tau_\rho) = \rho(F_n) = \rho(w)$. 
	For every decomposition $w = xy, x \Prefix w$, 
	% $\mu_x(F_n) = \mu_x(y)$ and, in addition, 
	$F_n$ contains an honest tine $\tau_x$ 
	so that the tine-pair $(\tau_\rho, \tau_x)$ witnesses $\mu_x(F_n) = \mu_x(y)$. 
	% It also contains a tine $\tau_w$ so that $\reach(\tau_w) = \mu_w(\varepsilon)$. 
	% The sequence $(F_n), n = 0, 1, \ldots, T$ is called a \emph{canonical sequence of forks for $w$}. 
	The (possibly non-distinct) designated tines $\tau_\rho$ and $\tau_x, x \Prefix w$ 
	are called the \emph{witness tines}.
\end{definition}
Note that if one's objective is to create a fork 
which contains many early-diverging tine-pairs witnessing large relative margins, 
a canonical fork is the best one can hope for. 






\section{An online strategy for building canonical forks}
	Let $w$ be a characteristic string, 
	written as $w = xy$, and 
	let $F$ be a fork for $w$. 
	If the tines $t_1, t_2 \in F$ are disjoint over $y$, 
	we say \emph{$t_1$ and $t_2$ are $y$-disjoint}, or equivalently, 
	\emph{$t_1$ is $y$-disjoint with $t_2$}. 
	Note that this means $\ell(t_1 \Intersect t_2) \leq |x|$. 
	Given two sets of tines $A, B$ in the same fork, 
	we say that a tine $t \in A$ \emph{diverges earliest with respect to $B$} if 
	$t = \arg \min_{t_a \in A}\left\{ \min_{t_b \in B} \ell(t_a \Intersect t_b) \right\}$.	
	Let $\leq_\pi$ be the lexicographical ordering of the tines where 
	each tine is represented as the list of vertex labels appearing in the tine's root-to-leaf path.
	If two tines have the same vertex labels, 
	we allow $\leq_\pi$ to break tie in an arbitrary but consistent way. 

	The fork-building strategy $\Adversary^*$ presented in Figure~\ref{fig:adv-opt} 
	builds canonical forks in an online fashion, i.e., 
	it scans the characteristic string $w$ once, from left to right, 
	maintains a ``current fork,'' 
	and updates it after seeing each new symbol by only adding new vertices. 
	Since the final fork $F \Fork w$ is canonical, 
	it satisfies $\mu_x(F) = \mu_x(y)$ simultaeneously for all decompositions $w = xy$; 
	hence we call $\Adversary^*$ the \emph{optimal online adversary}. 
	% In connection with Figure~\ref{fig:adv-opt}, note that 
	% $
	% % \reach_{F^\prime}(\tau_{w^\prime}) = 
	% \reach_{F^\prime}(\tau_\rho) = \rho(F^\prime)$ 
	% and $\reach_{F^\prime}(\tau_w) = \reach_{F^\prime}(t_\rho)$.





	% Define $\Tines(F)$ as the set of all tines in $F$ and 
	% \[
	% 	\MaxReachTines_F(A) = \{t \in A \SuchThat \text{$\reach_F(t)$ is maximum over all $t \in A$}\}
	% 	\,.
	% \]
	% % When the fork $F$ is understood from the context, we omit the subscript $F$. 
	% We write $\MaxReachTines(F)$ to denote the set of tines with the largest reach in fork $F$. 

	% For $A, B \subseteq \Tines(F)$, define
	% \[
	% 	\EarliestDiverging_B(A) = \min_{\leq_\pi} \{a^* \SuchThat \text{$a^* \in A$ minimizes $\ell(a \Intersect b), a \in A, b \in B$}\}
	% 	\,.
	% \]
	% If $B$ has a single element $b$, we write $\EarliestDiverging_b(A)$.
	% % For any set $B \subseteq \Tines(F)$, define 
	% % \[
	% % 	\Disjoint_F(B,y) = \{ t \in F \SuchThat \text{$t$ is $y$-disjoint with every tine $b \in B$ }\}
	% % 	\,.
	% % \]
	% % When $B$ has a single element $b$, we overload the above notation and write $\Disjoint_F(b, y)$. 
	% Finally, for any tine $\tau \in F$, define 
	% \[
	% 	\Disjoint_F(\tau,y) = \{ t \in F \SuchThat \text{$t$ is $y$-disjoint with $\tau$ }\}
	% 	\,.
	% \]
	% % When $B$ has a single element $b$, we overload the above notation and write $\Disjoint_F(b, y)$. 


\begin{figure}[!h]
	\begin{center}
	  \fbox{
	    \begin{minipage}{.9 \textwidth}
	      \begin{center}
	        \textbf{The strategy $\Adversary^*$}
	      \end{center}
				Let $w = w_1 \ldots w_n \in \{0,1\}^n$ and $w_{n + 1} \in \{0, 1\}$. 
				If $n = 0$, set $F_0 \Fork \varepsilon$ as 
				the trivial fork comprising a single vertex. 
				Otherwise, for $n \geq 0$, 
				let $F_n$ be the closed fork 
				built recursively by $\Adversary^*$ for the string $w$. 
				If $w_{n + 1} = 1$, set $F_{n + 1} = F_n$.
				Otherwise, 
				the closed fork $F_{n + 1} \Fork w0$ 
				is the result of a 
				single conservative extension 
				of a tine $s \in F_n$ into a new honest tine 
				$\sigma \in F_{n+1}, \ell(\sigma) = n + 1$; 
				The tine $s$ can be identified as follows. 
				If $F_n$ contains no tine with reach zero, 
				$s$ is the unique longest tine in $F_n$. 
				Otherwise, 
				$s$ is the reach-zero tine 
				that diverges earliest with respect to 
				the set of maximal-reach tines in $F_n$. 
				If there are multiple candidates for $s$, 
				select the one with the smallest $\leq_\pi$-rank.
	      \begin{center}
	        \textbf{Designating the witness tines}
	      \end{center}
		% \paragraph{Designating the witness tines.}
			Writing $w^\prime = w w_{n + 1}, F = F_n$, and $F^\prime = F_{n + 1}$, 
			identify the tines 
			$
			\tau_\rho, \tau_w, 
			% , \tau_{w^\prime} 
			\tau_x \in F^\prime, x \Prefix w
			$ as follows. 
			% If there are multiple candidates for any designated tine then 
			% select the one with the smallest $\leq_\pi$-rank. 
			Let $R^\prime$ be the set of $F'$-tines with the maximal $F'$-reach. 
			Set $\tau_\rho$ as the element of $R'$ with smallest $\leq_\pi$-rank. 
			% Set $\tau_{w'}$ as the element of $R'$ that diverges earliest from $\tau_\rho$; 
			% if there are multiple candidates, select the one with the smallest $\leq_\pi$-rank.
			Let $A$ be the set of $F$-tines that attain the reach $\max_{t \in F} \reach_{F'}(t)$. 
			Set $\tau_{w}$ as the element in $A$ that diverges earliest with respect to $R'$; 
			if there are multiple candidates, select the one with the smallest $\leq_\pi$-rank.
			For every decomposition $w = xy, |y| \geq 1, |x| \geq 0$, do as follows. 
			Let $B_x$ be the set of $F'$-tines that are $yw_{n+1}$-disjoint 
			with \emph{some} maximal-reach tine $r' \in R'$. 
			Let $C_x$ be the set of tines in $B_x$ that attain the reach $\max_{t \in B_x} \reach_{F'}(t)$. 
			Set $\tau_{x}$ as the element in $C_x$ that diverges earliest with respect to $R'$; 
			if there are multiple candidates, select the one with the smallest $\leq_\pi$-rank. 

			% \begin{align}\label{eq:canonical-tines}
			% 	\begin{cases}
			% 	&\tau_\rho = \min_{\leq_\pi} \, R^\prime \,, \\
			% 	&\tau_x = \EarliestDiverging_{R^\prime}\bigl(\MaxReachTines_{F^\prime}(\Union_{r^\prime \in R^\prime}\Disjoint_{F^\prime}(r^\prime, y w_{n+1}))\bigr) \,,\\ 
			% 	&\tau_w = \EarliestDiverging_{\tau_\rho}\bigl( \MaxReachTines_{F^\prime}(\Tines(F)) \bigr) \,, \\
			% 	&\tau_{w^\prime} = \EarliestDiverging_{\tau_\rho}\bigl( R^\prime \bigr) 
			% 	\,.			
			% 	\end{cases}
			% \end{align}
			
			% \begin{enumerate}
			% 	\item Set $\tau_\rho$ as the maximal-reach $F'$-tine. 
			% 	Set $\tau_{w'}$ as the maximal-reach $F'$-tine that diverges earliest from $\tau_\rho$.
				
			% 	\item Let $M_w$ be the set of $F$-tines that attain the reach $\max_{t \in F} \reach_{F'}(t)$. 
			% 	Set $\tau_{w}$ as the element in $M_w$ that diverges earliest from $\tau_\rho$.

			% 	\item For every decomposition $w = xy, |y| \geq 1, |x| \geq 0$, do as follows. 
			% 	Let $M'_x$ be the set of $F'$-tines that are $yw_{n+1}$-disjoint with \emph{some} maximal-reach tine $r' \in R'$. 
			% 	Let $M_x$ be the tines in $M'_x$ that attain the reach $\max_{t \in M'_x} \reach_{F'}(t)$. 
			% 	Set $\tau_{x}$ as the element in $M_x$ that diverges earliest with respect to $R'$.

			% \end{enumerate}

	    \end{minipage}
	  }
	\end{center}
	\caption{Optimal online adversary $\Adversary^*$}
	\label{fig:adv-opt}
 \end{figure}
	% \paragraph{The strategy $\Adversary^*$.}

	


	\begin{theorem}[$\Adversary^*$ builds canonical forks]\label{thm:canonical-fork}
		Let $w \in \{0,1\}^n$ and $b \in \{0,1\}$. 
		Let $F \Fork w$ and $F^\prime \Fork wb$ be two closed forks 
		built by the strategy $\Adversary^*$ 
		so that $F \ForkPrefix F^\prime$ and suppose, in addition, 
		that $F$ is canonical. 
		Then $F^\prime$ is canonical as well.
	\end{theorem}
	We remark that the fork-building strategy $\Adversary^*$ 
	would certainly satisfy Proposition~\ref{prop:muxy0-lowerbound-adv} and, therefore, 
	satisfy the recurrence relation~\eqref{eq:mu-relative-recursive} as well.

\section{Winning the \texorpdfstring{$(\Distribution,T;s,k)$-}{}settlement game, optimally}\label{sec:adv-winning-settlement-game} 
	Consider the player in the $(\Distribution,T;s,k)$-settlement game 
	who, at the first step, samples 
	a characteristic string $w \sim \Distribution, w = w_1 w_2 \ldots w_T$. 
	Since the challenger is deterministic, 
	the game is completely determined by the characteristic string 
	and the choices of the player. 
	In particular, for a given prefix $x \Prefix w, |x| = s - 1$, 
	consider the decompositions $w = xyz$. 
	The player's chance of winning the game will be maximized if, 
	for every $y, |y| \geq k + 1$ (so that $n = |xy|\geq s + k$),
	the fork $F_n \Fork xy$ 
	contains a tine-pair $(\tau_\rho, \tau_x)$ that witnesses $\mu_x(y)$. 
	In fact, 
	if $\mu_x(y) \geq 0$ for some $y$ then, 
	as shown in Fact~\ref{fact:margin-balance}, 
	the player wins the game by
	augmenting $F_n$ to an $x$-balanced fork $A_n \Fork xy$. 
	% In particular, he can do so 
	% by adding only $\gap(\tau_n)$ adversarial vertices to the tine $\tau_n \in F_n$. 
	% See the discussion in Section~\ref{sec:args-survey} and 
	% Observation~\ref{obs:settlement-balanced-fork} 
	% for the relationship between a balanced fork and the settlement violation.

	Note, in addition, that if $F_n$ is canonical, 
	the player can optimally play $(\Distribution, T; s, k)$-settlement games 
	\emph{simultaneously} for every $s \in [n - k]$. 
	That is, given a distribution $\Distribution$, 
	a canonical fork $F_n$ gives the player 
	the largest probability 
	of causing a settlement violation at as many slots $s \in [n - k]$ as possible, 
	at once.
	% Therefore, we shift our attention to characterizing an online strategy to build canonical forks.

\section{Proof of Theorem~\ref{thm:canonical-fork}}
	By assumption, $F$ is a canonical fork. 
	Thus $\reach_F(t_\rho) = \rho(w)$ 
	% $\reach_F(t_w) = \mu_w(\varepsilon) = \rho(w)$ 
	and 
	for every prefix $x \Prefix w$, 
	$\reach_F(t_x) = \mu_x(y)$. 
	Let $w' = wb$ and let $\tau_\rho, \tau_w, 
	% \tau_{w^\prime}, 
	\tau_x \in F^\prime, x \Prefix w$ 
	be the purported witness tines in $F'$. 
	Note that $\tau_x$ must be 
	$yb$-disjoint with $\tau_\rho$ by construction. 
	We wish to show that 
	$\reach_{F^\prime}(\tau_\rho) = \rho(w b)$, 
	$\reach_{F^\prime}(\tau_w) = \mu_{w}(b)$,  
	% $\reach_{F^\prime}(\tau_{w^\prime}) = \mu_{w^\prime}(\varepsilon)$, 
	and 
	$\reach_{F^\prime}(\tau_x) = \mu_x(y b)$ for $x \Prefix w$.	
	% It would be helpful to note that in any fork, 
	% the reach of a tine 
	% is no more than the reach of the last honest vertex on that tine. 

	\begin{description}[font=\normalfont\itshape\space]
		\item[If $b = 1$.]
			In this case, $F^\prime = F$ and $w^\prime = w 1$. 
			Examining the rule for assigning $\tau_\rho, \tau_x$, and $\tau_w$, 
			we see that 
			$\tau_\rho = 
			% \tau_{w^\prime} 
			= t_\rho$, 
			$\tau_w = t_\rho$, and 
			$\tau_x = t_x$ for all decompositions $w = xy, x \Prefix w$. 
			Since $F^\prime = F$ and $b = 1$, 
			the $F^\prime$-reach of every $F$-tine is one plus its $F$-reach. 
			Thus for any $x, x \Prefix w$, writing $w^\prime = xy1$, we have 
			$\mu_x(y1) 
			= 1 + \mu_x(y) 
			= 1 + \reach_F(t_x) 
			= \reach_{F^\prime}(t_x)
			= \reach_{F^\prime}(\tau_x)$.
			Similarly, 
			$\rho(w1) 
			= 1 + \rho(w) 
			% = \rho(w^\prime)
			= \reach_{F^\prime}(t_\rho)
			= \reach_{F^\prime}(\tau_\rho)
			$.
			By construction, 
			$\tau_w$ has the largest reach in $F$; 
			but this means $\reach_{F'}(\tau_w) = \rho(F') = \rho(w1)$ but, 
			on the other hand, $\mu_w(1) = 1 + \mu_w(\varepsilon) = 1 + \rho(w) = \rho(w1)$; 
			hence $\reach_{F^\prime}(\tau_w) = \mu_w(1)$. 
			% Finally, by examining the rule for assigning $\tau_{w'}$, 
			% $\reach_{F^\prime}(\tau_{w^\prime}) = \rho(F^\prime) = \rho(w^\prime) = \mu_{w^\prime}(\varepsilon)$. 

		\item[If $b = 0$.]
			The contingencies of this case 
			is covered by Propositions~\ref{prop:optadv-tau-rho},~\ref{prop:optadv-tau-mu-x}, and~\ref{prop:optadv-tau-mu-w}.

	\end{description}

	% To complete the proof, we only need to state and prove 
	% Propositions~\ref{prop:optadv-tau-rho},~\ref{prop:optadv-tau-mu-x}, and~\ref{prop:optadv-tau-mu-w}.
	For convenience, let us record the following fact 
	which compacts Claims~\ref{claim:ex} and~\ref{claim:nex}.

	\begin{fact}\label{fact:reach-fork-ext}
		Let $F \Fork w$ and $F^\prime \Fork w0$ be closed forks so that 
		$F \ForkPrefix F^\prime$ and 
		$F^\prime$ differs from $F$ by a single conservative extension 
		$\sigma \in F^\prime, \ell(\sigma) = |w| + 1$.
		Then $\reach_{F^\prime}(t) = \reach_F(t) - 1$ for every $t \in F$ and,  
		in addition, $\reach_{F^\prime}(\sigma) = 0$.
	\end{fact}

	In the rest of the exposition, we will frequently use 
	the above fact along with Lemma~\ref{lem:margin} and Lemma~\ref{lem:relative-margin}, 
	often without an explicit reference.

	% \begin{lemma}[Recursive description of relative margin]\label{lem:relative-margin}
	% \end{lemma}

	\begin{proposition}\label{prop:optadv-tau-rho}
		Assume the premise of Theorem~\ref{thm:canonical-fork} with $b = 0$.
		Then $F^\prime$ contains witness tines $\tau_\rho, \tau_{w0}$ 
		so that 
		$\reach_{F^\prime}(\tau_\rho) = \rho(w0)$ and 
		$\reach_{F^\prime}(\tau_{w0}) = \mu_{w 0}(\varepsilon)$. 
	\end{proposition}
	\begin{proof}~
		Recall that $\sigma \in F^\prime, \ell(\sigma) = |w| + 1$ 
		is a conservative extension to 
		a tine $s \in F, \reach_F(s) = 0$ 
		so that $\reach_{F^\prime}(\sigma) = 0$. 
		Also recall that $\mu_z(\varepsilon) = \rho(z)$ for any characteristic string $z$. 
		Finally, note that it suffices to show that 
		$\reach_{F^\prime}(\tau_\rho) \geq \rho(w0)$ and 
		$\reach_{F^\prime}(\tau_{w0}) \geq \mu_{w 0}(\varepsilon)$. 

		% Consider the following contingencies based on $\rho(w)$.

		% \begin{description}[font=\normalfont\itshape\space]
			% \item[If $\rho(w) > 0$.]
				Suppose $\rho(w) > 0$. 
				Using Fact~\ref{fact:reach-fork-ext}, Lemma~\ref{lem:relative-margin}, 
				and examining the rule for assigning $\tau_\rho$
				$\reach_{F^\prime}(\tau_\rho) 
				\geq \reach_{F^\prime}(t_\rho) 
				= \reach_F(t_\rho) - 1 
				= \rho(w) - 1 
				 = \rho(w0)
				$. 
			% \item[If $\rho(w) = 0$.]
				Otherwise, suppose $\rho(w) = 0$.
				We know that $\rho(w0)$ is zero as well. 
				By construction, 
				$\reach_{F^\prime}(\tau_\rho) 
				\geq \reach_{F^\prime}(\sigma) 
				= 0 = \rho(w0)
				$. 
		% \end{description}

		Examining the rule for assigning $\tau_{w0}$, we have 
		$\reach_{F^\prime}(\tau_{w0}) = \rho(F^\prime) 
		= \rho(w0) = \mu_{w0}(\varepsilon)$. 	
	\end{proof}



	\begin{proposition}\label{prop:optadv-tau-mu-w}
		Assume the premise of Theorem~\ref{thm:canonical-fork} with $b = 0$.
		Then $F^\prime$ contains a witness tine $\tau_w$ 
		so that 
		$\reach_{F^\prime}(\tau_w) = \mu_w(0)$. 
	\end{proposition}
	\begin{proof}
		Recall that $\sigma \in F^\prime, \ell(\sigma) = |w| + 1$ 
		is a conservative extension to 
		a tine $s \in F, \reach_F(s) = 0$ 
		so that $\reach_{F^\prime}(\sigma) = 0$. 
		Consider the following contingencies based on $\rho(w)$.

		\begin{description}[font=\normalfont\itshape\space]
			\item[If $\rho(w) > 0$.]
				Thus $\mu_w(0) = \mu_w(\varepsilon) - 1 = \rho(w) - 1 = \rho(w0)$.
				There are two mutually exclusive scenarios 
				based on $\tau_\rho$ and $\sigma$.
				If $\tau_\rho = \sigma$ then, by construction, 
				$\tau_w \neq \sigma$ and, in addition, 
				$\tau_w$ is the $F$-tine with the largest $F^\prime$-reach; 
				by Fact~\ref{fact:reach-fork-ext}, 
				$\tau_w$ must have the largest $F$-reach as well, i.e.,	
				$\reach_{F}(\tau_w) = \rho(w)$. 
				This implies 
				$\reach_{F^\prime}(\tau_w) = \reach_{F}(\tau_w) - 1 = \rho(w) - 1 = \mu_w(0)$. 
				On the other hand, if $\tau_\rho \neq \sigma$ then $\tau_\rho \in F$ and, 
				Examining the rule for assigning $\tau_w$, 
				$\reach_{F^\prime}(\tau_w) = \reach_{F^\prime}(\tau_\rho) = \rho(w0) = \mu_w(0)$. 

			\item[If $\rho(w) = 0$.]
				Since $\rho(F) = \rho(w) = 0$, 
				Fact~\ref{fact:reach-fork-ext} tells us that 
				every $F$-tine must have a negative reach in $F^\prime$. 
				Since $\rho(F^\prime)$ is non-negative, 
				it must be the case that $\tau_\rho = \sigma$. 
				We can reuse the argument from the subcase ``$\tau_\rho = \sigma$'' of the preceding case and conclude that $\reach_{F^\prime}(\tau_w) = \mu_w(0)$.
		\end{description}
	\end{proof}



	\begin{proposition}\label{prop:optadv-tau-mu-x}
		Assume the premise of Theorem~\ref{thm:canonical-fork} with $b = 0$.
		Let $x \Prefix w$ and write $w = xy$.
		Then $F^\prime$ contains a witness tine $\tau_x$ 
		so that 
		$\reach_{F^\prime}(\tau_x) = \mu_x(y0)$. 
	\end{proposition}
	\begin{proof}~
		Note that it suffices to show that 
		$\reach_{F^\prime}(\tau_x) \geq \mu_x(y0)$. 
		Let $R$ be the set of $F$-tines with the maximal $F$-reach and 
		let $R^\prime$ be the set of $F'$-tine with the maximal $F'$-reach. 
		We know that $t_x$ is $y$-disjoint with $t_\rho$ in $F$. 
		Consider the following mutually exclusive cases.

		\begin{description}[font=\normalfont\itshape\space]
			\item[If $\rho(w) > 0$ and $\mu_x(y) = 0$.]
				We know that $s \neq t_\rho$, and 
				that $\mu_x(y0) = 0$ using Lemma~\ref{lem:relative-margin}. 
				By our choice of $s$, 
				$\ell(s \Intersect t_\rho) \leq \ell(t_x \Intersect t_\rho)$ 
				since $\reach_F(t_x) = \mu_x(y) = 0$.
				Since $t_x$ is $y$-disjoint with $t_\rho$, so is $s$. 			
				Recall that $\reach_{F^\prime}(\tau_x)$ is the largest among all tines 
				that are $y0$-disjoint with $\tau_\rho$. 

				\begin{description}[font=\normalfont\itshape\space]
					\item[If $\tau_\rho = t_\rho$.]
						Thus $t_x$ is $y0$-disjoint with $\tau_\rho$. 
						It follows that $\reach_{F^\prime}(\tau_x) \geq \reach_{F^\prime}(\sigma) = 0$ 
						(since $\sigma$ must be $y0$-disjoint with $\tau_\rho = t_\rho$). 
						Since $0 = \mu_x(y0)$ by~\ref{eq:mu-relative-recursive}, 
						$\reach_{F^\prime}(\tau_x) \geq \mu_x(y0)$, as desired.
						% Now, either $\ell(\tau_x) = n + 1$ or $\ell(\tau_x) \leq n$. 
						% In the former case, $\sigma = \tau_x$ 
						% and hence $\reach_{F^\prime}(\tau_x) = 0$. 
						% In the latter case, 
						% we have $\reach_{F^\prime}(\tau_x) \leq \reach_{F}(\tau_x) \leq \mu_x(y) = 0$. 
						% Here, the first inequality follows from Fact~\ref{fact:reach-fork-ext} 
						% and the second inquality follows since $\tau_x$ is $y$-disjoint with $\tau_\rho = t_\rho$. 
						% Hence $\reach_{F^\prime}(\tau_x) = 0 = \mu_x(y0)$.

					\item[If $\tau_\rho \neq t_\rho$.]
						This is possible if $\rho(w) = 1, \rho(w0) = 0$, and $t_\rho, \sigma \in R^\prime$. 
						Note that $|R^\prime| \geq 2$ (since $\sigma, t_\rho \in R^\prime$). 
						If there are two $y0$-disjoint tines $r_1^\prime, r_2^\prime \in R^\prime$ 
						then $\reach_{F^\prime}(\tau_x) \geq 0 = \mu_x(y0)$. 
						% However, since $\rho(w0) = 0$, we actually have 
						% $\reach_{F^\prime}(\tau_x) = 0 = \mu_x(y0)$.
						Otherwise, all tines $r^\prime \in R^\prime$ share a vertex indexed by $y$. 
						Since $t_x$ is $y$-disjoint with $t_\rho$, 
						it must be $y$-disjoint (and thus $y0$-disjoint) with 
						every $r^\prime \in R^\prime$ as well. 
						Examining the rule for assigning $\tau_x$, we conclude that $\tau_x = t_x$ and, 
						therefore,  
						$\reach_{F^\prime}(\tau_x) = \reach_{F^\prime}(t_x) = \mu_x(y) = 0 = \mu_x(y0)$. 
			\end{description}

		
			\item[If $\rho(w) = 0$.]
				Let $x \Prefix w$ and note that 
				$\mu_x(y0) = \mu_x(y) - 1$. 
				In addition, by Fact~\ref{fact:reach-fork-ext}, 
				$\sigma$ will be the unique tine in $F^\prime$ 
				with the maximal reach $\rho(F^\prime) = \rho(w0) = 0$, 
				i.e., $\tau_\rho = \sigma$.
				By construction, $\tau_x$ has the largest reach 
				among all $F^\prime$-tines that are $y0$-disjoint with $\tau_\rho$. 
				If $s = t_\rho$ 
				then all $F$-tines will have a negative reach in $F^\prime$. 
				In this case, we have $\tau_\rho = \sigma$ and, consequently, 
				$\tau_x = t_x$. 
				Hence $\reach_{F^\prime}(\tau_x) = \reach_{F^\prime}(t_x) 
				= \reach_{F}(t_x) - 1 = \mu_x(y) - 1 = \mu_x(y0)$. 
				
				Now suppose $s \neq t_\rho$ and notice that 
				$s$ cannot be a prefix of $t_\rho$ since, in that case, $\reach_F(s) < \reach_F(t_\rho) = 0$. 
				Hence $s$ must have diverged from $t_\rho$ at some slot $d = \ell(s \Intersect t_\rho)$.
				
				\begin{description}[font=\normalfont\itshape\space]
					\item[If $d > |x|$.]
						Then $t_x$ and $s$ will be $y$-disjoint in $F$ 
						i.e., $t_x$ will be $y0$-disjoint in $F^\prime$ 
						with the unique maximal-reach tine $\sigma$.
						By construction, 
						$\reach_{F^\prime}(\tau_x) \geq \reach_{F^\prime}(t_x) = \mu_x(y) - 1 = \mu_x(y0)$.

					\item[If $d \leq |x|$.]
						Now $s$ has the largest reach (zero) in $F$ and, 
						in particular, it is a uniquely identifiable tine that diverges earliest 
						with respect to $R$; 
						we conclude that $s = t_x$, i.e., $\mu_x(y) = \reach_F(s) = 0$. 					
						This means the tine $t_\rho$ (which must be in $R$) 
						is $y0$-disjoint with $\tau_\rho = \sigma$ in $F^\prime$ 
						and, in addition, it has the largest $F^\prime$-reach of all tines $t \in F^\prime, t \neq \sigma = \tau_\rho$. 
						Examining the rule for assigning $\tau_x$, we conclude that 
						$\reach_{F^\prime}(\tau_x) = \reach_{F^\prime}(t_\rho) = \reach_F(t_\rho) - 1 = -1$. 
						However, by Lemma~\ref{lem:relative-margin}, we also have 
						$\mu_x(y0) = \mu_x(y) - 1 = -1 = \reach_{F^\prime}(\tau_x)$ 
						since $\mu_x(y) = 0$.

				\end{description}

			\item[If $\rho(w) > 0$ and $\mu_x(y) \neq 0$.]
				We know that $\reach_{F^\prime}(\tau_x) \leq \mu_x(y0) = \mu_x(y) - 1$. 
				We wish to show that $\reach_{F^\prime}(\tau_x) \geq \mu_x(y0)$. 
				% \textbf{What if $t_x = t_\rho$?}

				\begin{description}[font=\normalfont\itshape\space]
					\item[If $\reach_F(s) = 0$.]
						Then $s \not \in \{t_\rho, t_x\}$  
						and 
						$\reach_{F^\prime}(t_\rho) = \reach_F(t_\rho) - 1 = \rho(w) - 1 = \rho(w0)$. 
						It follwos that $t_\rho \in R^\prime$. 
						Note that $t_x$ is $y0$-disjoint with $t_\rho$ in $F^\prime$ and, in addition, that 
						$\tau_x$ has the largest reach among all tines 
						that are $y0$-disjoint with some tine $r^\prime \in R^\prime$. 
						Therefore, 
						$\reach_{F^\prime}(\tau_x) \geq \reach_{F^\prime}(t_x) = \reach_{F}(t_x) - 1 = \mu_x(y) - 1 = \mu_x(y0)$. 

					\item[If $\reach_F(s) \neq 0$.]	
					% \item[If $\reach_F(s) \neq 0$ and $s = t_\mu_x$.]	
					% 	Thus $F$ contains no tine with reach zero and $s$ is the longest tine in $F$. 			
					% 	Suppose $s = t_x$. 
					% 	Then 
					% 	$\sigma$ will be $y0$-disjoint with $t_\rho$ in $F^\prime$ 
					% 	and, as before, 
					% 	$\reach_{F^\prime}(t_\rho) = \rho(w0)$. 
					% 	Since $\sigma$ and $t_\rho$ are $y0$-disjoint, 
					% 	a similar argument as in the preceding case would complete the claim.

					% \item[If $\reach_F(s) \neq 0, s = t_\rho \neq t_x$.]	
					% \item[If $\reach_F(s) \neq 0, s = t_\rho \neq t_x$, and $|R| \geq 2$.]	
					% 	There must be a tine $t^* \in R$ with the largest $F$-reach 
					% 	which is $y$-disjoint with $t_x$. 
					% 	Retracing our argument above, the claim will follow since 
					% 	these tines are $y0$-disjoint in $F^\prime$ as well. 

					% \item[If $\reach_F(s) \neq 0, s = t_\rho \neq t_x$, and $R = \{t_\rho\}$.]	
					% 	% If $\rho(w) = 1$ then $t_\rho \in R^\prime$ and, as a result, 
					% 	% $t_x$ will be $y0$-disjoint with $t_\rho$ in $F^\prime$ and the claim would follow. 
					% 	% Otherwise, suppose $\rho(w) \geq 2$. 
						Then $s$ must be the longest tine in $F$.
						Considering fork $F^\prime$,
						if some tine $r^\prime \in R^\prime$ is $y0$-disjoint with $t_\mu$ 
						then, as in the preceding case, 
						$\reach_{F^\prime}(\tau_x) \geq \mu_x(y0)$. 
						Otherwise, $\ell(r^\prime \Intersect t_x) > |x|$ for every tine $r^\prime \in R^\prime$, 
						i.e., no maximal-reach $F^\prime$-tine is $y0$-disjoint with $t_x$. 
						Since $\ell(t_x, t_\rho) \leq |x|$ by assumption 
						and $\tau_\rho \in R^\prime$, it follows that 
						$\ell(\tau_\rho \Intersect t_\rho) \leq |x|$, i.e., 
						$t_\rho$ is $y0$-disjoint with $\tau_\rho$.
						Therefore, 
						$\reach_{F^\prime}(\tau_\mu) \geq \reach_{F^\prime}(t_\rho) = \reach_{F}(t_\rho) - 1 = \rho(w) - 1 \geq \mu_x(y) - 1 = \mu_x(y0)$. 
						Here, the second inequality is true since $\mu_x(y) \leq \rho(xy) = \rho(w)$.

				\end{description}


		\end{description}
		
	\end{proof}

	This completes the proof of Theorem~\ref{thm:canonical-fork}. 
	\hfill $\qed$



