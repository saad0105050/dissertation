A significant finding of~\citet{KRDO17} is that the margin of a
characteristic string $\mu(w)$---the maximum value of a quantity taken
over a (typically) exponentially-large family of forks---can be given
a simple, mutually recursive formulation with the associated quantity
of reach $\rho(w)$. Specifically, they prove the following lemma.

%%% original lemma
\begin{lemma}[{\cite[Lemma~4.19]{KRDO17}}]\label{lem:margin} 
  $\rho(\varepsilon) = 0$ 
  where $\varepsilon$ is the empty string, and, for all nonempty strings $w\in\{0,1\}^*$,
  \begin{equation}
    \rho(w1) = \rho(w)+1\,, \qquad\text{and}\qquad
    \rho(w0) = \begin{cases} 0 & \text{if $\rho(w) = 0$,}\\
      \rho(w)-1 & \text{otherwise.}
    \end{cases}
		\label{eq:rho-recursive}
  \end{equation}
  Furthermore, margin satisfies the mutually recursive relationship
  $\mu(\varepsilon) = 0$ and for all $w \in \{0,1\}^*$,
  \begin{equation}
    \mu(w1) = \mu(w)+1\,,\qquad\text{and}\qquad
    \mu(w0) = \begin{cases}
      0 & \text{if $\rho(w)>\mu(w)=0$,} \\
%      \mu(w)-1 & \text{if $\rho(w)=0$,} \\
      \mu(w)-1 & \text{otherwise.}
    \end{cases}
		\label{eq:mu-recursive}
  \end{equation}
  Additionally, there exists a closed fork $F\vdash w$ such that
  $\rho(F)=\rho(w)$ and $\mu(F)=\mu(w)$.
  %(It is convenient to separate the case $\rho(w) = 0$ from the other case which also yields $\mu(w) - 1$ in the proof, so we reflect that in the statement of the theorem.)
\end{lemma}

We prove an analogous recursive statement for relative margin, recorded below.

\begin{lemma}[Relative margin]\label{lem:relative-margin}
  Given a fixed string $x\in\{0,1\}\text{\emph{*}}$,
  $\mu_x(\varepsilon) =\rho(x)$ 
  where $\varepsilon$ is the empty string, and, for all nonempty strings $w=xy\in\{0,1\}\text{\emph{*}},$
  \begin{equation}
    \mu_x(y1)= \mu_x(y)+1\,,\qquad\text{and}\qquad
    \mu_x(y0)= \begin{cases}
      0 & \text{if } \rho(xy) > \mu_x(y)=0\,, \\
%      \mu_x(y)-1 &  \text{if } \rho(xy)=0\,, \\
      \mu_x(y)-1 & \text{otherwise.}
    \end{cases}
		\label{eq:mu-relative-recursive}
  \end{equation}
  Additionally, there exists a closed fork $F\vdash xy$ such that
  $\rho(F)=\rho(xy)$ and $\mu_x(F)=\mu_x(y)$.
  %(It is convenient to
  %separate the case $\rho(w) = 0$ from the other case which also
  %yields $\mu(w) - 1$ in the proof, so we reflect that in the
  %statement of the lemma.)
\end{lemma}

We delay the proof of Lemma~\ref{lem:relative-margin} to
\Section~\ref{sec:margin-proof}, preferring to immediately focus on the
application to settlement times in \Section~\ref{sec:estimates}.

\Paragraph{Discussion.} The proof of Lemma~\ref{lem:relative-margin}
shares many technical similarities with the proof of
Lemma~\ref{lem:margin} given by~\citet{KRDO17}. However, there is an
important respect in which the proofs differ. Each of the proofs
requires the definition of a particular adversary (which, in effect,
constructs a fork achieving the worst case reach and margin guaranteed
by the lemma). The adversary constructed by~\cite{KRDO17} can create a
balanced fork for $w$ whenever $\mu(w) \geq 0$ (i.e., $w$ is
``forkable''). However, the adversary only focuses on the problem of
producing disjoint tines over the \emph{entire string} $w$ (consistent
with the definition of $\mu(\cdot)$). The ``optimal online adversary,''
developed in \Section~\ref{sec:canonical-forks},
% developed during the proof of Lemma~\ref{lem:relative-margin},
%, in
%contrast,
uses a more sophisticated rule for extending chains (tines) of the
fork. 
Notably, this adversary can \emph{simultaneously maximize relative margin
  over all prefixes of the string}. 

% This adversary is a purist: if $w$ is not forkable, he gives up. As an
% illuminating example, consider the characteristic string
% $w=0^n0^k1^k0^k$, for $n\gg k$. The purist adversary will produce a
% single tine containing only the honest nodes, but we can imagine that
% a different adversary could notice that it is possible to build a fork
% with divergence of $2k$, and act differently than the purist
% adversary.
%
% Here, we develop and optimal online adversary
% that makes decisions with the goal of maximizing divergence.
% Crucially, if $w$ is forkable, we want our new adversary to still successfully fork $w$.


%%% Local Variables:
%%% mode: latex
%%% TeX-master: "main"
%%% End:
