With the recursive formulation for relative margin in hand, 
we study the stochastic process that arises when the
characteristic string $w$ is chosen from a distribution 
satisfying the $\epsilon$-martingale condition. 
Let us write $w = xy$ (where the decomposition is arbitrary) and 
let $E$ be the event that the relative margin $\mu_x(y)$ is non-negative. 
As Fact~\ref{fact:margin-balance} and Observation~\ref{obs:settlement-balanced-fork} point out, 
this event has a direct bearing on the settlement violation on $w$. 

In this section, we prove two bounds on the probability of the event $E$.
The first bound corresponds to the distribution 
$\mathcal{B}_\epsilon$ 
whereas the second bound applies to any distribution that 
satisfies the $\epsilon$-martingale condition. 
(Recall that the distribution $\mathcal{B}_\epsilon$, mentioned in Theorem~\ref{thm:main}, 
satisfies the $\epsilon$-martingale condition with equality.)
Our exposition in this section culminates in the proofs of our main theorems. 

We start with the following theorem 
which is a direct consequences of these bounds; see Section~\ref{sec:bounds} for a proof.
\begin{theorem}\label{thm:plain-main}
  Let $T, k \in \NN$.
  Let $w \in \{0,1\}^T$ be a random variable
  satisfying the $\epsilon$-martingale condition. 
  Consider the decomposition $w = xy, |y| = k$.  
  Then
  \[
    \Pr_{w = xy}[\text{there is an $x$-balanced fork for $xy$}] 
    = \Pr_{w = xy}[\mu_x(y) \geq 0] 
    \leq \exp(-\Omega(k))
    % = \exp({-\epsilon^3 (1 - O(\epsilon)) k/2})
    \,.
  \]
  (The asymptotic notation hides constants that depend only on $\epsilon$.)
\end{theorem}
Notice how the final bound does not depend on $|x|$. 
Indeed, as we show in Lemma~\ref{lemma:rho-stationary}, 
the reach of a Boolean string $x$ 
drawn from the distribution $\mathcal{B}_\epsilon$ 
% mentioned in Theorem~\ref{thm:main} 
converges to a fixed exponential distribution as
$|x| \rightarrow \infty$. 
This limiting distribution ``stochastically dominates'' 
any distribution that satisfies the $\epsilon$-martingale condition; 
see Section~\ref{sec:dominance-rho-stationary}.
The following corollary is immediate.
% An appeal to Fact~\ref{fact:margin-balance} yields the following corollary.
\begin{corollary}\label{cor:main} 
  Let $T, s, k \in \NN$.
  Let $w \in \{0,1\}^T$ be a 
  random variable satisfying the $\epsilon$-martingale condition. 
  Then
  \begin{align}\label{eq:cor-main}
    \Pr_w\left[\parbox{65mm}{
      there is a decomposition $w = xyz$, 
      where $|x| = s - 1$ and $|y| \geq k$, 
      so that $\mu_x(y) \geq 0$ 
    }\right] 
      \leq O(1) \cdot \exp(-\Omega(k))
    \,.
  \end{align}
\end{corollary}
\begin{proof}
  Notice that Theorem~\ref{thm:plain-main} works for \emph{any} prefix $x$ 
  of the characteristic string $w = xy$.
  Thus we can fix the prefix $x$ with length $s - 1$ and 
  sum the bound in Theorem~\ref{thm:plain-main} 
  over all suffixes $y$ with length at least $k$. 
  This would give an upper bound to the left-hand side of our claim, 
  the bound being 
  $\sum_{t \geq k} \exp(-\Omega(t)) = O(1)\cdot \exp(-\Omega(k))$. 
\end{proof}

We obtain another imporant corollary by setting $|x| = 0$ and $|y| = n$ in Theorem~\ref{thm:plain-main}. 
\begin{corollary}\label{coro:forkable-rare}%[cf. \cite{KRDO17}]
  Let $w \in \{0,1\}^n$ be a random variable satisfying the $\epsilon$-martingale condition. Then
  \[
    \Pr[\text{$w$ is forkable}] = \Pr[\mu(w) \geq 0] \leq \exp(-\Omega(n))
    \,.
  \]
\end{corollary}
Thus \emph{forkable strings are rare} 
where ``forkable'' is defined in Definition~\ref{def:forkable}.
This result 
significantly strengthens the $\exp(-\Omega(\sqrt{n}))$ 
bound obtained in Theorem 4.13 of~\cite{KRDO17}. 
The improvement comes in two respects: 
first, Corollary~\ref{cor:main} improves the exponent from $\sqrt{n}$ to $n$, 
and second, the characteristic string is allowed to be drawn 
from any distribution satisfying the $\epsilon$-martingale condition. 
For comparison, the characteristic string in Theorem 4.13 of~\cite{KRDO17} 
has the distribution $\mathcal{B}_\epsilon$, i.e., 
the bits were i.i.d.\ Bernoulli random variables 
with expectation $(1 - \epsilon)/2$.




\subsection{Two bounds for non-negative relative margin}\label{sec:bounds}
The main ingredients to proving Theorem~\ref{thm:plain-main} 
are two bounds on the event that for a characteristic string $xy$, 
the relative margin $\mu_x(y)$ is non-negative. 

\begin{bound}\label{bound:analytic}
  Let $x \in \{0,1\}^m$ and $y \in \{0,1\}^k$ be independent random
  variables, each chosen according to $\mathcal{B}_\epsilon$. Then
  \[
    \Pr[\mu_x(y) \geq 0] 
      \leq \exp({-\epsilon^3 (1 - O(\epsilon)) k/2})
    \,.
  \]
\end{bound}
% We are also interested in characteristic
% strings drawn from a distribution $\mathcal{W}$ 
% which satisfies $\epsilon$-martingale condition. 
% There are settings, 
% such as Genesis~\cite{DBLP:journals/iacr/BadertscherGKRZ18}, 
% where this flexibility is important.  


\begin{bound}\label{bound:geometric}
  Let $x \in \{0,1\}^m$ and $y \in \{0,1\}^k$ be random variables
  (jointly) satisfying the $\epsilon$-martingale condition with
  respect to the ordering $x_1, \ldots, x_m, y_1, \ldots, y_k$.  Let
  $x^\prime \in \{0,1\}^m$ and $y^\prime \in \{0,1\}^k$ be independent
  random variables, each chosen independently according to
  $\mathcal{B}_\epsilon$.  Then
  \[
    \Pr[\mu_x(y) \geq 0] \leq \Pr[\mu_{x^\prime}(y^\prime) \geq 0]
%    \,.
%  \]
%  As a result, 
%  \[
%    \Pr[\mu_x(y) \geq 0] 
      \leq \exp({-\epsilon^3 (1 - O(\epsilon)) k/2})
    \,.
  \]
\end{bound}

\paragraph{Proof of Theorem~\ref{thm:plain-main}.}
The equality is Fact~\ref{fact:margin-balance} 
and the inequality is Bound~\ref{bound:geometric}. $\qed$


% \hfill $\qed$ 



% \subsection{$\mathcal{B}$ stochastically dominates $\Distribution$; stationary distribution for reach}
\subsection{A stochastically dominant prefix distribution}\label{sec:dominance-rho-stationary}

Stochastic dominance plays an important role in the arguments
below. First of all, we observe that the distribution
$\mathcal{B}_\epsilon$ stochastically dominates any distribution
satisfying the $\epsilon$-martingale condition; this yields the first
inequality in Theorem~\ref{thm:main}. A more delicate application of
stochastic dominance is used in order to achieving bounds, such as
those of Section~\ref{sec:bounds}, that are independent of the length of
$x$. This follows from the fact that $\reach(B_{\epsilon})$ converges to a
particular, dominant distribution as its argument increases in length.

For notational convenience, we denote
the probability distribution associated with a random variable using
uppercase script letters; for example, the distribution of a random
variable $R$ is denoted by $\mathcal{R}$.  This usage should be clear
from the context.


\begin{definition}[Monotonicity and stochastic dominance]\label{def:dominance}
  Let $\Omega$ be a set endowed with a partial order $\leq$. A subset
  $A \subset \Omega$ is monotone if for all $x \leq y$, $x \in A$
  implies $y \in A$.  Let $X$ and $Y$ be random variables taking
  values in $\Omega$.
  % Let $\mathcal{X}$ and $\mathcal{Y}$ be distributions associated with 
  % $X$ and $Y$, respectively. 
  We say that $X$ \emph{stochastically dominates} $Y$, 
  written $Y \dominatedby X$, if 
  $
    \mathcal{X}(A) \geq \mathcal{Y}(A)
    % \,.
    $ for all monotone $A \subseteq \Omega$.  As a special case, when
    $\Omega = \R$, $Y \dominatedby X$ if
    $\Pr[X \geq \Lambda] \geq \Pr[Y \geq \Lambda]$ for every
    $\Lambda \in \R$.  We extend this notion to probability
    distributions in the natural way.
\end{definition}
% \begin{definition}[Stochastic dominance]\label{def:dominance} 
% Let $X$ and $Y$ be random
%   variables taking values in $\R$. We say that $X$ \emph{stochastically
%   dominates} $Y$, written $Y \dominatedby X$ if
%   \[
%     \Pr[X \geq \Lambda] \geq \Pr[Y \geq \Lambda]
%   \]
%   for every $\Lambda \in \R$.  We extend this notion to probability
%   distributions in the natural way.
% %  In addition, for two distributions $\mathcal{X}, \mathcal{Y}$,
% %  we say that $\mathcal{X}$ stochastically dominates $\mathcal{Y}$ 
% %  if and only if there are random variables $X \sim \mathcal{X}, Y \sim %\mathcal{Y}$ such that $Y \dominatedby X$;
% %  this is written as $\mathcal{Y} \dominatedby \mathcal{X}$.
% \end{definition}
Observe that for any non-decreasing function $u$ defined on $\Omega$,
$Y \dominatedby X$ implies $u(Y) \leq u(X)$. Finally, we note that for
real-valued random variables $X$, $Y$, and $Z$, if $Y \dominatedby X$
and $Z$ is independent of both $X$ and $Y$, then
$Z + Y \dominatedby Z + X$.

% Let $m \in \NN$ and suppose 
% $W = (W_1, \ldots, W_m) \in \{0,1\}^m$ satisfies the 
% $\epsilon$-martingale condition. 
% It turns out that $\rho(W)$ 
% is stochastically dominated by 
% the distribution of $\rho(B_1, \ldots, B_m)$, 
% where each $B_i \in \{0, 1\}$ is 
% an independent Bernoulli random variable with parameter $(1 - \epsilon)/2$.
% In addition, $\rho(B_1, \ldots, B_m)$ is stochastically dominated by 
% its limiting (stationary) distribution where we take $m \rightarrow \infty$.

%=======================================================
\begin{lemma}\label{lemma:rho-stationary}
  % Let $n \in \NN$ and consider a sequence of random variables
  % $W = (W_1, \ldots, W_n) \in \{0,1\}^n$ satisfying the
  % $\epsilon$-martingale condition. 
  Suppose $W = (W_1, \ldots, W_n) \in \{0,1\}^n$ satisfies the 
  $\epsilon$-martingale condition. 
  Let $\epsilon \in (0, 1)$ and $B = (B_1, \ldots, B_n) \in \{0,1\}^n$ 
  where each $B_i$ is independent with expectation $(1- \epsilon)/2$.
  Let $R_\infty \in \{0, 1, \ldots\}$ be a random variable 
  whose distribution $\StationaryRho$ is defined as 
    \begin{equation}
      \label{eq:stationary}
      \StationaryRho(k) 
        = \Pr[R_\infty = k] 
        \defeq \left(\frac{2\epsilon}{1+\epsilon}\right)\cdot \left(\frac{1-\epsilon}{1 + \epsilon}\right)^k
        \qquad \text{for $k = 0, 1, 2, \ldots$}\ 
      \,.
    \end{equation}
  Then $\rho(W) \dominatedby \rho(B) \dominatedby R_\infty$.
\end{lemma}

\begin{proof}
  We begin by observing that $B$ stochastically dominates $W$. As a
  matter of notation, for any fixed values
  $w_1, \ldots, w_k \in \{0,1\}^k$, let
  \[
    \theta[w_1, \ldots, w_k] = \Pr[ W_{k+1} = 1 \mid
    \text{$W_i = w_i$, for $i \leq k$}] \leq (1 - \epsilon)/2
  \]
  and $\theta[\varepsilon] = \Pr[W_1 = 1]$ 
  where $\varepsilon$ is the empty string. Then consider $n$ uniform and
  independent real numbers $(A_1, \ldots, A_n)$, each taking a value
  in the unit interval $[0,1]$; we use these random variables to construct a monotone
  coupling between $W$ and $B$. 
  Specifically, define $\beta: [0,1]^n \rightarrow \{0,1\}^n$
  by the rule $\beta(\alpha_1, \ldots, \alpha_n) = (b_1, \ldots, b_n)$
  where
  \[
    b_t = \begin{cases} 1 & \text{if $\alpha_t \leq (1-\epsilon)/2$},\\
      0 & \text{if $\alpha_t > (1 - \epsilon)/2$},
    \end{cases}
  \]
  and define
  $B = (B_1, \ldots, B_n) = \beta(A_1, \ldots, A_n)$; these
  $B_i$s are independent zero-one Bernoulli random variables with expectation
  $(1-\epsilon)/2$. Likewise define the function
  $\omega:[0,1]^n \rightarrow \{0,1\}^n$ so that
  $\omega(\alpha_1, \ldots, \alpha_n) = (w_1, \ldots, w_n)$
  where each $w_t$ is assigned by the iterative rule
  \[
    w_{t+1} = \begin{cases} 1 & \text{if $\alpha \leq \theta[w_1, \ldots, w_t]$},\\
      0 & \text{if $\alpha > \theta[w_1, \ldots, w_t]$},
    \end{cases}
  \]
  and observe that the probability law of
  $\omega(A_1, \ldots, A_n)$ is precisely that of
  $W = (W_1, \ldots, W_n)$. For convenience, we simply identify the
  random variable $W$ with $\omega(A_1, \ldots, A_n)$. Note
  that for any $\alpha = (\alpha_1, \ldots, \alpha_n)$ and for each
  $i$, the $i$th coordinates of $\beta(\alpha)$ and $\omega(\alpha)$ satisfy
  $\omega(\alpha)_i \leq \beta(\alpha)_i$ 
  % (which is to say that $W_i \leq B_i$). 
  % It follows immediately that
  % $\rho(\omega(\alpha)) \leq \rho(\beta(\alpha))$ with probability 1 and
  % hence $\rho(W) \dominatedby \rho(B)$. 
  % See~\cite[Lemma 22.5]{LevinPeres}. 
  (which is to say that $W_i \leq B_i$ with probability 1). 
  But this is equivalent to saying $W \dominatedby B$. 
  (See~\cite[Lemma 22.5]{LevinPeres}.) 
  Now consider the following partial order $\leq$ on the $n$-bit Boolean strings: 
  for $x,y \in \{0,1\}^n$, 
  we write $x \leq y$ if and only if $x_i = 1$ implies $y_i = 1, i \in [n]$.
  Since $\rho$ is non-decreasing with respect to this partial order, 
  we have  
  $\rho(\omega(\alpha)) \leq \rho(\beta(\alpha))$ with probability 1 and
  hence $\rho(W) \dominatedby \rho(B)$ as well. 
  

  %  \paragraph{$R_\infty$ stochastically dominates $\rho(B)$.}
  To complete the proof, we now establish that
  $\rho(B) \dominatedby R_\infty$.  We remark that the random variables
  $\rho(B)$ (and $R_\infty$) have an immediate interpretation in terms
  of the Markov chain corresponding to a biased random walk on $\Z$
  with a ``reflecting boundary'' at -1. Specifically, consider the
  Markov chain on $\{0, 1, \ldots\}$ given by the transition diagram
  \begin{center}
    \begin{tikzpicture}[scale=1,>=stealth', auto, semithick,
      flat/.style={circle,draw=black,thick,text=black,font=\small}]
      \node[flat] (n0) at (0,0)  {$0$};
      \node[flat] (n1) at (1,0)  {$1$};
      \node[flat] (n2) at (2,0)  {$2$};
      \node[flat,white] (n3) at (3,0) {$ \ \ \ $};
      \node[] at (3,0) {$\ldots$};
      \draw[thick,->,bend left] (n0) to (n1);
      \draw[thick,->,bend left] (n1) to (n2);
      \draw[thick,->,bend left] (n2) to (n3);
      \draw[thick,->,loop left] (n0) to (n0);
      \draw[thick,->,bend left] (n1) to (n0);
      \draw[thick,->,bend left] (n2) to (n1);
      \draw[thick,->,bend left] (n3) to (n2);
    \end{tikzpicture}
  \end{center}
  where edges pointing right have probability $(1-\epsilon)/2$ and edges pointing left---including the loop at 0---have probability $(1+\epsilon)/2$. Examining the recursive description of $\rho(w)$, it is easy to confirm that the random variable $\rho(B_1, \ldots, B_n)$ is precisely given by the result of evolving the Markov chain above for $n$ steps with all probability initially placed at 0. It is further easy to confirm that the distribution given by~\eqref{eq:stationary} above is stationary for this chain.

  % The above Markov chain is \emph{irreducible} since every state is
  % reachable from every state.  In addition, it is easy to see that this
  % chain is \emph{aperiodic}---at time $n$, the walk visits every state
  % $s \in \{0, 1, \cdots, n\}$ with a strictly positive probability.
  % According to~\citep[Theorem 21.12]{LevinPeres}, an irreducible Markov
  % chain is positive recurrent if and only if there exists a distribution
  % $\pi$ on the state space satisfying $\pi P = \pi$, where $P$ is the
  % transition matrix of the chain.  We can take $\pi$ to be the
  % distribution given by~\eqref{eq:stationary}, implying that the chain
  % is positive recurrent.  Consequently, by ~\citep[Theorem
  % 21.14]{LevinPeres}, this irreducible, aperiodic, positive recurrent
  % Markov chain converges to the unique stationary distribution $\pi$.
  % It follows that the distributions $R_n$ limit to $R_\infty$.

  To establish stochastic dominance, it is convenient to work with the
  underlying distributions and consider walks of varying lengths: let
  $\DistRho_n: \Z \rightarrow \R$ denote the probability distribution given by
  $\rho(B_1, \ldots, B_n)$; likewise
  define $\DistRho_\infty$. For a distribution $\DistRho$ on $\Z$, we define $[\DistRho]_0$
  to denote the probability distribution obtained by shifting all
  probability mass on negative numbers to zero; that is, for $x \in \Z$,
  \[
    [\DistRho]_0(x) = \begin{cases} \DistRho(x) & \text{if $x > 0$},\\
      \sum_{t \leq 0} \DistRho(t) & \text{if $x = 0$},\\
      0 & \text{if $x < 0$.}
    \end{cases}
  \]
  We observe that if $A \dominatedby C$ then $[A]_0 \dominatedby [C]_0$ for any
  distributions $A$ and $C$ on $\Z$. It will also be convenient to
  introduce the shift operators: for a distribution
  $\DistRho: \Z \rightarrow \R$ and an integer $k$, we define $S^k\DistRho$ to be the
  distribution given by the rule $S^k\DistRho(x) = \DistRho(x-k)$. With these
  operators in place, we may write
  \[
    \DistRho_t = \left(\frac{1 - \epsilon}{2}\right) S^1 \DistRho_{t-1} +
      \left(\frac{1 + \epsilon}{2}\right) \left[S^{-1}\DistRho_{t-1} \right]_0\,,
  \]
  with the understanding that $\DistRho_0$ is the distribution placing unit probability at $0$. The proof now proceeds by induction. It is clear that $\DistRho_0 \dominatedby \DistRho_\infty$. Assuming that $\DistRho_n \dominatedby \DistRho_\infty$, we note that for any $k$
  \[
    S^k \DistRho_n \dominatedby S^k \DistRho_\infty \qquad \text{and, additionally, that
    }\qquad [S^{-1}\DistRho_n]_0 \dominatedby [S^{-1}\DistRho_{\infty}]_0\,.
  \]
  Finally, it is clear that stochastic dominance respects convex combinations, 
  in the sense that if $A_1 \dominatedby C_1$ and $A_2 \dominatedby C_2$ then 
  $\lambda A_1 + (1-\lambda) A_2 \dominatedby \lambda C_1 + (1-\lambda) C_2$ (for $0 \leq \lambda \leq 1$). We conclude that
  \[
    \DistRho_{t+1} = \left(\frac{1 - \epsilon}{2}\right) S^1 \DistRho_{t} +
      \left(\frac{1 + \epsilon}{2}\right) \left[S^{-1}\DistRho_{t} \right]_0 \dominatedby \left(\frac{1 - \epsilon}{2}\right) S^1 \DistRho_{\infty} +
      \left(\frac{1 + \epsilon}{2}\right) \left[S^{-1}\DistRho_{\infty} \right]_0 
      \,.
  \]
  By inspection, the right-hand side equals $\DistRho_{\infty}$, as desired. 
  Hence $\rho(B) \dominatedby R_\infty$.
  % $\qedhere$
\end{proof}

  \paragraph{Remark.} 
  In fact, the random variable $\rho(B)$
  actually converges to $R_\infty$ as $n \rightarrow \infty$. 
  This can be seen, for example, 
  by solving for the stationary distribution of the Markov chain in the proof above. 
  However, we will only require the dominance for our exposition. 
  Importantly, since $\mu_x(\varepsilon) = \rho(x)$, and 
  $\Pr[\mu_x(y) \geq 0]$ increases monotonically 
  with an increase in $\Pr[\mu_x(\varepsilon) \geq r]$ for any $r \geq 0$, 
  it suffices to take $|x| \rightarrow \infty$ 
  when reasoning about an upper bound on $\Pr[\mu_x(y) \geq 0]$. 

%=======================================================
%=======================================================
\subsection{Proof of Bound~\ref{bound:analytic}}\label{sec:gf-proof}

%\begin{proof}[of Bound~\ref{bound:analytic}]
% \begin{proof}
  Anticipating the proof, we make a few remarks about generating
  functions and stochastic dominance.  We reserve the term
  \emph{generating function} to refer to an ``ordinary'' generating
  function which represents a sequence $a_0, a_1, \ldots$ of
  non-negative real numbers by the formal power series
  $\gf{A}(Z) = \sum_{t = 0}^\infty a_t Z^t$. When
  $\gf{A}(1) = \sum_t a_t = 1$ we say that the generating function is
  a \emph{probability generating function}; in this case, the
  generating function $\gf{A}$ can naturally be associated with the
  integer-valued random variable $A$ for which $\Pr[A = k] = a_k$. If
  the probability generating functions $\gf{A}$ and $\gf{B}$ are
  associated with the random variables $A$ and $B$, it is easy to
  check that $\gf{A} \cdot \gf{B}$ is the generating function
  associated with the convolution $A + B$ (where $A$ and $B$ are
  assumed to be independent).  Translating the notion of stochastic
  dominance to the setting with generating functions, we say that the
  generating function $\gf{A}$ \emph{stochastically dominates}
  $\gf{B}$ if $\sum_{t \leq T} a_t \leq \sum_{t \leq T} b_t$ for all
  $T \geq 0$; we write $\gf{B} \dominatedby \gf{A}$ to denote this state of
  affairs. If $\gf{B}_1 \dominatedby \gf{A}_1$ and
  $\gf{B}_2 \dominatedby \gf{A}_2$ then
  $\gf{B}_1 \cdot \gf{B}_2 \dominatedby \gf{A}_1 \cdot \gf{A}_2$ and
  $\alpha \gf{B}_1 + \beta \gf{B}_2 \dominatedby \alpha \gf{A}_1 + \beta
  \gf{A}_2$ (for any $\alpha, \beta \geq 0$).  Moreover, if
  $\gf{B} \dominatedby \gf{A}$ then it can be checked that
  $\gf{B}(\gf{C}) \dominatedby \gf{A}(\gf{C})$ for any probability
  generating function $\gf{C}(Z)$, where we write $\gf{A}(\gf{C})$ to
  denote the composition $\gf{A}(\gf{C}(Z))$.


  Finally, we remark that
  if $\gf{A}(Z)$ is a generating function which converges as a
  function of a complex $Z$ for $|Z| < R$ for some non-negative $R$, 
  $R$ is called the \emph{radius of convergence} of $\gf{A}$.  
  It follows from \citep[Theorem 2.19]{WilfGF} that 
  $\lim_{k \rightarrow \infty} {a_k}R^k = 0$ and $|a_k| = O(R^{-k})$. 
	In addition, if $\gf{A}$ is a probability generating function associated with the
  random variable $A$ then it follows that
  $\Pr[A \geq T] = O(R^{-T})$.
  
  We define $p = (1 - \epsilon)/2$ and $q = 1 - p$ and 
  as in the proof of Bound~\ref{bound:geometric},
  consider the independent $\{0,1\}$-valued random variables
  $w_1, w_2, \ldots$ where $\Pr[w_t = 1] = p$. We also define the
  associated $\{\pm1\}$-valued random variables $W_t =
  (-1)^{1+w_t}$.
	
  
	Although our actual interest is in the random variable $\mu_x(y)$ 
	from~\eqref{eq:mu-relative-recursive} on a characteristic string $w=xy$, 
  we begin by analyzing the case when $|x|=0$. 

  %\vspace{-2ex}
	\paragraph{Case 1: $x$ is the empty string.}
  In this case, the random variable $\mu_x(y)$ is identical to $\mu(w)$ 
  from~\eqref{eq:mu-recursive} with $w = y$. 
	Our strategy is to study the probability generating
  function
  \[
    \gf{L}(Z) = \sum_{t = 0}^\infty \ell_t Z^t
  \]
  where $\ell_t = \Pr[\text{$t$ is the last time $\mu_t =
    0$}]$. Controlling the decay of the coefficients $\ell_t$ suffices
  to give a bound on the probability that $w_1\ldots w_k$ is forkable
  because
  \[
    \Pr[\text{$w_1 \ldots w_k$ is forkable}] \leq 1 - \sum_{t =
      0}^{k-1} \ell_t = \sum_{t = k}^\infty \ell_t\,.
  \]
  It seems challenging to give a closed-form algebraic expression for
  the generating function $\gf{L}$; our approach is to develop a
  closed-form expression for a probability generating function
  $\gf{\hat{L}} = \sum_t \hat{\ell}_t Z^t$ which stochastically
  dominates $\gf{L}$ and apply the analytic properties of this closed
  form to bound the partial sums $\sum_{t \geq k} \hat{\ell}_k$.
  Observe that if $\gf{L} \dominatedby \gf{\hat{L}}$ then the series
  $\gf{\hat{L}}$ gives rise to an upper bound on the probability that
  $w_1\ldots w_k$ is forkable as
  $\sum_{t=k}^\infty \ell_t \leq \sum_{t=k}^\infty \hat{\ell}_t$.

  The coupled random variables $\rho_t$ and $\mu_t$ are Markovian
  in the sense that values $(\rho_s, \mu_s)$ for $s \geq t$ are
  entirely determined by $(\rho_t, \mu_t)$ and the subsequent
  values $W_{t+1}, \ldots$ of the underlying variables $W_i$. We
  organize the sequence
  $(\rho_0, \mu_0), (\rho_1, \mu_1), \ldots$ into ``epochs''
  punctuated by those times $t$ for which $\rho_t = \mu_t =
  0$. With this in mind, we define $\gf{M}(Z) = \sum m_t Z^t$ to be
  the generating function for the first completion of such an epoch,
  corresponding to the least $t > 0$ for which
  $\rho_t = \mu_t = 0$. As we discuss below, $\gf{M}(Z)$ is not a
  probability generating function, but rather
  $\gf{M}(1) = 1 - \epsilon$. It follows that
  \begin{equation}\label{eq:L-def}
    \gf{L}(Z) = \epsilon(1 + \gf{M}(Z) + \gf{M}(Z)^2 + \cdots) =
    \frac{\epsilon}{1 - \gf{M}(Z)}\,.
  \end{equation}
  Below we develop an analytic expression for a generating function
  $\gf{\hat{M}}$ for which $\gf{M} \dominatedby \gf{\hat{M}}$ and define
  $\gf{\hat{L}} = \epsilon/(1 - \gf{\hat{M}}(Z))$. We then proceed as
  outlined above, noting that $\gf{L} \dominatedby \gf{\hat{L}}$ and using
  the asymptotics of $\gf{\hat{L}}$ to upper bound the probability
  that a string is forkable.

  In preparation for defining $\gf{\hat{M}}$, we set down two
  elementary generating functions for the ``descent'' and ``ascent''
  stopping times. Treating the random variables $W_1, \ldots$ as
  defining a (negatively) biased random walk, define $\gf{D}$ to be
  the generating function for the \emph{descent stopping time} of the
  walk; this is the first time the random walk, starting at 0, visits
  $-1$. The natural recursive formulation of the descent time yields a
  simple algebraic equation for the descent generating function,
  $\gf{D}(Z) = qZ + pZ \gf{D}(Z)^2$, and from this we may conclude
  \[
    \gf{D}(Z) = \frac{1 - \sqrt{1 - 4pqZ^2}}{2pZ}\,.
  \]
  We likewise consider the generating function $\gf{A}(Z)$ for the
  \emph{ascent stopping time}, associated with the first time the
  walk, starting at 0, visits 1: we have
  $\gf{A}(Z) = pZ + qZ \gf{A}(Z)^2$ and
  \[
    \gf{A}(Z) = \frac{1 - \sqrt{1 - 4pqZ^2}}{2qZ}\,.
  \]
  Note that while $\gf{D}$ is a probability generating function, the
  generating function $\gf{A}$ is not: according to the classical
  ``gambler's ruin'' analysis~\cite{Grinstead:1997ng}, the probability
  that a negatively-biased random walk starting at 0 ever rises to 1
  is exactly $p/q$; thus $\gf{A}(1) = p/q$.

  Returning to the generating function $\gf{M}$ above, we note that an
  epoch can have one of two ``shapes'': in the first case, the epoch
  is given by a walk for which $W_1 = 1$ followed by a descent (so
  that $\rho$ returns to zero); in the second case, the epoch is
  given by a walk for which $W_1 = -1$, followed by an ascent (so that
  $\mu$ returns to zero), followed by the eventual return of $\rho$
  to 0. Considering that when $\rho_t > 0$ it will return to zero
  in the future almost surely, it follows that the probability that
  such a biased random walk will complete an epoch is
  $p + q(p/q) = 2p = 1 - \epsilon$, as mentioned in the discussion
  of~\eqref{eq:L-def} above. One technical difficulty arising in a
  complete analysis of $\gf{M}$ concerns the second case discussed
  above: while the distribution of the smallest $t > 0$ for which
  $\mu_t = 0$ is proportional to $\gf{A}$ above, the distribution of
  the smallest subsequent time $t'$ for which $\rho_{t'} = 0$
  depends on the value $t$. More specifically, the distribution of the
  return time depends on the value of $\rho_t$. Considering that
  $\rho_t \leq t$, however, this conditional distribution (of the
  return time of $\rho$ to zero conditioned on $t$) is
  stochastically dominated by $\gf{D}^t$, the time to descend $t$
  steps. This yields the following generating function $\gf{\hat{M}}$
  which, as described, stochastically dominates $\gf{M}$:
  \[
    \gf{\hat{M}}(Z) = pZ\cdot \gf{D}(Z) + qZ \cdot \gf{D}(Z) \cdot
    \gf{A}(Z\cdot \gf{D}(Z))\,.
  \]
  
  It remains to establish a bound on the radius of convergence of
  $\gf{\hat{L}}$. Recall that if the radius of convergence of
  $\gf{\hat{L}}$ is $\exp(\delta)$ it follows that
  $\Pr[\text{$w_1 \ldots w_k$ is forkable}] = O(\exp(-\delta k))$. A
  sufficient condition for convergence of
  $\gf{\hat{L}}(z) = \epsilon/(1 - \gf{\hat{M}}(z))$ at $z$ is that
  that all generating functions appearing in the definition of
  $\gf{\hat{M}}$ converge at $z$ and that the resulting value
  $\gf{\hat{M}}(z) < 1$.
  
  The generating function $\gf{D}(z)$ (and $\gf{A}(z)$) converges when
  the discriminant $1 - 4pqz^2$ is positive; equivalently
  $|z| < 1/\sqrt{1 - \epsilon^2}$ or
  $|z| < 1 + \epsilon^2/2 + O(\epsilon^4)$. Considering
  $\gf{\hat{M}}$, it remains to determine when the second term,
  $qz D(z) \gf{A}(z \gf{D}(z))$, converges; this is likewise determined by
  positivity of the discriminant, which is to say that
  \[
    1 - (1 - \epsilon^2)\left(\frac{1 - \sqrt{1 - (1 - \epsilon^2)z^2}}{1 - \epsilon}\right)^2 > 0\,.
  \]
  Equivalently,
  \[
    |z| <  \sqrt{\frac{1}{1 + \epsilon}\left(\frac{2}{\sqrt{1 - \epsilon^2}} - \frac{1}{1+\epsilon}\right)} = 1 + \epsilon^3/2 + O(\epsilon^4) 
		\, .	
  \]
  Note that when the series $pz \cdot \gf{D}(z)$ converges, it
  converges to a value less than $1/2$; the same is true of
  $qz \cdot \gf{A}(z)$. It follows that for
  $|z| = 1 + \epsilon^3/2 + O(\epsilon^4)$, $|\gf{\hat{M}}(z)| < 1$
  and $\gf{\hat{L}}(z)$ converges, as desired. We conclude that
	\begin{align}
	  \Pr[\text{$w_1 \ldots w_k$ is forkable}] &= \exp(-\epsilon^3(1 + O(\epsilon))k/2)\,.
	\label{eq:prob_forkable_gf}
	\end{align}

          %\vspace{-2ex}
	\paragraph{Case 2: $x$ is non-empty.}
        The relative margin before $y$ begins is $\mu_x(\varepsilon)$.
        Recalling that $\mu_x(\varepsilon) = \rho(x)$ and conditioning on the event that $\rho(x) = r$, 
    let us define the random variables $\left\{ \tilde{\mu}_t \right\}$ for $t = 0, 1, 2, \cdots$ as follows: $\tilde{\mu}_0 = \rho(x)$ and
    \[
      %\, , \qquad \text{and} \qquad
      \Pr[\tilde{\mu}_t = s]\, =\, \Pr[\mu_x(y) = s \mid \rho(x) = r \text{ and } |y| = t ]
      \, .
    \]
    If the $\tilde{\mu}$ random walk makes the $r$th descent at some time $t < n$, then $\tilde{\mu}_t = 0$ and the remainder of the walk is 
	identical to an $(k-t)$-step $\mu$ random walk which we have already analyzed. 
	Hence we investigate the probability generating function
	\[
			\gf{B}_r(Z) = \gf{D}(Z)^r \gf{L}(Z) \quad \text{with coefficients} \quad
      b^{(r)}_t := \Pr[t \text{ is the last time } \tilde{\mu}_t = 0 \mid \tilde{\mu}_0 = r]	
  \]
  where $t = 0, 1, 2, \cdots$. Our interest lies in the quantity 
    \[
      b_t 
      := \Pr[t \text{ is the last time } \tilde{\mu}_t = 0] 
      = \sum_{r\geq 0}{  b^{(r)}_t \DistRho_m(r) } 
      \,,
     \]
  where the \emph{reach distribution} 
  $\DistRho_m : \Z \rightarrow [0,1]$ 
  associated with the random variable $\rho(x), |x| = m$ is defined as 
  \begin{align}\label{eq:dist-rho}
    % \DistRho_m(r) &= \Pr_{W \sim \mathcal{D}}[\rho(W) = r \Given W \text{ has length } m]
    \DistRho_m(r) &= \Pr_{x \SuchThat |x| = m}[\rho(x) = r]
    \, .
  \end{align}
     % from~\eqref{eq:dist-rho}.
     Let $\gf{R}_m(Z)$ be the probability generating function
     for the distribution $\DistRho_m$. 
     Using Lemma~\ref{lemma:rho-stationary} and Definition~\ref{def:dominance}, we deduce that
     $\gf{R}_m \dominatedby \gf{R}_\infty$ for every $m \geq 0$ since 
    $\DistRho_m \dominatedby \StationaryRho$.
     In addition, it is easy to check from~\eqref{eq:stationary} that
     the probability generating function for $\StationaryRho$ is in fact
     $\gf{R}_\infty(Z) = (1-\beta)/(1-\beta Z)$ where $\beta := (1-\epsilon)/(1+\epsilon)$. 
    Thus the generating function corresponding to the
     probabilities $\{b_t\}_{t=0}^\infty$ is
	\begin{align}
		\gf{B}(Z) 
		&= \sum_{t=0}^\infty{b_t Z^t} = \sum_{r=0}^\infty{\DistRho_m(r) \sum_{t=0}^\infty{b_t^{(r)} Z^t} } = \sum_{r=0}^\infty{\DistRho_m(r) \gf{B}_r(Z) } \nonumber \\
    &= \gf{L}(Z) \sum_{r=0}^\infty{\DistRho_m(r) \gf{D}(Z)^r}    \nonumber 
		= \gf{L}(Z)\  \gf{R}_m (\gf{D}(Z)) \nonumber 
		\dominatedby \gf{\hat{L}}(Z)\  \gf{R}_\infty (\gf{D}(Z))  \nonumber \\
    &= \frac{(1-\beta)\,\gf{\hat{L}}(Z) }{1 - \beta \gf{D}(Z)}
		\, .
	\label{eq:gf-mu-relative}
	\end{align}
  The dominance notation above follows because
  $\gf{L} \dominatedby \gf{\hat{L}}$ and $\gf{R}_m \dominatedby \gf{R}_\infty$.


  For $\gf{B}(Z)$ to converge, we need to check that $\gf{D}(Z)$
  should never converge to $1/\beta$.  One can easily check that
  the radius of convergence of $\gf{D}(Z)$---which is
  $\displaystyle 1/\sqrt{1-\epsilon^2}$---is strictly less than $1/\beta$ when
  $\epsilon > 0$.  We conclude that $\gf{B}(Z)$ converges if
  both $\gf{D}(Z)$ and $\gf{L}(Z)$ converge.  The radius of
  convergence of $\gf{B}(Z)$ would be the smaller of the radii
  of convergence of $\gf{D}(Z)$ and $\gf{L}(Z)$.  We already
  know from the previous analysis that $\gf{\hat{L}}(Z)$ has the
  smaller radius of the two; therefore, the bound
  in~\eqref{eq:prob_forkable_gf} applies to the relative margin $\mu_x(y)$
  for $|x|\geq 0$. 
  \hfill $\qed$  
  % $\qedhere$
  % $\qed$
% \end{proof}


%=======================================================
\subsection{Proof of Bound~\ref{bound:geometric}}\label{sec:martingale-proof-new}

Let $\epsilon \in (0, 1)$,  
$W \in \{0,1\}^m, W^\prime \in \{0,1\}^k$ 
where both $(W_1, \ldots, W_n)$ and $(W^\prime_1, \ldots, W^\prime_n)$ 
satisfy the $\epsilon$-martingale condition. 
Let $B \in \{0,1\}^m, B^\prime \in \{0,1\}^k$ where the components of $B, B^\prime$ 
are independent with expectation $(1 - \epsilon)/2$.
By Lemma~\ref{lemma:rho-stationary}, 

\begin{equation*}\label{eq:WB-dominance}\tag{$*$}
  W \dominatedby B\quad \text{and} \quad W^\prime \dominatedby B^\prime
  \,. 
\end{equation*}

Let us define the partial order $\leq$ on Boolean strings $\{0,1\}^k, k \in \NN$ 
as follows: 
$a \leq b$ if and only if
for all $i \in [k]$, $a_i = 1$ implies $b_i = 1$. 
Let $\mu : \{0,1\}^k \rightarrow \Z$ be the margin function 
from Lemma~\ref{lem:relative-margin}. 
Observe that for Boolean strings $a, a^\prime, b, b^\prime$ 
with $|a| = |a^\prime|$ and $|b| = |b^\prime|$, 
(i.) $b \leq b^\prime$ implies $\mu_a(b) \leq \mu_a(b^\prime)$ and 
(ii.) $a \leq a^\prime$ implies $\mu_a(b) \leq \mu_{a^\prime}(b)$. 
That is, 
\begin{equation*}\label{eq:mu-WB-dominance}\tag{$\dagger$}
  \text{$\mu_a(b)$ is non-decreasing in both $a$ and $b$}
  \,.   
\end{equation*}

Using~\eqref{eq:WB-dominance} and~\eqref{eq:mu-WB-dominance}, 
it follows that $\mu_W(W^\prime) \dominatedby \mu_{B}(B^\prime)$. 
% Recall that $P \dominatedby Q$ is true if, 
% for all monotone sets $E$, 
% by $\Pr[ P \in E ] \leq \Pr[Q \in E]$. 
% In particular, concerning the dominance $\mu_W(W^\prime) \dominatedby \mu_{B}(B^\prime)$, 
Writing $x = W$ and $y = W^\prime$, we have 
\begin{align*}
  \Pr[\mu_x(y) \geq 0]\, 
    = \Pr[\mu_W(W^\prime) \geq 0]\, 
    \leq \Pr[\mu_{B}(B^\prime) \geq 0]
\end{align*}
where the inequality comes from the definition of stochastic dominance. 
A bound on the right-hand side 
is obtained in Bound~\ref{bound:analytic}. 
\hfill $\qed$






In Appendix~\ref{sec:martingale-proof}, 
we present a weaker bound 
on $\Pr[\mu_x(y) \geq 0]$ where the sequence 
$x_1, \ldots, x_m, y_1, \ldots, y_k$ satisfies $\epsilon$-martingale conditions. 
The proof directly uses the properties of the martingale 
and Azuma's inequality  but 
it does not use a stochastic dominance argument. 
Although it gives a bound of $3 \exp\left( -\epsilon^4 (1 - O(\epsilon) ) k/64 \right)$, 
the reader might find the proof of independent interest. 


%
% The following lemma shows that $W$ is, in fact, 
% dominated by $B = (B_1, \ldots, B_k)$ where 
% each $B_i$ is an independent Bernoulli random variable 
% with parameter $(1 - \epsilon)/2$.
%
%--------------------------------
% \begin{lemma}\label{lem:dominance}
%   Let $X = (X_1, \ldots, X_n)$ be a family of random variables taking
%   values in $\{0,1\}$ with the property that, for each $i > 0$,
%   $\Exp[X_i \mid X_1, \ldots, X_{i-1}] \leq p$. Let
%   $B = (B_1, \ldots, B_n)$ be a family of independent random
%   variables, taking values in $\{0,1\}$, for which
%   $\Exp[B_i = 1] = p$.  Then $X \dominatedby B$.
% \end{lemma}
%
% \begin{proof}
%   We proceed by induction. The statement is clear for $n=1$. In
%   general, consider a random variable $X$ satisfying the conditions of
%   the theorem and taking values in $\{0,1\}^{n+1}$; let
%   $E \subset \{0,1\}^{n+1}$ be a monotone event. We wish to prove that
%   $\Pr[X \in E] \leq \Pr[B \in E]$.
%
%   We write $X = (Y, Z)$, where $Y$ takes values in $\{0,1\}^n$ and $Z$
%   in $\{0,1\}$. By induction, we may assume that
%   $Y \dominatedby (B_1, \ldots, B_n)$. Consider the events
%   \[
%     E_0 = \{ (y_1, \ldots, y_n) \mid (y_1, \ldots, y_n, 0) \in
%     E\}\qquad\text{and}\qquad E_1 = \{ (y_1, \ldots, y_n) \mid (y_1,
%     \ldots, y_n, 1) \in E\}\,;
%   \]
%   observe that the monotonicity of $E$ implies that
%   $E_0 \subseteq E_1$ and that $E_0, E_1$ are monotone. 
%   To study
%   $\Pr[X \in E]$, for an element
%   $y = (y_1, \ldots, y_n) \in \{0,1\}^n$ define
%   \[
%     q(y) = \Pr[ X \in E \mid Y = y]\,.
%   \]
%   Observe that $\Pr[X \in E] = \Exp[q(Y)]$ and, recalling that
%   $E_0 \subset E_1$, that
%   \begin{align*}
%     y \in E_0 & \Rightarrow q(y) = 1,\\
%     y \in E_1 \setminus E_0 & \Rightarrow q(y) \leq p\quad\text{by assumption, and}\\
%     y \not\in E_1 & \Rightarrow q(y) = 0\,.
%   \end{align*}
%   We conclude that
%   \begin{align}
%     \Pr[X \in E] & \leq \Pr[Y \in E_0] + p\cdot
%                    \Pr[Y \in E_1 \setminus E_0]
%                    = p \Pr[Y \in E_1] + (1-p) \Pr[Y \in E_0] \nonumber\\
%                  & \leq p \Pr[(B_1, \ldots, B_n) \in E_1] + (1-p) \Pr[(B_1, \ldots, B_n) \in E_0] \label{eq:XBdominance}\\
%                  &= \Pr[(B_1, \ldots, B_n) \in E_0] + p \Pr[(B_1, \ldots, B_n) \in E_1 \setminus E_0] = \Pr[B \in E]\,, \nonumber
%   \end{align}
%   as desired. The inequality of line~\eqref{eq:XBdominance} follows by
%   the induction hypothesis.
% \end{proof}
% ---------------------------------




\subsection{Proof of main theorems}\label{sec:thm-proofs}

\paragraph{Proof of Theorem~\ref{thm:main}.}

Let us start with the following observation.
It allows us to formulate the
$(s, k)$-settlement insecurity of a distribution $\Distribution$
directly in terms of the relative margin.

\begin{lemma}\label{lemma:settlement-margin}
  Let $s, k, T \in \NN$. 
  Let $\Distribution$ be any distribution on $\{0,1\}^T$. 
  Then
  \[
    \mathbf{S}^{s,k}[\Distribution] \leq
      \Pr_{w \sim \Distribution} \left[\parbox{65mm}{
          there is a decomposition $w = x y z$, 
          where $|x| = s - 1$ and $|y| \geq k + 1$, 
          so that $\mu_x(y) \geq 0$
      }\right]
    \,.
  \]
\end{lemma}
\begin{proof}
  Lemma~\ref{lem:main-forks} implies that 
  $\mathbf{S}^{s,k}[\Distribution]$ is no more than 
  the probability that slot $s$ is not $k$-settled 
  for the characteristic string $w$. 
  By Observation~\ref{obs:settlement-balanced-fork}, 
  this probability, in turn, is no more than 
  the probability that there exists an $x$-balanced fork 
  $F \Fork xy$
  where we write $w = xyz, |x| = s - 1, |y| \geq k + 1, |z| \geq 0$. 
  Finally, Fact~\ref{fact:margin-balance} states that 
  for any characteristic string $xy$, 
  the two events ``exists an $x$-balanced fork $F \Fork xy$'' 
  and ``$\mu_x(y)$ is non-negative'' have the same measure. 
  Hence the claim follows. 
\end{proof}

If the distribution $\mathcal{D}$ in the lemma above 
satisfies the $\epsilon$-martingale condition, 
the probability in this lemma is no more than the probability 
in the left-hand side of Corollary~\ref{cor:main}. 
Finally, by retracing the proof of Corollary~\ref{cor:main} 
using the explicit probability from Bound~\ref{bound:geometric}, 
we see that the bound in Corollary~\ref{cor:main} is 
$O(1) \cdot \exp\bigl(-\Omega(\epsilon^3 (1 - O(\epsilon))k)\bigr)$. 
Since $\mathcal{B}_\epsilon$ satisfies the $\epsilon$-martingale condition, 
we conclude that $\mathbf{S}^{s,k}[\mathcal{B}_\epsilon]$ is no more than 
this quantity as well.
% \[
%   \mathbf{S}^{s,k}[\mathcal{B}_\epsilon] 
%       \leq O(1) \cdot \exp\bigl(-\Omega(\epsilon^3 (1 - O(\epsilon))k)\bigr)
%     \,.
% \]


For any player playing the settlement game, 
the set of strings on which the player wins is monotone 
with respect to the partial order $\leq$ defined in Section~\ref{sec:martingale-proof-new}. 
To see why, note that if the adversary wins with a specific string $w$, 
he can certainly win with any string $w^\prime$ where $w \leq w^\prime$. 
As $\mathcal{B}_\epsilon$ stochastically dominates $\mathcal{W}$, it follows that 
$
  \mathbf{S}^{s,k}[\mathcal{W}] \leq \mathbf{S}^{s,k}[\mathcal{B}_\epsilon]
$.

\hfill$\qed$


\paragraph{Proof of Theorem~\ref{thm:main-CP}}
For the first inequality, observe that if $w$ violates $\kCP$, it must violate $\kSlotCP$ as well. 
It remains to prove the second inequality. 
Let $\Distribution$ be any distribution on $\{0,1\}^T$. 
We can apply Fact~\ref{fact:margin-balance} on the statement of Theorem~\ref{thm:cp-fork} 
to deduce that 
\begin{equation*}\label{eq:main-argument-1}
  \Pr_{w \sim \Distribution}[\text{$w$ violates $\kSlotCP$}] 
    \leq 
% \Pr_{\substack{w = x y z \\ |y| \geq k + 1} }[\text{exists a $k$-settlement violation} ] 
%    = 
    \Pr_{w \sim \Distribution}\left[\parbox{55mm}{
      there is a decomposition $w = xyz$, 
      where $|y| \geq k$, 
      so that $\mu_x(y) \geq 0$ 
    } \right] 
    \,.
\end{equation*}
By using a union bound over $|x|$, the above probability is at most 
\[
    \sum_{s = 1}^{T - k + 1} 
    \quad 
      \Pr_w\left[\parbox{60mm}{
        there is a decomposition $w = xyz$, 
        where $|x| = s - 1$ and $|y| \geq k$, 
        so that $\mu_x(y) \geq 0$ 
      }\right] 
    \,.
\]
Since $w$ satisfies the $\epsilon$-martingale condition, 
we can upper bound the probability inside the sum 
using Corollary~\ref{cor:main}. 
As we have seen in the proof of Theorem~\ref{thm:main}, 
the bound in Corollary~\ref{cor:main} is 
$
  O(1) \cdot \exp\bigl(-\Omega(\epsilon^3 (1 - O(\epsilon))k)\bigr)
  % \,.
$.
It follows that the sum above is at most $T \exp\bigl(-\Omega(\epsilon^3 (1 - O(\epsilon))k)\bigr)$.
\hfill $\qed$


It remains to prove the recursive formulation of the relative margin 
from Section~\ref{sec:recursion}; 
we tackle it in the next section.



%%% Local Variables:
%%% mode: latex
%%% TeX-master: "main"
%%% End:
