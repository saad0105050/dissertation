We rely on the elementary framework of forks and margin
from~\citet{KRDO17}. We restate and briefly discuss the pertinent
definitions below. With these basic notions behind us, we then define
a new ``relative'' notion of margin, which will allow us to
significantly improve the efficacy of these tools for reasoning about
settlement times.
%In particular, these tools will allow us to reason
%about the possibility that an adversary can produce two alternate
%histories of the blockchain that diverge prior to a particular block.

Recall that for a given execution of the protocol, we record the
result of the leader election process via a \emph{characteristic
  string} $w \in \{0,1\}^T$, defined such that $w_i = 0$ when a unique
and honest party is assigned to slot $i$ and $w_i = 1$ otherwise.
A vertex of a fork is said to be \emph{honest}
  if it is labeled with an index $i$ such that $w_i=0$.

\begin{definition}[Tines, length, and height]
  Let $F \vdash w$ be a fork for a characteristic string.  A
  \emph{tine} of $F$ is a directed path starting from the root. For
  any tine $t$ we define its \emph{length} to be the number of edges
  in the path, and for any vertex $v$ we define its \emph{depth} to be
  the length of the unique tine that ends at $v$. 
  If a tine $t_1$ is a strict prefix of another tine $t_2$, we write $t_1 \Prefix t_2$. 
  Similarly, if $t_1$ is a non-strict prefix of $t_2$, we write $t_1 \PrefixEq t_2$.
  The longest common prefix of two tines $t_1, t_2$ is denoted by $t_1 \Intersect t_2$. 
  That is, $\ell(t_1 \Intersect t_2) = \max\{\ell(u) \SuchThat \text{$u \PrefixEq t_1$ and $u \PrefixEq t_2$} \}$. 
  The \emph{height} of
  a fork (as usual for a tree) is the length of the longest tine,
  denoted $\height(F)$. 
\end{definition}

\begin{definition}[The $\sim_x$ relations]
  For two tines $t_1$ and $t_2$ of a fork $F$, we write $t_1 \sim t_2$
  when $t_1$ and $t_2$ share an edge; otherwise we write
  $t_1 \nsim t_2$. We generalize this equivalence relation to reflect
  whether tines share an edge over a particular suffix of $w$: for
  $w = xy$ we define $t_1 \sim_x t_2$ if $t_1$ and $t_2$ share an edge
  that terminates at some node labeled with an index in $y$;
  otherwise, we write $t_1 \nsim_x t_2$ (observe that in this case the
  paths share no vertex labeled by a slot associated with $y$).  We
  sometimes call such pairs of tines \emph{disjoint} (or, if
  $t_1 \nsim_x t_2$ for a string $w = xy$, \emph{disjoint over
    $y$}). Note that $\sim$ and $\sim_\varepsilon$ are the same
  relation.
\end{definition}

%Informally, $t_1\sim_x t_2$ indicates that when we restrict our view of history to only blocks \emph{after}
%the prefix $x$, $t_1$ and $t_2$ share an edge (and thus agree on at least one block after that point).

The basic structure we use to use to reason about settlement times is
that of a ``balanced fork.''

\begin{definition}[Balanced fork; cf.\ ``flat'' in \cite{KRDO17}]\label{def:balanced-fork} A
  fork $F$ is \emph{balanced} if it contains a pair of tines $t_1$ and
  $t_2$ for which $t_1\nsim t_2$ and
  $\length(t_1)=\length(t_2)=\height(F)$. We define a relative notion
  of balance as follows: a fork $F \vdash xy$ is \emph{$x$-balanced}
  if it contains a pair of tines $t_1$ and $t_2$ for which
  $t_1 \not\sim_x t_2$ and $\length(t_1) = \length(t_2) = \height(F)$.
\end{definition}

Thus, balanced forks contain two completely disjoint, maximum-length
tines, while $x$-balanced forks contain two maximum-length tines that
may share edges in $x$ but must be disjoint over the rest of the
string. 
See Figures~\ref{fig:balanced} and~\ref{fig:x-balanced} 
for examples of balanced forks.
\begin{figure}[ht]
  \centering
  \begin{tikzpicture}[>=stealth', auto, semithick,
    honest/.style={circle,draw=black,thick,text=black,double,font=\small},
   malicious/.style={fill=gray!10,circle,draw=black,thick,text=black,font=\small}]
    \node at (0,-2) {$w =$};
  \node at (1,-2) {$0$}; \node[honest] at (1,-.7) (b1) {$1$};
  \node at (2,-2) {$1$}; \node[malicious] at (2,.7) (a1) {$2$};
  \node at (3,-2) {$0$}; \node[honest] at (3,.7) (a2) {$3$};
  \node at (4,-2) {$1$}; \node[malicious] at (4,-.7) (b2) {$4$};
  \node at (5,-2) {$0$}; \node[honest] at (5,-.7) (b3) {$5$};
  \node at (6,-2) {$1$}; \node[malicious] at (6,.7) (a3) {$6$};
    \node[honest] at (-1,0) (base) {$0$};
  \draw[thick,->] (base) to[bend left=10] (a1);
      \draw[thick,->] (a1) -- (a2);
      \draw[thick,->] (a2) -- (a3);
  \draw[thick,->] (base) to[bend right=10] (b1);
      \draw[thick,->] (b1) -- (b2);
      \draw[thick,->] (b2) -- (b3);
    \end{tikzpicture} 
  \caption{A balanced fork}
  \label{fig:balanced}
\end{figure}

\begin{figure}[ht]
  \centering
  \begin{tikzpicture}[>=stealth', auto, semithick,
    honest/.style={circle,draw=black,thick,text=black,double,font=\small},
   malicious/.style={fill=gray!10,circle,draw=black,thick,text=black,font=\small}]
    \node at (0,-2) {$w =$};
  \node at (1,-2) {$0$}; \node[honest] at (1,0) (ab1) {$1$};
  \node at (2,-2) {$0$}; \node[honest] at (2,0) (ab2) {$2$};
  \node at (3,-2) {$0$}; \node[honest] at (3,.7) (a3) {$3$};
  \node at (4,-2) {$1$}; \node[malicious] at (4,-.7) (b3) {$4$};
  \node at (5,-2) {$0$}; \node[honest] at (5,-.7) (b4) {$5$};
  \node at (6,-2) {$1$}; \node[malicious] at (6,.7) (a4) {$6$};
    \node[honest] at (-1,0) (base) {$0$};
  \draw[thick,->] (base) -- (ab1);
  \draw[thick,->] (ab1) -- (ab2);
  \draw[thick,->] (ab2) to[bend left=10] (a3);
    \draw[thick,->] (a3) -- (a4);
  \draw[thick,->] (ab2) to[bend right=10] (b3);
    \draw[thick,->] (b3) -- (b4);
    \end{tikzpicture} 
  \caption{An $x$-balanced fork, where $x=00$}
  \label{fig:x-balanced}
\end{figure}


\Paragraph{Balanced forks and settlement time.}
A fundamental question arising in typical blockchain settings is how
to determine \emph{settlement time}, the delay after which the
contents of a particular block of a blockchain can be considered
stable. The existence of a balanced fork is a precise indicator for
``settlement violations'' in this sense. Specifically, consider a
characteristic string $xy$ and a transaction appearing in a block
associated with the first slot of $y$ (that is, slot $|x| + 1$). One
clear violation of settlement at this point of the execution is the
existence of two chains---each of maximum length---which diverge
\emph{prior to $y$}; in particular, this indicates that there is an
$x$-balanced fork $F$ for $xy$. Let us record this observation below.

\begin{observation}\label{obs:settlement-balanced-fork}
  Let $s, k \in \NN$ be given and 
  let $w$ be a characteristic string. 
  Slot $s$ is not $k$-settled for the characteristic string $w$ 
  if 
  there exist a decomposition $w = xyz$, 
  where $|x| = s - 1$ and $|y| \geq k+1$, 
  and an $x$-balanced fork for $xy$. 
\end{observation}

In fact, every $\kSlotCP$ violation produces a balanced fork as well;
see Theorem~\ref{thm:cp-fork} in \Section~\ref{sec:cp-forks}.  In
particular, to provide a rigorous $k$-slot settlement
guarantee---which is to say that the transaction can be considered
settled once $k$ slots have gone by---it suffices to show that with
overwhelming probability in choice of the characteristic string
determined by the leader election process (of a full execution of the
protocol), no such forks are possible. Specifically, if the protocol
runs for a total of $T$ time steps yielding the characteristics string
$w = xy$ (where $w \in \{0,1\}^T$ and the transaction of interest
appears in slot $|x| + 1$ as above) then it suffices to ensure that
there is no $x$-balanced fork for $x\hat{y}$, where $\hat{y}$ is an
arbitrary prefix of $y$ of length at least $k + 1$; see
Corollary~\ref{cor:main} in \Section~\ref{sec:estimates}.  Note that
for systems adopting the longest chain rule, this condition must
necessarily involve the \emph{entire future dynamics} of the
blockchain. We remark that our analysis below will in fact let us take
$T = \infty$.

\begin{definition}[Closed fork]
A fork $F$ is \emph{closed} if every leaf is honest. For convenience, we say the trivial fork is closed.
\end{definition}

Closed forks have two nice properties that make them especially useful in reasoning about the view of honest parties.
First, a closed fork must have a unique longest tine (since honest parties are aware of all previous honest blocks, and honest
parties observe the longest chain rule). Second, recalling our description of the
settlement game, closed forks intuitively capture decision points for the adversary.
The adversary can potentially show many tines to many honest parties, but once an honest node has been placed on top of 
a tine, any adversarial blocks beneath it are part of the public record and are visible to all honest parties. For these
reasons, we will often find it easier to reason about closed forks than arbitrary forks. % (without loss of generality).

The next few definitions are the start of a general toolkit for reasoning about an adversary's capacity to build highly diverging paths in forks, based on the underlying characteristic string.
%current state of a fork.

%%%%Reach and margin
\begin{definition}[Gap, reserve, and reach]\label{def:gap-reserve-reach}
For a closed fork $F \vdash w$ and its unique longest tine $\hat{t}$, we define the \emph{gap} of a tine $t$ to be $\gap(t)=\length(\hat{t})-\length(t)$.
Furthermore, we define the \emph{reserve} of $t$, denoted $\reserve(t)$, to be the number of adversarial indices in $w$ that appear after the terminating vertex of $t$. More precisely, if $v$ is the last vertex of $t$, then
\[
  \reserve(t)=|\{\ i \mid w_i=1 \ and \ i > \ell(v)\}|\,.
  \]
These quantities together define the \emph{reach} of a tine: $
\reach(t)=\reserve(t)-\gap(t)$.
\end{definition}

The notion of reach can be intuitively understood as a measurement of
the resources available to our adversary in the settlement
game. Reserve tracks the number of slots in which the adversary has
the right to issue new blocks.  When reserve exceeds gap (or
equivalently, when reach is nonnegative), such a tine could be
extended---using a sequence of dishonest blocks---until it is as long
as the longest tine. Such a tine could be offered to an honest player
who would prefer it over, e.g., the current longest tine in the
fork. In contrast, a tine with negative reach is too far behind to be
directly useful to the adversary at that time.

\begin{definition}[Maximum reach]
For a closed fork $F\vdash w$, we define $\rho(F)$ to be the largest reach attained by any tine of $F$, i.e., 
\[
\rho(F)=\underset{t}\max \ \reach(t)\,.
\]
Note that $\rho(F)$ is never negative (as the longest tine of any fork always has reach at least 0). We overload this notation to denote the maximum reach over all forks for a given characteristic string: 
\[
\rho(w)=\underset{\substack{F\vdash w\\\text{$F$ closed}}}\max\big[\underset{t}\max \ \reach(t)\big]\,.
\]
\end{definition}

\begin{definition}[Margin]\label{def:margin}
The \emph{margin} of a fork $F\vdash w$, denoted $\mu(F)$, is defined as 
\begin{equation}\label{eq:margin-absolute}
\mu(F)=\underset{t_1\nsim t_2}\max \bigl(\min\{\reach(t_1),\reach(t_2)\}\bigr)\,,
\end{equation}
where this maximum is extended over all pairs of disjoint tines of
$F$; thus margin reflects the ``second best'' reach obtained over all
disjoint tines. In order to study splits in the chain over particular portions of a
string, we generalize this to define a ``relative'' notion of margin:
If $w = xy$ for two strings $x$ and $y$ and, as above, $F \vdash w$,
we define
\[
  \mu_x(F)=\underset{t_1\nsim_x t_2}\max \bigl(\min\{\reach(t_1),\reach(t_2)\}\bigr)\,.
\]
Note that $\mu_\varepsilon(F) = \mu(F)$.

For convenience, we once again overload this notation to denote the
margin of a string. $\mu(w)$ refers to the maximum value of $\mu(F)$
over all possible closed forks $F$ for a characteristic string $w$:
\[
\mu(w)=\underset{\substack{F\vdash w,\\ \text{$F$ closed}}}\max \, \mu(F)\,.
\]
Likewise, if $w = xy$ for two strings $x$ and $y$ we define
\[
\mu_x(y)=\underset{\substack{F\vdash w,\\ \text{$F$ closed}}} \max \, \mu_x(F)\,.
\]
%(Cf.~\cite{KRDO17}, which defined and studied the ``absolute'' version
%$\mu(\cdot)$ of this quantity of~\eqref{eq:margin-absolute}.)
\end{definition}
Note that, at least informally, ``second-best'' tines are of natural
interest to an adversary intent on the construction of an $x$-balanced fork, 
which involves two (partially disjoint) long tines.


\Paragraph{Balanced forks and relative margin.}
\citet{KRDO17} showed that a balanced fork can be constructed for a
given characteristic string $w$ if and only if there exists some
closed $F\vdash w$ such that $\mu(F)\geq 0$.  We record a relative
version of this theorem below, which will ultimately allow us to
extend the analysis of \cite{KRDO17} to more general class of
disagreement and settlement failures.

\begin{fact}\label{fact:margin-balance}
  Let $xy \in \{0,1\}^n$ be a characteristic string. Then there is an
  $x$-balanced fork $F \vdash xy$ if and only if $\mu_x(y) \geq 0$.
\end{fact}

\begin{proof}
  The proof is immediate from the definitions. We sketch the details for completeness.
  
  Suppose $F$ is an $x$-balanced fork for $xy$. Then $F$ must contain a pair of tines $t_1$ and $t_2$ for which
  $t_1 \not\sim_x t_2$ and $\length(t_1) = \length(t_2) = \height(F)$. We observe that (1) $\gap(t_i)=0$ for both $t_1$ and $t_2$, and (2) reserve is always a nonnegative quantity. Together with the definition of $\reach$, these two facts immediately imply $\reach(t_i) \geq 0$. Because $t_1$ and $t_2$ are edge-disjoint over $y$ and $\min\{\reach(t_1),\reach(t_2)\}\geq0,$ we conclude that $\mu_x(y)\geq 0$, as desired. 

  Suppose $\mu_x(y)\geq 0$. Then there is some closed fork $F$ for $xy$ such that $\mu_x(F)\geq0$. By the definition of
   relative margin, we know that $F$ has two tines $t_1$, $t_2$ such that $t_1\nsim_x t_2$ and 
  $\reach(t_i)\geq0$. Recall that we define reach by $\reach(t)=\reserve(t)-\gap(t)$, and so in this case 
 it follows that $\reserve(t_i) - \gap(t_i)\geq0$. Thus, an $x$-balanced fork $F'\vdash xy$ can be constructed from $F$ by 
 appending a path of $\gap(t_i)$ adversarial vertices to each $t_i$.
\end{proof}

As indicated above, we can define the ``forkability'' of a
characteristic string in terms of its margin.
\begin{definition}[Forkable strings]\label{def:forkable}
  A charactersitic string $w$ is \emph{forkable} if its margin is non-negative, i.e., $\mu(w) \geq 0$.
  Equivalently, $w$ is forkable if there is a balanced fork for $w$.
\end{definition}
Although this definition is not necessary for our presentation, it
reflects the terminology of existing literature.

%%% Local Variables:
%%% mode: latex
%%% TeX-master: "main"
%%% End: