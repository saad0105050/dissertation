
A blockchain is a data structure consisting of a collection of data
blocks placed in linear order. It further requires that
% It has the characteristic that
each block contains a collision-free hash of the previous block: thus
blocks implicitly commit to the entire prefix of the blockchain
preceding them. This elementary data structure has remarkable
applications in distributed computing, and now appears as an essential
component of consensus protocols in a wide variety of models and
settings; this notably includes both the ``permissionless'' setting
popularized by Bitcoin and the classic ``permissioned'' model.

Such consensus protocols call for players to collaboratively assemble
a blockchain by repeatedly selecting players to add blocks.
%A blockchain is a hash-chain that is collaboratively constructed by a
%set of players.
%
%New blocks are added to the chain by players chosen
Specifically, the protocol determines a stochastic process resembling
a lottery: each ``leader'' selected by the lottery is then responsible
for broadcasting a new block. While the algorithmic details of this
lottery depend heavily on the protocol, the outcome can be privately
determined and provides the winning player a proof of leadership that
can be publicly demonstrated. Assuming that the expected wait time for
some player to win the lottery is constant, the blockchain experiences
steady growth when players follow the protocol.

Network infelicities, adversarial behavior, or the possibility that
two players simultaneously win the lottery can lead to disagreements
among the players about the current blockchain. Thus protocols adopt a
``chain selection rule'' that determines how players should break ties
among the various chains they observe on the network; ideally, the
combination of the chain selection rule and the lottery should
guarantee that the players' blockchains agree, perhaps with the
exception of a short suffix. The emblematic chain selection strategy
among such systems is the \emph{longest-chain rule}, which calls for
players to adopt the longest chain among various contenders.

%Note that each player runs its own
%lottery attempting to extend a certain hash-chain it is aware of.
%Players choose which hash-chain to extend via some greedy criterion
%such as picking the longest hash-chain they have seen thus far.
%
%Implementing the lottery mechanism and the criterion for
%chain selection yields a particular blockchain protocol instantiation. 

The first blockchain protocol was the core of the sensational Bitcoin
system~\cite{Nakamoto2008}; it adopted a lottery mechanism based on a
cryptographic puzzle~\cite{C:DwoNao92,hashcash}---also
known as \emph{proof-of-work or PoW, for short}---and a chain
selection rule favoring chains that represent more work. The system is
particularly notable for its ability to survive in a
permissionless setting---where players may freely join and
depart---even when a portion of the players are actively attacking the
protocol. Unfortunately, the proof-of-work mechanism makes quite
striking energy demands: the system currently consumes as much
electricity as a small country.\footnote{See e.g., 
\url{https://digiconomist.net/bitcoin-energy-consumption} where it is reported
that Bitcoin 
annual energy consumption is on the order of at least 50 Twhr at the time of writing. } This motivated the blockchain
community to exploring alternative lottery mechanisms, e.g.,
proof-of-stake
(PoS)~\cite{DBLP:journals/corr/BentovGM14,DBLP:conf/asiacrypt/PassS17,KRDO17},
proof of
space~\cite{C:DFKP15,DBLP:journals/iacr/ParkPAFG15}
and others~\cite{cryptoeprint:2016:035}. The proof-of-stake mechanism
is particularly attractive from the perspective of efficiency, as it
makes no assumption of external computational resources.

The fundamental consistency property---critical in all these
blockchain systems---is \emph{common-prefix}
(cf.~\cite{DBLP:conf/eurocrypt/GarayKL15}). It precisely captures the
intuition described above: by trimming a $k$-block suffix from the
chain held by any honest player the resulting blockchain is a prefix
of the blockchain possessed by any honest party at any future point of
the execution. A principal goal in the analysis of these systems is a
to guarantee common prefix, for an appropriate value of $k$, even if
some of the players collude to disrupt the protocol. Common prefix is
typically only shown to hold with high probability $1-\varepsilon$,
where $\varepsilon$ is an error term that is a function of
$k$. The exact dependency of $\varepsilon$ on $k$ is critically
important: it determines the length of the suffix that is to be
removed from a blockchain in order to ensure that the remaining prefix
will be retained at any future point of the execution. This directly
imposes a lower bound on how long one has to wait for information in
the blockchain (such as a payment transaction) to ``settle.''
Additionally, many blockchain protocols internally rely on common
prefix for correctness; thus the relationship between $\varepsilon$ and
$k$ is critical to establishing the regime of correctness of the
entire protocol.

%Under mild assumptions one must have
A relatively straightforward lower bound for $\varepsilon$ is 
$\varepsilon \geq \exp(-\alpha k )$ for some $\alpha>0$.  This lower
bound applies when there is a coalition of adversarial players of
constant fraction,
% that may be adversarial which is
the case of primary interest in practice. The result is easy to infer
from the analysis of~\cite{Nakamoto2008}, where a strategy is
demonstrated that violates common prefix with such probability (this
is referred to as a ``double-spending'' attack in that paper). The
tightness of this bound is an important open problem. For the special
case of proof-of-work an upper bound of $\exp(-\Omega(k))$ was shown
first in \cite{DBLP:conf/eurocrypt/GarayKL15} and further verified in
extended security models
by~\cite{DBLP:conf/crypto/GarayKL17,DBLP:conf/eurocrypt/PassSS17}.
%(here the parameter $T$ corresponds to the total run-time of the
%blockchain).
In the proof-of-stake setting, on the other hand, the tightness of the
bound remains open.  While recent proof-of-stake algorithms have been
presented with rigorous analyses that rival proof-of-work in many
regards, they suffer from a quadratic relationship between $k$ and
$\log(\varepsilon)$. For example, the Ouroboros
protocols~\cite{KRDO17,DBLP:conf/eurocrypt/DavidGKR18,DBLP:journals/iacr/BadertscherGKRZ18},
as well as Snow White~\cite{DBLP:journals/iacr/BentovPS16a}, provide
an upper bound on $\varepsilon$ of $\exp(-\Omega(\sqrt{k}))$; this
should be compared with $\varepsilon = \exp(-\Theta(k))$ for
proof-of-work. The significant gap from the known lower bound was
attributed to a notable, general attack that distinguished PoS from
PoW: Known as the \emph{nothing-at-stake} problem, this refers to the
ability of an adversarial coalition of players to strategically reuse
a winning PoS lottery to extend multiple blockchains.


\ignore{%%%%%%  
The success of Bitcoin and, in general, usage of blockchains for
supporting and archiving the results of consensus has led to
a concerted effort to develop rigorous formal tools for reasoning
about blockchain dynamics. These efforts were motivated both by the
Bitcoin proof-of-work blockchain itself and the desire for alternative
blockchain protocols that are organized around other resources (e.g.,
proof-of-space, proof-of-stake, etc.).
% In particular,
% Bitcoin adopts a ``proof-of-work'' convention which demands that
% participants compete in a continuing race to solve computational
% problems. This mechanism is remarkably successful, trusted, and
% well-understood; on the other hand, it demands unsatisfactory resource
% commitment that grows with the population and commitment of its
% participants. One influential counter-proposal calls for participants
% to be weighted by their stake; a quantity determined by the blockchain
% itself.
In this article, we establish rigorous, quantitative bounds on the
time necessary for transactions to settle for a broad family of
blockchain protocols adopting the \emph{longest-chain rule}, notably
including proof-of-stake blockchains such as 
Snow White~\cite{DBLP:journals/iacr/BentovPS16a} and 
the Ouroboros family~\cite{KRDO17,DBLP:conf/eurocrypt/DavidGKR18,DBLP:journals/iacr/BadertscherGKRZ18}. 
The principal feature of our new analysis is that it applies to
\emph{proof-of-stake} based blockchain systems,
which must contend
with challenging adversarial behavior that does not exist in
proof-of-work systems:
\begin{itemize}
\item \emph{Nothing at stake attacks.} When an adversary has the right
  to extend a proof-of-stake blockchain, he may produce many different
  blocks that extend different chains of the system or, similarly,
  yield many different extensions of a particular longest chain---this
  corresponds to ``nothing at stake'' attacks that can permit an
  adversary to construct, on-the-fly, competing blockchains at no
  cost. In contrast, the \emph{total} number of blocks produced by a
  minority adversary in a proof-of-work based system is dominated by
  the number of blocks that are honestly produced, and constructing additional blocks
  requires additional work.
\item \emph{Known leader schedules.} In some proof-of-stake based
  blockchains, (part or all of) the future schedule of participants
  permitted to add to the chain is public. In contrast, the right to
  add a block in proof-of-work settings is determined in a stochastic
  online fashion that does not permit the adversary significant
  ``look-ahead''.
\end{itemize}
}%%%%%%%%%%%%%%%%

\paragraph{Our results.}
Our objective is to control the common-prefix error
$\varepsilon$ as tightly as possible while
making minimal assumptions on the underlying blockchain
protocol.  We work in a general model formulated by a simple family of
\emph{blockchain axioms}. The axioms themselves are easy to interpret
and few in number. This permits us to abstract many features of the
underlying blockchain protocol (e.g., the details of the
leader-election process, the cryptographic security of the relevant
signature schemes and hash functions, and randomness generation),
while still establishing results that are strong enough to directly
incorporate into existing specific analyses.
%%(For example, our techniques can directly improve
%the analysis of Snow White, Ouroboros, Ouroboros Praos, and Ouroboros
%Genesis.)

Our most interesting finding is a quite tight theory of common prefix
that depends \emph{only on the schedule of participants certified to
  add a block}. Under common assumptions about this schedule, we
achieve the optimal relationship $\varepsilon =
\exp(-\Theta(k))$. This directly improves the common prefix guarantees
(and settlement times) of existing proof-of-stake blockchains such as
Snow White~\cite{DBLP:journals/iacr/BentovPS16a},
Ouroboros~\cite{KRDO17}, Ouroboros
Praos~\cite{DBLP:conf/eurocrypt/DavidGKR18}, and Ouroboros
Genesis~\cite{DBLP:journals/iacr/BadertscherGKRZ18}.
%
%analysis of~\cite{KRDO17}, which established that the probability of
%a ``depth-$k$'' settlement failure at time $T$ was no more than
%$T \exp(-\Omega(\sqrt{k}))$. Our new techniques establish
%that the probability of a settlement failure at time $T$ is no more
%than $\exp(-\Omega(k))$. 
%%Note that this is independent of the
%%the time elapsed $T$, 
Specifically, this improves the scaling in the exponent from
$\sqrt{k}$ to $k$ and establishes a tight characterization for
$\varepsilon = \exp(-\Theta(k))$. (In fact, we even obtain reasonable
control of the constants.)  We remark that our assumptions about the
schedule distribution can be weakened---without any effect on the
final bounds---to apply to martingale-style distributions such
as those that arise in the analysis of adaptive
adversaries~\cite{DBLP:conf/eurocrypt/DavidGKR18,DBLP:journals/iacr/BadertscherGKRZ18}.
%While we discuss this in detail later, we remark that
%this is important for applying our techniques to security proofs
%involving adaptive adversaries.% and, in particular, to the
%analysis of the Ourorboros Genesis protocol.

Our new analysis offers an additional, but lower order, improvement for
several of these blockchains. The existing analysis of, e.g.,
Ouroboros Praos~\cite{DBLP:conf/eurocrypt/DavidGKR18}, required a
union bound to be taken over the entire lifetime of the protocol in
order to rule out a common prefix violation at a particular point of
time; thus such events were actually bounded above by a function of
the form $T \exp(-\Omega(\sqrt{k}))$, where $T$ is the lifetime of the
protocol. While this event \emph{does} depend on the entire dynamics
of the protocol, we show how to avoid this pessimistic tail bound to
achieve a ``single shot'' common prefix violation---at a particular
time of interest---of form $\exp(-\Theta(k))$; this removes the
dependence on $T$.

From a technical perspective, we contrast the structure of our proofs
with existing techniques for the PoW case. The PoW results find a
direct connection between common-prefix and the behavior of a biased,
one-dimensional random walk. Interestingly, our results give a tight
relationship between the general (e.g., PoS) case and a pair of
\emph{coupled} biased random walks. A major challenge in the analysis
is to bound the behavior of this richer stochastic process.  Our tools
yield precise, explicit upper bounds on the probability of persistence
violations that can be directly applied to tune the parameters of
deployed PoS systems. See Appendix~\ref{sec:exact-prob} where we
record some concrete results of the general theory. The importance of
these results in the practice of PoS blockchain systems cannot be
overstated: they provide, for the first time, concrete error bounds
for settlement times for PoS blockchains that follow the longest chain
rule.



\paragraph{Further analytic details.} Our approach begins with the
graph-theoretic framework of \emph{forks} and \emph{margin} developed
for the analysis of the Ouroboros~\cite{KRDO17} protocol.  (A fork is
an abstraction of the protocol execution given the outcomes of the
leader-election process.)  We begin by generalizing the notion of
margin to account for local, rather than global, features of a leader
schedule, and provide an exact, recursive closed form for this new
quantity (see Section~\ref{sec:recursion}). This proof identifies an
optimal online adversary (i.e., a fork-building strategy whose current
decisions do not depend on the future) for PoS blockchain algorithms
%when
%divergence is measured in slots. 
%In addition, we present an \emph{optimal} online adversary 
with the remarkable property that the sequence of forks produced by
this adversary \emph{simultaneously} achieve the worst-case (slot)
common-prefix violations associated with all slots (see
Section~\ref{sec:canonical-forks}). We then study the stochastic
process generated when the \emph{characteristic string}---a Boolean
string representing the outcome of the leader election scheme---is
given by a family of i.i.d.\ Bernoulli random variables. In this case,
we identify a generating function that bounds the tail events off
interest, and analytically upper bound the growth of the function. We
then show how to extend the analysis to the setting where the
characteristic string is drawn from a martingale sequence.  As it
happens, this more general distribution arises naturally in the
analyses of PoS protocols that survive adaptive adversaries; e.g.,
Ouroboros Genesis~\cite{DBLP:journals/iacr/BadertscherGKRZ18}.  We
obtain the pleasing result that the common prefix error probability in
the martingale case is no more than that in the i.i.d.\ Bernoulli
case.

\paragraph{Direct consequences.} 
Our results establish consistency bounds in a quite general
setting---see below: In particular, they directly imply
$\exp(-\Theta(k))$ consistency for the Sleepy consensus (Snow
White)~\cite{DBLP:conf/asiacrypt/PassS17}, Ouroboros~\cite{KRDO17},
Ouroboros Praos~\cite{DBLP:conf/eurocrypt/DavidGKR18}, and Ouroboros
Genesis~\cite{DBLP:journals/iacr/BadertscherGKRZ18} blockchain
protocols. (The Ouroboros Praos and Ouroboros Genesis analyses in fact
directly relied on an earlier e-print version of the present article
for their settlement estimates.)

\paragraph{Related work.} 
Blockchain protocol analysis in the PoW-setting was initiated
in~\cite{DBLP:conf/eurocrypt/GarayKL15} and further improved in
\cite{DBLP:conf/eurocrypt/PassSS17,DBLP:conf/crypto/GarayKL17}.  The
established security bounds for consistency are linear in the security
parameter.  Sleepy
consensus~\cite[Theorem~13]{DBLP:conf/asiacrypt/PassS17} provides a
consistency bound of the form
$\exp(-\Omega(\sqrt{k}))$. %\footnote{We note that the term $T$ stems from a union bound applied over all communication rounds (CHECK XXX).}
Note that~\cite{DBLP:conf/asiacrypt/PassS17} is not a PoS protocol per
se, but it is possible to turn it into one (as was demonstrated in
\cite{DBLP:journals/iacr/BentovPS16a}). The analysis of the Ouroboros
blockchain~\cite{KRDO17} achieves $\exp(-\Omega(\sqrt{k}))$. We remark
that the analyses of Ouroboros
Praos~\cite{DBLP:conf/eurocrypt/DavidGKR18} and Ouroboros
Genesis~\cite{DBLP:journals/iacr/BadertscherGKRZ18} developed
significant new machinery for handling other challenges (e.g.,
adaptive adversaries, partial synchrony), but directly referred to a
preliminary version of this article to conclude their guarantees of
$ \exp(-\Omega(k))$.



%which were only experimentally verified. 

Our results complement the recent results of
\cite{DBLP:journals/corr/abs-1809-06528}, which also considers
longest-chain PoS protocols. \cite{DBLP:journals/corr/abs-1809-06528}
focuses on identifying dynamics unique to longest-chain PoS
protocols. In particular, they show that longest-chain PoS protocols
that are \emph{predictable} (i.e., for which some portion of the
schedule of slot leaders is known ahead of time) are necessarily
vulnerable to ``predictable double-spends.''  The conventional defense
against such attacks is to wait for the specified settlement time to
elapse before accepting a transaction, which (until now) has resulted
in slow confirmation times. As such,
\cite{DBLP:journals/corr/abs-1809-06528} raised the question of
whether long confirmation times are a necessary evil in longest-chain
PoS blockchains.  As double-spending attacks imply a consistency
violation, our results show that PoS protocols can safely decrease
settlement times to asymptotically match PoW protocols without
sacrificing security against double-spends.

Because we focus on the longest-chain rule, our
analysis is not applicable to protocols like
Algorand~\cite{DBLP:journals/corr/Micali16} which, in fact, offer
settlement in expected constant time without invoking blockchain
reorganisation or forks; however, Algorand lacks the ability to
operate in the ``sleepy''~\cite{DBLP:conf/asiacrypt/PassS17} or
``dynamic availability''~\cite{DBLP:journals/iacr/BadertscherGKRZ18}
setting.
% (which permits an evolving population of participants).
In our combinatorial analysis, synchronous operation is assumed
against a rushing adversary; this is without loss of generality
vis-a-vis the result of \cite{DBLP:conf/eurocrypt/DavidGKR18} where it
was shown how to reduce the combinatorial analysis in the partially
synchronous setting to the synchronous one.  We note that a number of
works have shown how to use a blockchain protocol to bootstrap a
cryptographic protocol that can offer faster settlement time under
stronger assumptions than honest majority, e.g., Hybrid
Consensus~\cite{DBLP:conf/wdag/PassS17} or
Thunderella~\cite{DBLP:conf/eurocrypt/PassS18}; our results are
orthogonal and synergistic to those since they can be used to improve
the settlement time bounds of the blockchain protocol that operates as
a fallback mechanism.

\paragraph{Outline.} We begin in Section~\ref{sec:model} by describing
a simple general model for blockchain
dynamics. Section~\ref{sec:definitions} builds on this model to set
down a number of basic definitions required for the proofs. The first
part of the main proof is described in Section~\ref{sec:recursion},
which develops a ``relative'' version of the theory of margin
from~\cite{KRDO17}; most details are then relegated to
Section~\ref{sec:margin-proof} in order to move quickly to the
consistency estimates in Section~\ref{sec:estimates}.  In
Section~\ref{sec:canonical-forks}, we present an optimal online
adversary who can simultaneously maximize the relative margins for all
prefixes of the characteristic string.  \ignore{We then provide two
  different settlement estimates in Section~\ref{sec:estimates};
  roughly, the two bounds trade off generality with the strength of
  the final estimates.} Finally, in Appendix~\ref{sec:exact-prob}, we
compute exact upper bounds on $k$-settlement error probabilities for
various values of $k$ and describe a simple $O(k^3)$-time algorithm to
compute these probabilities in general.
%The \texttt{C++} source code is publicly available 
%at~\href{https://github.com/saad0105050/forkable-strings-code}{https://github.com/saad0105050/forkable-strings-code}~\cite{PrForkableCode}.
 

%%% Local Variables:
%%% mode: latex
%%% TeX-master: "main"
%%% End:
