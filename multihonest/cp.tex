% \subsection{The Common Prefix  (CP) property}\label{sec:cp-model} 
  For the sake of simplicity, 
  assume the synchronous communication model from Section~\ref{sec:game-mh}; 
  the $\Delta$-synchronous setting can be handled in the same way 
  as delineated in \Section~\ref{sec:async-multihonest}.

  The common prefix property with parameter $k$ asserts
  that, for any slot index $s$, if an honest observer at slot $s + k$
  adopts a blockchain $\Chain$, the prefix $\Chain[0 : s]$ will be
  present in every honestly-held blockchain at or after slot $s + k$.
  (Here, $\Chain[0 : s]$ denotes the prefix of the blockchain $\Chain$
  containing only the blocks issued from slots $0, 1, \ldots, s$.)

  We translate this property into the framework of forks.  Consider a
  tine $t$ of a fork $F \vdash w$.  The \emph{trimmed} tine
  $t\TrimSlot{k}$ is defined as the portion of $t$ labeled with slots
  $\{ 0, \ldots, \ell(t) - k\}$. For two tines, we use the notation
  $t_1 \PrefixEq t_2$ to indicate that the tine $t_1$ is a
  prefix of tine $t_2$.

  \begin{definition}[Common Prefix Property with parameter $k \in \NN$]\label{def:cp-slot-mh}
    Let $w$ be a characteristic string. A fork $F \vdash w$ satisfies
    $\kSlotCP$ if, for all pairs $(t_1, t_2)$ of viable tines $F$ for
    which $\ell(t_1) \leq \ell(t_2)$, we have $t_1\TrimSlot{k} \PrefixEq t_2$. 
    Otherwise, we say that the tine-pair $(t_1, t_2)$ is a witness to a $\kSlotCP$ violation.
    Finally, \emph{$w$ satisfies $\kSlotCP$} if every fork $F \vdash w$ satisfies $\kSlotCP$.
    %We
    %denote this property by $\kSlotCP$.
    % Let $\Chain_1$ and $\Chain_2$ be two blockchains adopted by 
    % two (not necessarily distinct) honest players 
    % at the onset of slots $r_1$ and $r_2$, respectively, with $r_1 \leq r_2$. 
    % Then $\Chain_1\TrimSlot{k} \PrefixEq \Chain_2$. 
    % We denote this property by $\kSlotCP$.
  \end{definition} 
  If a string $w$ does not possess the $\kSlotCP$ property, 
  we say that \emph{$w$ violates $\kSlotCP$}.
  Observe that traditionally
  (cf. \cite{GKL17}), 
  the truncated chain 
  % $\Chain\TrimSlot{k}$ 
  is defined in terms of
  deleting a suffix of (block-)length $k$ from $\Chain$. 
  We denote this traditional version of the common prefix property as the
  $\kCP$ property. Note, however, that a $\kCP$ violation immediately
  implies a $\kSlotCP$ violation; hence, bounding the probability of a
  $\kSlotCP$ violation is sufficient to rule out both events.

  \paragraph{Connection with the UVP.}
  Note that 
  if $w$ admits a $\kSlotCP$ violation, 
  then there must be a fork $F$ containing 
  two distinct viable tines $t_1, t_2, \ell(t_1) \leq \ell(t_2)$ 
  so that $\ell(t_1) - \ell(t_1 \Intersect t_2) \geq k + 1$. 
  Then $t_1$ must contain a vertex $v, \ell(t_1 \Intersect t_2) < \ell(v) \leq \ell(t_1) - k$ 
  so that $v$ does not belong to $t_2$. 
  % Recall the UVP from Section~\ref{sec:catalan}.   
  If every substring $x$ of $w$ with $|x| \geq k$, contained a slot with the UVP then 
  we would never have a $\kSlotCP$ violation. 
  Therefore, 
  \begin{align}\label{eq:cp-uvp}
    \parbox{20mm}{\centering $w$ violates $\kSlotCP$}
    &\implies
    \parbox{45mm}{\centering
      $w$ has a substring $y, |y| \geq k$ so that 
      no slot indexed by $y$ has the UVP in $w$.
    }
  \end{align}


  Recall that a uniquely honest Catalan slot has the UVP. 
  This fact allows us to bound 
  the probability of common prefix violations by 
  reasoning only about Catalan slots.\footnote{ 
  One can also prove Theorem~\ref{thm:main-mh-CP} 
  by 
  directly showing---as is done in~\cite{LinearConsistency}---that 
  a $\kSlotCP$ violation implies a $k$-settlement violation 
  and then appealing to Theorem~\ref{thm:main-mh}. 
  The proof of the implication 
  turns out to be quite long and complicated 
  compared to the short proof above; 
  see Appendix~\ref{sec:cp-forks}.
  A positive side of this this alternate proof, however,  
  is that it shows how arguments in~\cite{LinearConsistency} 
  can be adapted to our generalized fork framework.
  }

  \begin{theorem}[Main theorem; CP version] \label{thm:main-mh-CP} Let
    $\epsilon, p_\h \in (0,1)$ and $T, k \in \NN, T \geq k$. 
    Let $w$ be a length-$T$ characteristic string satisfying the $(\epsilon, p_\h)$-Bernoulli condition. 
    % be a random variable satisfying the $\epsilon$-martingale condition.
    Then
    $$
      \Pr_{w}[\text{$w$ violates $\kCP$}] 
        \leq 
      \Pr_{w}[\text{$w$ violates $\kSlotCP$}] 
        \leq T \cdot 
        \exp\bigl(-k\cdot \Omega(\min(\epsilon^3, \epsilon^2 p_\h))\bigr)
        \,.
    $$
    
    Next, suppose that axiom~\ref{axiom:tie-breaking} is satisfied. 
    If $w$ is a length-$T$ bivalent characteristic string satisfying the $(\epsilon, 0)$-Bernoulli condition 
    then
    $$
      \Pr_{w}[\text{$w$ violates $\kCP$}] 
        \leq 
      \Pr_{w}[\text{$w$ violates $\kSlotCP$}] 
        \leq T \cdot 
        \exp\bigl(-k \cdot \Omega(\epsilon^3 (1 + O(\epsilon)))\bigr)
        \,.
    $$
  \end{theorem}
  \begin{proof}
% \paragraph{Proof of Theorem~\ref{thm:main-mh-CP}}
    \newcommand{\EpsCat}{\mathsf{\varepsilon}}
    (The first claim.) 
    % Let $w \in \{\h, \H, \A\}^T$ be a (trivalent) characteristic string 
    % sampled from a distribution satisfying the $\epsilon$-martingale condition.
    Let $s \in [T - k]$.
    Let $\EpsCat_{k}$ be 
    the probability that $y = w_s \ldots w_{s+k-1}$ contains no slot with the UVP in $w$. 
    Then, recalling~\eqref{eq:cp-uvp}, we can apply a union bound 
    over all substrings of $w$ of length at least $k$ to get 
    $
      \Pr[\text{$w$ violates $\kSlotCP$}] 
        \leq T\, \sum_{r \geq  k} \EpsCat_{r}
    $
    where the factor $T$ represents a summation over all $s \in [T - k + 1]$. 
    By Theorem~\ref{thm:unique-honest}, 
    if a substring $y$ of $w$ does not contain a slot with the unique vertex property in $w$, 
    $y$ cannot contain a uniquely honest slot that is Catalan in $w$.
    Therefore, $\EpsCat_k$ is no more than the error probability from Bound~\ref{bound:unique-honest-catalan}. 
    Since $\EpsCat_k$ decreases exponentially in $k$, 
    we can write 
    \begin{align*}
      \Pr[\text{$w$ violates $\kSlotCP$}] 
        &\leq T\cdot O(1) \cdot \EpsCat_{k}
        % \\&\leq T 
        %   \begin{cases}
        %     \exp\left(-k\cdot \Theta(\min(\epsilon^3, \epsilon^2 q_\h) \right) & \text{if $q_\h \rightarrow 0$}\,,\\
        %     \exp\left(-k\cdot \Theta( \min(\epsilon^3, \epsilon q_\h/q_\H) \right) & \text{otherwise}
        %     \,.
        %   \end{cases}
        \,.
    \end{align*}
    This proves the second inequality. 
    The first inequality follows since, 
    in a given characteristic string, 
    a $\kCP$ violation implies a $\kSlotCP$ violation.

    (The second claim.) 
    The proof in this case 
    is identical to the preceding argument except that 
    we need to refer to Theorem~\ref{thm:multiple-honest} in lieu of Theorem~\ref{thm:unique-honest}
    and Bound~\ref{bound:two-catalans} in lieu of Bound~\ref{bound:unique-honest-catalan}.
    % \hfill $\qed$  
  \end{proof}
  \paragraph{The $\Delta$-synchronous setting.} 
  A $\kCP$ violation in a $\Delta$-fork for a string $w \in \{\perp, \h, \H, \A\}^*$ 
  would imply 
  a $\kCP$ violation in the corresponding synchronous fork 
  in the string $\Reduce(w) \in \{\h, \H, \A\}^*$ 
  and, consequently, a $\kSlotCP$ violation in $\Reduce(w)$. 
  % Hence the probability of the former event 
  % is no more than that in Theorem~\ref{thm:main-mh-async} 
  % applied on the string $\Reduce(w)$. 
  We omit further details.
