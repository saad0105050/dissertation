We study the behavior of the elementary \emph{longest-chain rule}
algorithm, carried out by a collection of participants:
\begin{itemize}
  \item In each round,
each participant collects all valid blockchains from the network; if a
participant is a leader in the round, he adds a block to the longest
chain and broadcasts the result.
\end{itemize}
Here, ``valid'' indicates that any block appearing in the
chain was indeed issued by a leader from the associated slot; in the
PoS setting, this property is guaranteed with digital
signatures.

We begin by studying this algorithm in the simple, synchronous model
posited by Blum et. al~\cite{LinearConsistency}. The model adopts a
synchronous communication network in the presence of a \emph{rushing}
adversary: in particular,
\begin{enumerate}[label={\textbf{A\arabic*}}., ref={\textbf{A\arabic*}}, series=axiom, start = 0]
  \item\label{axiom:message-delivery} 
  Any message broadcast by an honest participant at the beginning of a
  particular slot is received by the adversary first, who may decide
  strategically and individually for each recipient in the network
  whether to inject additional messages and in which order all messages
  are to be delivered prior to the conclusion of the slot. 
\end{enumerate}
See the comments prior to Section~\ref{sec:model-settlement} for
further discussion of this network assumption.  A variant of this
adversarial message-ordering is presented in
Section~\ref{sec:lcr-model}.  The $\Delta$-synchronous communication
model is handled in \Section~\ref{sec:async-multihonest}.
% the full version~\cite{MultiHonestFullVersion}.


Given this, it is easy to describe the behavior of the longest-chain
rule when carried out by a group of honest participants with the extra
guarantee that exactly one is elected as leader in a slot: Assuming
that the system is initialized with a common ``genesis block''
corresponding to $\slot_0$, the players observe a common, linearly
growing blockchain:
\begin{center}
  \begin{tikzpicture}[>=stealth', auto, semithick,
    flat/.style={circle,draw=black,thick,text=black,font=\small}]
    \node[flat]    at (0,0)  (base) {$0$};
    \node[flat]    at (1,0)  (n1) {$1$};
    \node[flat]    at (2,0)  (n2) {$2$};
    \node[flat,white]    at (3,0)  (n3) {$\ \ \  $};
    \node at (3,0) {$\ldots$};
    \draw[thick,->] (base) to (n1);
    \draw[thick,->] (n1) to (n2);
    \draw[thick,->] (n2) to (n3);
  \end{tikzpicture}
\end{center}
\noindent
Here node $i$ represents the block broadcast by the leader of slot $i$
and the arrows represent the direction of increasing time.
%(Note that
%the requirement of a single leader per slot is important in this
%simple picture; it is possible for a network adversary to induce
%divergent views between the players by taking advantage of slots where
%more than a single honest participant is elected a leader.)

\paragraph{The blockchain axioms: Informal discussion.}
The introduction of adversarial participants or multiple slot leaders
complicates the family of possible blockchains that could emerge from
this process. To explore this in the context of our protocols, we work
with an abstract notion of a blockchain which
% (as informally suggested above)
ignores all internal structure. We consider a fixed assignment of
leaders to time slots, and assume that the blockchain uses a proof
mechanism to ensure that any block labeled with slot $\slot_t$ was
indeed produced by a leader of slot $\slot_t$; this is guaranteed in
practice by appropriate use of a secure digital signature scheme.

Specifically, we treat a \emph{blockchain} as
a sequence of abstract blocks, each labeled with a slot number, so
that:
\begin{enumerate}[label={\textbf{A\arabic*}}., ref={\textbf{A\arabic*}}, resume=axiom]
  \item\label{axiom:root} 
  The blockchain begins with a fixed ``genesis'' block, assigned to slot $\slot_0$.
  
  \item\label{axiom:labels} 
  The (slot) labels of the blocks are in strictly increasing order.
\end{enumerate}
It is further convenient to introduce the structure of a directed
graph on our presentation, where each block is treated as a vertex; in
light of the first two axioms above, a blockchain is a path beginning
with a special ``genesis'' vertex, labeled $0$, followed by vertices
with strictly increasing labels that indicate which slot is associated
with the block. %(See the example below.)
\begin{center}
  \begin{tikzpicture}[>=stealth', auto, semithick,
    flat/.style={circle,draw=black,thick,text=black,font=\small}]
    \node[flat]    at (0,0)  (base) {$0$};
    \node[flat]    at (1,0)  (n1) {$2$};
    \node[flat] at (2,0)  (n2) {$4$};
    \node[flat] at (3,0)  (n3) {$5$};
    \node[flat] at (4,0)  (n4) {$7$};
    \node[flat] at (5,0)  (n5) {$9$};
    \draw[thick,->] (base) to (n1);
    \draw[thick,->] (n1) to (n2);
    \draw[thick,->] (n2) to (n3);
    \draw[thick,->] (n3) -- (n4);
    \draw[thick,->] (n4) -- (n5);
  \end{tikzpicture}
\end{center}
The protocols of interest call for honest players to add a
\emph{single} block %(to a single previous chain in its local state)
during any slot. In particular:
% \begin{enumerate}[label={\textbf{A\arabic*}}., resume=axiom]
% \item If a slot $\slot_t$ was assigned to a single honest player, 
% then a single block is created---during the entire protocol---with the label $\slot_t$.
% \end{enumerate}
\begin{enumerate}[label={\textbf{A\arabic*}}., ref={\textbf{A\arabic*}}, resume=axiom]
  \item\label{axiom:honest}
   Let $k \geq 1$ be an integer. 
  If a slot $\slot_t$ was assigned to $k$ honest players but no adversarial players, 
  then $k$ blocks are created---during the entire protocol---each having the label $\slot_t$.
\end{enumerate}
Recall that blockchains are \emph{immutable} in the sense that any
block in the chain commits to the entire previous history of the
chain; this is achieved in practice by including with each block a
collision-free hash of the previous block. These properties imply that
% if a specific slot $\slot_t$ was assigned to a unique honest player,
% then any chain that includes the unique block from $\slot_t$ 
% if a specific slot $\slot_t$ was assigned to a unique honest player,
any chain that includes a block issued by an honest player 
must also include that block's associated prefix in its entirety.

As we analyze the dynamics of blockchain algorithms, it is convenient
to maintain an entire family of blockchains at once. As a matter of
bookkeeping, when two blockchains agree on a common prefix, we can
glue together the associated paths to indicate this, as shown
below.
\begin{center}
  \begin{tikzpicture}[>=stealth', auto, semithick,
    flat/.style={circle,draw=black,thick,text=black,font=\small}]
    \node[flat]    at (0,0)  (base) {$0$};
    \node[flat]    at (1,0)  (n1) {$2$};
    \node[flat] at (2,0)  (n2) {$4$};
    \node[flat] at (3,0)  (n3) {$5$};
    \node[flat] at (4,.5)  (n4a) {$7$};
    \node[flat] at (5,.5)  (n5a) {$9$};
    \node[flat] at (4,-.5)  (n4b) {$8$};
    \node[flat] at (5,-.5)  (n5b) {$9$};
    \draw[thick,->] (base) to (n1);
    \draw[thick,->] (n1) to (n2);
    \draw[thick,->] (n2) to (n3);
    \draw[thick,->] (n3) to (n4a);
    \draw[thick,->] (n4a) to (n5a);
    \draw[thick,->] (n3) to (n4b);
    \draw[thick,->] (n4b) to (n5b);
  \end{tikzpicture}
  \end{center}
  When we glue together many chains to form such a diagram, we call it
  a ``fork''---the precise definition appears below. Observe that
  while these two blockchains agree through the vertex (block) labeled
  5, they contain (distinct) vertices labeled 9; this reflects two
  distinct blocks associated with slot 9 which, in light of the axiom
  above, 
  % must have been produced by an adversarial participant.
  may be produced by either an adversarial participant assigned to slot 9 or 
  two honest participants, both assigned to slot 9.
  
  Finally, as we assume that messages from honest players are
  delivered before the next slot begins, we note a direct consequence of the longest
  chain rule:
\begin{enumerate}[label={\textbf{A\arabic*}}., ref={\textbf{A\arabic*}}, resume=axiom]
  \item\label{axiom:honest-depth} 
  If two honestly generated blocks $B_1$ and $B_2$ are labeled
  with slots $\slot_1$ and $\slot_2$ for which $\slot_1 < \slot_2$,
  then the length of the unique blockchain terminating at $B_1$ is
  strictly less than the length of the unique blockchain terminating at $B_2$.
\end{enumerate}
Recall that the honest participant(s) assigned to slot
$\slot_2$ will be aware of the blockchain terminating at $B_1$ that
was broadcast by an honest player in slot $\slot_1$ as a result of
synchronicity; according to the longest-chain rule, 
$B_2$ must have been placed on a chain that was at least this long. In contrast, not
all participants are necessarily aware of all blocks generated by
dishonest players, and indeed dishonest players may often want to
delay the delivery of an adversarial block to a participant or show
one block to some participants and show a completely different block
to others.

\paragraph{Characteristic strings, forks, and the formal axioms.}
Note that with the axioms we have discussed above, whether or not a
particular fork diagram (such as the one just above) corresponds to a valid
execution of the protocol depends on how the slots have been awarded to the parties by the
leader election mechanism. We introduce the notion of a ``characteristic'' string as a convenient
means of representing information about slot leaders in a given execution.
% \begin{definition}[Characteristic string]
%   Let $\slot_1, \ldots, \slot_{n}$ be a sequence of slots. A \emph{characteristic string} $w$ is an element of $\{0,1\}^n$ defined for a particular execution of a blockchain protocol so that
%   \[
%     w_t =   \begin{cases}
%     0 & \text{if $\slot_{t}$ was assigned to a single honest participant},\\
% %    1 & \text{if $\slot_{t}$ was assigned to an adversarial participant},
%     1 & \text{otherwise.}
%   \end{cases}
% \]
% \end{definition}
\begin{definition}[Characteristic string]\label{def:trivalent-char-string}
  Let $\slot_1, \ldots, \slot_{n}$ be a sequence of slots.  A
  \emph{characteristic string} $w$ is an element of
  $\{\h,\H,\A\}^n$. The string $w$ is consistent with a particular
  execution of a blockchain protocol on these slots if for each
  $t \in [n]$, (i) if $w_t = \A$, the slot $\slot_t$ is assigned to
  at least one adversarial participant, (ii) if $w_t = \h$, the slot
  $\slot_{t}$ is assigned to a unique, honest participant, and (iii)
  if $w_t = \H$, the slot $\slot_{t}$ is assigned to at least one
  honest participant and no adversarial participants.

  Observe that when an execution corresponds to a characteristic
  string $w$, it also corresponds to any string obtained from $w$ by
  replacing $\h$ symbols with $\H$ symbols.
% 
  % \[
  %   w_t =   \begin{cases}
  %   \h & \text{if $\slot_{t}$ was assigned to a single participant who is honest},\\
  %   \H & \text{if $\slot_{t}$ was assigned to two or more participants, all honest},\\
  %   \A & \text{otherwise.}
  %   \end{cases}
  % \]
\end{definition}
For two strings $x$ and $w$ on the same alphabet, 
we write $x \Prefix w$ if and only if $x$ is a strict prefix of $w$. 
Similarly, 
we write $x \PrefixEq w$ if and only if either $x = w$ or $x \Prefix w$. 
The empty string $\varepsilon$ is a prefix to any string. 
If $w_t \in \{\h, \H\}$, we say that ``$\Slot_t$ is honest'' and 
otherwise, we say that ``$\Slot_t$ is adversarial.'' 
With this discussion behind us, we set down the formal object we use
to reflect the various blockchains adopted by honest players during
the execution of a blockchain protocol. This definition formalizes the blockchains axioms discussed above.

%%%%Forks

\begin{definition}[Fork]\label{def:fork}
  Let $w\in \{\h, \H, \A\}^n$, $P = \{ i : w_i = \h\}$, and $Q = \{ j : w_j = \H\}$. 
  A \emph{fork} for the string $w$ consists of a directed and rooted
  tree $F=(V,E)$ with a labeling $\ell:V\to\{0,1,\dots,n\}$. We insist
  that each edge of $F$ is directed away from the root vertex and
  further require that
  \begin{enumerate}[label=(F{\arabic*})]
    \item\label{fork:root-mh} the root vertex $r$ has label $\ell(r)=0$;

    \item\label{fork:monotone-mh} the labels of vertices along any directed path are strictly increasing;

    \item\label{fork:unique-honest-mh}\label{fork:multiply-honest}
    each index $i\in P$ 
    is the label of exactly one vertex of $F$ 
    and 
    % \item
    each index $j\in Q$ 
    is the label of at least one vertex of $F$; and 

    \item\label{fork:honest-depth-mh} 
    for any indices $i,j\in P \Union Q$, 
    if $i<j$ then 
    the depth of a vertex with label $i$ 
    is strictly less than 
    the depth of a vertex with label $j$.
  \end{enumerate}
\end{definition}

If $F$ is a fork for the characteristic string $w$, we write
$F\vdash w$.  The conditions~\ref{fork:root-mh}--\ref{fork:honest-depth-mh}
are analogues of the axioms~\ref{axiom:root}--\ref{axiom:honest-depth}
above. The formal reflection of axiom~\ref{axiom:honest} by
condition~\ref{fork:multiply-honest} deserves further comment: We have
chosen a definition of characteristic string that does not indicate
the number of honest victories in cases where there may be many; in
particular, the symbol $\H$ may be associated with any positive number
of (honest) vertices in the fork. Indeed, we even permit a fork to
have a \emph{single} honest vertex associated with such a symbol,
which enlarges the class of forks under consideration for a particular
characteristic string. This strengthens our results by effectively
giving the adversary the option to treat $\H$ symbols as $\h$
symbols. See Fig.~\ref{fig:fork-mh}
% in Section~\ref{sec:figures} 
for an example fork. 

\begin{figure}[t]
  \centering
  \begin{tikzpicture}[scale=0.85,>=stealth', auto, semithick,
    honest/.style={circle,draw=black,thick,text=black,double,font=\small},
    malicious/.style={fill=gray!10,circle,draw=black,thick,text=black,font=\small}]
    \node at (0,-2) {$w =$};
    \node at (1,-2) {$\h$};
    \node[honest]    at (1,-.5)  (ab1) {$1$};
    \node at (2,-2) {$\A$};
    \node[malicious] at (2,0)  (b2) {$2$}; \node[malicious] at (2,1) (c1) {$2$};
    \node at (3,-2) {$\h$};    \node[honest]    at (3,1)  (c2) {$3$};
    \node at (4,-2) {$\A$};    \node[malicious] at (4,0)  (b3) {$4$}; \node[malicious] at (4,-1) (a2) {$4$};
    \node[malicious] at (4,1) (c3) {$4$};
    \node at (5,-2) {$\h$};    \node[honest]    at (5,-1) (a3) {$5$};
    \node at (6,-2) {$\H$};    \node[honest]    at (6,0)  (b4) {$6$}; \node[honest]    at (6,-1)  (a4) {$6$};
    \node at (7,-2) {$\A$};    \node[malicious] at (7,1)  (c4) {$7$};
    \node at (8,-2) {$\A$};    \node[malicious] at (8,1)  (c5) {$8$};
    \node at (9,-2) {$\H$};    \node[honest]    at (9,0)  (b5) {$9$}; \node[honest]    at (9,-1)  (a5) {$9$};
    \node[honest] at (-1,0) (base) {$0$};
    % \node[state,honest] at (3,-1) (bottom) {};
    % \node[state,honest] at (7,1) (top) {$H$};
    \draw[thick,->] (base) to[bend left=10] (c1);
    \draw[thick,->] (base) to[bend right=10] (ab1);
    \draw[thick,->] (ab1) to[bend right=10] (a2);
    \draw[thick,->] (a2) -- (a3);
    \draw[thick,->] (a3) -- (a4);
    \draw[thick,->] (ab1) to[bend left=10] (b2);
    \draw[thick,->] (b2) -- (b3);
    \draw[thick,->] (b3) -- (b4);
      % \draw[thick,->] (b4) -- (b5);
    \draw[thick,->] (c4) to[bend right=10] (b5);
    \draw[thick,->] (a4) -- (a5);
    \draw[thick,->] (c1) -- (c2);
    \draw[thick,->] (c2) -- (c3);
    \draw[thick,->] (c3) -- (c4);
    \draw[thick,->] (c4) -- (c5);
    % \draw[thick,<->] (3,0) -- (7,0) node[pos=.5] {$\gap(f)$};
  \end{tikzpicture}
  \caption{A fork $F$ for the characteristic string $w = \h\A\h\A\h\H\A\A\H$;
    vertices appear with their labels and honest vertices are
    highlighted with double borders. Note that the depths of the
    (honest) vertices associated with the honest indices of $w$ are
    strictly increasing. Note, also, that this fork has three disjoint
    paths of maximum depth. 
    In addition, two honest vertices have label 6 and two more have label 9, 
    indicating the fact that two honest leaders are associated with each of the (honest) slots 6 and 9. 
    Honest vertices with the same label are concurrent and, therefore, cannot extend each other.
    Note that the two honest vertices with label 6 extend different vertices with the same depth. 
    This is allowed since any tie in the longest-chain rule is broken by the adversary. 
    }
  \label{fig:fork}
\end{figure}


A final notational convention: If $F \vdash x$ and
$\hat{F} \vdash w$, we say that $F$ is a \emph{prefix} of $\hat{F}$,
written $F \fprefix \hat{F}$, if 
% the string $x \in \{0,1\}^\ell$ is a prefix of the string $w \in \{0,1\}^{\ell + m}$ 
$x \PrefixEq w$
and $F$ appears as a
consistently-labeled subgraph of $\hat{F}$. 
(Specifically, each path of $F$ appears, with identical labels, in $\hat{F}$.) 


% Observe that any string of the form $0^k$ has a unique
% fork consisting of a single path:
% \begin{center}
%   \begin{tikzpicture}[scale=1,>=stealth', auto, semithick,
%     flat/.style={circle,draw=black,thick,text=black,font=\small}]
%     \node[flat] at (0,0)  (base) {$0$};
%     \node[flat] at (1,0)  (n1) {$1$};
%     \node[flat] at (2,0)  (n2) {$2$};
%     \node[flat] at (5,0)  (n5) {$k$};
%     \draw[thick,->] (base) to (n1);
%     \draw[thick,->] (n1) to (n2);
%     \draw[thick,->,dashed] (n2) to (n5);
%   \end{tikzpicture}
% \end{center}
% On the other hand, there are in fact an infinite number of forks for
% any string with at least one ``1,'' as slots for which $w_i = 1$ can
% be associated with any number of vertices. Axiom~\ref{fork:monotone-mh}
% reflects that any legal chain must consist of blocks with increasing
% slot labels. Axiom~\ref{fork:unique-honest} reflects the fact that honest plays produce a single block. Axiom~\ref{fork:honest-depth} reflects that each new
% honest vertex is always placed at a depth strictly greater than all
% previous honest vertices, because honest users always choose to add
% their block to the longest visible chain, and we assume honest blocks
% can be seen by all users. (By contrast, adversaries may play on
% shorter tines, or may ``hide'' dishonest blocks from other users until
% later slots.)



  Let $w$ be a characteristic string.  The directed paths in the fork
  $F \Fork w$ originating from the root are called \emph{tines}; these
  are abstract representations of blockchains. (Note that a tine may 
  not terminate at a leaf of the fork.)  We naturally extend the label
  function $\ell$ for tines: i.e., $\ell(t) \triangleq \ell(v)$ where
  the tine $t$ terminates at vertex $v$. The length of a tine $t$ is
  denoted by $\length(t)$.

 \paragraph{Viable tines.}
 The longest-chain rule dictates that honest players build on chains
 that are at least as long as all previously broadcast honest
 chains. It is convenient to distinguish such tines in the analysis:
 specifically, a tine $t$ of $F$ is called \emph{viable} if its length
 is no smaller than the depth of any honest vertex $v$ for which
 $\ell(v) \leq \ell(t)$. A tine $t$ is \emph{viable at slot $s$} if
 the length of the portion of $t$ appearing over slots $0,\ldots, s$ 
 is no smaller than the depths of any honest vertices labeled from these slots. (As noted,
 the properties~\ref{fork:multiply-honest} and~\ref{fork:honest-depth-mh}
 together imply that an honest observer at slot $s$ will only adopt a
 viable tine.)  
 % The \emph{honest depth} function
 % $\hdepth : H \rightarrow [n]$ gives the depth of the (unique) vertex
 % associated with an honest slot; by~\ref{fork:honest-depth},
 % $\hdepth(\cdot)$ is strictly increasing.
 The \emph{honest depth} function
 $\hdepth : P \Union Q \rightarrow [n]$, 
 defined as $\hdepth(i) = \max_{t \in F} \left\{ \length(t) \SuchThat \ell(t) = i \right\}$, 
 gives the largest depth of the (honest) vertices 
 associated with an honest slot; by~\ref{fork:honest-depth-mh},
 $\hdepth(\cdot)$ is strictly increasing.


% \subsection{Comments on the model}\label{sec:model-comments}





    % \begin{definition}[Conservative fork]\label{def:conservative-fork}
    %   The term \emph{conservative fork} is defined inductively, as follows.
    %   The trivial fork $F_0$ for the empty string $\varepsilon$ 
    %   is conservative. 
    %   A fork $F_{T} \Fork w0$ is conservative if 
    %   the fork prefix $F_{T_1} \ForkPrefix F_T, F_{T-1} \Fork w$ 
    %   is conservative 
    %   and every honest tine $t \in F_T$ at slot $T$ 
    %   is a conservative extension with respect to $F_{T-1}$.
    %   Likewise, 
    %   a fork $F_{T} \Fork w1$ is conservative if 
    %   the fork prefix $F_{T_1} \ForkPrefix F_T, F_{T-1} \Fork w$ 
    %   is conservative and $F_T = F_{T-1}$. 
    % \end{definition}
    % Observe that a conservative fork is closed. 

    % We record the following lemma from [LinearConsistencyPaper].
    % \begin{lemma}[Optimal online adversary]\label{lemma:online-adversary}
    %   For every characteristic string $w = xy$ 
    %   there is a closed fork $F \Fork xy$ 
    %   so that 
    %   $\reach(F) = \reach(xy)$ and $\mu_x(F) = \mu_x(y)$. 
    % \end{lemma}
    % % Given a charactersistic string $w = xy$, 
    % % the ``online adversary'' given in the LinearConsistencyPaper 
    % % creates a conservative fork 
    % % which achieves the reach and relative margin of the string $xy$. 


\section[Slot Settlement and UVP]{Slot settlement and the Unique Vertex Property}\label{sec:model-settlement}
  
  We are now ready to explore the power of an adversary in this
  setting who has corrupted a (perhaps evolving) coalition of the
  players. We focus on the possibility that such an adversary can
  violate the consistency of the honest players'
  blockchains. In particular, we consider the possibility that, at
  some time $t$, the adversary conspires to produce two maximum-length blockchains 
  that diverge prior to a previous slot $s \leq t$; in
  this case honest players adopting the longest-chain rule may clearly
  disagree about the history of the blockchain after slot $s$. We call
  such a circumstance a \emph{settlement violation}.

  To express this in our abstract language, let $F \Fork w$ be a fork
  corresponding to an execution with characteristic string $w$. Such a
  settlement violation induces two viable tines $t_1, t_2$ with the
  same length that diverge prior to a particular slot of interest. We
  record this below.
    
  \begin{definition}[Settlement with parameters $s,k \in \NN$]\label{def:settlement-mh}
    Let $n \in \NN$ and let $w$ be a characteristic string of length $n$. 
    Let $t \in [s + k, n]$ be an integer, $\hat{w} \PrefixEq w, |\hat{w}| = t$, and 
    let $F$ be any fork for $\hat{w}$. 
    We say that a slot $s$ is \emph{not $k$-settled} in $F$ if 
    $F$ contains two maximum-length tines $\Chain_1, \Chain_2$ 
    that ``diverge prior to $s$,'' i.e., they either
    contain different vertices labeled with $s$, or one contains a
    vertex labeled with $s$ while the other does not. 
    Otherwise, we say that \emph{slot $s$ is $k$-settled in $F$}. 
    We say that \emph{slot $s$ is $k$-settled in $w$} if, 
    for each $t \geq s+k$, 
    it is $k$-settled in every fork $F \Fork \hat{w}$ where $\hat{w} \PrefixEq w, |\hat{w}| = t$.
  \end{definition}




  \begin{figure*}[t]
    \singlespacing
    \begin{center}
      \fbox{
        \begin{minipage}{\textwidth}
          \begin{center}
            \textbf{The $(\Distribution,T;s,k)$-settlement game}
          \end{center}
          \begin{enumerate}

          \item A characteristic string $w \in \{\h,\H,\A\}^T$ is drawn from
            $\mathcal{D}$. (This reflects the results of the leader
            election mechanism.)

          \item Let $A_0 \vdash \varepsilon$ denote the initial fork for
            the empty string $\varepsilon$ consisting of a single node
            corresponding to the genesis block.

          \item For each slot $\Slot_t, t = 1, \ldots, T$ in increasing order:
            \begin{enumerate}
            \item\label{game:honest} (Honest slot.)  This case
              pertains to $w_t \in \{\h, \H\}$.  If $w_t = \h$ then
              $\Adversary$ sets $k = 1$.  If $w_t = \H$ then
              $\Adversary$ chooses an arbitrary integer $k \geq 1$.
              The challenger is then given $k$ and the fork
              $A_{t-1} \vdash w_1 \ldots w_{t-1}$.  He must determine
              a new fork $F_{t} \vdash w_1 \ldots w_t$ by adding $k$
              new vertices (all labeled with $t$) to $A_{t-1}$.  Each
              new vertex is added at the end of a maximum-length path
              in $A_{t-1}$.  If there are multiple
              candidates\footnote{ It is possible that all maximum-length 
                tines are honest. In the settlement game
                considered in~\cite{LinearConsistencySODA}, at least
                one of these tines was adversarial.} 
              for this path,
              $\Adversary$ may break the tie.  If $k \geq 2$, multiple
              vertices (all with label $t$) may be added at the end of
              the same path.



            \item 
            (Adversarial slot.)
            If $w_t = \A$, this is an adversarial slot. $\Adversary$
              may set $F_t \vdash w_1\ldots w_t$ to be an arbitrary fork
              for which $A_{t-1} \fprefix F_t$.
              
            \item (Adversarial augmentation.) $\Adversary$ determines an
              arbitrary fork $A_t \vdash w_1 \ldots, w_{t}$ for
              which $F_{t} \fprefix A_{t}$.
            \end{enumerate}
             Recall that $F \fprefix F'$ indicates that $F'$
              contains, as a consistently-labeled subgraph, the fork $F$.
          \end{enumerate}
          $\Adversary$ \emph{wins the settlement game} if slot $s$ is not
          $k$-settled in some fork $A_t, t \geq s+k$.
        \end{minipage}
      }
    \end{center}
  \end{figure*}



  \begin{definition}[Bottleneck Property (BP) and Unique Vertex Property (UVP)]\label{def:bottleneck-property}\label{def:unique-vertex-property}
    Let $w \in \{\h, \H, \A\}^T$ be a characteristic string.  
    A slot $s \in [T]$ is said to have the 
    \emph{bottleneck property in $w$} 
    if, 
    for any fork $F \Fork w$ 
    and any $k \geq s + 1$, 
    every tine viable at the onset of slot $k$ 
    contains, as its prefix, some vertex with label $s$.       
    Slot $s$ is said to have the \emph{Unique Vertex Property} 
    if, 
    for any fork $F \Fork w$, 
    there is a unique vertex $u \in F$ with label $s$ 
    so that 
    for any $k \geq s + 1$, 
    all tines viable at the onset of slot $k$ 
    contain, as their common prefix, 
    the vertex $u$.
  \end{definition}
  Thus 
  if a uniquely honest slot in $w$ has the bottleneck property, 
  it has the UVP as well.
  As a consistency property, UVP has several advantages over slot settlement. 
  First, it easily implies the slot settlement property:
  let $w \in \{\h, \H, \A\}^T, s \in [T]$, and $k \in [T - s]$. 
  \begin{equation}\label{eq:settlement-uvp}
    \text{If a slot $t \in [s, s + k]$ has UVP in $w$ 
    then $s$ is $k$-settled in $w$.}     
  \end{equation}  
  In addition, UVP has a straightforward characterization 
  using ``Catalan slots'' 
  (see Theorem~\ref{thm:unique-honest}) 
  and ``relative margin'' 
  (see Lemma~\ref{lemma:uvp-margin}); 
  these characterizations are amenable to stochastic analysis. 
  Finally, since UVP is structurally reminiscent of the traditional common prefix (CP) violations, 
  % (see Section~\ref{sec:cp-model}), 
  UVP easily implies CP. 
  The analogous statement ``settlement implies CP,'' however, 
  requires a lengthy proof both in~\cite{LinearConsistency} and in 
  our framework. See 
  \Section~\ref{sec:cp-multihonest}
  % the full version~\cite{MultiHonestFullVersion}
  for details.




\section[Attacks on Settlement]{Adversarial attacks on settlement time; the settlement game}\label{sec:game-mh} 


  To clarify the relationship between forks and the chains at play in a
  canonical blockchain protocol, we define a game-based model below that
  explicitly describes the relationship between forks and executions.
  By design, the probability that the adversary wins this game is at
  most the probability that a slot $s$ is not $k$-settled. 
  % We remark
  % that while we focus on settlement violations for clarity, one could
  % equally well have designed the game around common prefix violations.

  Consider the \emph{$(\Distribution,T;s,k)$-settlement game} 
  (presented in the box), played
  between an adversary $\Adversary$ and a challenger $\Challenger$ with
  a leader election mechanism modeled by an ideal distribution
  $\Distribution$. Intuitively, the game should reflect the ability of
  the adversary to achieve a settlement violation; that is, to present
  two maximum-length viable blockchains to a future honest observer,
  thus forcing them to choose between two alternate histories which
  disagree on slot $s$.
  The challenger plays the role(s) of the honest players during the
  protocol. 

  It is important to note that the game bestows the player $\Adversary$ 
  with the power to choose the number of honest vertices in 
  a multiply honest slot. 
  Note that this setting makes the player strictly more powerful and, 
  importantly, implies that 
  the game is completely determined 
  by the choices made by $\Adversary$ 
  (i.e., the actions of the challenger are deterministic). 
  Consequently, in Definition~\ref{def:settlement-mh-insecurity}, 
  we can use a single, implicit universal quantifier 
  over all strategies $\Adversary$; no choices of the challenger are actually necessary to fully describe the game.


  \begin{definition}[Settlement insecurity]\label{def:settlement-mh-insecurity}
    Let $\Distribution$ be a distribution on $\{\h, \H, \A\}^T$. 
    Let $w \sim \Distribution$ be the string used in the 
    first step of a $(\Distribution, T; s, k)$-settlement game $G$. 
    The \emph{$(s,k)$-settlement insecurity} of $\Distribution$ 
    is defined as 
    \[
      \mathbf{S}^{s,k}[\Distribution] \triangleq 
        \max_{\substack{\hat{w} \PrefixEq w \\ |\hat{w}| \geq s + k}}\,
        \max_{F \Fork \hat{w}}\, 
        \Pr\left[\parbox{50mm}{\centering $F$ has two maximum-length tines that diverge prior to slot $s$}\right]
      \,.
    \]
    % this maximum taken over all adversaries $\Adversary$.
    Note that the probability in the right-hand side is the same as 
    the probability that $\Adversary$ wins $G$.
  \end{definition}

  \section{Main theorem for PoS consistency}
  Note that in typical PoS settings the distribution $\Distribution$
  is determined by the combined stake held by the adversarial players,
  the leader election mechanism, and the dynamics of the protocol. The
  most common case (as seen in Snow White~\cite{SnowWhite},
  Ouroboros~\cite{Ouroboros}, and Ouroboros Praos~\cite{Praos})
  guarantees that the characteristic string $w = w_1 \ldots w_T$ is
  drawn from an i.i.d.\ distribution for which
  $\Pr[w_i = \A] \leq (1 - \epsilon)/2$ for some $\epsilon \in (0, 1)$;
  here the constant $(1-\epsilon)/2$ is directly related to the stake
  held by the adversary. Some settings involving adaptive adversaries
  (e.g., Ouroboros Praos~\cite{Praos}) yield a weaker martingale-type
  guarantee that
  $\Pr[w_i = \A \mid w_1, \ldots, w_{i-1}] \leq (1 - \epsilon)/2$.  We
  can easily handle both types of distributions in our analysis since
  the former distribution ``stochastically dominates'' the latter; 
  see Section~\ref{sec:martingale-dominance} for a precise definition.
  % Let us define below the notion of stochastic dominance. 
  As a rule, we denote the
  probability distribution associated with a random variable using
  uppercase script letters. 
  \begin{definition}[Stochastic dominance]\label{def:dominance-mh} 
    Let $X$ and $Y$ be random variables taking values in some set $\Omega$ 
    endowed with a partial order $\leq$. 
    We say that $X$ \emph{stochastically dominates} $Y$, 
    written $Y \dominatedby X$, if 
    $
      \mathcal{X}(A) \geq \mathcal{Y}(A)
      % \,.
    $ 
    for all \emph{monotone sets} $A \subseteq \Omega$, 
    where a set $A \subseteq \Omega$ is called 
    monotone if $a \in A$ implies $a' \in A$ for all $a \leq a'$.
    As a special case, when $\Omega = \R$,  $Y \dominatedby X$ if
    $\Pr[X \geq \Lambda] \geq \Pr[Y \geq \Lambda]$
    for every $\Lambda \in \R$.  
    We extend this notion to probability
    distributions in the natural way.
  \end{definition}

  Throughout the paper, we adopt the following partial order on
  $\{\h, \H, \A\}^T$: If $T = 1$, define $\h < \H < \A$.  Otherwise,
  for two strings $xa, yb \in \{\h, \H, \A\}^T, |a| = |b| = 1$,
  $xa \leq yb$ if and only if $x \leq y$ and $a \leq b$. When
  $x \leq y$, one might say that $y$ is ``more adversarial'' than $x$:
  indeed, if $F \vdash x$ and $x \leq y$ then $F \vdash y$ so that any
  settlement violation for $x$ induces a settlement violation for $y$.

  \begin{definition}[$(\epsilon, p_\h)$-Bernoulli
    condition]\label{def:bernoulli-cond}
    Let $T \in \NN, \epsilon \in (0,1)$, and
    $p_\h \in [0,(1+\epsilon)/2]$. Define $p_\A = (1-\epsilon)/2$ and
    $p_\H = 1- p_\A - p_\h$.  A random variable $w = w_1 \ldots w_T$
    taking values in $\{\h, \H, \A\}^T$ is said to satisfy the
    \emph{$(\epsilon, p_\h)$-Bernoulli condition} if each
    $w_i, i \in [T]$, is independent and identically distributed as
    follows: $\Pr[w_i = \sigma] = p_\sigma$ for
    $\sigma \in \{\h, \H, \A\}$.  The distribution of $w$ is also said
    to satisfy the $(\epsilon, p_\h)$-Bernoulli condition.

    We frequently use the notation $p_\H$ and $p_\A$ in the context of
    such a random variable when $\epsilon$ and $p_\h$ can be inferred
    from context.
  \end{definition}
  
  \begin{theorem}[Main theorem]\label{thm:main-mh}
    Let $\epsilon, p_\h \in (0, 1)$ and $s, k, T \in \NN$.  
    Let $\mathcal{B}$ be a distribution 
    on length-$T$ characteristic strings satisfying 
    the $(\epsilon, p_\h)$-Bernoulli condition.
    Then 
    $
      \mathbf{S}^{s,k}[\mathcal{B}] 
        \leq 
        \exp\left(-k\cdot \Omega( 
          \min(\epsilon^3, \epsilon^2 p_\h) 
        \right)
        % \,.
    $.
    Furthermore, 
    let 
    $\mathcal{W}$ be a distribution on
    $\{\h, \H, \A\}^T$ so that 
    $\mathcal{W} \DominatedBy \mathcal{B}$. 
    Then $\mathbf{S}^{s,k}[\mathcal{W}] 
        \leq \mathbf{S}^{s,k}[\mathcal{B}]$.   
    (Here, the asymptotic notation hides constants that do not depend on $\epsilon$ or $k$.)
  \end{theorem}
  Note that the quantity $p_\h$ above cannot be zero.
  We present the proof in \Section~\ref{sec:bounds-main-proofs-multihonest}. 
  In \Section~\ref{sec:fork-framework}, 
  we give a characterization of the UVP which 
  allows us to explicitly compute $\mathbf{S}^{s,k}[\mathcal{B}]$; 
  see 
  % Lemma~\ref{lemma:uvp-margin}, 
  Theorem~\ref{thm:relative-margin} and 
  \Section~\ref{sec:exact-prob-multihonest}.


  \paragraph{Analysis in the $\Delta$-synchronous setting.} The security
   game above most naturally models a blockchain protocol over a
   synchronous network with immediate delivery (because each ``honest''
   play of the challenger always builds on a fork that contains the fork
   generated by previous honest plays). However, the model can be easily
   adapted to protocols in the $\Delta$-synchronous model by applying
   the $\Delta$-reduction mapping of~\cite{Praos} (which is specifically
   designed to lift the synchronous analysis to the $\Delta$-synchronous
   setting).
   These details appear in \Section~\ref{sec:async-multihonest}.


  \paragraph{Public leader schedules.} One attractive feature of this
  model is that it gives the adversary full information about the future
  schedule of leaders. The analysis of some protocols indeed demand this
  (e.g., Ouroboros, Snow White). Other protocols---especially those
  designed to offer security against adaptive adversaries (Praos,
  Genesis)---in fact contrive to keep the leader schedule private. Of
  course, as our analysis is in the more difficult ``full information''
  model, it applies to all of these systems.

  \paragraph{Bootstrapping multi-phase algorithms; stake shift.} We remark that
  several existing proof-of-stake blockchain protocols proceed in
  phases, each of which is obligated to generate the randomness (for
  leader election, say) for the next phase based on the current stake
  distribution. The blockchain security properties of each phase are
  then individually analyzed---assuming clean randomness---which yields
  a recursive security argument; in this context the game outlined above
  precisely reflects the single phase analysis.






%%% Local Variables:
%%% mode: latex
%%% TeX-master: "main"
%%% End:
