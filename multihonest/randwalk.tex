% Stochastic dominance plays an important role in the arguments
% below. First of all, we observe that the distribution
% $\mathcal{B}_\epsilon$ stochastically dominates any distribution
% satisfying the $\epsilon$-martingale condition; this yields the first
% inequality in Theorem~\ref{thm:main}. A more delicate application of
% stochastic dominance is used in order to achieving bounds, such as
% those of Section~\ref{sec:bounds}, that are independent of the length of
% $x$. This follows from the fact that $\reach(B_{\epsilon})$ converges to a
% particular, dominant distribution as its argument increases in length.


%=======================================================

In this chapter, we first define a simple, biased random walk on a characteristic string 
which satisfies $(\epsilon, p_\h)$-Bernoulli condition (Definition~\ref{def:bernoulli-cond}).
In later chapters, we will derive tail bounds (e.g., Bound~\ref{bound:unique-honest-catalan}) 
for certain bad events on this walk.


\section{\texorpdfstring{The random walks $W$ and $X$}{Two random walks on integers}}
  Let $\epsilon \in (0,1)$. 
  Let $p = (1 - \epsilon)/2$ and $q = (1 + \epsilon)/2$ 
  so that $q - p = \epsilon$. 
  Let $q_\h \in (0, q]$ and $q_\H = q - q_\h$. 
  Let $B = B_1 B_2 \cdots$ be a characteristic string 
  satisfying the $(\epsilon, p_\h)$-Bernoulli condition. 
  Define the process $W = (W_t : t \in \NN), W_t \in \{\pm 1\}$ as $W_t = 1$ if and only if $B_t = \A$. 
  % That is, for all $t$, $\Pr[W_t = \h] = q_\h, \Pr[W_t = \H] = q_\H$, and $\Pr[W_t = \A] = p$. 
  Let 
  \begin{align*}
    S(B) &= \#_\A(B) - \#_\h(B) - \#_\H(B)\,.
  \end{align*}
  We say that the process $S$ is a \emph{simple biased random walk on integers with $\epsilon$ negative (i.e., downward) bias}. 
  $S_t$ is the position of the particle (i.e., the random walker) at time $t$. 
  By convention, set $B_0 = S_0 = 0$. 

  Specifically, the random variables
  $X(B)$ (and $X_\infty$) have an immediate interpretation in terms
  of the Markov chain corresponding to a biased random walk on $\Z$
  with a ``reflecting boundary'' at -1. Specifically, consider the
  Markov chain on $\{0, 1, \ldots\}$ given by the transition diagram 
  in Figure~\ref{fig:walk}
  \begin{figure}[tb]
    \centering
    \begin{minipage}{0.5 \textwidth}
        \centering
        \begin{tikzpicture}[scale=1.5,>=stealth', auto, semithick,
          flat/.style={circle,draw=black,thick,text=black,font=\small}]
          \node[] at (-2,0) {$\ldots$};
          \node[flat,white] (n_2) at (-2,0) {$ \ \ \ $};
          \node[flat] (n_1) at (-1,0)  {$-1$};
          \node[flat] (n0) at (0,0)  {$0$};
          \node[flat] (n1) at (1,0)  {$1$};
          \node[flat,white] (n2) at (2,0) {$ \ \ \ $};
          \node[] at (2,0) {$\ldots$};
          
          \draw[thick,->,bend left] (n_2) to (n_1);
          \draw[thick,->,bend left] (n_1) to (n0);
          \draw[thick,->,bend left] (n0) to (n1);
          \draw[thick,->,bend left] (n1) to (n2);

          \draw[thick,->,bend left] (n2) to (n1);
          \draw[thick,->,bend left] (n1) to (n0);
          \draw[thick,->,bend left] (n0) to (n_1);
          \draw[thick,->,bend left] (n_1) to (n_2);
        \end{tikzpicture}
      \end{minipage}%
      \begin{minipage}{0.5 \textwidth}
        \centering
        \begin{tikzpicture}[scale=1.5,>=stealth', auto, semithick,
          flat/.style={circle,draw=black,thick,text=black,font=\small}]
          \node[flat] (n0) at (0,0)  {$0$};
          \node[flat] (n1) at (1,0)  {$1$};
          \node[flat] (n2) at (2,0)  {$2$};
          \node[flat,white] (n3) at (3,0) {$ \ \ \ $};
          \node[] at (3,0) {$\ldots$};
          \draw[thick,->,bend left] (n0) to (n1);
          \draw[thick,->,bend left] (n1) to (n2);
          \draw[thick,->,bend left] (n2) to (n3);
          \draw[thick,->,loop left] (n0) to (n0);
          \draw[thick,->,bend left] (n1) to (n0);
          \draw[thick,->,bend left] (n2) to (n1);
          \draw[thick,->,bend left] (n3) to (n2);
        \end{tikzpicture}    
      \end{minipage}
    \caption{Markov chain for the random variable $S(B)$ \emph{(left)} and $X(B)$ \emph{(right)}
    with parameter $\epsilon \in (0,1)$.  
    A right-arrow has probability $(1-\epsilon)/2$ and a left-arrow (and the only self loop) 
    has probability $(1+\epsilon)/2$.}
    \label{fig:walk}  
  \end{figure}
  where edges pointing right have probability $(1-\epsilon)/2$ and edges pointing left---including the loop at 0---have probability $(1+\epsilon)/2$.


  Consider the random variables 
  \begin{align}
    M(B) &= \min_{t \leq |B|} S(B_1 \cdots B_t)\,, \quad\text{and } \\
    X(B) &= S(B) - M(B)\,. \label{eq:X-walk}
  \end{align}
  When $B$ is understood from the context, we often omit $B$ and simply write $X$ and $X_t$.
  Here, $X_t$ has the interpretation that it is the relative hight of $S_t$ 
  with respect to its minimum over the prefix $S_1 \cdots S_t$. 
  Let $\mathcal{X}_t(B)$ be the distribution of $X(B_1 \cdots B_t)$ 
  and define $X_\infty = X_t$ where $t = \infty$. 


  \section{The limiting distribution for reach}\label{sec:stationary-dist-reach}
  By examining~\eqref{eq:rho-recursive-mh}, 
  we can confirm that $X_t = \rho(W_1 \cdots W_t)$. 
  Thus it is not surprising that $X_t$ will play an important role in our analysis.
  In this section, 
  we compute the limiting distribution $\mathcal{X}_\infty$ 
  and show that it stochastically dominates $\mathcal{X}_t(B), t \geq 0$.


  % \begin{lemma}\label{lemma:rho-stationary}
  %   Let $\epsilon, p_\h \in (0,1)$ and consider 
  %   two equally long characteristic strings $W$ and $B$, 
  %   so that $W$ satisfies the $(\epsilon, p_\h)$-martingale condition 
  %   and $B$ satisfies the $(\epsilon, p_\h)$-Bernoulli condition. 
  %   Let $X_\infty \in \{0, 1, \ldots\}$ be a random variable 
  %   whose distribution $\mathcal{X}_\infty$ is defined as 
  %     \begin{equation}
  %       \label{eq:stationary}
  %       \mathcal{X}_\infty(k) 
  %         = \Pr[X_\infty = k] 
  %         \defeq \left(\frac{2\epsilon}{1+\epsilon}\right)\cdot \left(\frac{1-\epsilon}{1 + \epsilon}\right)^k
  %         \qquad \text{for $k = 0, 1, 2, \ldots$}\ 
  %       \,.
  %     \end{equation}
  %   Then $X(W) \DominatedBy X(B) \DominatedBy X_\infty$.
  % \end{lemma}


  \begin{lemma}\label{lemma:rho-stationary}
    Let $\epsilon, p_\h \in (0,1)$. 
    Le $B$ be a characteristic string which satisfies the $(\epsilon, p_\h)$-Bernoulli condition. 
    Let $X_\infty \in \{0, 1, \ldots\}$ be a random variable 
    whose distribution $\mathcal{X}_\infty$ is defined as 
      \begin{equation}
        \label{eq:stationary}
        \mathcal{X}_\infty(k) 
          = \Pr[X_\infty = k] 
          \defeq \left(\frac{2\epsilon}{1+\epsilon}\right)\cdot \left(\frac{1-\epsilon}{1 + \epsilon}\right)^k
          \qquad \text{for $k = 0, 1, 2, \ldots$}\ 
        \,.
      \end{equation}
    Then $X(B) \DominatedBy X_\infty$.
  \end{lemma}
  Note that $p_\h$ does not appear in \eqref{eq:stationary}.

  \begin{proof}  
    %  \paragraph{$R_\infty$ stochastically dominates $X(B)$.}
   
    Examining~\eqref{eq:X-walk}, it is easy to confirm that 
    the random variable $X_n$ is precisely given by 
    the result of evolving the Markov chain in Figure~\ref{fig:walk} for $n$ steps 
    with all probability initially placed at 0. 
    It is further easy to confirm that the distribution given by~\eqref{eq:stationary} above is stationary for this chain.

    % The above Markov chain is \emph{irreducible} since every state is
    % reachable from every state.  In addition, it is easy to see that this
    % chain is \emph{aperiodic}---at time $n$, the walk visits every state
    % $s \in \{0, 1, \cdots, n\}$ with a strictly positive probability.
    % According to~\citep[Theorem 21.12]{LevinPeres}, an irreducible Markov
    % chain is positive recurrent if and only if there exists a distribution
    % $\pi$ on the state space satisfying $\pi P = \pi$, where $P$ is the
    % transition matrix of the chain.  We can take $\pi$ to be the
    % distribution given by~\eqref{eq:stationary}, implying that the chain
    % is positive recurrent.  Consequently, by ~\citep[Theorem
    % 21.14]{LevinPeres}, this irreducible, aperiodic, positive recurrent
    % Markov chain converges to the unique stationary distribution $\pi$.
    % It follows that the distributions $R_n$ limit to $R_\infty$.


    To establish stochastic dominance, it is convenient to work with the
    underlying distributions and consider walks of varying lengths: let
    $\mathcal{X}_n: \Z \rightarrow \R$ denote the probability distribution given by
    $X(B_1, \ldots, B_n)$; likewise
    define $\mathcal{X}_\infty$. For a distribution $\mathcal{X}$ on $\Z$, we define $[\mathcal{X}]_0$
    to denote the probability distribution obtained by shifting all
    probability mass on negative numbers to zero; that is, for $x \in \Z$,
    \[
      [\mathcal{X}]_0(x) = \begin{cases} \mathcal{X}(x) & \text{if $x > 0$},\\
        \sum_{t \leq 0} \mathcal{X}(t) & \text{if $x = 0$},\\
        0 & \text{if $x < 0$.}
      \end{cases}
    \]
    We observe that if $A \dominatedby C$ then $[A]_0 \dominatedby [C]_0$ for any
    distributions $A$ and $C$ on $\Z$. It will also be convenient to
    introduce the shift operators: for a distribution
    $\mathcal{X}: \Z \rightarrow \R$ and an integer $k$, we define $S^k\mathcal{X}$ to be the
    distribution given by the rule $S^k\mathcal{X}(x) = \mathcal{X}(x-k)$. With these
    operators in place, we may write
    \[
      \mathcal{X}_t = \left(\frac{1 - \epsilon}{2}\right) S^1 \mathcal{X}_{t-1} +
        \left(\frac{1 + \epsilon}{2}\right) \left[S^{-1}\mathcal{X}_{t-1} \right]_0\,,
    \]
    with the understanding that $\mathcal{X}_0$ is the distribution placing unit probability at $0$. The proof now proceeds by induction. It is clear that $\mathcal{X}_0 \dominatedby \mathcal{X}_\infty$. Assuming that $\mathcal{X}_n \dominatedby \mathcal{X}_\infty$, we note that for any $k$
    \[
      S^k \mathcal{X}_n \dominatedby S^k \mathcal{X}_\infty \qquad \text{and, additionally, that
      }\qquad [S^{-1}\mathcal{X}_n]_0 \dominatedby [S^{-1}\mathcal{X}_{\infty}]_0\,.
    \]
    Finally, it is clear that stochastic dominance respects convex combinations, 
    in the sense that if $A_1 \dominatedby C_1$ and $A_2 \dominatedby C_2$ then 
    $\lambda A_1 + (1-\lambda) A_2 \dominatedby \lambda C_1 + (1-\lambda) C_2$ (for $0 \leq \lambda \leq 1$). We conclude that
    \[
      \mathcal{X}_{t+1} = \left(\frac{1 - \epsilon}{2}\right) S^1 \mathcal{X}_{t} +
        \left(\frac{1 + \epsilon}{2}\right) \left[S^{-1}\mathcal{X}_{t} \right]_0 \dominatedby \left(\frac{1 - \epsilon}{2}\right) S^1 \mathcal{X}_{\infty} +
        \left(\frac{1 + \epsilon}{2}\right) \left[S^{-1}\mathcal{X}_{\infty} \right]_0 
        \,.
    \]
    By inspection, the right-hand side equals $\mathcal{X}_{\infty}$, as desired. 
    Hence $X(B) \dominatedby X_\infty$.
    % $\qedhere$
  \end{proof}
  \paragraph{Remark.} 
  In fact, the random variable $X(B)$
  actually converges to $X_\infty$ as $|B| \rightarrow \infty$. 
  This can be seen, for example, 
  by solving for the stationary distribution of the Markov chain in the proof above. 
  However, we will only require the dominance for our exposition. 
  % Importantly, since $\mu_x(\varepsilon) = X(x)$, and 
  % $\Pr[\mu_x(y) \geq 0]$ increases monotonically 
  % with an increase in $\Pr[\mu_x(\varepsilon) \geq r]$ for any $r \geq 0$, 
  % it suffices to take $|x| \rightarrow \infty$ 
  % when reasoning about an upper bound on $\Pr[\mu_x(y) \geq 0]$.

