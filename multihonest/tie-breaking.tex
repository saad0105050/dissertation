We can think of the longest chain tie-breaking rule (LCR) 
as a function applied by the adversary on a set of longest chains within the view of an honest observer. 
In this chapter, 
we assume that this rule is \emph{consistent}, i.e., 
if the same set of longest chains is served to two honest observers, 
they always select the same chain. 
It turns out that 
we can prove stronger consistency bounds under this assumption 
(compared to Theorem~\ref{thm:main-mh}); 
in particular, we can achieve consistency even when 
uniquely honest slots are rare (or even non-existent).

In Section~\ref{sec:lcr-model}, we specify the model and state our main theorem. 
In the subsequent section, we state a tail bound, and prove this theorem. 
Finally, in the last section, we prove this tail bound using a random walk analysis.

% \section{A consistent longest-chain selection rule}\label{sec:lcr-model} 
  % Let $\leq_\pi$ be the lexicographical ordering of the tines in a fork where 
  % each tine is represented as the list of vertex labels appearing in the tine's root-to-leaf path.
  % If two tines have the same vertex labels, 
  % $\leq_\pi$ must break tie in an arbitrary but consistent way. 
  %
  % \begin{definition}[Conservative fork]\label{def:conservative-fork}
  %   The term \emph{conservative fork} is defined inductively, as follows.
  %   The trivial fork $F_0$ for the empty string $\varepsilon$ 
  %   is conservative. 
  %   A fork $F_{T} \Fork w0$ is conservative if 
  %   the fork prefix $F_{T_1} \ForkPrefix F_T, F_{T-1} \Fork w$ 
  %   is conservative 
  %   and every honest tine $t \in F_T$ at slot $T$ 
  %   is a conservative extension with respect to $F_{T-1}$.
  %   Likewise, 
  %   a fork $F_{T} \Fork w1$ is conservative if 
  %   the fork prefix $F_{T_1} \ForkPrefix F_T, F_{T-1} \Fork w$ 
  %   is conservative and $F_T = F_{T-1}$. 
  % \end{definition}
  % Observe that a conservative fork is closed. 

  % The rushing adversary 
  % described at the outset of Section~\ref{sec:model} 
  % can always reorder messages before delivering to a recipient. 
  Let us 
  modify axiom~\ref{axiom:message-delivery} as follows:
    % the adversary uses an arbitrary (but fixed) 
    % ordering for each subset of honest blockchains. 
    % When he delivers honest blockchains 
    % to an honest player, 
    % he delivers them in this pre-specified order. 

  \begin{enumerate}[label={\textbf{A\arabic*}\ensuremath{^\prime}}., ref={\textbf{A\arabic*}\ensuremath{^\prime}}, start=0]
  \item\label{axiom:tie-breaking} 
  %     if all maximally long blockchains to be received by an honest party 
  %     are honest, 
  %     the adversary delivers these blockchains
    % in an arbitrary but consistent order.
    % Suppose two honest recipients receive 
    % the same set $L$ of maximally long blockchains. 
    % If all chains in $L$ end in honest blocks then 
    % the adversary delivers the elements of $L$ to these honest recipients 
    % in an arbitrary but consistent order.
    In addition to axiom~\ref{axiom:message-delivery},  
    an arbitrary but consistent 
    longest-chain tie-breaking rule 
    is used by all honest participants.
  \end{enumerate}
  % Note that 
  % the adversary is free to deliver adversarial blockchains 
  % in any order, interleaving the honest chains if he so wishes. 
  As a consequence, 
  if two honest participants observe 
  the same set of blockchains of maximum length, 
  they will extend the same blockchain.
  % there is no competitive adversarial blockchain 
  % when delivering to an honest recipient, 
  % he relinquishes his right to 
  % reorder the maximally long honest blockchains. 



  \begin{definition}[Bivalent characteristic
    string]\label{def:bivalent-char-string}
    Let $\slot_1, \ldots, \slot_{n}$ be a sequence of slots. 
    A \emph{bivalent characteristic string} $w$ 
    is an element of $\{\H,\A\}^n$ 
    defined for a particular execution of a blockchain protocol on these slots so that 
    for $t \in [n]$, 
    $w_t = \A$ if $\slot_{t}$ is assigned to an adversarial participant, 
    and $w_t = \H$ otherwise.
    % \[
    %   w_t =   \begin{cases}
    %   \H & \text{if $\slot_{t}$ was assigned to one or more participants, all honest},\\
    %   \A & \text{otherwise.}
    %   \end{cases}
    % \]
  \end{definition}
  The definition of a fork for a bivalent characteristic string is
  identical to Definition~\ref{def:fork} (somewhat simplified as a
  bivalent string does not contain any $\h$ symbol).
  Also note that the $(\epsilon, 0)$-condition 
  from Definition~\ref{def:bernoulli-cond} 
  is well-defined for bivalent characteristic strings.
  %  except the following: 
  % \begin{enumerate}[label=(F{\arabic*}.),start=3]
  %   \item\label{fork:multiply-honest-bivalent}
  %     % \label{fork:unique-honest}\label{fork:multiply-honest} 
  %     each index in $\{j : w_j = \H\}$ 
  %     is the label of at least one vertex of $F$.
  % \end{enumerate}
  % However, condition~\ref{fork:multiply-honest-bivalent} above 
  % is identical to the condition~\ref{fork:multiply-honest} 
  % in Definition~\ref{def:fork} 
  % since $w$ does not contain the symbol $\h$.

  Let $w$ be a bivalent characteristic string, $F$ a fork for $w$, and
  $F'$ a fork for $w\H$ so that $F \ForkPrefix F'$ and any honest
  vertex in $F' \setminus F$ has label $|w| + 1$.  If $F$ contains a
  maximum-length adversarial tine, there is no guarantee that two
  honest observers at slot $|w| + 1$ will agree on the longest chain:
  the adversary may chose to expose the adversarial chain to one and
  not the other.
  In this case, we say that \emph{$F$ has a tie for the longest-chain
  rule}---or, in short, that \emph{$F$ has an LCR tie}.  
  % For example, consider the two honest vertices 
  % with label 6 in Figure~\ref{fig:fork}. 
  % Although both observe 
  % the (honest) vertex with label 5, 
  % only the top one observes 
  % the middle (adversarial) vertex with label 4.
  When there
  is no LCR tie (that is, no maximum-length adversarial tine), all
  honest slot leaders at slot $|w| + 1$ necessarily extend the same
  honest tine determined by the consistent longest-chain tie-breaking
  rule.



  \begin{theorem}[Main theorem; consistent tie-breaking]\label{thm:main-mh-bivalent}
    % Let $\epsilon \in (0, 1), s, k, T \in \NN$.  Let
    % $\mathcal{W}$ and $\tilde{\mathcal{B}}_\epsilon$ be two distributions on
    % $\{\H, \A\}^T$ where 
    % $\tilde{\mathcal{B}}_\epsilon$ is defined above and
    Let $\epsilon \in (0, 1)$ and $s, k, T \in \NN$.  
    Let $\mathcal{B}$ be a distribution 
    on length-$T$ bivalent characteristic strings 
    satisfying the $(\epsilon, 0)$-Bernoulli condition. 
    Let 
    $\mathcal{W}$ be a distribution on
    $\{\H, \A\}^T$ so that 
    $\mathcal{W} \DominatedBy \mathcal{B}$. 
    Then 
    $
      \mathbf{S}^{s,k}[\mathcal{W}] 
        \leq \mathbf{S}^{s,k}[\mathcal{B}] 
        \leq 
        \exp\bigl(-k \cdot \Omega(\epsilon^3 (1 + O(\epsilon) ) )\bigr)
      % \,.
    $. 
    % for large $k$. 
    (Here, the asymptotic notation hides constants that do not depend on $\epsilon$ or $k$.)
  \end{theorem}

  The proof is deferred to section~\ref{sec:lcr-tie-bound-main-thm-proof}.
  Note that the theorem above states that a 
  PoS protocol can achieve optimal consistency error 
  even with a leader election scheme 
  that produces no uniquely honest slots. 
  In contrast, Theorem~\ref{thm:main-mh} requires 
  a non-zero probability for uniquely honest slots.


\section{Catalan slots and the UVP}\label{sec:catalan-uvi-lcr-tie}
  

  The following theorem shows that under axiom~\ref{axiom:tie-breaking}, 
  two consecutive Catalan slots 
  imply that the first slot has the UVP.

  \begin{theorem}\label{thm:multiple-honest}
    Let $w \in \{\H, \A\}^T$ be a bivalent characteristic string 
    and axiom~\ref{axiom:tie-breaking} is satisfied.
    % where 
    % all honest slot leaders use the consistent longest-chain selection rule $\LCR(\cdot)$.
    Let $s \in[2, T]$ be an integer such that 
    $s$ and $s - 1$ are two honest slots in $w$. 
    The following statements are equivalent:
    \begin{enumerate*}[label=(\textit{\roman*})]
      \item\label{thm:part-catalan-multiple-honest} 
      Slots $s, s - 1$ are Catalan.

      \item\label{thm:part-cp-multiple-honest} 
      If $s \leq T - 1$, both $s$ and $s - 1$ have the UVP. 
      Otherwise, slot $T - 1$ has the UVP but 
      slot $T$ has the bottleneck property.

      % \item\label{thm:part-margin-multiple-honest} 
      % Slots $s, s - 1$ are margin-critical.
    \end{enumerate*}
  \end{theorem}
  \begin{proof}
    Since the slots $s, s - 1$ satisfy the (weaker) bottleneck property, 
    Fact~\ref{fact:almost-cp-implies-catalan} implies that 
    they must be Catalan slots. 
    This proves 
    ~\ref{thm:part-cp-multiple-honest} implies~\ref{thm:part-catalan-multiple-honest}.
    
    Now let us prove that ~\ref{thm:part-catalan-multiple-honest} implies~\ref{thm:part-cp-multiple-honest}. 
    Slots $s, s - 1$ are Catalan. 
    Let $V_{s}$ (resp. $V_{s+1}$) be the set of all viable tines 
    at the onset of slot $s$ (resp. slot $s + 1$). 
    Since $s - 1$ (resp. $s$) is a Catalan slots, we use 
    Fact~\ref{fact:catalan-unique-longest} and conclude that 
    $V_s$ (resp. $V_{s+1}$) can contain 
    only maximum-length honest tines $t, \ell(t) = s - 1$ (resp. $\ell(t) = s$). 
    Let $u_s \in V_s$ be the unique vertex determined 
    by the consistent tie-breaking rule when applied to the set $V_s$. 
    Define $u_{s+1} \in V_{s+1}$ in an analogous way for the set $V_{s+1}$.

    Let $k \in [s + 1, T+1]$ be an integer.
    We wish tho show that for every tine $t$ viable at the onset of slot $k$, 
    the following holds: 
    (i) if  $s \leq T - 1$ then $u_s \Prefix u_{s+1} \PrefixEq t$, and 
    (ii) if $s = T$ then 
    $u_{T-1} \Prefix t$ where $\ell(t) = T$. 

    % Since $V_s$ contains only honest tines, 
    All tines at the honest slot $s$ build upon $u_s$. 
    If $s = T$, we are done. 
    Otherwise, i.e., if $s \leq T - 1$, 
    let $\tau = u_{s + 1}$ and note that $u_s \Prefix u_{s + 1} = \tau$. 
    If $k = s + 1$, we are done since by Fact~\ref{fact:catalan-unique-longest}, 
    every tine at the honest slot $k$ will build upon $\tau$.  

    It remains to reason about the case $s \leq T - 2$ and $k \geq s + 2$. 
    Consider a tine $t$ which is viable at the onset of slot $k$. 
    (All we know about $t$'s label is that $\ell(t) \leq k - 1$.) 
    We claim that $\tau \Prefix t$. 
    Suppose, towards a contradiction, that 
    $\tau$ is not a prefix of $t$. 
    Let $B_1$ be the last honest vertex on $t$ such that $\ell(B_1) \leq s - 1$.
    (If no such vertex can be found, take $B_1$ as the root vertex.) 
    Likewise, let $B_2$ be the first honest vertex on $t$ such that $\ell(B_2) \in [s + 1, k - 1]$. 
    
    Below, we show that every choice for $B_1, B_2$ leads to a contradiction 
    and, therefore, $\tau$ must be a prefix of $t$. 
    If $B_2$ exists then, 
    by construction, $\ell(B_1) < s < \ell(B_2) \leq k - 1$.
    If $\ell(B_2) = s + 1$ then, as we have argued earlier, 
    $B_2$ must have built on $\tau$. 
    This contradicts our assumption that 
    $\tau$ is not a prefix of $t$. 
    Otherwise, suppose $\ell(B_2) \geq s + 2$.  
    Let $I$ be the interval $[\ell(B_1) + 1, \ell(B_2) - 1]$ and note that  
    $I$ contains $s$. 
    There can be two scenarios.
    If $t$ contains an adversarial vertex between $B_1$ and $B_2$ then,  
    by Corollary~\ref{coro:interval-honest-vertices}, 
    $I$ must be $\Aheavy$; but 
    this contradicts the assumption that $s$ is a Catatan slot. 
    Otherwise,
    $B_2$ builds on top of $B_1$ and, 
    in particular, 
    $B_1$ must be viable at the onset of slot $\ell(B_2) \geq s + 1$. 
    Since $\ell(\tau) = s$, this means $\length(B_1) \geq \length(\tau)$. 
    However, since $\ell(B_1) < s$, 
    by the monotonicity of the honest-depth function $\hdepth(\cdot)$, 
    $\length(\tau) \geq 1 + \length(B_1)$. 
    This contradicts the inequality above.           

    If $B_2$ does not exist then 
    we claim that $t$ is an adversarial tine. 
    To see why, note that if $t$ were honest and $\ell(t) \geq s + 1$ 
    then there would have been a $B_2$. 
    If $t$ were honest with $\ell(t) = s, t \neq \tau$ 
    then $t$ would not be viable at the onset of slot $s + 2$. 
    This is because $s$ is a Catalan slot and as such, 
    each vertex from slot $s + 1$ builds on $\tau, \length(\tau) \geq \length(t)$. 
    Hence tines viable at the onset of slot $s + 2$ must have length at least $1 + \length(\tau) > \length(t)$. 
    Finally, if $t$ is honest and $\ell(t) \leq s - 1$ then, 
    by Fact~\ref{fact:catalan-unique-longest}, 
    $t$ cannot be viable at the onset of slot $s + 1$ 
    since $s$ is Catalan.  
    Since $s + 1$ is an honest slot, 
    honest tines with label $s + 1$ will be strictly longer than $t$ 
    and, therefore, 
    $t$ cannot be viable at the onset of slot $k \geq s + 2$ either. 
    We conclude that $t$ must be an adversarial tine viable at the onset of slot $k$. 
    By Proposition~\ref{prop:fork-structure},       
    the interval $I = [\ell(B_1) + 1, k - 1]$ must be $\Aheavy$. 
    However, since $I$ contains $s$, it contradicts the fact that $s$ is a Catalan slot.     
  \end{proof}


\section{A tail bound on Catalan slots; Proof of Theorem~\ref{thm:main-mh-bivalent}}\label{sec:lcr-tie-bound-main-thm-proof}

  \begin{bound}\label{bound:two-catalans}
    Let $T, s, k \in \NN, T \geq s + k$ and  $\epsilon \in (0, 1)$. 
    Let $w$ be a bivalent characteristic string satisfying 
    the $(\epsilon, 0)$-Bernoulli condition 
    and let $y = w_s \ldots w_{s+k-1}$.
    Then 
    \[
      \Pr_w[\text{$w$ does not contain two consecutive Catalan slots in $y$}]  
        \leq 
        \exp\left(
          - k\cdot \Omega(\epsilon^3(1 + O(\epsilon))) 
        \right)
        \,.
    \]
  \end{bound}

\Paragraph{Proof of Theorem~\ref{thm:main-mh-bivalent}.}
  This proof is identical to the proof of Theorem~\ref{thm:main-mh} 
  except that 
  we need to refer to Theorem~\ref{thm:multiple-honest} in lieu of Theorem~\ref{thm:unique-honest}
  and Bound~\ref{bound:two-catalans} in lieu of Bound~\ref{bound:unique-honest-catalan}.
  \hfill $\qed$  


\section{Proof of \texorpdfstring{Bound~\ref{bound:two-catalans}}{the Tail Bound} }\label{sec:lcr-tie-estimates}
  % \input{estimates-two}
  % \begin{proof}
  Let $p = (1 - \epsilon)/2$ and $q = 1 - p$; 
  thus $q - p = \epsilon$. 
  Let $B$ denote the event that 
  $w$ does not contain two consecutive Catalan slots in $y$. 
  We would like to bound $\Pr_w[B]$ from above.

  Define the process $W = (W_t : t \in \NN), W_t \in \{\pm 1\}$ as $W_t = 1$ if and only if $w_t = \A$. 
  Let $S = (S_t : t \in \NN), S_t = \sum_{i \leq t} W_i$ be the position of the particle at time $t$. 
  Thus $S$ is a random walk on $\ZZ$ with $\epsilon$ negative (i.e., downward) bias. 
  By convention, set $W_0 = S_0 = 0$. 

  \paragraph{Case 1: $x$ is an empty string.} 
  In this case, we write $w = yz$ so that $|y| = k$. 
  Let $m_t$ denote the probability that 
  $t$ is the first index so that both $t$ and $t+1$ are Catalan slots in $w$, 
  with $m_0 = 0$, and consider the probability generating function 
  $\{m_t\} \SeqGF \gfM(Z) = \sum_{t = 0}^\infty m_t Z^t$. 
  Controlling the decay of the coefficients $m_t$ suffices
  to give a bound on $\Pr[B]$, i.e., 
  the probability that 
  $y$ \emph{does not} contain two consecutive Catalan slots, 
  because this probability is at most 
  $
    1 - \sum_{t =0}^{k-1} m_t 
      = \sum_{t = k}^\infty m_t
      % \,.
  $. 
  % It seems challenging to give a closed-form algebraic expression for
  % the generating function $\gfM$; 
  To this end, we develop a
  closed-form expression for a related probability generating function
  $\gfMhat(Z) = \sum_t \hat{m}_t Z^t$ which stochastically
  dominates $\gfM(Z)$. 
  Recall that this means that for any $k, \sum_{t \geq k} m_k \leq \sum_{t \geq k} \hat{m}_k$. 
  Finally, bound the latter sum  
  by using the analytic properties of $\gfMhat(Z)$. 


  % \paragraph{Generating function for the first instance of consecutive Catalan slots.}
  Recall the ``first ascent'' and ``first descent'' 
  generating functions $\gfA(Z)$ and $\gfD(Z)$ 
  from the proof of Bound~\ref{bound:unique-honest-catalan}. 
  We wish to devise the generating function for 
  the first occurrence of a left-Catalan slot 
  immediately followed by a right-Catalan slot. 
  To that end, 
  note that $\gfD(Z)$ is the generating function for 
  the first left-Catalan slot. 
  The generating function for the first right-Catalan slot 
  can be devised as follows. 
  Consider the walk $S$ starting at the origin. 
  With probability $q(1 - p/q) = \epsilon$, the walk will 
  immediately descend a step and never return to the origin. 
  But this means $S_1 \leq S_t, t \geq 2$ and hence 
  the first slot is a right-Catalan slot and we are done. 
  Otherwise, i.e., with probability $1 - \epsilon$, 
  the walk makes a (guaranteed) return to the origin in future. 
  In this case, we will have to restart our search 
  for the next consecutive Catalan slots but, 
  before that, 
  we will have to ensure that we are in a ``safe position.'' 
  In particular, we can safely restart our search if 
  Specifically, if the current position (i.e., level) of the walk is at its historical minimum, 
  we can restart our search by applying $\gfD(Z)$ to find the next left-Catalan slot.
  Thus an ``epoch'' begins with a guaranteed return and 
  ends when the walk descends to a new level for the first time. 
  Let $\gfE(Z)$ be the generating function of an epoch. 
  Thus we can write 
  \begin{align}\label{eq:gfM}
    \gfM(Z) 
    &= \gfD(Z) \cdot \{\epsilon + (1-\epsilon)\gfE(Z)\gfM(Z) \} \nonumber \\
    &= \frac{\epsilon \gfD(Z)}{1 - (1 - \epsilon) \gfE(Z) }
    \,.
  \end{align}

  
  An epoch can have two shapes. 
  If an epoch starts with an up-step (i.e., an ``up'' shape), 
  it is easy to see that the epoch ends as soon as the walk 
  returns to the origin from above and, importantly, 
  that the walk will (eventually) return to the origin with probability one. 
  However, if the epoch starts with a down-step (i.e., a ``down'' shape), 
  we have to ``remember'' the lowest level $\ell$ touched 
  by the walk in its way to its (sure) ascent to the origin 
  and then descend $\ell$ levels to end the epoch. 
  In particular, we have to ensure that we return to the origin with probability one. 
  
  A generating function of a stopping time of a random walk 
  is ill suited to ``remember'' its historical minimum/maximum. 
  However, it can remember the length of the walk for free. 
  Thus, instead of working directly with $\gfE(Z)$, 
  we work with a generating function $\gfEhat(Z)$ 
  which is identical to $\gfE(Z)$ for the up shape 
  but differs in the down shape. 
  Specifically, in the down shape, 
  the walk represented by $\gfEhat(Z)$ descends as many levels 
  as the number of steps it took to return to the origin. 
  Clearly, $\gfE \DominatedBy \gfEhat$ where 
  \[
      \gfEhat(Z) \triangleq p Z \gfD(Z) + q Z \gfA(Z \gfD(Z) )/\gfA(1)
      \,.
  \] 
  Here, the first term denotes the ``return to origin from above'' shape. 
  An individual term in $\gfA(Z \gfD(Z)) = \sum_t a_t Z^t \gfD(Z)^t$ 
  has the interpretation 
  ``if the first ascent took $t$ steps then follow it by descending $t$ levels.''
  Since $\gfA(Z)$ is not a probability generating function, 
  we have to normalize it by $\gfA(1)$ to denote that 
  the ascent happens with certainty. 
  This implies, 
  \[
      \gfM(Z) 
          \DominatedBy \gfMhat(Z) 
          \triangleq \frac{\epsilon \gfD(Z)}{1 - (1 - \epsilon) \gfEhat(Z) }
  \]

  % \paragraph{Convergence of $\gfM(Z)$.}
  It remains to establish a bound on the radius of convergence of
  $\gfMhat$. 
  A sufficient condition for the convergence of
  $\gfMhat(z)$ for some $z \in \RR$ is 
  that all generating functions appearing in the definition of
  $\gfMhat$ converge at $z$ and 
  that $(1-\epsilon) \gfEhat(Z) \neq 1$. 

  By retracing our footsteps as in the proof of Bound~\ref{bound:unique-honest-catalan}, 
  we can see that $\gfD(z), \gfA(z)$, and $\gfA(z \gfD(z))$ converge 
  when $|z|$ satisfies~\eqref{eq:roc-AZDZ}. 
  Moreover, since $\gfD(Z)$ is a probability generating function, 
  it follows that $\gfEhat(Z)$ is stochastically dominated by 
  $p Z \gfD(Z) + q Z \gfA(Z \gfD(Z) )/\gfA(1) \cdot \gfD(Z)$.
  Therefore, when $\gfEhat(z)$ converges for some $z$, it satisfies 
  \begin{align*}
      \gfEhat(z)
      &\leq pz\gfD(z) + (q/p) (q z\gfD(z))\gfA(z\gfD(z)) \\
      &< 1/2 + (q/p)/2
  \end{align*}
  since $\gfA(1) = p/q, pz\gfD(z) < 1/2$, 
  and $qx\gfA(x) < 1/2$ for any $z, x$ so that $\gfA(x)$ and $\gfD(z)$ converge, respectively. 
  Therefore, $(1-\epsilon)\gfEhat(z) = 2p \gfEhat(z) < p + q = 1$. 
  It follows that 
  $\gfMhat(z)$ converges for
  $|z| < 1 + \epsilon^3/2 + O(\epsilon^4) \leq \exp(\epsilon^3/2 + O(\epsilon^4))$. 
  Recall that if the radius of convergence of
  $\gfMhat$ is $\exp(\delta)$ then 
  $\Pr[B]$ is 
  $O(1) \cdot e^{-\delta k}$. 
  We conclude that
  \begin{align}
    \Pr_w[B] 
      &\leq O(1) \cdot e^{-\epsilon^3(1 + O(\epsilon))k/2} \,.
  \label{eq:prob_two_catalan_gf}
  \end{align}


  \paragraph{Case 2: $x$ is non-empty.}
  This part of the proof is the same as the $|x| \geq 1$ case 
  in the proof of Bound~\ref{bound:unique-honest-catalan}. 
  The only difference is that 
  $\gfChat(Z)$ and $\gfCtilde(Z)$ would be replaced by 
  $\gfMhat(Z)$ and $\gfMtilde(Z)$, respectively, where 
  \[
    \gfMtilde(Z)\DominatedBy \sum_{h = 0}^\infty \mathcal{X}_\infty(h) \gfD(Z)^h \gfMhat(Z)
    \,.
  \]
  We conclude that the bound
  in~\eqref{eq:prob_two_catalan_gf} holds when $|x| \geq 0$. 
  \hfill$\qed$
 % \end{proof}



