

In this chapter, 
we describe a simple adversarial strategy to build a fork 
which simultaneously maximizes the relative margins for all slots. 
Although the proof is completed in Section~\ref{sec:opt-adversary-thm-proof}, 
in Section~\ref{sec:opt-adv-lowerbound}, 
we show that the relative margin achieved by the adversary $\Adversary^*$ 
is at least as high as the best-possible value 
(assuming Theorem~\ref{thm:relative-margin}, which is later proved in Section~\ref{sec:relative-margin-thm-proof}).

\section{An optimal online adversary against slot settlement}
\label{sec:opt-adversary}
Let $w$ be a characteristic string. 
For a fixed decomposition $w = xy$, 
there is an adversary\footnote{
  Specifically, 
  let $w' = xyb$ 
  where $b \in \{\h, \H, \A\}$. 
  This strategy recursively builds a closed fork $F \Fork xy$. 
  Then, upon encountering $b$, 
  it augments $F$ 
  by making zero, one, or two conservative extensions, as follows: 
  If $b = \A$, it does nothing. 
  If $b = \h$, it extends a zero-reach tine if possible; 
  otherwise,it extends a maximum-reach tine. 
  If $b = \H$, it extends a pair of tines that 
  witness $\mu_x(F)$. 
  By following the arguments in~\cite{LinearConsistencySODA}, 
  one can show that 
  if $\mu_x(F) = \mu_x(y)$ then 
  $\mu_x(F')$ is 
  at least as large as 
  the right-hand side in~\eqref{eq:mu-relative-recursive-mh}.   
} 
who builds a fork $F \Fork xy$ 
so that the $\mu_x(F)$ is 
at least as large as 
the right-hand side of~\eqref{eq:mu-relative-recursive-mh}. 
However, 
in light of Lemma~\ref{lemma:uvp-margin}, 
if an adversary wants to violate the settlement 
\emph{of all possible slots of $w$ at once}, 
% to show that the lowerbound is satisfied 
% \emph{simultaneously} 
% for all decompositions $w = xy$, 
% one has to do more work.
he needs to produce a fork $F$ for $w$ 
so that $\mu_x(F) \geq 0$ 
for every prefix $x \PrefixEq w$. 
In Figure~\ref{fig:adv-opt-mh}, 
we describe a strategy $\Adversary^*$ 
which does even better: 
it produces a fork $F$ so that $\mu_x(F) = \mu_x(y)$ 
for every prefix $x \PrefixEq w$. 


$\Adversary^*$ builds a fork for $w = w_1 \ldots w_{n+1}$ 
in an online fashion, i.e., 
it scans $w$ once, from left to right, 
maintains a fork $F_n$ after scanning 
the first $n$ symbols, 
and augments $F_n$ by conservatively extending 
zero-reach tine(s) using label $n + 1$.
Specifically, if $w_{n+1} = \A$, $\Adversary^*$ does nothing. 
If $w_{n+1} = \h$, it (obviously) makes a single extension. 
Now suppose $w_{n+1} = \H$. 
It still makes a single extension
if either $F_n$ contains exactly one zero-reach tine 
or $F_n$'s reach is positive. 
Otherwise, 
if $\rho(F_n) = 0$ 
and there are at least two zero-reach tines in $F_n$, 
$\Adversary^*$ extends two zero-reach tines 
that diverge earliest in $F_n$.



\begin{figure}[!h]
  \begin{center}
    \fbox{
      \begin{minipage}{.9 \textwidth}
        \begin{center}
          \textbf{The strategy $\Adversary^*$}
        \end{center}
        Let $n$ be a non-negative integer, 
        $w \in \{\h, \H, \A\}^n$, 
        and $w_{n + 1} \in \{\h, \H, \A\}$. 
        If $n = 0$, set $F_0 \Fork \varepsilon$ as 
        the trivial fork comprising a single vertex. 
        Otherwise, 
        let $F_n$ be the closed fork 
        built recursively by $\Adversary^*$ for the string $w$. 
        If $w_{n + 1} = \A$, 
        output $F_n$ (as a fork for $w w_{n+1}$). 
        Otherwise, 
        let $Z$ and $R$ be the set of zero-reach tines 
        and maximum-reach tines in $F_n$, respectively.

        \begin{enumerate}
          \item 
          Identify a set $S$ as follows: 
          If $|Z| = 1$ then set $S = Z$. 
          Otherwise, 
          let $r_1 \in R, z_1 \in Z$ be two tines so that 
          $\ell(r_1 \Intersect z_1) = 
          \min\{ \ell(r \Intersect z) :  r \in R, z \in Z \}$ 
          and set  
          \[
          S = \begin{cases}
            \{z_1\} & \text{
              if $w_{n + 1} = \h$ 
              or $\rho(F_n) \geq 1$ 
              % or for all prefix $x \Prefix w$, $\mu_x(F_n) < 0$}
              % or $|Z| = 1$
              }\,, \\
            \{z_1, r_1\} & \text{otherwise}\,.
          \end{cases}
          \]

          \item
          Conservatively extend 
          each tine in $S$ 
          using label $n + 1$. 
          Let $F_{n + 1} \Fork w w_{n+1}$ 
          be the new closed fork. 
          Output $F_{n+1}$.
        \end{enumerate}
      \end{minipage}
    }
  \end{center}
  \caption{Optimal online adversary $\Adversary^*$}
  \label{fig:adv-opt-mh}
\end{figure}






\begin{definition}[Canonical fork]
  A \emph{canonical fork} for $w \in \{\h, \H, \A\}^*$ 
  is a closed fork $F \Fork w$ so that 
  $\rho(F) = \rho(w)$ 
  and, for all prefixes $x \Prefix w$, $\mu_x(F) = \mu_x(y)$. 
  If $|w| = 0$, $F$ is 
  the unique fork with a single (honest) vertex and no edge. 
  % Let $w_1 \ldots w_T \in \{0,1\}^T$. 
  % For $n = 0, 1, \ldots, T$, a \emph{canonical fork $F_n$ for $w = w_1\ldots w_n$} 
  % is inductively defined as follows. 
  % If $n = 0$ then $F_0$ is the trivial fork for the empty string; 
  % it consists of a single (honest) vertex and no edge. 
  % If $n \geq 1$, the following holds: 
  % $F_n$ is a closed fork so that $F_{n-1} \ForkPrefix F_n$. 
  % $F_n$ contains an honest tine $\tau_\rho$ so that 
  % $\reach(\tau_\rho) = \rho(F_n) = \rho(w)$. 
  % For every decomposition $w = xy, x \Prefix w$, 
  % % $\mu_x(F_n) = \mu_x(y)$ and, in addition, 
  % $F_n$ contains two honest tines $\tau_x, \tau_{\rho x}$ 
  % so that the tine-pair $(\tau_{\rho x}, \tau_x)$ witnesses $\mu_x(F_n) = \mu_x(y)$. 
  % The (possibly non-distinct) designated tines $\tau_\rho, \tau_{\rho x}, \tau_x, x \Prefix w$ 
  % are called the \emph{witness tines}. 
  % For the sake of completeness, define $\tau_w = \tau_{\rho w} = \tau_\rho$.
\end{definition}

It is not obvious whether a canonical fork always exists 
or whether it can be found algorithmically. 
The theorem below gives us the assurance:


\begin{theorem}\label{thm:opt-adversary-canonical}
  Let $w \in \{\h, \H, \A\}^*$. 
  The strategy $\Adversary^*$ in Figure~\ref{fig:adv-opt-mh}
  outputs a canonical fork for $w$.  
\end{theorem}
That is, for every characteristic string $w$ 
there is a fork $F \Fork w$ so that 
for every prefix $x \PrefixEq w$, $\mu_x(F) = \mu_x(y)$. 
Note that if one's objective is to create a fork 
which contains many early-diverging tine-pairs (that witness large relative margins), 
a canonical fork is the best one can hope for. 
This is why $\Adversary^*$ is called an \emph{optimal} online adversary. 
The proof of Theorem~\ref{thm:opt-adversary-canonical} 
is given in Section~\ref{sec:margin-proof-multihonest}.



\section{\texorpdfstring{$\Adversary^*$}{The optimal adversary} simultaneously maximizes all relative margins}\label{sec:opt-adv-lowerbound}

\begin{proposition}\label{prop:mu-lowerbound}
  Let $w \in \{\h, \H, \A\}^*$ 
  and $b \in \{\h, \H, \A\}$. 
  Assume that Theorem~\ref{thm:opt-adversary-canonical} 
  holds for characteristic strings of length $|w|$.
  Let $F'$ be the fork built by $\Adversary^*$ 
  for the characteristic string $wb$. 
  Then 
  \begin{equation}
    \rho(F') \geq \begin{cases}
     \rho(xy) + 1 & \text{if $b = \A$} \,, \\
     0 & \text{if $b \in \{\h, \H\}$ and $\rho(xy) = 0$}\,, \\
     \rho(xy) - 1 & \text{otherwise}\,.
    \end{cases}
    \label{eq:rho-lowerbound}    
  \end{equation}
  Furthermore, for any decomposition $w = xy, |y| \geq 0$, 
  \begin{equation}
    \mu_x(F') \geq \begin{cases}
      \mu_x(y) + 1 & \text{if $b = \A$}\,, \\
      0 & \text{if $b \in \{\h, \H\}$ and $\rho(xy) > \mu_x(y)=0$}\,, \\
      0 & \text{if $b = \H$ and $\rho(xy) = \mu_x(y) = 0$}\,, \\
      \mu_x(y)-1 & \text{otherwise.}
    \end{cases}
    \label{eq:mu-lowerbound}
  \end{equation}
  
\end{proposition}
\begin{proof}  
  Let $w' = wb$.
  Let $F$ and $F'$ be the forks built by $\Adversary^*$ 
  for the characteristic string $w$ and $wb$, respectively, 
  so that $F \fprefix F'$.
  By assumption, $F$ is a canonical fork for $w$; 
  this means $\rho(F) = \rho(w)$ and
  for all $x \Prefix w$, $\mu_x(F) = \mu_x(y)$. 
  It will be helpful for the reader to recall Fact~\ref{fact:reach-fork-ext-mh} before proceeding.


  \paragraph{Proving~\eqref{eq:rho-lowerbound}.} 
  We wish to show that 
  $\rho(F')$ satisfies~\eqref{eq:rho-lowerbound}. 
  If $b = \A$ then, by construction,
  $F' = F$. 
  The symbol $b = \A$ increases the reserve 
  of every tine by one. 
  Thus  
  $
  \rho(F') 
  = \rho(F) + 1
  = \rho(xy)  + 1
  $. 
  % Here, the first equality follows from our choice of $\tau_\rho$. 
  % and the last equality follows from our assumption about $t_\rho$. 
  Now suppose $b \in \{\h, \H\}$. 
  Since all tines $\sigma \in F'$ with label $|xy| + 1$ 
  are conservative extensions, $\reach_{F'}(\sigma) = 0$ 
  and the $F'$-reach of all $F$-tines decreases by one. 
  Let $t$ be a maximum-reach tine in $F$; 
  since $F$ is canonical, $\reach_F(t) = \rho(F) = \rho(xy)$.
  Therefore, 
  $\rho(F') \geq \reach_{F'}(t) = \reach_{F}(t) - 1 = \rho(xy) - 1$. 
  If $\rho(F) = 0$ then this inequality can be tightened, as follows. 
  As all $F$-tines have negative $F'$-reach, 
  any maximum-reach $F'$-tine 
  must be one of the extensions; 
  it follows that $\rho(F') = 0$. 
  Thus we have proved~\eqref{eq:rho-lowerbound}. 



  \paragraph{Proving~\eqref{eq:mu-lowerbound}.} 
  Let $w = xy$ be an arbitrary decomposition; 
  this $x$ remains fixed for the remainder of the proof. 
  (Note that $\Adversary^*$ is unaware of this decomposition.) 
  % Let 
  % \begin{align*}
  %   R' &=  \{t \in F' : \reach_{F'}(t) = \rho(F') \}\,\quad\text{and} \\
  %   B_x &= \{t \in F' : \exists t' \in R'\,, \ell(t \Intersect t') \leq |x| \}\,.
  %   \,.
  % \end{align*}
  % % and set $C_x = \{t : t \in B_x, \reach_{F'}(t) = r_x\}$.
  % % Finally, identify the tine-pair 
  % % $(\tau_{\rho x}, \tau_x) = \EarlyDivWitness(R', C_x)$. 
  % Finally, let $\tau_{\rho x} \in R', \tau_x \in B_x$ be two tines 
  % so that 
  % \begin{align*}
  % \reach_{F'}(\tau_x) &= \max\{\reach_{F'}(t) : t \in B_x\}\quad\text{and} \\
  % \ell(\tau_{\rho x} \Intersect \tau_x) &= \min \{ \ell(t_1 \Intersect t_2) : t_1 \in R', t_2 \in B_x\}
  % \,.  
  % \end{align*}
  % % Note that 
  % % $\reach_{F'}(\tau_\rho) = \reach_{F'}(\tau_{\rho x}) = \rho(F')$ 
  % % and that 
  % % $\tau_\rho$ and $\tau_{\rho x}$ may be the same tine.
  % We say that the tines $\tau_x, \tau_{\rho x} \in F'$ \emph{witness} $\mu_x(F')$.
  

  Let $\tau_x, \tau_{\rho x} \in F'$ be two $yb$-disjoint tines 
  so that 
  $\reach_{F'}(\tau_{\rho x}) = \rho(F')$, 
  $\reach_{F'}(\tau_x) = \mu_x(F')$, 
  and, of all $yb$-disjoint tine pairs in $F'$
  that attain this requirement, 
  these two tines diverge the earliest. 
  We say that the tines $\tau_x, \tau_{\rho x}$ \emph{witness} $\mu_x(F')$. 


  Designate the witness tines $t_x, t_{\rho x} \in F$ 
  in the same way as we have designated $\tau_x, \tau_{\rho x} \in F'$; 
  specifically, 
  $w, y$, and $F$ would substitute $w', yb$, and $F'$ 
  in the recipe above. 
  By assumption, $F$ is a canonical fork for $xy$. 
  Therefore, 
  $\rho(F) = \reach_F(t_{\rho x}) = \rho(xy)$, 
  $t_x$ is $y$-disjoint with $t_{\rho x}$, 
  and 
  $\mu_x(F) = \reach_F(t_x) = \mu_x(y)$. 
  % (It is possible that $t_\rho = t_{\rho x}$.)
  We wish to show that 
  $\mu_x(F')$ satisfies~\eqref{eq:mu-lowerbound}. 

  If $b = \A$ then, by construction,
  $F' = F$ and, therefore,
  % $\tau_\rho = 
  % t_\rho$, 
  % $\tau_x = t_x$, and $\tau_{\rho x} = t_{\rho x}$. 
  % Thus 
  % $\tau_x$ and $\tau_{\rho x}$ are $yb$-disjoint.
  % Since the $F'$-reach of every $F$-tine is one plus its $F$-reach, 
  % $
  % \mu_x(F') \geq 
  % \reach_{F'}(\tau_x)
  % = \reach_{F'}(t_x)
  % = \reach_F(t_x) + 1
  % = \mu_x(y) + 1
  % $. 
  $t_x$ and $t_{\rho x}$ are $yb$-disjoint in $F'$.
  Note that the $F'$-reach of every $F$-tine is one plus its $F$-reach. 
  Therefore, 
  $
  \mu_x(F') 
  \geq \min(\reach_{F'}(t_{\rho x}), \reach_{F'}(t_x))
  = \reach_{F'}(t_x)
  = \reach_F(t_x) + 1
  = \mu_x(y) + 1
  $. 


  If $b \in \{\h, \H\}$, 
  all tines in $F'$ with label $|w| + 1$ arise from conservative extensions. 
  Since the tines $t_x, t_{\rho x}$ are $yb$-disjoint in $F'$, 
  it follows that $\mu_x(F') \geq \min(\reach_{F'}(t_x), \reach_{F'}(t_{\rho x})) \geq \reach_{F'}(t_x) = \reach_{F}(t_x) - 1 = \mu_x(y) - 1$. 
  Here, the first inequality follows from the definition of relative margin and 
  the second one from the fact that $\reach(t_x) \leq \reach(t_{\rho x})$ by assumption. 
  The first equality follows from Fact~\ref{fact:reach-fork-ext-mh} and 
  the second one follows from our assumption that the tines $t_{\rho x}, t_x\in F$ witness $\mu_x(F) = \mu_x(y)$. 

  However, we can tighten the above inequality when $\mu_x(y)$ is zero, as follows. 
  Recall the sets $Z, S, R$, 
  the zero-reach tine $z_1$, 
  and the maximum-reach tine $r_1$ 
  from Figure~\ref{fig:adv-opt-mh}. 
  Also recall that $z_1$, of all zero-reach tines, 
  diverges earliest from any maximum-reach tine.
  As $\reach_F(z_1) = \mu_x(F) = \mu_x(y) = 0$, 
  it follows that 
  % $z_1 \in Z$. 
  % Thus there must be a tine $t^* \in R$ 
  % $\ell(z_1 \Intersect t^*) = \min\{\ell(z \Intersect t) : z \in Z, t \in R\}$.
  % In particular, 
  $z_1$ and $r_1$ must be $y$-disjoint.
  Let $\sigma_1 \in F'$ be the conservative extension of $z_1$.


  \begin{description}[font=\normalfont\itshape\space]
    \item[If $\rho(xy) \geq 1$ and $\mu_x(y) = 0$]
      then $\sigma_1$ is the only new extension in $F'$ 
      and it has reach zero in $F'$.
      % then $\sigma_1$ must be $yb$-disjoint with $t^*$ in $F'$.
      Note that 
      $\reach_{F'}(r_1) = \reach_F(r_1) - 1 = \rho(F) - 1 \geq 0$ 
      since $\rho(F) = \rho(xy) \geq 1$ by assumption. 
      It follows that $\mu_x(F') \geq \min(\reach_{F'}(\sigma_1), \reach_{F'}(r_1) ) \geq \reach_{F'}(\sigma_1) = 0$. 




    \item[If $\rho(xy) = 0$ and $\mu_x(y) = 0$] 
      then $Z = R$ and $|Z| \geq 2$. 
      % % there must be two zero-reach $y$-disjoint tines in $F$.
      % % Specifically, 
      % there must be a zero-reach tine $z \in R$ so that 
      % % and 
      % the pair $z_1, z$ are $y$-disjoint. 
      % %  and, 
      % % in particular,
      % % $\ell(z_1 \Intersect z) = \min_{t_1, t_2 \in R} \ell(t_1 \Intersect t_2)$. 
      If $b = \h$, 
      $\sigma_1$
      is the only tine in $F'$ with the maximum reach, zero. 
      Note that 
      $\reach_{F'}(r_1) = \reach_F(r_1) - 1 = \rho(F) - 1 \geq -1$. 
      Since $\sigma_1$ and $r_1$ are $yb$-disjoint, 
      it follows that 
      $\mu_x(F') \geq \min(\reach_{F'}(\sigma_1), \reach_{F'}(r_1) ) 
      \geq \reach_{F'}(r_1) \geq = -1$.

      On the other hand, if $b = \H$ then 
      $F'$ contains two new conservative extensions, 
      $\sigma_1$ and $\sigma_2$, 
      both with label $|xy| + 1$, 
      where $z_1 \Prefix \sigma_1$ and $r_1 \Prefix \sigma_2$.  
      These extensions, therefore, are $yb$-disjoint 
      and have zero reach.
      It follows that $\mu_x(F') \geq 0$.
  \end{description}
\end{proof}


Note that 
if we want \eqref{eq:mu-lowerbound} 
to hold only for \emph{a given prefix} $x \PrefixEq w$ 
(a scenario pertinent in~\cite{LinearConsistencySODA}), 
the adversary $\Adversary^*$ 
% in Proposition~\ref{prop:mu-lowerbound} 
(which produces a canonical fork) 
would be an overkill. 
Instead, 
we can use a simpler, prefix-aware adversary 
such as the one mentioned 
at the outset of Section~\ref{sec:opt-adversary}; 
let us call this strategy $\Adversary$. 
In addition, 
instead of assuming Theorem~\ref{thm:opt-adversary-canonical}, 
it suffices to assume 
Proposition~\ref{prop:mu-lowerbound} inductively 
for all strings of length $|w|$. 
Let $F$ be the fork 
built by $\Adversary$ for the string $w = xy$.
In conjunction with Proposition~\ref{prop:mu-upperbound}, 
this would imply ``$\rho(F) = \rho(w)$ and 
$\mu_x(F) = \mu_x(y),$'' 
a critical property used inside the above proof. 
We omit further details.



