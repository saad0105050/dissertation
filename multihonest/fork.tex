\section{Viable blockchains and adversarial extensions}
  % Let us lay down the elements from the fork framework of~\cite{LinearConsistencySODA}. 
  % For completeness, we restate and briefly discuss the pertinent definitions below.
  A vertex of a fork is said to be \emph{honest} 
  if it is labeled with an index $i$ such that $w_i \in \{\h, \H\}$; 
  otherwise, it is said to be \emph{adversarial}.

  \begin{definition}[Tines, length, and height]
    Let $F \vdash w$ be a fork for a characteristic string.  A
    \emph{tine} of $F$ is a directed path starting from the root. For
    any tine $t$ we define its \emph{length} to be the number of edges
    in the path, and for any vertex $v$ we define its \emph{depth} to be
    the length of the unique tine that ends at $v$. 
    If a tine $t_1$ is a strict prefix of another tine $t_2$, we write $t_1 \Prefix t_2$. 
    Similarly, if $t_1$ is a non-strict prefix of $t_2$, we write $t_1 \PrefixEq t_2$.
    The longest common prefix of two tines $t_1, t_2$ is denoted by $t_1 \Intersect t_2$. 
    That is, $\ell(t_1 \Intersect t_2) = \max\{\ell(u) \SuchThat \text{$u \PrefixEq t_1$ and $u \PrefixEq t_2$} \}$. 
    The \emph{height} of
    a fork (as is usual for a tree) 
    is the length of the longest tine,
    denoted by $\height(F)$. 
  \end{definition}
  Let $F \Fork xy$ and 
  two tines $t_1, t_2 \in F$ are disjoint over $y$. 
  We say that these tines are \emph{$y$-disjoint}; 
  equivalently, we also say that \emph{$t_1$ is $y$-disjoint with $t_2$}.

  When an adversary builds a fork, it is natural to imagine that 
  he ``grows'' an existing fork by adding new vertices and edges. 
  % We can study this notion via fork prefixes and extensions.
  \begin{definition}[Fork prefixes]
    Let $w, x \in \{\h, \H, \A\}^*$ so that $x\PrefixEq w$. 
    Let $F, F'$ be two forks for $x$ and $w$, respectively. 
    We say that $F$ is a \emph{prefix} of $F'$ if 
    $F$ is a consistently labeled subgraph of $F'$. 
    That is, all vertices and edges of $F$ also appear in $F'$ and 
    the label of any vertex appearing in both $F$ and $F'$ is identical. 
    We denote this relationship by $F\fprefix F'$.
  \end{definition}

  \noindent
  When speaking about a tine that appears in both $F$ and $F'$, 
  we place the fork in the subscript of relevant properties.
  % , e.g., writing $\reach_F$, etc.

  % \subsection{Viable tines, reach, and $\Aheavy$ intervals}

    For any string $x$ (on any alphabet) and a symbol $\sigma$ in that alphabet, 
    define $\#_\sigma(x)$ 
    as the number of appearances of $\sigma$ in $x$. 
    % For example, for a characteristic string $w \in \{\h, \H, \A\}^*$, 
    % $\#_\h(w)$ is the number of $\h$ appearing in $w$. 
    When a characteristic string $w \in \{\h, \H, \A\}^T$ is fixed from the context, 
    we extend this notation to sub-intervals of $[T]$ in a natural way: 
    For integers $i, j \in [T], i \leq j$, 
    let $I = [i, j] \subset [T]$ be a closed interval 
    and define $\#_\sigma(I) = \#_\sigma(w_i \ldots w_j)$ for $\sigma \in \{\h, \H, \A\}$. 
    A characteristic string $w$ is called $\Hheavy$\ if $\#_\h(w) + \#_\H(w) > \#_1(x)$; 
    otherwise, it is called $\Aheavy$. 
    For a given characteristic string $w$ of length $T$, 
    an interval $I = [i,j] \subseteq [T]$ is called $\Aheavy$\ 
    if the substring $w_i \ldots w_j$ is $\Aheavy$.


    \paragraph{Adversarial extensions.} 
    Let $x,y$ be two characteristic strings and $|y| \geq 0$.
    Let $F$ be a fork for $x$ and let $B$ be an honest tine in $F$. 
    We say that \emph{$B$ has an adversarial extension} 
    if there is a fork $F' \Fork xy, F \ForkPrefix F'$ and 
    an adversarial tine $t \in F'$ 
    so that $B \Prefix t$ and  
    $B$ is the last honest vertex on $t$. 
    Note that $t$ can be made disjoint with any $F$-tine 
    over the interval $[\ell(B) + 1, \ell(t)]$. 
    






    \paragraph{Viable adversarial extensions and $\A$-heaviness.} 
    Let $w \in \{\h, \H, \A\}^T$,   
    $s \in [T + 1]$, and 
    $F \Fork w_1 \ldots w_{s - 1}$ an arbitrary fork. 
    Let $B \in F$ be an honest vertex 
    and $t$ a maximum-length \emph{honest tine} in $F$.
    Consider the following statements: 
    \begin{enumerate}[label=(\alph*)]
      \item \label{fact-part:viable-adv-ext} $B$ has an adversarial extension viable at the onset of slot $s$.
      \item \label{fact-part:Aheavy} The interval $I = [\ell(B) + 1, s - 1]$ is $\Aheavy$. 
      \item \label{fact-part:conservative} $\length(t) = \#_\h(I) + \#_\H(I) + \length(B)$.     
    \end{enumerate}



    \begin{proposition}[]\label{prop:fork-structure}
    ~\ref{fact-part:viable-adv-ext} $\Longrightarrow$
    \ref{fact-part:Aheavy}.
    In addition, if we assume~\ref{fact-part:conservative}, then 
    ~\ref{fact-part:Aheavy} $\Longrightarrow$ 
    ~\ref{fact-part:viable-adv-ext}.
    \end{proposition}

    \begin{proof}~
      \begin{description}[font=\normalfont\itshape\space]
        \item[\ref{fact-part:viable-adv-ext} implies~\ref{fact-part:Aheavy}.]
        Let $F' \Fork w_1 \ldots w_{s-1}$ be 
        a fork so that $F \ForkPrefix F'$ 
        and $B$ has an adversarial extension $t' \in F'$ 
        viable at the onset of slot $s$. 
        Considering the interval $I$, 
        the longest honest tine in $F'$ 
        grows by at least $\#_\h(I) + \#_\H(I)$ vertices. 
        Since the viable tine $t'$ 
        contains only adversarial vertices from the interval $I$, 
        it follows that $\#_\A(I)$ must be at least $\#_\h(I) + \#_\H(I)$. 
        Hence, $I$ is $\Aheavy$.

        \item[\ref{fact-part:conservative} and \ref{fact-part:Aheavy} implies~\ref{fact-part:viable-adv-ext}.]
        Since $I$ is $\Aheavy$,  
        $I$ contains at least $\#_\h(I) + \#_\H(I) = \length(t) - \length(B)$ 
        adversarial slots. 
        Thus, we can augment $B$ by adding 
        $\length(t) - \length(B)$ adversarial vertices 
        from these slots 
        so that 
        the resulting adversarial extension is viable 
        at the onset of slot $s$.
      \end{description}
    \end{proof}






    \begin{corollary}\label{coro:interval-honest-vertices}
      Let $w$ be a characteristic string, 
      $F$ be any fork for $w$, 
      and let $t$ be any tine in $F$.
      Let $B_1$ and $B_2$ be two honest vertices on $t$ such that 
      (i) $\ell(B_1) < \ell(B_2)$, 
      (ii) $t$ contains only adversarial vertices from $I = [\ell(B_1) + 1, \ell(B_2) - 1]$, and 
      (iii) $t$ contains at least one vertex from $I$.
      Then $I$ is $\Aheavy$. 
      % In addition, let $\phi \Fork w_1\ldots w_{\ell(B_2)}$ 
      % be a prefix of $F$ so that 
      % $\phi$ contains all honest vertices of $F$ with labels at most $\ell(B_2)$. 
      % Then $\reach_{\phi}(B_1) \geq 0$.
    \end{corollary}
    \begin{proof}
      By assumption, 
      the honest vertex $B_2$ builds on some adversarial tine $t^\prime$ 
      that is viable at the onset of slot $\ell(B_2)$ and, importantly, 
      contains $B_1$ as its last honest vertex. 
      By Proposition~\ref{prop:fork-structure}, 
      the interval $I$ is $\Aheavy$.
      % and, in addition, $\reach_{\phi}(B_1) \geq 0$.
    \end{proof}
    









  % \begin{corollary}[Multiple honest leaders in a slot]\label{obs:multi-honest}
  %   Let $w \in \{0,1\}^T$ be a charactersitic string 
  %   and 
  %   let $s$ be a Catalan slot. 
  %   All honest vertices from slot $s$ builds on the same honest tine.
  % \end{corollary}
  % \begin{proof}
  %   Let $F$ be any fork for $w$.
  %   Since $s$ is a Catalan slot, 
  %   $F$ admints no LCR tie at the onset of slot $s + 1$. 
  %   Thus the set of all viable longest tines 
  %   at the onset of slot $s$
  %   Hence all honest vertices from slot $s$ builds upon 
  %   the unique honest tine $\tau = \LCR(C)$ 
  % \end{proof}
