Proof-of-Stake (PoS) blockchain protocols have emerged as a viable
alternative to resource-intensive Proof-of-Work (PoW) blockchain
protocols such as Bitcoin and Ethereum. These PoS protocols are
organized in rounds (which we call \emph{slots} in this paper); their
most critical algorithmic component is a leader election procedure
which determines---for each slot---a subset of participants with the
authority to add a block to the blockchain. Existing security analyses
of these protocols are logically divided into two components: the
first reasons about the properties of the leader election process, the
second reasons about the combinatorial properties of the blockchains
that can be produced by an \emph{idealized} leader schedule in the
face of adaptive adversarial control of some participants. 


An
attractive side effect of 
the separation between the leader election process 
and the blockchain dynamic 
is that the combinatorial
considerations can be treated independently of other aspects of the
protocol. 
% A recent article of Blum et al.~\cite{LinearConsistency}
% gave an axiomatic treatment of this combinatorial portion of the
% analysis which we extend in this paper.
These common combinatorial arguments can be formulated with very
little information about the leader election process. 

% Below, we briefly outline the two 

% As a result, 
% the security guarantees of PoS blockchains 
% are stated in the context of the outcome of the distribution process. 
% An example of such a statement is   
% e.g., ``PoS protocol X will satisfy property Y with probability Z 
% as long as on average, 
% a slot is more likely to be associated with a single honest leader 
% than with one or more adversarial leaders.''



An
attractive side effect of 
the separation between the leader election process 
and the blockchain dynamic 
is that the combinatorial
considerations can be treated independently of other aspects of the
protocol. 
% A recent article of Blum et al.~\cite{LinearConsistency}
% gave an axiomatic treatment of this combinatorial portion of the
% analysis which we extend in this paper.
These common combinatorial arguments can be formulated with very
little information about the leader election process. 


\Paragraph{Existing PoS security guarantees and assumptions.}
The analysis of the Nakamoto-style PoS protocol Ouroboros~\cite{Ouroboros} 
is in the synchronous model, against a static adversary. 
The analyses in Sleepy Consensus~\cite{Sleepy}, SnowWhite~\cite{SnowWhite}, 
Praos~\cite{Praos}, and Genesis~\cite{Genesis} are 
in the semi-synchronous model, against an adaptive adversary. 
There are two salient features of these analyses; 
to appreciate these, recall that 
a slot can be either uniquely honest, multihonest, or adversarial, 
based on the outcome of the leader election in that slot.

\begin{description}[font=\normalfont\itshape\space]
  \item[Assumptions on leader election outcomes:] The analyses in Ouroboros and Praos assume that on average, 
  a slot is more likely to be uniquely honest than the other two types. 
  That is, $p_\h > p_\A + p_\H$. The analyses in Sleepy Consensus and Snow White assume that on average, 
  a slot is more likely to be uniquely honest than adversarial. 
  That is, $p_\h > p_\A$.

  \item[Consistency guarantee:] Let $\EpsP$ be the probability of a $\kSlotCP$ violation.
  Under the assumptions above, 
  Ouroboros, Sleepy, and Snow White guarantee that $\EpsP \leq e^{-\Omega(\sqrt{k})}$.
\end{description}

It can be easily seen that 
the PoS consistency error bounds 
in these analyses 
are an order-of-magnitude worse than the known PoW bounds of $e^{-\Omega(k)}$. 
As we discussed before, 
this would imply that for a fixed consistency error tolerance, 
a transaction would take far longer ($k^2$ rounds) to settle in a PoS blockchain 
than on a PoW blockchain with similar parameters (just $k$ rounds).

A deeper concern lies with the treatment of multihonest slots.
Multihonest slots do not arise in Ouroboros since it publicly elects a single leader for each slot. 
However, they arise naturally when each player
checks privately whether he is a leader (e.g., Praos, SnowWhtie, and Sleepy). 
They may also occur
naturally in the non-synchronous setting when the time between the
broadcast of two blocks is exceeded by network delay---in this case
the party issuing the later block may not be aware of the earlier
block which can result the two blocks sharing the same chain history,
a de facto incidence of multiple honest leaders. The role of these
slots is rather delicate: while it is good for the system to have many
honest blocks, \emph{concurrent} blocks can help the adversary in
creating two long, diverging blockchains that might jeopardize the
consistency property. 

Note that while Sleepy and SnowWhite 
simply discards the multihonest probability $p_\H$ from their assumptions
(i.e., treating a multihonest slot as a neutral event), 
the analysis in Praos treats multihonest slots as if they are bad events
(e.g., Praos need $p_\h - p_\H > p_\A$). 
Finally, notice that all existing
analyses break down if $p_\h < p_\A$, i.e., when the uniquely
honest slots are less probable than the adversarial slots.

Finally, the existing analyses only give asymptotic results; 
there is no way to explicitly compute the error probabilities.
These concerns leave a lot to be desired from 
the analyses of Nakamoto-style PoS blockchains.



\Paragraph{Our goals.}
Our goal is to establish a rigorous modeling and analysis 
framework to overcome the above concerns. 
In particular, 
we want to achieve a consistency error $\exp(-\Theta(k))$, 
matching that of PoW.\footnote{
  The constants hidden under the asymptotic notation may differ between PoW and PoS, 
  but this is acceptable since 
  PoS, as a model, gives the adversary more leeway than does PoW.
} 
In addition, 
we want to show that Nakamoto-style PoS, like PoW, 
can achieve consistency
under the most desirable assumption\footnote{ 
  Consistency is unachievable in the case $p_\h + p_\H < p_\A$.
  See \cite{GK18} for a detailed discussion of the honest majority
  assumption. 
} $p_\h + p_\H > p_\A$, even allowing $p_\h < p_\A$. 

We also want to be able to go beyond asymptotic bounds 
and actually compute the error probabilities.




% % {\color{red} XXXX I got to here.}

% Our new analysis shows that this second effect
% can be mitigated, achieving consistency error bound of
% $e^{-\Theta(k)}$ under the (tight) assumption $p_\h + p_\H > p_\A$.



\section{Our results and contributions} 
As described above, we show for the first time that Nakamoto-style PoS blockchain
protocols can achieve a $k$-consistency error
of $e^{-\Theta(k)}$ under the desirable condition
$p_\h + p_\H > p_\A$. 
We analyzed both the synchronous setting and the semi-synchronous setting.
This improves the security guarantee of all existing 
Nakamoto-style PoS protocols such as Ouroboros~\cite{Ouroboros}, Praos~\cite{Praos},
Genesis~\cite{Genesis}, and Snow White~\cite{SnowWhite}.
(We remark that other PoS protocols such as Algorand~\cite{Algorand} 
operate in a different setting where explicit participation bounds are assumed
and forks can be prevented). 

To put it in the perspective, 
our analysis reduces the state-of-the-art estimate 
of the $k$-settlement time for Nakamoto-style PoS 
by an order of magnitude (from $k^2$ to $k$), 
under a much weaker assumption.
Our results show that Nakamoto-style PoS protocols can achieve
consistency even when $p_\h < p_\A$; 
this parameter regime was not treated by any existing analyses. 

In the synchronous setting, 
we give an algorithm (with C++ implementation) to explicitly compute the probability 
that a given slot encounters a consistency violation 
under the idealized leader election mechanism. 
The time and space required by this algorithm is cubic 
in the length of the protocol execution.

We also consider a variant model where the honest players use a
consistent tie-breaking rule when selecting the longest chain.  (I.e.,
when a fixed set of blockchains of equal length are presented to a
collection of honest players, they all select the same chain.
In previous models, the adversary had the right to break such ties by influencing
network delivery.)
Assuming $p_\h + p_\H > p_\A$, we achieve 
$e^{-\Theta(k)}$ consistency error in this case as well. 
Moreover, we can take $p_\h =0$; 
no existing analysis can handle this parameter regime. 
This can be especially helpful for leader election schemes
where a uniquely honest slot almost never occurs. 
Such leader elections are deployed in Filecoin~\cite{Filecoin}, for example.


  % \item 
Our results mentioned above can be transferred to
the $\Delta$-synchronous setting using the standard 
\emph{$\Delta$-synchronous to synchronous reduction
approach} used in the Ouroboros Praos analysis~\cite{Praos}. Thus, we
can achieve a consistency error probability of $e^{-\Theta(k)}$ in this
setting as well. 




  % \item 
\section{A technical overview}
We initially work in the synchronous communication model and extend
the synchronous combinatorial framework
of~\cite{LinearConsistency} to accommodate multihonest
slots. 

First, our analysis focuses on a combinatorial event called a ``Catalan
slot.''\footnote{The name is a nod to the \emph{Catalan number} in
  combinatorics: The $n$th Catalan number $C_n$ is the number of
  strings $w \in \{0, 1\}^{2n}$ so that every prefix $x$ of $w$
  satisfies $\#_0(x) \geq \#_1(x)$.} Catalan slots are honest slots
$c$ with the property that any interval containing $c$ possesses
strictly more honest slots---with any number of honest leaders---than
adversarial ones. The analysis of~\cite{SnowWhite} and ~\cite{Sleepy}
introduced this basic concept, though they counted only uniquely
honest slots. In comparison with their analysis, then, our treatment
has two important advantages: first of all, we let multihonest
slots count in the analysis and, additionally, we achieve strikingly
stronger error bounds: specifically, we achieve optimal settlement
error of $\exp(-\theta(k))$ rather than $\exp(-\theta(\sqrt{k}))$.

A Catalan slot $c$ acts as a barrier for the adversary in that if an
honest blockchain from a slot $h < c$ is padded with adversarial
blocks and presented to an honest observer at slot $c + 1$, the
observer will never adopt this blockchain.  As a result, the chains
adopted by this honest observer must contain \emph{some} block from
slot $c$.  Note that this is true \emph{even if $c$ is
  multihonest}.  A critical observation is that \emph{a slot is
  Catalan if and only if all competitive blockchains in future slots
  contain at least one block from this slot}.  Thus, if a Catalan slot
$c$ is uniquely honest, all blockchains that are eligible to be
adopted by future honest players must contain the (only) honest block
issued from slot $c$.  We call this the ``Unique Vertex Property''
(UVP).  Note how the UVP is reminiscent of the ``Common Prefix
Property'' (CP) in the literature. Thus, together, the UVP and 
Catalan slots act as a conduit between consistency
violations and the underlying stochastic process. 

Our major technical challenge is to bound the probability that Catalan
slots are infrequent. Here we break away entirely from the analysis
of~\cite{SnowWhite} and approach the question using the theory of
generating functions and stochastic dominance. We find an exact
generating function for a related event and use this, by dominance, to
control the undesirable event that a long window of slots is devoid of
Catalan slots. This yields
% permits us to achieve
asymptotically optimal settlement bounds.

Finally, it follows from the discussion above that if two consecutive
slots are Catalan then any subsequent honest block must contain, in
its prefix, a block from each of these slots.  In a setting where all
honest players use a consistent longest-chain selection rule,
% Theorem~\ref{thm:multiple-honest} further states
we show that both slots have UVP as well.  Since Catalan slots can be
multihonest, PoS protocols can achieve a consistency error bound
of $e^{-\Theta(k)}$ in this model even if $p_\h = 0$.

In a separate line of reasoning, in \Section~\ref{sec:fork-framework}, 
we generalize the fork-theoretic framework of~\citet{LinearConsistency} for the multihonest setting. 
Here, we characterize the UVP 
in terms of the so-called ``relative margin,'' 
a combinatorial property of a given slot. 
We describe an adversary who optimally attacks the UVP 
of all slots, simultaneously. 
Next, we prove a recurrence relation for relative margin. 
Suppose each slot is 
independently and identically chosen 
(by the leader election mechanism) 
to be either uniquely honest, multihonest, or adversarial. 
The recurrence relation mentioned above then 
leads to an algorithm to explicitly compute 
the probability that 
a given slot encounters a consistency violation; 
see \Section~\ref{sec:exact-prob-multihonest}. 
In contrast, the Catalan slot-centric characterization of the UVP 
gives us only an asymptotic bound on this probability. 
It can be concluded that the fork-framework, after all, 
is expressive enough to capture consistency violations 
in the multi-leader setting.
%No existing
%analysis can handle this parameter regime.


\paragraph{Outline.}
{\color{red} Need update}

We specify our model in \Section~\ref{sec:model-multihonest} and focus on a
specific consistency property called ``$k$-settlement.''  This section
also contains our main theorems; the proofs are deferred to
\Section~\ref{sec:bounds-main-proofs-multihonest}.  In
Section~\ref{sec:definitions}, we describe amplifications
% further necessary elements of
to the fork framework of~\cite{LinearConsistency} in order to
%so that we can
explore the relationship between Catalan slots and the UVP. 
In \Section~\ref{sec:bounds-main-proofs-multihonest},
we present two bounds on the stochastic events of interest, e.g., the
rarity of a Catalan slot; these bounds lead to short proofs of the
main theorems.  The proofs of these bounds are presented next in
Section~\ref{sec:estimates-multihonest} which contains all of our probabilistic
arguments.  

\Section~\ref{sec:recursion-multihonest} contains an alternative treatment 
of the UVP via fork-theoretic notions of~\cite{LinearConsistency}. 
Along the way, it describes an optimal adversary who simultaneously attacks the consistency of all slots. 
It also describes an algorithm to compute explicit values 
for the probability of consistency violations. 
The proofs of two important theorems from this section 
are presented subsequently in \Section~\ref{sec:margin-proof-multihonest}.


Our treatment of the $\Delta$-synchronous setting is
presented in \Section~\ref{sec:async-multihonest}.  In \Section~\ref{sec:cp-multihonest}, we
treat the traditional Common Prefix (CP) violations using our bounds
on the UVP.  

In \Section~\ref{sec:cp-multihonest}, 
we characterize the Common Prefix Property (CP) violations. 
We give a simple proof using the UVP as well as 
an alternative (lengthy) proof which uses fork-theoretic constructs (such as ``balanced forks'') 
but not the UVP or Catalan slots. 
This is an example where Catalan slots (and the UVP) simplifies existing proofs.



%%% Local Variables:
%%% mode: latex
%%% TeX-master: "main"
%%% End:
