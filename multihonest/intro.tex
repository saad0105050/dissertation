Proof-of-Stake (PoS) blockchain protocols have emerged as a viable
alternative to resource-intensive Proof-of-Work (PoW) blockchain
protocols such as Bitcoin and Ethereum. These PoS protocols are
organized in rounds (which we call \emph{slots} in this paper); their
most critical algorithmic component is a leader election procedure
which determines---for each slot---a subset of participants with the
authority to add a block to the blockchain. Existing security analyses
of these protocols are logically divided into two components: the
first reasons about the properties of the leader election process, the
second reasons about the combinatorial properties of the blockchains
that can be produced by an \emph{idealized} leader schedule in the
face of adaptive adversarial control of some participants. An
attractive side effect of this structure is that the combinatorial
considerations can be treated independently of other aspects of the
protocol. A recent article of Blum et al.~\cite{LinearConsistency}
gave an axiomatic treatment of this combinatorial portion of the
analysis which we extend in this paper.

These common combinatorial arguments can be formulated with very
little information about the leader election process.  Specifically,
current analyses focus on three parameters:
\begin{itemize}
\item $p_\h$, the probability that a slot is \emph{uniquely honest}, having a single honest leader;
\item $p_\H$, the probability that a slot is \emph{multiply honest}, having multiple, but honest,
  leaders; and
\item $p_\A$, the probability that a slot has at least one adversarial leader.
\end{itemize}
Our major contribution is a generic, rigorous guarantee of consistency
under the most desirable assumption\footnote{ 
  Consistency is unachievable in the case $p_\h + p_\H < p_\A$.
  %leaves little room for consistency
%since in this case  
%the adversary can create a private protocol execution identically
%distributed to that of the honest parties and by selectively disclosing it, it
%can break consistency.
  See \cite{GK18} for a detailed discussion of the honest majority
  assumption. } $p_\h + p_\H > p_\A$ that achieves optimal consistency
error $\exp(-\Theta(k))$ as a function of confirmation time $k$. Our
analysis can be directly applied to existing protocols to improve
their consistency guarantees.

To contrast this with existing literature, the analysis of Ouroboros
Praos~\cite{Praos} and Ouroboros Genesis~\cite{Genesis}
%\footnote{The chain selection
%  rule in Genesis is a combination of the usual longest-chain rule and
%  a chain-density based rule.}
require the threshold assumption $p_\h - p_\H > p_\A$ to achieve the
optimal consistency error of $e^{-\Theta(k)}$. Note how multiply
honest slots actually \emph{detract} from security, appearing
negatively in the basic security threshold. The consistency analyses
in Snow White~\cite{SnowWhite} and Sleepy Consensus~\cite{Sleepy}
assume an improved threshold $p_\h > p_\A$; however, they only
establish a consistency error bound of $e^{-\Theta(\sqrt{k})}$. Note
here that multiply honest slots appear neutrally. All existing
analyses break down if $p_\h < p_\A$, i.e., when the uniquely
honest slots are less probable than the adversarial slots.

% {\color{red} XXXX I got to here.}

Multiply honest slots may arise by design, e.g., when each player
checks privately whether he is a leader.  They may also occur
naturally in the non-synchronous setting when the time between the
broadcast of two blocks is exceeded by network delay---in this case
the party issuing the later block may not be aware of the earlier
block which can result the two blocks sharing the same chain history,
a de facto incidence of multiple honest leaders. The role of these
slots is rather delicate: while it is good for the system to have many
honest blocks, \emph{concurrent} blocks can help the adversary in
creating two long, diverging blockchains that might jeopardize the
consistency property. Our new analysis shows that this second effect
can be mitigated, achieving consistency error bound of
$e^{-\Theta(k)}$ under the (tight) assumption $p_\h + p_\H > p_\A$.



\paragraph{Our results and contributions.} 
As described above, we show for the first time that PoS blockchain
protocols using the longest-chain rule can achieve a consistency error
of $e^{-\Theta(k)}$ under the desirable condition
$p_\h + p_\H > p_\A$.  This improves the security guarantee of all
``longest chain rule''  PoS protocols such as Praos~\cite{Praos},
Genesis~\cite{Genesis}, and Snow White~\cite{SnowWhite}
(we remark that other PoS protocols such as Algorand~\cite{DBLP:journals/corr/Micali16} 
operate in a different setting where explicit participation bounds are assumed
and forks can be prevented).
We discuss
our results in more detail before turning to the model and proofs.

% \begin{itemize}
% \item
Our analysis in the simple synchronous model achieves the same
asymptotic error bound as in~\cite{LinearConsistencySODA}---the
tightest result in the literature---under a much weaker assumption,
namely $p_\h + p_\H > p_\A$.  Thus PoS protocols can in fact achieve
consistency with $p_\h < p_\A$, a regime beyond reach of all previous analyses. 
When uniquely honest slots are rare 
(i.e., when $p_\h$ is very small), 
our bound has the desired dependence on $p_\h$. 
Moreover, when $p_\H = 0$ (i.e., all honest slots are in fact
uniquely honest), we exactly recover the bound
in~\cite{LinearConsistencySODA}. 
We also give an algorithm to explicitly compute the probability 
that a given slot encounters a consistency violation 
under the idealized leader election mechanism. 
The time and space required by this algorithm is cubic 
in the length of the protocol execution.

Next, we consider a variant model where the honest players use a
consistent tie-breaking rule when selecting the longest chain.  (I.e.,
when a fixed set of blockchains of equal length are presented to a
collection of honest players, they all select the same chain.
In previous models, the adversary had the right to break such ties by influencing
network delivery.)
Assuming $p_\h + p_\H > p_\A$, we prove that the consistency error
bound in this model is identical to the $e^{-\Theta(k)}$ bound
in~\cite{LinearConsistencySODA} \emph{even when $p_\h =
  0$}. No existing analysis survives in this regime.

% Thus not only is our analysis a first in this parameter regime, but
% our bound is also the tightest in any regime.

  % \item 
\paragraph{$\Delta$-synchronous setting.}
In the $\Delta$-synchronous
communication setting, all messages are delivered with at most
a $\Delta$ delay. Our results mentioned above can be transferred to
this setting using the \emph{$\Delta$-synchronous to synchronous reduction
approach} used in the Ouroboros Praos analysis~\cite{Praos}. Thus, we
can achieve a consistency error probability of $e^{-\Theta(k)}$ in this
setting as well. 
This analysis is presented in 
Section~\ref{sec:async-multihonest}.

  % \item 
\paragraph{A technical overview.}
We initially work in the synchronous communication model and extend
the synchronous combinatorial framework
of~\cite{LinearConsistency} to accommodate multiply honest
slots. 
% Many of the important constructs and proofs from their
% development break down
% and some critical combinatorial notions 
% % in their framework 
% % of ``relative margin'' and ``balanced forks'' 
% do not retain their direct significance in the multi-leader setting. 
% It appears that we need new tools with the right expressive properties.

First, our analysis focuses on a combinatorial event called a ``Catalan
slot.''\footnote{The name is a nod to the \emph{Catalan number} in
  combinatorics: The $n$th Catalan number $C_n$ is the number of
  strings $w \in \{0, 1\}^{2n}$ so that every prefix $x$ of $w$
  satisfies $\#_0(x) \geq \#_1(x)$.} Catalan slots are honest slots
$c$ with the property that any interval containing $c$ possesses
strictly more honest slots---with any number of honest leaders---than
adversarial ones. The analysis of~\cite{SnowWhite} and ~\cite{Sleepy}
introduced this basic concept, though they counted only uniquely
honest slots. In comparison with their analysis, then, our treatment
has two important advantages: first of all, we let multiply honest
slots count in the analysis and, additionally, we achieve strikingly
stronger error bounds: specifically, we achieve optimal settlement
error of $\exp(-\theta(k))$ rather than $\exp(-\theta(\sqrt{k}))$.

A Catalan slot $c$ acts as a barrier for the adversary in that if an
honest blockchain from a slot $h < c$ is padded with adversarial
blocks and presented to an honest observer at slot $c + 1$, the
observer will never adopt this blockchain.  As a result, the chains
adopted by this honest observer must contain \emph{some} block from
slot $c$.  Note that this is true \emph{even if $c$ is
  multiply honest}.  A critical observation is that \emph{a slot is
  Catalan if and only if all competitive blockchains in future slots
  contain at least one block from this slot}.  Thus, if a Catalan slot
$c$ is uniquely honest, all blockchains that are eligible to be
adopted by future honest players must contain the (only) honest block
issued from slot $c$.  We call this the ``Unique Vertex Property''
(UVP).  Note how the UVP is reminiscent of the ``Common Prefix
Property'' (CP) in the literature. Thus, together, the UVP and 
Catalan slots act as a conduit between consistency
violations and the underlying stochastic process. 

Our major technical challenge is to bound the probability that Catalan
slots are infrequent. Here we break away entirely from the analysis
of~\cite{SnowWhite} and approach the question using the theory of
generating functions and stochastic dominance. We find an exact
generating function for a related event and use this, by dominance, to
control the undesirable event that a long window of slots is devoid of
Catalan slots. This yields
% permits us to achieve
asymptotically optimal settlement bounds.

Finally, it follows from the discussion above that if two consecutive
slots are Catalan then any subsequent honest block must contain, in
its prefix, a block from each of these slots.  In a setting where all
honest players use a consistent longest-chain selection rule,
% Theorem~\ref{thm:multiple-honest} further states
we show that both slots have UVP as well.  Since Catalan slots can be
multiply honest, PoS protocols can achieve a consistency error bound
of $e^{-\Theta(k)}$ in this model even if $p_\h = 0$.

In a separate line of reasoning, in Section~\ref{sec:recursion}, 
we generalize the fork-theoretic framework of~\citet{LinearConsistency} for the multi-leader setting. 
Here, we characterize the UVP 
in terms of the so-called ``relative margin,'' 
a combinatorial property of a given slot. 
We describe an adversary who optimally attacks the UVP 
of all slots, simultaneously. 
Next, we prove a recurrence relation for relative margin. 
Suppose each slot is 
independently and identically chosen 
(by the leader election mechanism) 
to be either uniquely honest, multiply honest, or adversarial. 
The recurrence relation mentioned above then 
leads to an algorithm to explicitly compute 
the probability that 
a given slot encounters a consistency violation; 
see Section~\ref{sec:exact-prob}. 
In contrast, the Catalan slot-centric characterization of the UVP 
gives us only an asymptotic bound on this probability. 
It can be concluded that the fork-framework, after all, 
is expressive enough to capture consistency violations 
in the multi-leader setting.
%No existing
%analysis can handle this parameter regime.


\paragraph{Outline.}
We specify our model in Section~\ref{sec:model} and focus on a
specific consistency property called ``$k$-settlement.''  This section
also contains our main theorems; the proofs are deferred to
\Section~\ref{sec:bounds-main-proofs-multihonest}.  In
Section~\ref{sec:definitions}, we describe amplifications
% further necessary elements of
to the fork framework of~\cite{LinearConsistency} in order to
%so that we can
explore the relationship between Catalan slots and the UVP. 
In \Section~\ref{sec:bounds-main-proofs-multihonest},
we present two bounds on the stochastic events of interest, e.g., the
rarity of a Catalan slot; these bounds lead to short proofs of the
main theorems.  The proofs of these bounds are presented next in
Section~\ref{sec:estimates} which contains all of our probabilistic
arguments.  

\Section~\ref{sec:recursion-multihonest} contains an alternative treatment 
of the UVP via fork-theoretic notions of~\cite{LinearConsistency}. 
Along the way, it describes an optimal adversary who simultaneously attacks the consistency of all slots. 
It also describes an algorithm to compute explicit values 
for the probability of consistency violations. 
The proofs of two important theorems from this section 
are presented subsequently in \Section~\ref{sec:margin-proof-multihonest}.


Our treatment of the $\Delta$-synchronous setting is
presented in \Section~\ref{sec:async-multihonest}.  In \Section~\ref{sec:cp-multihonest}, we
treat the traditional Common Prefix (CP) violations using our bounds
on the UVP.  

In \Section~\ref{sec:cp-forks}, 
we characterize common prefix violations 
in the presence of multiply honest slots 
using ``balanced forks'' from~\cite{LinearConsistency} (and, importantly, 
without using Catalan slots).



%%% Local Variables:
%%% mode: latex
%%% TeX-master: "main"
%%% End:
