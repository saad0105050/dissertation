\hrule



      \begin{theorem}[No LCR tie iff the last slot is left-Catalan]\label{thm:lcr-tie-left-catalan}
        Let $w \in \{0,1\}^T$ be a charactersitic string. 
        Slot $T$ is left-Catalan 
        if and only if 
        $w$ admits no fork with an LCR tie at the onset of slot $T + 1$.
      \end{theorem}
      \begin{proof}
        ~
        \begin{description}[font=\normalfont\itshape\space]
          \item[Forward direction.]
          Suppose $F \Fork w$ is a fork which contains 
          at least two longest tines at the onset of slot $T + 1$ so that 
          one of them is adversarial. 
          We claim that $T$ is not a left-Catalan slot.
          Let $t$ be this adversarial tine and  
          let $B$ be the last honest block on $t$. 
          Since $t$ is viable at the onset of slot $T + 1$, 
          by Fact~\ref{fact:fork-structure}, 
          the interval $[\ell(B) + 1, T]$ must be $\Aheavy$. 



          \item[Reverse direction.]
          Suppose $T$ is \emph{not} a left-Catalan slot; 
          we claim that there is a fork for $w$ with an LCR tie.
          In particular, $w$ has a decomposition $w = xy$ so that 
          $y$ is the longest $\Aheavy$ suffix of $w$. 
          Let $I$ be the interval indexed by the slots of $y$.
          If $w$ is the all-ones string, we trivially get an LCR tie. 
          Hence assume that $w$ contains at least one honest slot and, 
          in particular, 
          observe that by construction, 
          $I$ must contain the last honest slot in $w$. 
          Next, observe that if $w$ ends in one or more adversarial slots, 
          it is always possible to create an adversarial tine which is 
          at least as long as the longest honest tine, 
          leading to an LCR tie at the onset of slot $T + 1$. 
          Thus assume that $T$ is an honest slot.
          Finally, let $F$ be a fork for $w$ 
          and let $C$ be the set of all longest tines in $F$.

          \begin{description}[font=\normalfont\itshape\space]
            \item[If $|C| \geq 2$ and $C$ contains an adversarial tine:]
            We are done; $F$ produces an LCR tie at the onset of slot $T + 1$.

            \item[If $C$ contains only honest tines:] 
            If $|C| = 1$, let $t$ be the only member in $C$.
            Otherwise, let $t = \LCR(C)$. 
            In either case, $t$ is an honest tine.
            %  and notice that 
            % either $t$ is  or it is an adversarial tine longer than $t$.
            % In either case, 
            Below, we construct a new adversarial longest tine $t^\prime$ 
            so that $\length(t^\prime) = \length(t)$. 
            This would imply that the modified fork admits 
            an LCR tie at the onset of slot $T + 1$.

            Let $B$ be the last honest block 
            on tine $t$ issued from the prefix $x$.
            (If no such block is present, we can take $B$ to be the genesis block.)
            We augment the fork $F$ as follows: 
            i) Extend $B$ by duplicating the (adversarial) blocks 
            that appear on $t$ after $\ell(B)$ but within the slots indexed by $x$; 
            let $B^\prime$ be the last of these newly created blocks. 
            Let $n = \length(t) - \depth(B^\prime)$.
            ii) Extend $B^\prime$ using the first $n$ adversarial slots in $I$;  
            let $t^\prime$ be the 
            newly created (adversarial) tine. 
            Such an extension is always possible since $I$ is $\Aheavy$.
            By construction, $t \neq t^\prime$ and $\length(t^\prime) = \length(t)$. 

            \item[If $C = \{t\}$ where $t$ is an adversarial tine.]
            Let $\tau$ be a tine ending in (the honest) slot $T$. 
            It follows that $\length(t) > \length(\tau)$ 
            since, otherwise, $C$ would contain $\tau$.
            In addition, $\ell(t) < \ell(\tau) = T$ and, as a result, 
            $t$ cannot contain any block from slot $T$. 
            In fact, we can say more: 
            the number of trailing adversarial blocks in 
            every tine $\hat{t} \in F$ so that $\length(\hat{t}) > \length(\tau)$ 
            is at least $n_{\hat{t}} \triangleq \length(\hat{t}) - \length(\tau)$.
            Let us delete $n_{\hat{t}}$ trailing adversarial blocks 
            from every tine $\hat{t}$ above. 
            The resulting fork must contain a longest honest tine $\tau$ 
            plus at least one longest adversarial tine. 
            As a result, this new fork is a witness to an LCR tie at the onset of slot $T + 1$.
          \end{description}

        \end{description}

      \end{proof}

    % \subsection{CP implies adjacent Catalan slots}

    %   \begin{fact}[CP implies adjacent Catalan slots]\label{fact:cp-implies-catalan-multiple-honest}
    %     Let $w \in \{0,1\}^T$ be a characteristic string 
    %     and let $s \in [T]$ so that $s-1, s$ are honest slots in $w$.  
    %     Suppose $w$ has the following property: 
    %     for any fork $F \Fork w$ 
    %     % constructed under the LCR tie-breaking rule $\LCR(\cdot)$ 
    %     and any slot $k, k \geq s + 1$, 
    %     every tine $t$ viable at the onset of slot $k$ 
    %     contains, in its prefix, 
    %     a unique tine 
    %     $\tau, \ell(\tau) = s$ which 
    %     builds upon a unique tine $\rho, \ell(\rho) = s - 1$.
    %     Then the two slots $s-1$ and $s$ are Catalan.
    %   \end{fact}
    %   \begin{proof}
    %     Fact~\ref{fact:almost-cp-implies-catalan} implies that 
    %     $s$ and $s - 1$ must be Catalan slots. 

    %     % Fact~\ref{fact:almost-cp-implies-catalan} implies that 
    %     % $s - 1$ must be Catalan. 
    %     % We claim that 
    %     % there can be no $\Aheavy$ interval $I \subset [s, T]$.
    %     % Suppose, toward a contradiction, 
    %     % that $I$ is the longest of such $\Aheavy$ intervals. 
    %     % Necessarily, $I$ can be written as $I = [s, h - 1]$ 
    %     % where either $h$ is an honest slot or $h = T$.
    %     % Consider a fork $F$ 
    %     % containing two blocks $\rho, \rho^\prime$ from slot $s - 1$ 
    %     % where, by our assumption, all relevant viable tines 
    %     % contain $\rho$ as the common prefix.
    %     % However, since $I$ is $\Aheavy$, 
    %     % $F$ can be augmented with an adversarial tine $t$ 
    %     % that is viable at the onset of slot $h$ and---importantly--- 
    %     % contains $\rho^\prime$ as its last honest block. 
    %     % This would violate the uniqueness of the common prefix $\rho$.
    %     % Therefore, every interval of the form $[s, b], b \geq s$ must be $\Hheavy$. 

    %     % The above argument shows that $s$ is a right-Catalan slot.
    %     % Since $s - 1$ is a Catalan slot, the honest slot $s$ is left-Catalan and, therefore, Catalan.
    %   \end{proof}

    % \subsubsection{CP implies negative relative margin}

  %   \begin{fact}[CP implies negative relative margin]\label{fact:cp-implies-neg-margin-multiple-honest}
  %     Let $w \in \{0,1\}^T$ be a characteristic string 
  %     and let $s \in[2, T]$ be an integer such that 
  %     both $s$ and $s - 1$ are honest slots in $w$.  
  %     Suppose $w$ has the following property: 
  %     For any slot $k, k \geq s + 1$ and 
  %     any fork $F \Fork w$
  %     where honest nodes use $\LCR(\cdot)$ to select the longest chain, 
  %     every tine $t$ viable at the onset of slot $k$ 
  %     contains, in its prefix, a 
  %     unique tine $\rho, \ell(\rho) = s - 1$.
  %     Then 
  %     \begin{enumerate}[label=\roman*.]
  %       \item\label{margin-s-1-multihonest} 
  %       $\mu_{x}(y) < 0$ 
  %       for all decompositions $w = xyz$ 
  %       so that $|x| < s, |xy| \geq s$; and 
        
  %       \item\label{margin-s-multihonest} 
  %       $\mu_{x^\prime}(y^\prime) < 0$ 
  %       for all decompositions $w = x^\prime y^\prime z$ 
  %       so that $|x^\prime| < s - 1, |x^\prime y^\prime| \geq s - 1$. 
  %     \end{enumerate} 
  %   \end{fact}
  %   \begin{proof}
  %     Recall that the statements in Part~\ref{thm:part-cp-multiple-honest} 
  %     and Part~\ref{thm:part-catalan-multiple-honest} 
  %     of Theorem~\ref{thm:multiple-honest} 
  %     are equivalent. 
  %     Thus we can assume that the slots $s, s - 1$ are in fact Catalan slots. 

  %     \begin{description}[font=\normalfont\itshape\space]
  %       \item[Part~\ref{margin-s-1-multihonest}]
  %       Let $m^\prime, n^\prime \in \NN, m^\prime < s - 1 \leq n^\prime \leq T$.
  %       Write $w = x^\prime y^\prime z$ with $|x^\prime| = m^\prime$ and $|x^\prime y^\prime| = n^\prime$. 
  %       (Thus $|y^\prime| \geq 1$ and $y^\prime_1 = w_{s-1}$.)
  %       Let $F$ be any fork for $x^\prime y^\prime$. 
  %       The reach of the longest honest tine $\tilde{t}$ in $F$ is always zero 
  %       and by assumption, it contains $\rho$ as a prefix.
  %       Let $A^\prime$ be the set of tines $t$ such that 
  %       $t$ is $y^\prime$-disjoint with $\tilde{t}$. 
  %       Thus $t$ cannot contain $\rho$ as a prefix and, in fact, 
  %       $t$ cannot be viable at the onset of slot $|x^\prime y^\prime| + 1$.
  %       We wish to show that $\reach(t) < 0$ for all $t \in A^\prime$. 
  %       Note that the reach of an adversarial tine 
  %       is no more than the reach of the last honest block on that tine. 
  %       Let $B$ be an honest block on $t$ so that $\ell(B) \leq s - 2$; 
  %       we wish to show that $\reach(B) < 0$.
  %       Since $s - 1$ is a Catalan slot, the interval $I = [\ell(B) + 1, T]$ 
  %       is $\Hheavy$.
  %       By Part~\ref{reach:$\Hheavy$-neg-reach}, $\reach(B) < 0$.
  %       Since the above argument is true for any fork $F \Fork x^\prime y^\prime$, 
  %       we conclude that $\mu_{x^\prime}(y^\prime) < 0$. 

  %       \item[Part~\ref{margin-s-multihonest}]
  %       Let $m, n \in \NN, m < s \leq n \leq T$.
  %       Write $w = x y z$ with $|x| = m$ and $|x y| = n$. 
  %       (Thus $|y| \geq 1$ and $y_1 = w_{s}$.)
  %       Let $F$ be any fork for $x y$. 
  %       The reach of the longest honest tine $\tilde{t}$ in $F$ is always zero 
  %       and by assumption, it contains $\rho$ as a prefix. 
  %       Let $\tau$ be the block on $\tilde{t}$ so that $\ell(\tau) = s$.

  %       Let $h$ be an honest slot, $h \leq s - 1$ 
  %       and let $B_h$ be an honest block from slot $h$. 
  %       Since $s$ is a Catalan slot,         
  %       every interval $[h + 1, T]$ is $\Hheavy$. 
  %       By Part~\ref{reach:$\Hheavy$-neg-reach} of Fact~\ref{fact:ext-reach}, 
  %       $\reach(B_h) < 0$. 
  %       Note that the reach of an adversarial tine is no more than the reach of its last honest block.
  %       Consider a tine $t$ whose last honest vertex comes from the interval $[1, s - 1]$. 
  %       (Necessarily, $t$ is $y$-disjoint with $\tilde{t}$.)
  %       It follows that $\reach(t) < 0$. 

  %       Let $A$ be the set of tines $t$ such that 
  %       $t$ is $y$-disjoint with $\tilde{t}$ 
  %       and $t$ contains an honest block from the interval $[s, T]$. 
  %       (Note that $t$ does not contain $\tau$ as a prefix.)
  %       It suffices to consider only honest tines $t$ since 
  %       the reach of an adversarial tine is no more 
  %       than the reach of its last honest block.
  %       Suppose $t$ is viable at the onset of slot $|xy| + 1$.
  %       By assumption, $t$ contains $\rho$ as a prefix but, by construction, 
  %       $t$ does not contain $\tau$.
  %       As $t, \tilde{t}$ are $y$-disjoint and $t$ is viable, 
  %       the interval $s, T$ must be
  %       Let $B$ be the first honest block on $t$ in the interval $[s, T]$. 


  %       By Fact~\ref{fact:catalan-unique-longest}, 
  %       there are no viable adversarial tine at the onset of slot $s + 1$. 
  %       Let $h$ be the first honest slot after slot $s$. 
  %       Since the interval $[s-1, h - 1]$ is $\Aheavy$, 
  %       every honest leader at slot $h$ builds on a unique tine $\tau,\ell(\tau) = s$ 
  %       determined by the longest-chain selection rule. 
  %       Furthermore, if an honest slot leader at slot $h^\prime \geq h + 1$ 
  %       builds on a viable adversarial tine $t^\prime$ 
  %       that does not contain a block from slot $h$ then, 
  %       by assumption, 
  %       the last honest block on $t^\prime$ must contain $\rho$




  %       \begin{description}
  %         \item[If $t$ is viable.]
  %         ~
  %         \begin{description}
  %           \item[If $t$ contains no block from slot $s$.]
  %           We claim that this is impossible. 
  %           Let $B_2$ be the first honest block on $t$ after slot $s$ 
  %           and let $h = \ell(B_2)$.
  %           (If $B_2$ does not exist, take $h = T + 1$.)
  %           $t$ must contain, as its prefix, some adversarial tine $t^\prime$ 
  %           such that $t^\prime$ is viable at the onset of slot $h$ and, 
  %           importantly, the last honest block on $t^\prime$ 
  %           is $\rho$ (instead of $\tau$). 
  %           By Fact~\ref{fact:viable-adv-tine-$\Aheavy$}, 
  %           the interval $I = [s, h - 1]$ must be $\Aheavy$. 
  %           However, it contradicts the fact that $s$ is a Catalan slot. 
  %           Thus $t$ must contain some block from slot $s$.
            
  %           \item[If $t$ contains a block from slot $s$.]
  %           {\color{red} This looks like okay.}
  %         \end{description}

  %         \item[If $t$ is not viable.]
  %         To do
  %       \end{description}


  %       We claim that $t$ must contain some block from slot $s$: 
  %       to see this, note that
  %       it may (or may not) contain some other block from slot $s$.

  %        as a prefix and, in fact, 
  %       $t$ cannot be viable at the onset of slot $|x^\prime y^\prime| + 1$.
  %       We wish to show that $\reach(t) < 0$ for all $t \in A^\prime$. 
  %       Note that the reach of an adversarial tine 
  %       is no more than the reach of the last honest block on that tine. 
  %       Let $B$ be an honest block on $t$ so that $\ell(B) \leq s - 2$; 
  %       we wish to show that $\reach(B) < 0$.
  %       Since $s - 1$ is a Catalan slot, the interval $I = [\ell(B) + 1, T]$ 
  %       is $\Hheavy$.
  %       By Part~\ref{reach:$\Hheavy$-neg-reach}, $\reach(B) < 0$.
  %       Since the above argument is true for any fork $F \Fork x^\prime y^\prime$, 
  %       we conclude that $\mu_{x^\prime}(y^\prime) < 0$. 

  %     \end{description}



  %   \end{proof}


  %   Below is another proof which does not use the fact that the slots $s, s - 1$ 
  %   the statement of Fact~\ref{fact:cp-implies-neg-margin-multiple-honest} 
  %   are in fact Catalan slots.
  %   \begin{proof}
  %     Let $k \in [s + 1, |x^\prime y^\prime| + 1]$. 

  %     \begin{description}[font=\normalfont\itshape\space]
  %       \item[Part~\ref{margin-s-1-multihonest}]
  %       Let $m^\prime, n^\prime \in \NN, m^\prime < s - 1 \leq n^\prime \leq T$.
  %       Write $w = x^\prime y^\prime z$ with $|x^\prime| = m^\prime$ and $|x^\prime y^\prime| = n^\prime$. 
  %       (Thus $|y^\prime| \geq 1$ and $y^\prime_1 = w_{s-1}$.)
  %       Let $F$ be any fork for $x^\prime y^\prime$.
  %       By assumption, 
  %       all tines in $F$ viable at the onset of slot $|x^\prime y^\prime| + 1$ 
  %       have, as their common prefix, 
  %       {\color{blue}a unique honest tine} $\rho, \ell(\rho) = s - 1$. 
        
  %       Let $A$ be the set of tines that are 
  %       $y^\prime$-disjoint with the longest honest tine in $F$. 
  %       We claim that any tine $t \in A$ 
  %       must have $\reach(t) < 0$. 
  %       Let us identify a block $B$ on $t$, as follows.
  %       \begin{description}[font=\normalfont\itshape\space]
  %         \item[If $t$ is honest.]
  %         If $\ell(t) \leq s - 1$ then, trivially, 
  %         $t$ must come from a slot indexed by $x^\prime$;
  %         in this case, we set $B = t$.
  %         If $\ell(t) \geq s + 1$ then 
  %         $t$ must build on some viable tine which, by assumption, 
  %         contains $\rho$ as a prefix and thus $t \not \in A$; 
  %         hence this is not possible. 
  %         If $\ell(t) = s$ and $t$ is viable at the onset of slot $s + 1$, 
  %         $t$ must contain $\rho$ as its prefix (and thus $t \not \in A$).
  %         Hence if $\ell(t) = s$ and $t \in A$, 
  %         $t$ is not viable at the onset of slot $s + 1$. 
  %         Since all honest nodes employ a consistent longenst-chain tie-breaking rule, 
  %         this can ony happen if $t$ builds on 
  %         some adversarial tine $t^\prime, t^\prime \in A, \ell(t^\prime) \leq s - 2$. 
  %         Let $B$ be the last honest block on $t^\prime$. 
  %         In any case, $\ell(B) \leq s - 2$.

  %         \item[If $t$ is adversarial.]
  %         Let $B$ be the last honest block on $t$.
  %         By the discussion above, $\ell(B) \leq s - 2$. 
  %       \end{description}
  %       \noindent
  %       Thus $B$ is an honest block with $\ell(B) \leq s - 2$.
  %       By our assumption, there can be no augmentation of $F$ 
  %       (via adding new vertices and edges) 
  %       so that the new fork would contain an adversarial tine $\tilde{t}$ 
  %       viable at the onset of slot $|x^\prime y^\prime| + 1$ 
  %       so that $B$ is the last honest block on $\tilde{t}$.
  %       Using the contrapositive of 
  %       Part~\ref{reach:$\Aheavy$} of Fact~\ref{fact:ext-reach}, 
  %       we conclude that $\reach(B) < 0$.
  %       As the reach of an adversarial tine is no more than the reach of its last vertex, 
  %       it follows that in any fork $\phi \Fork x^\prime y^\prime$, 
  %       $\reach(u) < 0$ where 
  %       the tine $u$ is $y^\prime$-disjoint 
  %       with the longest honest tine in the fork $\phi$.
  %       It follows that $\mu_{x^\prime}(y^\prime) < 0$.

  %       \item[Part~\ref{margin-s-multihonest}]
  %       To do.

  %     \end{description}
      


  %     Since slot $s$ belongs to $y$, 
  %     $F$ cannot contain two tines 
  %     such that 
  %     i) both tines are viable at the onset of slot $|xy| + 1$ 
  %     and, at the same time, 
  %     ii) disjoint over the length of $y$. 
  %     % As any $x$-balanced fork for $xy$ requires two maximally long tines 
  %     % that are disjoint over $y$, 
  %     In particular, 
  %     $F$ cannot be $x$-balanced. 
  %     As $F$ was an arbitrary fork for $xy$, 
  %     no fork for $xy$ can be $x$-balanced (for our choice of $m, k$).
  %     According to Fact~\ref{fact:margin-balanced}, 
  %     we conclude that 
  %     the relative margin $\mu_x(y)$ must be strictly negative.
  %     Since $m, k$ satisfy $m < s \leq k \leq T$, 
  %     the above conclusion applies to 
  %     all decompositions $w = xyz$ where $|x| < s$ and $|xy| \geq s$.
  %   \end{proof}




  \subsubsection{Negative relative margin implies {\color{blue}left-Catalan} slot}


    \begin{fact}[Negative relative margin implies a left-Catalan slot]\label{fact:neg-margin-implies-catalan}
      Let $w \in \{0,1\}^T$ be a characteristic string 
      and let $s \in [T]$ be an honest slot in $w$.  
      If slot $s$ is margin-critical then 
      it is left-Catalan.
    \end{fact}
    \begin{proof}
      Recall that $s$ is an honest slot by assumption. 
      If $s = 1$, the claim follows since, by definition, 
      slot $1$ is left-Catalan if and only if $w_1 = 0$. 
      Now suppose $s \geq 2$.
      We will show that $s$ must be a left-Catalan slot. 
      For a contradiction, suppose $s$ is \emph{not} a left-Catalan slot. 
      Then there must be a $\Aheavy$ (open) interval $I = (a, b) \ni s$ 
      such that both $a, b$ are honest slots, $a \leq s - 1$, and $b \geq s + 1$. 
      Let us write $w = xy^\prime z$ 
      so that $|x| = s - 1$ and $|xy^\prime| = b - 1$. 
      (Thus the slot $s$ is the first slot in $y^\prime$. 
      In addition, $\mu_x(y^\prime)$ is negative by assumption.)
      Let $F^\prime$ be a fork for $xy^\prime$ 
      and let $t^\prime$ be the longest honest tine in $F^\prime$. 
      Necessarily, $\reach(t^\prime) \geq 0$.
      Since $I$ is $\Aheavy$, we can 
      extend the honest block at slot $a$---using 
      only the adversarial slots in $I$---to 
      create a tine $t^+$ so that $\length(t^+) = \length(t^\prime)$, 
      i.e., $\gap(t^+) = 0$. 
      Thus $\reach(t^+) = \reserve(t^+) - \gap(t^+) \geq 0$ 
      since reserve is always non-negative.
      Thus the augmented fork $F^+$, 
      which is comprised of $F^\prime$ plus the new tine $t^+$, 
      contains two tines---each with a non-negative reach---that are also 
      disjoint over the length of $y^\prime$. 
      Thus $F^+ \Fork xy^\prime$ is an $x$-balanced fork and, 
      by Fact~\ref{fact:margin-balanced}, 
      $\mu_x(y^\prime)$ must be non-negative; 
      but this violates our prior observation that $\mu_x(y^\prime)$ is negative; 
      hence the desired contradiciton.
      Therefore, $s$ must be a left-Catalan slot.
      
    \end{proof}


\subsection{CP and margin with multiple honest slots}
  \begin{conjecture}[CP and relative margin]\label{thm:multiple-honest-cp-margin}
    Let $w \in \{0,1\}^T$ be a characteristic string 
    and let $s \in[2, T]$ be an integer such that 
    both $s$ and $s - 1$ are honest slots in $w$.  
    Denote by $M$ the following event:
    \begin{enumerate}[label=\roman*.]
      \item $\mu_{x}(y) < 0$ 
      for all decompositions $w = xyz$ 
      so that $|x| < s, |xy| \geq s$; and 
      \item $\mu_{x^\prime}(y^\prime) < 0$ 
      for all decompositions $w = x^\prime y^\prime z$ 
      so that $|x^\prime| < s - 1, |x^\prime y^\prime| \geq s - 1$. 
    \end{enumerate}       
    A strong-CP event with parameter $s$ implies $M$, 
    and 
    $M$ implies a weak-CP event with parameter $s$.
    {\color{blue} Equivalence?}
  \end{conjecture}

  \begin{proof}
    ~
    \paragraph{Event $M$ implies weak-CP.} 
    Consider any slot $k, k \geq s + 1$ and consider any fork $F$ for $w$.
    By Lemma~\ref{lemma:neg-margin-implies-bottleneck}, 
    every tine $t$ viable at the onset of slot $k$
    must contain a block from slot $s$. 
    Similarly, for any slot $k^\prime, k^\prime \geq s$,
    every tine $t$ viable at the onset of slot $k^\prime$
    must contain a block from slot $s - 1$. 
    As $s, s - 1$ are honest slots and 
    honest nodes use $\LCR(\cdot)$ to select the longest chain to build upon, 
    all honest tines $\tau, \ell(\tau) = s$ 
    must build on a unique tine $\rho, \ell(\rho) = s - 1$.
    {\color{red} But it does not show that $\tau$ is unique.}


    % By the equivalence of 
    % Part~\eqref{thm:part-margin-multiple-honest} and Part~\eqref{thm:part-cp-multiple-honest} 
    % of our claim, 
    % slot $s - 1$ must be Catalan. 
    % Hence, 
    % by Fact~\ref{fact:catalan-unique-longest}, 
    % all blocks from slot $s$ 
    % must build on a unique honest tine $\rho$ from slot $s - 1$. 
    % Thus $\rho$ is a common prefix for all tines viable at the onset of slot $k$.

    \paragraph{Strong-CP implies event $M$:} 
    Using Theorem~\ref{thm:multiple-honest-cp-catalan}, 
    assume that the first two statements are equivalent.

    \begin{description}[font=\normalfont\itshape\space]
      \item[Strong-CP implies Part (i) of event $M$.]
      Let $m^\prime, n^\prime \in \NN, m^\prime < s - 1 \leq n^\prime \leq T$.
      Write $w = x^\prime y^\prime z$ with $|x^\prime| = m^\prime$ and $|x^\prime y^\prime| = n^\prime$. 
      (Thus $|y^\prime| \geq 1$ and $y^\prime_1 = w_{s-1}$.)
      Let $F$ be any fork for $x^\prime y^\prime$. 
      The reach of the longest honest tine $\tilde{t}$ in $F$ is always zero 
      and by assumption, it contains $\rho$ as a prefix.
      Let $A^\prime$ be the set of tines $t$ such that 
      $t$ is $y^\prime$-disjoint with $\tilde{t}$. 
      Thus $t$ cannot contain $\rho$ as a prefix and, in fact, 
      $t$ cannot be viable at the onset of slot $|x^\prime y^\prime| + 1$.
      We wish to show that $\reach(t) < 0$ for all $t \in A^\prime$. 
      Note that the reach of an adversarial tine 
      is no more than the reach of the last honest block on that tine. 
      Let $B$ be an honest block on $t$ so that $\ell(B) \leq s - 2$; 
      we wish to show that $\reach(B) < 0$.
      Since $s - 1$ is a Catalan slot, the interval $I = [\ell(B) + 1, T]$ 
      is $\Hheavy$.
      By Part~\ref{reach:$\Hheavy$-neg-reach}, $\reach(B) < 0$.
      Since the above argument is true for any fork $F \Fork x^\prime y^\prime$, 
      we conclude that $\mu_{x^\prime}(y^\prime) < 0$. 


      \item[Strong-CP implies Part (ii) of event $M$.]
      Let $m, n \in \NN, m < s \leq n \leq T$.
      Write $w = x y z$ with $|x| = m$ and $|x y| = n$. 
      (Thus $|y| \geq 1$ and $y_1 = w_{s}$.)
      Let $F$ be any fork for $x y$. 
      The reach of the longest honest tine $\tilde{t}$ in $F$ is always zero 
      and by assumption, it contains $\rho$ as a prefix. 
      Let $\tau$ be the block on $\tilde{t}$ so that $\ell(\tau) = s$.

      Let $h$ be an honest slot, $h \leq s - 1$ 
      and let $B_h$ be an honest block from slot $h$. 
      Since $s$ is a Catalan slot,         
      every interval $[h + 1, T]$ is $\Hheavy$. 
      By Part~\ref{reach:$\Hheavy$-neg-reach} of Fact~\ref{fact:ext-reach}, 
      $\reach(B_h) < 0$. 
      Note that the reach of an adversarial tine is no more than the reach of its last honest block.
      Consider a tine $t$ whose last honest vertex comes from the interval $[1, s - 1]$. 
      (Necessarily, $t$ is $y$-disjoint with $\tilde{t}$.)
      It follows that $\reach(t) < 0$. 

      Next, we wish to show that $\reach(t) < 0$ for all tines $t, \ell(t) \geq s$.
      Let $h$ be an honest slot from the interval $s, T$. 
      Let $t$ be a tine which contains 
      its last honest vertex (let us call it $B$) from slot $h$ and, 
      importantly, $t$ is $y$-disjoint with $\tilde{t}$, i.e., the longest tine in $F$. 
      Note that $t$ cannot be viable at the onset of slot $|xy| + 1$ since, otherwise,       
      it would contain $\tau$ from slot $s$ 
      (this would contradict the fact that $t$ is $y$-disjoint with $\tilde{t}$). 
      Thus $B$ is not viable at the onset of slot $|xy| + 1$. 
      Since $F$ is an arbitrary tine, the interval $I = [h + 1, T]$ must $\Hheavy$. 
      By Part~\ref{reach:$\Hheavy$-neg-reach} of Fact~\ref{fact:ext-reach}, 
      $\reach(B) < 0$.
      Note that the reach of an adversarial tine 
      is no more than the reach of the last honest block on that tine. 
      Therefore, 
      for all forks for $xy$ and all tines $t \in F$ 
      that are $y$-disjoint with the longest tine in $F$, 
      $\reach(t) < 0$. 
      We conclude that $\mu_x(y) < 0$.
    \end{description}

  \end{proof}
\hrule


  % -------------------------------------------
  % \begin{fact}[Catalan slot implies common prefix]\label{fact:catalan-convergence}
  %   Let $w \in \{0,1\}^T$ be a characteristic string 
  %   and $s$ a Catalan slot with $s \leq T - 1$. 
  %   Let $F$ be any fork for $w$ and 
  %   let $\tau = \LCR(C)$ 
  %   where $C$ is the set of tines (in $F$) from slot $s$. 
  %   For any slot $k, k \geq s + 1$, 
  %   every tine $t \in F$ 
  %   that is viable at the onset of slot $k$ 
  %   must contain $\tau$ as its prefix. 
  % \end{fact}
  % \begin{proof}
  %   As $s$ is a Catalan slot, slots $s$ and $s + 1$ must be honest. 
  %   By Fact~\ref{fact:catalan-unique-longest}, 
  %   the honest tine $\tau$ is the uniquely identified viable tine 
  %   at the onset of slot $s + 1$.    
  %   Let $t$ be a viable tine at the onset of some slot $k, k \geq s + 2$. 
  %   We claim that $\tau$ must be a prefix of $t$. 
  %   Suppose, for a contradiction, that 
  %   $t$ does not contain $\tau$ as its prefix. 
  %   Below, we argue that this leads to a contradiction. 

  %   Let $B_1$ be the last honest block on $t$ such that $\ell(B_1) \leq s - 1$.
  %   Likewise, let $B_2$ be the first honest block on $t$ such that $\ell(B_2) \geq s + 1$. 
  %   Let $I$ be the interval $[\ell(B_1) + 1, \ell(B_2) - 1]$. 
  %   If $I$ is empty, $\ell(B_2) = s + 1$ and by Observation~\ref{obs:multi-honest}, 
  %   $B_2$ builds on $\tau$. 
  %   Therefore, let us assume that $I$ is non-empty.

  %   \begin{description}[font=\normalfont\itshape\space]
  %     \item[If $t$ does not contain a block from slot $s$.]
  %     By construction, $t$ does not contain any honest blocks 
  %     from the interval $I$. 
  %     Thus the block $B_2$ is built upon 
  %     some adversarial tine which  
  %     i) contains $B_1$ as its last honest block and 
  %     ii) is viable at the onset of slot $\ell(B_2)$. 
  %     By Fact~\ref{fact:viable-adv-tine-$\Aheavy$},       
  %     the interval $I$ must be $\Aheavy$. 
  %     However, since $I \ni s$, it contradicts the fact that $s$ is a Catalan slot. 

  %     \item[If $t$ contains a block $B$ from slot $s$ so that $\depth(B) < \length(\tau)$.]
  %     Note that $t$ must contain adversarial blocks from the non-empty interval $I$ 
  %     (otherwise, $B_2$ would build on some honest block from $I$;  
  %     this contradicts the fact that $B_2$ is 
  %     the first honest block on $t$ from slots $s + 1, s+2, \ldots$).
  %     Let $B_3$ be the first adversarial block on $t$ 
  %     which precedes $B_2$ and 
  %     $\depth(B_3) = \hdepth(\ell(B_3))$. 
  %     Let $t^\prime$ be the tine ending with $B_3$; 
  %     thus $t^\prime$ 
  %     is a shortest viable adversarial tine 
  %     at the onset of slot $\ell(B_3) + 1$. 
  %     Necessarily, $t^\prime$ must be disjoint with the 
  %     longest honest tine from the interval $J = [s + 1, \ell(t^\prime)]$. 
  %     By Fact~\ref{fact:viable-adv-tine-$\Aheavy$}, 
  %     $J$ must be a $\Aheavy$ interval and, 
  %     in addition, 
  %     $\#_1(J) \geq \#_0(J) + k$ 
  %     where we define $k = \length(\tau) - \depth(B)$. 
  %     By assumption, $k \geq 1$ and hence 
  %     $\#_1(J) > \#_0(J)$. 
  %     It follows that the interval $[s, \ell(B_3)]$ is $\Aheavy$ as well, 
  %     contradicting the fact that $s$ is a Catalan slot.
  %     % every tine $t^\prime$ created at the honest slot $s + 1$ 
  %     % always builds upon $\tau$. 
  %     % Thus $t$ cannot contain $t^\prime$ as a prefix as well.
  %     % $B_2$ must builds on the


  %     \item[If $t$ contains a block $B$ from slot $s$ so that $B \neq \tau, \depth(B) = \length(\tau)$.]
  %     To do.


  %     \item[Blah blah]

  %     \item[If $t$ is an adversarial tine.]
  %     There must be a last honest block $B$ common to both $\tau$ and $t$. 
  %     Let $h = \ell(B)$ and note that $h \leq s - 1$. 
  %     (If $s = 1$, we can take $B$ as the genesis block and $h = 0$.) 
  %     Since $t$ is an adversarial tine viable at the onset of slot $k$, 
  %     by Fact~\ref{fact:viable-adv-tine-$\Aheavy$},       
  %     the interval $I = [h + 1, k - 1]$ must be $\Aheavy$. 
  %     However, 
  %     the interval $I$ contains $s$ and, 
  %     since $s$ is a Catalan slot, 
  %     $I$ is $\Hheavy$. 
  %     This is a contradiction.


  %     \item[If $t$ is an honest tine.]    
  %     % \textbf{LOOK HERE}
  %     Let $h_1 = \ell(B_1)$ and $h_2 = \ell(B_2)$.
  %     By construction, $t$ does not contain any honest blocks from the interval $J = [h_1 + 1, h_2 - 1]$. 
  %     Thus the block $B_2$ is built upon 
  %     some adversarial tine $t^\prime$ which  
  %     i) contains $B_1$ as its last honest block and 
  %     ii) is viable at the onset of slot $h_2$. 
  %     By repeating the argument of the preceding case 
  %     with adversarial tine $t^\prime$ and interval $J$, 
  %     we derive a contradiction. 
  %   \end{description}
  %   It follows that every viable tine $t \in F, \ell(t) \geq s + 1$ must contain $\tau$ as its prefix.
  % \end{proof}
  % ----------------------------------------------

  \begin{fact}[Last Catalan slot implies deepest balanced fork]\label{fact:catalan-implies-balanced-fork}
    Let $w \in \{0,1\}^T$ be a characteristic string which contains at least one Catalan slot. 
    If $s$ is the last Catalan slot, then 
    the deepest $x$-balanced fork for $w = xy$ 
    occurs when $|x| = s$. 
  \end{fact}
  \begin{proof}
    Suppose $s$ is the last Catalan slot in $w$. 
    Thus every slot $r, r \geq s + 1$ belongs to some $\Aheavy$ interval. 
    These intervals ``cover'' the slots $s + 1, \ldots, T$. 
    In particular,  
    consider an ordered collection of $\Aheavy$ intervals 
    $\mathcal{F} = \{I_1, I_2, \ldots\}$ where 
    the following conditions hold.
    i) Writing $I_i = [a_i, b_i] \subset [T]$, we require that $a _i \leq a_{i + 1}$
    ii) For $i \geq 2$, 
    the interval $I_i$  
    has non-empty overlaps with $I_{i-1}$. 
    For $i \leq |\mathcal{F}|$, 
    the interval $I_i$  
    has non-empty overlaps with $I_{i + 1}$. 
    iii) The union of these intervals is the interval $[s + 1, T]$.
    And iv) Of all collections of intervals that satisfy the anove conditions, 
    $\mathcal{F}$ has the smallest cardinality. 
    Let $k = |\mathcal{F}|$ and observe that 
    $s$ cannot be in any of these intervals; 
    in fact, $a_1 = s + 1$.

    Let $F$ be any fork for $w$ and let $t$ be a longest tine in $F$. 
    If there are multiple longest tines, we are done. 
    Otherwise, 
    By Fact~\ref{fact:catalan-convergence-multiple-honest}, 
    $t$ must contain, as its prefix, the unique honest tine $\tau$ with $\ell(\tau) = s$.
    Let $h$ be first honest slot in $s, s+1, \ldots, T$.
    We will construct a tine $t^\prime$ with only adversarial nodes 
    so that $\length(t^\prime) = \length(t)$ and the tines 
    $t, t^\prime$ will be disjoint over the interval $[s+1, T]$. \textbf{Double check} 
    
    Define $B_0^*$ as the last block on $\tau$.
    For every interval $I_i, i = 1, \ldots, k$, do the following. 
    \begin{itemize}
      \item Let $t[I_i]$ be the segment of the longest tine $t$ 
      pertaining to the interval $I_i$. 
      Let $n_i$ be the number of nodes in $t[I_i]$.

      \item Use the first $n_i$ adversarial slots in $I_i$ 
      to extend $B_{i-1}^*$ with a sequence of $n_i$ nodes. 
      This is possible since $I_i$ is $\Aheavy$. 
      Define $B_i^*$ as the last of these nodes 
      that falls in the interval $I_i \setminus I_{i+1}$; 
    \end{itemize}
    Let $t^\prime$ be 
    the tine ending in the adversarial block $B_{n_k, k}$. 
    By construction, it has the same length as the tine $t$ 
    and, in addition, it is completely disjoint with $t$ over the interval $[s + 1, T]$. 
    This the promised tine $t^\prime$. 
    Let $F^\prime$ be the augmented fork $F$, i.e., 
    it contains all tines of $F$ plus all nodes in the root-to-leaf path in $t^\prime$. 
    Writing $w = xy, |x| = s$, it is clear that $F^\prime$ is an $x$-balanced fork.  

    Recall from Fact~\ref{fact:catalan-convergence-multiple-honest} that 
    for all forks for $w$, all tines viable at the onset of any slot $r, r \geq s + 1$ 
    must contain $\tau$ as a prefix. 
    Thus there can be no balanced forks that contains two tines diverging beyond slot $s$. 
    Hence $F^\prime$ is the deepest balanced fork for $w$.
  \end{proof}




% \subsection{Sleepy terminology}
%   Consider an extended characteristic string $w^\prime \in \{0, 1, +, \perp\}^{T^\prime}$ 
%   and the reduced string $w = \rho_\Delta(w^\prime) \in \{0,1\}^T$.

%   \begin{definition}[Convergence opportunity]\label{def:conv-opp}
%     Let $c$ be a uniquely honest slot in $w^\prime \in \{0, 1, +, \perp\}^{T^\prime}$, 
%     where $c \leq T - \Delta$. 
%     The interval $I_c = [c - \Delta, c + \Delta]$ 
%     is called a \emph{convergence opportunity}---sometimes written CO in short---if 
%     the substring $w^\prime[c - \Delta, c + \Delta]$ 
%     satisfies the pattern $\{1, \perp \}^\Delta 0 \{1, \perp\}^\Delta$. 
%     If $c \leq \Delta$, 
%     the interval $I_c = [0, c + \Delta]$ and 
%     the pattern  
%     $\{1, \perp \}^{c - 1} 0 \{1, \perp\}^\Delta$ 
%     is used instead. 
%     When $I_c$ is understood from the context, 
%     we simply say that the slot $c$ is a convergence opportunity.
%   \end{definition}
%   Note that if a uniquely honest slot $c$ is accompanied by 
%   an honest slot within $\Delta$ slots in either direction, 
%   $c$ is not a convergence opportunity.
%   We use $\#_c(I)$ to indicate the number of convergence opportunities 
%   in an interval $I$ for the extended characteristic string $w^\prime$.
%   A block issued from an convergence opportunity is called a \emph{CO-block} in short. 
%   If $c, c^\prime$ are two COs such that there is no CO in the interval $[c + 1, c^\prime - 1]$, 
%   then $c, c^\prime$ are called \emph{adjacent} COs.

%   \begin{observation}[Convergence opportunity and the reduction map]
%     Let $c$ be a slot in $w = \rho_\Delta(w^\prime)$ 
%     so that $w_{c-1} w_c = 00$ 
%     and let $c^\prime$ be the corresponding slot in $w^\prime$. 
%     Then $c^\prime$ is a convergence opportunity. 
%     However, the converse does not hold. 
%   \end{observation}

%   Let $c$ be a convergence opportunity and let $h, h^\prime$ be two 
%   honest slots (unqiue or otherwise) such that $c - h > \Delta$ and 
%   $h^\prime - c > \Delta$. 
%   Let $B_h, B_c, B_{h^\prime}$
%   Then $\depth(B_c) \geq \depth(B_h) + 1$ and $\depth(B_{h^\prime}) \geq \depth(B_c) + 1$.

%   Let $F$ be a fork.
%   For any tine $t$ in $F$, 
%   let $B, B^\prime$ be two blocks issued from 
%   two consecutive convergence opportunities. 
%   Consider the interval $I = [\ell(B) + 1, \ell(B^\prime) - 1]$. 
%   We have $\depth(\hat{B}) \geq \depth(B) + \#_c(I) + 1$. 
%   In particular, $t$ contains at least $\#_c(I)$ blocks from the interval $I$. 

%   % \begin{observation}\label{obs:ext-$\Aheavy$-async}
%   %   Fix an extended characteristic string $w^\prime \in \{0, 1, +, \perp\}^*$ 
%   %   and 
%   %   let $B, B^\prime$ be two blocks issued from 
%   %   two consecutive convergence opportunities $c, \hat{c}$ in $w^\prime$. 
%   %   (There can be zero or more uniquely honest slots between the slots $c$ and $\hat{c}$.)
%   %   Then $\depth(B^\prime) \geq \depth(B) + 1$. 
%   %   Suppose these blocks are on the same tine and 
%   %   let $k = \depth(\hat{B}) - \depth(B) - 1$. 
%   %   If $k \geq 0$, then 
%   %   all $k$ blocks between $\hat{B}$ and $B$ must be adversarial. 
%   %   That is, let $I = [\ell(B) + 1, \ell(B^\prime) - 1]$; 
%   %   and hence $I$ contains at least $\depth(\hat{B}) - \depth(B) - 1$ adversarial slots.
%   %   and, in particular, $I$ must be $\Aheavy$.
%   % \end{observation}


%   \begin{definition}[Pivot]\label{def:pivot} 
%     Let $w^\prime \in \{0, 1, +, \perp\}^{T^\prime}$ be an extended chracteristic string. 
%     Let $a \in \NN$. 
%     A slot $s$ in $w^\prime \in \{0, 1, +, \perp\}^{T^\prime}$ 
%     is called an \emph{$a$-pivot} if, 
%     for every interval $I \subseteq [T^\prime]$ such that $c \in I$, 
%     either $\#_c(I) > \#_1(I)$ or $\#_1(I) = 0$ 
%     A slot is called a \emph{strong pivot} if it is an $a$-pivot for all $a \in [T^\prime]$.
%   \end{definition}

%   In other words, if a slot $s$ is contained in an size-$a$ interval $I$ 
%   where $\#_c(I) \leq \#_1(I)$, $s$ is not an $a$-pivot. 
%   Observe that for any strong pivot $s$, any interval $I \ni s$, 
%   and two adjacent COs $c, c^\prime \in I$, 
%   $\#_1([c + 1, c - 1]) = 0$.


%   \begin{lemma}\label{lemma:pivot-divergence}
%     Let $w \in \{0,1, +, \perp\}^T$ be 
%     an extended characteristic string and 
%     suppose the slot $s$ is a strong pivot. 
%     For any fork $F$ for $w$, all viable tines 
%     ending at slots after $s$ must 
%     pass through the unique honest block emitted from 
%     the first honest slot in the interval $s, s+1, \ldots, T$\ . 
%   \end{lemma}
%   \begin{proof}
%     ~
%     % \paragraph{Case: $s$ is an honest slot.} 
%     % Let $B^*$ be the unique honest block issued at slot $s$. 
%     % We claim that any viable tine $t \in F$ must contain $B^*$. 
%     % For a contradiction, suppose it does not. 
%     % On the tine $t$, identify 
%     % the last honest block $B$ in the interval $[1, s]$ 
%     % and the first honest block $\hat{B}$ in the interval $[s, T]$. 
%     % The interval $I = [\ell(B) + 1, \ell(\hat{B}) - 1]$ 
%     % is non-empty as it includes slot $s$ 
%     % and, in addition, $t$ contains no honest block from $I$.
%     % By Observation~\ref{obs:ext-$\Aheavy$}, 
%     % $I$ must be $\Aheavy$, contradicting the fact that $s$ is a strong pivot. 
%     % Hence all viable tines $t$, such that $\length(t) \geq \depth(B^*)$, 
%     % agree on the (unique) block issued at the honest slot $s$.


%     \paragraph{Case: $s$ is not a convergence opportunity.} 
%     Let $t$ be a viable tine with $\ell(t) \geq s$.
%     On the tine $t$, identify 
%     the block $B$ from the last CO in the interval $[1, s - 1]$ 
%     and 
%     the block $\hat{B}$ from the first CO in the interval $[s + 1, T^\prime]$. 
%     Suppose these blocks exist.
%     (If not, \textbf{to do}.) 
%     Let $I = [\ell(B) + 1, \ell(\hat{B}) - 1]$.
%     Clearly, $\depth(\hat{B}) = \depth(B) + 1 + \#_c(I)$ 
%     and in particular, 
%     $t[I]$ must contain at least $\#_c(I)$ non-CO blocks and, 
%     therefore, 
%     $I$ contains at least $\#_c(I)$ non-CO slots.
%     We claim that the interval  
%     cannot contain an honest slot. 
%     Specifically, by Observation~\ref{obs:ext-$\Aheavy$}, 
%     if $I$ is not an empty interval, it must be $\Aheavy$. 
%     But $I$ cannot be $\Aheavy$ since $s$ is a strong pivot; 
%     hence $I$ must be an empty interval and, in particular, 
%     we have $\ell(B) = s - 1, \ell(\hat{B}) = s + 1$.

%     Let us consider another tine $t^\prime$ 
%     and identify, on $t^\prime$, 
%     the closest honest blocks $C, \hat{C}$ on both sides of slot $s$. 
%     By repeating the argument above, 
%     In particular, we have $\ell(C) = s - 1 = \ell(B)$ and 
%     $\ell(\hat{C}) = s + 1 = \ell(\hat{B})$. 
%     Since there can be a unique honest vertex in an honest slot, 
%     we must have $B = C, \hat{B} = \hat{C}$ and, therefore, 
%     both $t_1$ and $t_2$ must agree about the (honest) block issued from slot $s - 1$.
%   \end{proof}

% \subsection{Pivots and balanced forks}
%   This takes us to the following interesting fact.

%   \begin{fact}\label{fact:pivot-fork}
%     Let $w \in \{0,1\}^T$ be a characteristic string. 
%     The following statements are equivalent:
%     1) Slot $s$ is the last strong pivot in $w$; and 
%     2) There exists a balanced fork for $w$; 
%     in particular, the deepest $x$-balanced fork for $w = xy$ 
%     occurs when $|x| = s$.
%   \end{fact}
%   \begin{proof}
%     (\emph{Strong pivot implies the deepest balanced fork.})
%     Suppose $s$ is the last strong pivot in $w$. 
%     (If no such slot is present, we can take slot $0$, 
%     i.e., the genesis slot, as $s$.) 
%     Thus every slot $r, r \geq s + 1$ belongs to some $\Aheavy$ interval. 
%     These intervals ``cover'' the slots $s + 1, \ldots, T$. 
%     In particular, for some $k \in \NN$, 
%     let $\mathcal{F} = (I_1, I_2, \ldots, I_k), I_i \subset [T]$ be 
%     an ordered collection of these $\Aheavy$ intervals 
%     so that 
%     1) $I_i$ has non-empty overlaps with $I_{i-1}$ and $I_{i+1}$
%     (boundary cases are handled naturally) 
%     and 
%     2) the union of the $I_i$s is the interval $[s + 1, T]$. 
%     (Notice that $s \not \in I_i$ as $s$ is a strong pivot and every $I_i$ is $\Aheavy$.)
%     Let $k = |\mathcal{F}|$.

%     Let $F$ be any fork for $w$ and let $t$ be a longest tine in $F$. 
%     If there are multiple longest tines, we are done. 
%     Otherwise, let $h$ be first honest slot in $s, s+1, \ldots, T$.
%     We will construct a tine $t^\prime$ with only adversarial blocks 
%     so that $\length(t^\prime) = \length(t)$ and the tines 
%     $t, t^\prime$ will be disjoint over the interval $[s+1, T]$. \textbf{Double check} 
    
%     For every interval $I_i, i = 1, \ldots, k$, do the following. 
%     \begin{itemize}
%       \item Let $t[I_i]$ be the segment of the longest tine $t$ 
%       pertaining to the interval $I_i$. 
%       Let $n_i = \length(t[I_i])$.

%       \item Use the first adversarial slots in $I_i$ 
%       to create a sequence of $n_i$ blocks 
%       $b_{1, i}, \ldots, b_{n_i, i}$. 
%       Let $b_i^*$ be the last block that falls in the interval $I_i \setminus I_{i+1}$; 
%       Define $b_0^*$ as the last block on the tine segment $t[0 : s]$.
%       Finally, make $b_{i,1}$ a child of $b_{i -1}^*$. 
%     \end{itemize}
%     Notice that the tine ending in the adversarial block $b_{n_k, k}$ has the same length as $t$, 
%     and, in addition, it is completely disjoint with $t$ over the interval $[s + 1, T]$. 
%     This the promised tine $t^\prime$.
%     Thus the fork $F$ augmented with $t^\prime$ is an $x$-balanced fork 
%     where we write $w = xy, |x| = s$. 
%     % This is the deepest balanced fork for $w$ since 
%     % by Fact~\ref{fact:pivot-divergence}, 
%     % every viable tine
    


%     For the other direction, 
%     let $F$ be the ``deepest'' $x$-balanced fork for $w = xy$ 
%     in the sense that $|y|$ is maximum over all decompositions $w = xy$ 
%     that yield an $x$-balanced fork.
%   \end{proof}
