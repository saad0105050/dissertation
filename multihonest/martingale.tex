In the definition of $(\epsilon, p_\h)$-Bernoulli condition (Definition~\ref{def:bernoulli-cond}), 
we assumed that the symbols of the characteristic string are independent. 
However, there are realistic election mechanisms where 
the election in the current slot depends on the past elections in an arbitrary way. 
For example, in Ouroboros Genesis~\cite{Genesis}, the model needs this structure 
so that it can handle ``dynamic availability,'' i.e., users joining and dropping off as they wish.

We can formalize the notion of ``dependence on the past'' 
in \emph{$(\epsilon, p_\h)$-martingale condition} below;
this is a generalization of Definition~\ref{def:bernoulli-cond}. 


\begin{definition}[$(\epsilon, p_\h)$-martingale condition]\label{def:martingale-cond}
  Let $T \in \NN, \epsilon \in (0,1)$, and $p_\h, p_\H \in  [0, (1+\epsilon)/2]$ so that 
  $p_\h + p_\H = (1+\epsilon)/2$. A random variable $w = w_1 \ldots w_T$
  taking values in $\{\h, \H, \A\}^T$ is said to satisfy the
  \emph{$\epsilon$-Martingale condition} if each
  $w_i, i \in [T]$, is independent and identically distributed as
  follows: 
  $\Pr[w_i = \A \mid w_1 \cdots w_{i-1}] \leq (1-\epsilon)/2$ and 
  $\Pr[w_i = \h \mid w_1 \cdots w_{i-1}] \geq p_\h$.
  The distribution of $w$ is also said
  to satisfy the $(\epsilon, p_\h)$-martingale condition.
\end{definition}


In Section~\ref{sec:martingale-dominance}, we prove that the tail bounds 
for the Bernoulli condition 
(e.g., Bounds~\ref{bound:unique-honest-catalan},\ref{bound:unique-honest-catalan-Delta}, 
and~\ref{bound:two-catalans})
can directly apply to characteristic strings that satisfy the analogous martingale condition. 
Our argument relies on stochastic dominance.

In Section~\ref{sec:martingale-proof}, 
we prove a a slightly weaker bound 
by directly using martingale techniques (without relying on stochastic dominance). 



\section{Handling martingale-like dependency via stochastic dominance}\label{sec:martingale-dominance}


  
  Define the partial order $\leq$ on the set $\{\h, \H, \A\}$:
  $\h \leq \A$ and $\H \leq \A$. 
  (The symbols $\h$ and $\H$ are not partially ordered.)
  We can extend this partial order on characteristic strings as well: 
  for $x,y \in \{\h, \H, \A\}^n$, 
  we write $x \leq y$ if and only if for all $i \in [n], x_i = \A$ implies $y_i = \A$.
  A function $X_n$ on $\{\h, \H, \A\}^n$ is \emph{monotone with respect to $\leq$} 
  if, for any two strings $W, B \in \{\h, \H, \A\}^n$, $W \leq B$ implies $X_n(W) \leq X_n(B)$. 

   

  \begin{lemma}\label{lemma:dominance-martingale}
    Let $\epsilon, p_\h \in (0,1)$ and consider 
    two equally long characteristic strings $W$ and $B$, 
    so that $W$ satisfies the $(\epsilon, p_\h)$-martingale condition 
    and $B$ satisfies the $(\epsilon, p_\h)$-Bernoulli condition. 
    Let $X$ be a monotone function on $\{\h, \H, \A\}^n$ with respect to $\leq$. 
    Then $X(W) \DominatedBy X(B)$.
  \end{lemma}

  Note that the important random variables studied so far -- e.g., 
  relative margin, the first Catalan slot, the settlement insecurity, etc. -- are 
  all monotone functions on characteristic strings. 
  It follows that our results for strings satisfying the $(\epsilon, p_\h)$-Bernoulli condition, 
  can be used directly for strings satisfying the analogous martingale condition.


  \begin{proof}
    As a
    matter of notation, for any fixed values
    $w_1, \ldots, w_k \in \{\h, \H, \A\}^k$, let
    \[
      \theta[w_1, \ldots, w_k] = \Pr[ W_{k+1} = \A \mid
      \text{$W_i = w_i$, for $i \leq k$}] \leq (1 - \epsilon)/2
    \]
    and $\theta[\varepsilon] = \Pr[W_1 = \A]$ 
    where $\varepsilon$ is the empty string. Then consider $n$ uniform and
    independent real numbers $(A_1, \ldots, A_n)$, each taking a value
    in the unit interval $[0,1]$; we use these random variables to construct a monotone
    coupling between $W$ and $B$. 
    Specifically, define $\beta: [0,1]^n \rightarrow \{\h, \H, \A\}^n$
    by the rule $\beta(\alpha_1, \ldots, \alpha_n) = (b_1, \ldots, b_n)$
    where
    \[
      b_t = \begin{cases} 
        \A & \text{if $\alpha_t \leq (1-\epsilon)/2$},\\
        \h & \text{if $(1-\epsilon)/2 < \alpha_t \leq (1-\epsilon)/2 + p_\h$},\\
        \H & \text{if $\alpha_t > (1-\epsilon)/2 + p_\h$}\,,
      \end{cases}
    \]
    and define
    $B = (B_1, \ldots, B_n) = \beta(A_1, \ldots, A_n)$; these
    $B_i$s are independent $\{\h,\H,\A\}$-valued random variables.
     % with expectation $(1-\epsilon)/2$. 
    Likewise define the function
    $\omega:[0,1]^n \rightarrow \{\h,\H,\A\}^n$ so that
    $\omega(\alpha_1, \ldots, \alpha_n) = (w_1, \ldots, w_n)$
    where each $w_t$ is assigned by the iterative rule
    \[
      w_{t+1} = \begin{cases} 
        \A & \text{if $\alpha_{t+1} \leq \theta[w_1, \ldots, w_t]$},\\
        \h & \text{if $\theta[w_1, \ldots, w_t] < \alpha_{t+1} \leq \theta[w_1, \ldots, w_t] + p_\h$},\\
        \H & \text{if $\alpha_{t+1} > \theta[w_1, \ldots, w_t] + p_\h$}\,,
      \end{cases}
    \]
    and observe that the probability law of
    $\omega(A_1, \ldots, A_n)$ is precisely that of
    $W = (W_1, \ldots, W_n)$. For convenience, we simply identify the
    random variable $W$ with $\omega(A_1, \ldots, A_n)$. Note
    that for any $\alpha = (\alpha_1, \ldots, \alpha_n)$ and for each
    $i$, the $i$th coordinates of $\beta(\alpha)$ and $\omega(\alpha)$ satisfy
    $\omega(\alpha)_i \leq \beta(\alpha)_i$ 
    % (which is to say that $W_i \leq B_i$). 
    % It follows immediately that
    % $X(\omega(\alpha)) \leq X(\beta(\alpha))$ with probability 1 and
    % hence $X(W) \dominatedby X(B)$. 
    % See~\cite[Lemma 22.5]{LevinPeres}. 
    (which is to say that $W_i \leq B_i$ with probability 1). 
    But this is equivalent to saying $W \DominatedBy B$. 
    (See~\cite[Lemma 22.5]{LevinPeres}.) 
    Since $X$ is non-decreasing with respect to this partial order, 
    we have  
    $X(\omega(\alpha)) \leq X(\beta(\alpha))$ with probability 1 and
    hence $X(W) \dominatedby X(B)$ as well. 

  \end{proof}




\section{A bound using martingale techniques}\label{sec:martingale-proof}

Recall that $\mu_x(y)$ is the relative margin (Definition~\ref{def:margin}) 
of a characteristic string $w = xy$, 
and that $\mu_x(y) \geq 0$ implies that there is an $x$-balanced fork for $w$. 
(This implies, slot $|x|+1$ is not $(|y|-1)$-settled).
\begin{bound}\label{bound:martingale}
  Let $\epsilon \in (0,1)$ and $p_\h \geq (1+\epsilon)/2$. 
  Let $x \in \{0,1\}^m$ and $y \in \{0,1\}^k$ be characteristic strings 
  satisfying the $(\epsilon, p_\h)$-martingale condition (with respect to the ordering $x_1, \ldots, x_m, y_1, \ldots, y_k$). 
  Then
  \[
    \Pr[\mu_x(y) \geq 0] \leq
    % \exp({-2\epsilon^4 (1 - O(\epsilon))n})
    3 \exp\left( -\epsilon^4 (1 - O(\epsilon) ) k/64 \right)  
    \, .
  \]
\end{bound}
Note that this bound is weaker than Bound~\ref{bound:unique-honest-catalan} and the likes.

Before we proceed, recall 
the following standard large deviation bound for supermartingales.
\begin{theorem}[Azuma's inequality (Azuma; Hoeffding). See {\cite[4.16]{Motwani:1995:RA:211390}} for a discussion]\label{thm:azuma}
  Let $X_0, \ldots, X_n$ be a sequence of real-valued random variables
  so that, for all $t$,
  $\Exp[X_{t+1} \mid X_0, \ldots, X_{t}] \leq X_t$ and
  $|X_{t+1} - X_t| \leq c$ for some constant $c$. Then  
  $
    \Pr[X_n - X_0 \geq \Lambda] \leq
    \exp\left(-{\Lambda^2}/{2nc^2}\right)
    %\,.
  $ 
  for every $\Lambda \geq 0$.
\end{theorem}

The proof uses the following results from \citet{Ouroboros} and \citet{LinearConsistencySODA}.

\begin{lemma}[{\cite[Lemma~4.19]{Ouroboros}}]\label{lem:margin-old} 
  $\rho(\varepsilon) = 0$ 
  where $\varepsilon$ is the empty string, and, for all nonempty strings $w\in\{0,1\}^*$,
  \begin{equation}
    \rho(w1) = \rho(w)+1\,, \qquad\text{and}\qquad
    \rho(w0) = \begin{cases} 0 & \text{if $\rho(w) = 0$,}\\
      \rho(w)-1 & \text{otherwise.}
    \end{cases}
    \label{eq:rho-recursive}
  \end{equation}
  Furthermore, margin satisfies the mutually recursive relationship
  $\mu(\varepsilon) = 0$ and for all $w \in \{0,1\}^*$,
  \begin{equation}
    \mu(w1) = \mu(w)+1\,,\qquad\text{and}\qquad
    \mu(w0) = \begin{cases}
      0 & \text{if $\rho(w)>\mu(w)=0$,} \\
%      \mu(w)-1 & \text{if $\rho(w)=0$,} \\
      \mu(w)-1 & \text{otherwise.}
    \end{cases}
    \label{eq:mu-recursive}
  \end{equation}
  Additionally, there exists a closed fork $F\vdash w$ such that
  $\rho(F)=\rho(w)$ and $\mu(F)=\mu(w)$.
  %(It is convenient to separate the case $\rho(w) = 0$ from the other case which also yields $\mu(w) - 1$ in the proof, so we reflect that in the statement of the theorem.)
\end{lemma}

and

\begin{lemma}[{\cite[Lemma~5.2]{LinearConsistencySODA}}]\label{lem:relative-margin-old}
  Given a fixed string $x\in\{0,1\}\text{\emph{*}}$,
  $\mu_x(\varepsilon) =\rho(x)$ 
  where $\varepsilon$ is the empty string, and, for all nonempty strings $w=xy\in\{0,1\}\text{\emph{*}},$
  \begin{equation}
    \mu_x(y1)= \mu_x(y)+1\,,\qquad\text{and}\qquad
    \mu_x(y0)= \begin{cases}
      0 & \text{if } \rho(xy) > \mu_x(y)=0\,, \\
%      \mu_x(y)-1 &  \text{if } \rho(xy)=0\,, \\
      \mu_x(y)-1 & \text{otherwise.}
    \end{cases}
    \label{eq:mu-relative-recursive}
  \end{equation}
  Additionally, there exists a closed fork $F\vdash xy$ such that
  $\rho(F)=\rho(xy)$ and $\mu_x(F)=\mu_x(y)$.
  %(It is convenient to
  %separate the case $\rho(w) = 0$ from the other case which also
  %yields $\mu(w) - 1$ in the proof, so we reflect that in the
  %statement of the lemma.)
\end{lemma}


%Now we are ready to prove Bound~\ref{bound:geometric-original}.

%\begin{proof}[of Bound~\ref{bound:geometric}]
\begin{proof}
  Let $w_1, w_2, \ldots$ be random variables obeying the
  $\epsilon$-martingale condition.  Specifically,
  $\Pr[w_t = 1 \mid E] \leq (1 - \epsilon)/2$ conditioned on any event
  $E$ expressed in the variables $w_1, \ldots, w_{t-1}$.  For
  convenience, define the associated $\{\pm1\}$-valued random
  variables $W_t = (-1)^{1+w_t}$ and observe that
  $\Exp[W_t] \leq -\epsilon$.

%\vspace{-2ex}
\paragraph{If $x$ is empty.}
Observe that in this case, the relative margin $\mu_x(y)$ reduces to 
the non-relative margin $\mu(y)$ from Lemma~\ref{lem:margin}. 
Since the sequence $y_1, y_2, \ldots$ in the statement of the claim 
is identical to the sequence $w_1, w_2, \ldots$ defined above, 
we focus on the reach and margin of the latter sequence. 
Specifically, define $\rho_t = \rho(w_1 \ldots w_t)$ and
$\mu_t = \mu(w_1 \ldots w_t)$ to be the two random variables from
Lemma~\ref{lem:margin} acting on the string $w=w_1 \ldots w_t$. The
analysis will rely on the ancillary random variables
$\overline{\mu}_t = \min(0,\mu_t)$.  Observe that $\Pr[\text{$w$
  forkable}] = \Pr[\mu(w) \geq 0] = \Pr[\overline{\mu}_k = 0]$, so we
may focus on the event that $\overline{\mu}_k = 0$. As an additional
preparatory step, define the constant
$\alpha = (1+\epsilon)/(2\epsilon) \geq 1$ and define the random
variables $\Phi_t \in \mathbb{R}$ by the inner product
  \[
    \Phi_t = (\rho_t, \overline{\mu}_t) \cdot
    \left(\begin{array}{c} 1\\ \alpha\end{array}\right) = \rho_t +
    \alpha \overline{\mu}_t\,.
  \]
  The $\Phi_t$ will act as a ``potential function'' in the analysis:
  we will establish that $\Phi_k < 0$ with high probability and,
  considering that
  $\alpha\overline{\mu}_k \leq \rho_k + \alpha \overline{\mu}_k =
  \Phi_k$, this implies $\overline{\mu}_k < 0$, as desired.
  
  Let $\Delta_t = \Phi_t - \Phi_{t-1}$; we claim that---conditioned
  on any fixed value $(\rho, \mu)$ for $(\rho_t, \mu_t)$---the
  random variable $\Delta_{t+1} \in [-(1 + \alpha),1+ \alpha]$ has
  expectation no more than $-\epsilon$. The analysis has four cases,
  depending on the various regimes of $\rho$ and $\mu$ from Lemma~\ref{lem:margin}. 
  When $\rho > 0$ and $\mu < 0$,
  $\rho_{t+1} = \rho + W_{t+1}$ and
  $\overline{\mu}_{t+1} = \overline{\mu} + W_{t+1}$, where
  $\overline{\mu} = \max(0,\mu)$; then
  $\Delta_{t+1} = (1 + \alpha)W_{t+1}$ and
  $\Exp[\Delta_{t+1} ] \leq -(1 + \alpha)\epsilon \leq -\epsilon$. When
  $\rho > 0$ and $\mu \geq 0$, $\rho_{t+1} = \rho + W_{t+1}$
  but $\overline{\mu}_{t+1} = \overline{\mu}$ so that
  $\Delta_{t+1} = W_{t+1}$ and $\Exp[\Delta_{t+1} ] \leq -\epsilon$. Similarly, when
  $\rho = 0$ and $\mu < 0$,
  $\overline{\mu}_{t+1} = \overline{\mu} + W_{t+1}$ while
  $\rho_{t+1} = \rho + \max(0, W_{t+1})$; we may compute
  \[
    \Exp[\Delta_{t+1} ] \leq \frac{1 - \epsilon}{2}(1 + \alpha) - \frac{1 +
      \epsilon}{2}\alpha = \frac{1 - \epsilon}{2} - \epsilon\alpha =
    \frac{1 - \epsilon}{2} - \epsilon\left(\frac{1}{\epsilon} \cdot
      \frac{1 + \epsilon}{2}\right) = -\epsilon\,.
  \]
  Finally, when $\rho = \mu = 0$ exactly one of the two random
  variables $\rho_{t+1}$ and $\overline{\mu}_{t+1}$ differs from
  zero: if $W_{t+1} = 1$ then
  $(\rho_{t+1}, \overline{\mu}_{t+1}) = (1,0)$; likewise, if
  $W_{t+1} = -1$ then
  $(\rho_{t+1}, \overline{\mu}_{t+1}) = (0,-1)$. It follows that
  \[
    \Exp[\Delta_{t+1} ] \leq \frac{1 - \epsilon}{2} - \frac{1 +
      \epsilon}{2}\alpha \leq -\epsilon\,.
  \]

  \noindent
  Thus $
  \Exp[\Phi_k] = \Exp \sum_{t=1}^k \Delta_t  
  \leq -\epsilon k
  $. 
  We wish to apply Azuma's inequality to conclude that
  $\Pr[\Phi_k \geq 0]$ is exponentially small. For this purpose, we
  transform the random variables $\Phi_t$ to a related supermartingale by
  shifting them: specifically, define
  $\tilde{\Phi}_t = \Phi_t + \epsilon t$ and
  $\tilde{\Delta}_t = \Delta_t + \epsilon$ so that
  $\tilde{\Phi}_t = \sum_i^t \tilde{\Delta}_t$. Then
  \[
    \Exp[\tilde{\Phi}_{t+1} \mid \tilde{\Phi}_1, \ldots,
    \tilde{\Phi}_{t}] = \Exp[\tilde{\Phi}_{t+1} \mid W_1, \ldots,
    W_{t}]\leq \tilde{\Phi}_t\,,
    \qquad
    \tilde{\Delta}_t \in [-(1 + \alpha) + \epsilon, 1+ \alpha +
    \epsilon]\,,
  \]
  and $\tilde{\Phi}_k = \Phi_k + \epsilon k$. It follows
  from Azuma's inequality that
  \begin{align}\label{eq:azuma-bound}
    \Pr[\text{$w$ forkable}] 
    &= \Pr[\overline{\mu}_k = 0] \leq \Pr[\Phi_k \geq 0] = \Pr[\tilde{\Phi}_k \geq \epsilon k] 
    \nonumber \\ 
    &\leq \exp\left(-\frac{\epsilon^2 k^2}{2k (1 + \alpha + \epsilon)^2}\right)
       = \exp\left(-\left(\frac{2 \epsilon^2}{1 + 3 \epsilon + 2\epsilon^2}\right)^2 \cdot \frac{k}{2}\right) \nonumber \\
    &\leq \exp\left(-\frac{2\epsilon^4}{1 + 35\epsilon} \cdot k\right)
    \,.
    %                  \qedhere
  \end{align}


\newcommand{\muxr}{\mu_x^{(r)}}
\newcommand{\Snr}{S_k^{(r)}}
\newcommand{\Sr}{S^{(r)}}
\newcommand{\Srstar}{S^{(r^*)}}
\newcommand{\event}[1]{\mathsf{#1}}
\newcommand{\notevent}[1]{\overline{\event{#1}}}

%\vspace{-2ex}
\paragraph{If $x$ is not empty.} 
In this case, we go back to study the sequences $x$ and $y$ as in the statement of the claim.
Recall the reach distribution (i.e., the distribution of the random variable $\rho(x)$) 
$\DistRho_m : \Z \rightarrow [0,1]$ from~\eqref{eq:dist-rho}. 
Since $x = (x_1, \ldots, x_m)$ satisfies the $\epsilon$-martingale condition, 
Lemma~\ref{lemma:rho-stationary} states that $\DistRho_m \dominatedby \StationaryRho$.
We reserve the symbol $\muxr$ for the relative margin 
random walk $\mu_x$ which starts at a non-negative initial position $r$. 
Thus $\rho(x) = \mu_x(\epsilon) = r$, and
\begin{align}\label{eq:azuma-generic}
\Pr[\mu_x(y) \geq 0] 
&= \sum_{r \geq 0}{\DistRho_m(r) \Pr[\muxr(y) \geq 0]} 
\leq \sum_{r \geq 0}{\StationaryRho(r) \Pr[\muxr(y) \geq 0]} 
\, 
\end{align}
since the sequence $( \, \Pr[\muxr(y) \geq 0] \, )_{r=0}^\infty$ is non-decreasing and $\DistRho_m \dominatedby \StationaryRho$. Fix a ``large enough'' positive integer $r^*$ whose value will be assigned later in the analysis. 
Let us define the following events:
 \begin{itemize}
  %\item Event $\event{A}_r$:~when $r > r^*$. 
  \item Event $\event{B}_r$:~it occurs when $r \in [0, r^*]$ and the $\muxr$ walk is strictly positive on every prefix of $y$ with length at most $k/2$; and 
  \item Event $\event{C}_{r,s}$:~it occurs when $r \in [0, r^*]$ and 
  $\hat{y}$ is the smallest prefix of $y$ of length $s \in [r, k/2]$ 
  such that $\muxr(\hat{y}) = 0$. 
  We say that $\hat{y}$ is a witnesses to the event $\event{C}_{r, s}$.
\end{itemize}
%Note that these two events cannot happen simultaneously. 
The right-hand side of~\eqref{eq:azuma-generic} can be written as
\begin{align*}
     &\quad \sum_{r>r^*}{\StationaryRho(r) \Pr[\muxr(y) \geq 0]} 
		+ \sum_{r \leq r^*}{\StationaryRho(r) \Pr[\event{B}_r] \cdot \Pr\left[\muxr(y) \geq 0 \mid \event{B}_r\right]} \\
    &\quad+ \sum_{r \leq r^*}{\StationaryRho(r) \sum_{s = r}^{k/2}{\Pr[\event{C}_{r,s}] \cdot \Pr[\muxr(y) \geq 0 \mid \event{C}_{r,s}]} }
    \, .
\end{align*} 
We observe that the probabilities $\Pr[\muxr(y) \geq 0]$ and $\Pr[\muxr(y) \geq 0 \mid \event{B}_r]$ are at most one. 
In addition, recall that for two non-negative sequences $(a_i), (b_i)$ of equal lengths, 
we have $\sum{a_i b_i} \leq \max b_i$ if $\sum{a_i} \leq 1$. 
Thus~\eqref{eq:azuma-generic} can be simplified as
\begin{align}\label{eq:three-terms}
\Pr[\mu_x(y) \geq 0] 
 &\leq 
    \sum_{r > r^*}{\StationaryRho(r)} 
  + \sum_{r \leq r^*}{\StationaryRho(r) \Pr[\event{B}_r]} \nonumber \\
  &\quad+ \sum_{r \leq r^*}{\StationaryRho(r)\, \max_{r \leq s \leq k/2}{\Pr[\muxr(y) \geq 0 \mid \event{C}_{r,s}]} }
  \nonumber \\
 &\leq    
      \sum_{r > r^*}{\StationaryRho(r)}            
  + 
      \max_{r \leq r^*}{\Pr[\event{B}_r]}          
  + 
      \max_{\substack{r \leq r^* \\ r \leq s \leq k/2}}{\Pr[\muxr(y) \geq 0 \mid \event{C}_{r,s}]}   
\, .
\end{align}


\emph{The first term in~\eqref{eq:three-terms} } is the right-tail of the distribution $\StationaryRho$. 
Using Lemma~\ref{lemma:rho-stationary}, 
this quantity is at most $\beta^{r^*}$ where $\beta := (1-\epsilon)/(1+\epsilon)$. 
Furthermore, it can be easily checked that the above quantity is at most $\exp(-5 \epsilon/3)$.

\emph{The second term in~\eqref{eq:three-terms} } concerns the event
$\event{B}_r$ and calls for more care.  Define
\[
  \Snr := \sum_{t=0}^k {W_t}
\]
where $W_0 = r$ and the random variables $W_t$ are defined at the
outset of this proof for $t \geq 1$.  We know that the $\muxr$ walk
starts with $\rho(x) = \mu(x) = r \geq 0$.  Since $\event{B}_r$ holds,
both the margin $\mu_x(\hat{y})$ and the reach $\rho(x\hat{y})$ remain
non-negative for all prefixes $\hat{y}$ of length
$t = 1, 2, \cdots, k/2$.  These two facts imply that the random
variable $\muxr(\hat{y})$ is identical to the sum $\Sr_t$ for all
prefixes $\hat{y}$ of length $t = 1, 2, \cdots, k/2$.

To be precise,
\[
  \Pr[\event{B}_r] = \Pr[\Sr_t \geq 0 \quad \text{for all } t \leq k/2]\,.
\]
The latter probability is at most $\Pr[\Sr_{k/2} \geq 0]$ because the
event $\Sr_{k/2} \geq 0$ does not constrain the intermediate sums
$\Sr_t$ for $t < k/2$.  Since $\Pr[\Sr_{k/2} \geq 0]$ increases
monotonically in $r$, we conclude that the second term
in~\eqref{eq:three-terms} is at most $\Pr[\Srstar_{k/2} \geq 0]$.  Now
we are free to shift our focus from the relative margin walk to the
sum of a martingale sequence.

For notational clarity, let us write $S := \Srstar_{k/2}$. 
Since the sequence $(w_t)$ obeys the $\epsilon$-martingale condition, 
$\Exp S$ is at most $M := r^* - k\epsilon/2$. 
Let us set $r^* = W_0 = k\epsilon/4$. Then $\Exp S$ is at most $-k\epsilon/4$ and Azuma's inequality gives us
\[
\Pr[S \geq 0] 
= \Pr[(S - \Exp S) \geq k\epsilon/4] 
\leq \exp\left( - \frac{(k\epsilon/4)^2}{2(k/2)\cdot 2^2}\right) 
= \exp\left( -\frac{k \epsilon^2}{64} \right)
\, .
\]
This is an upper bound on the second term in~\eqref{eq:three-terms}.

\emph{The third term in~\eqref{eq:three-terms}} concerns the event $\event{C}_{r,s}$ and it can be bounded using 
our existing analysis of the $|x|=0$ case. 
Specifically, suppose $y = \hat{y} z$ where
$\hat{y}$ is a witness to the event $\event{C}_{r,s}$. 
Since the $\muxr$ walk remains non-negative over the entire string $\hat{y}$, 
it follows that $\rho(x\hat{y}) = \mu(x\hat{y}) = 0$ 
and as a consequence, the $\mu_{x\hat{y}}$ walk on $z$ is identical to 
the $\mu$ walk on $z$. 
Our analysis in the $|x| = 0$ case suggests that 
$\Pr[\mu(z) \geq 0]$ is at most $A(k-s, \epsilon)$ 
where $|z| = k - s$ and $A(k, \epsilon)$ is the bound in~\eqref{eq:azuma-bound}. 
Since $A(\cdot,\epsilon)$ decreases monotonically in the first argument, 
$A(k-s, \epsilon)$ is at most $A(k/2, \epsilon)$. 
However, since the last quantity is independent of $r$, 
the third term in~\eqref{eq:three-terms} is at most 
$A(k/2, \epsilon) = \exp\left( -k \epsilon^4/(1+35\epsilon) \right)$. 


Returning to~\eqref{eq:three-terms} and using $r^* = k\epsilon/4$, we get
\begin{align*}
\Pr[\mu_x(y) \geq 0] 
 &\leq    \exp\left(-\frac{5 \epsilon}{3} \cdot \frac{k\epsilon}{4} \right)  
        + \exp\left(-\frac{2\epsilon^4}{1 + 35\epsilon} \cdot \frac{n}{2}\right)
        + \exp\left( -\frac{k \epsilon^2}{64} \right)
\, .
\end{align*}
It is easy to check that the above quantity is at most
$$
  3 \exp\left( - k \epsilon^4/(64 + 35 \epsilon) \right) 
= 3 \exp\left( - \epsilon^4 (1 - O(\epsilon) ) k/64 \right)
\,.
$$

% $\qed$
\end{proof}
