  % \input{estimates-two}
  % \begin{proof}
  Let $p = (1 - \epsilon)/2$ and $q = 1 - p$; 
  thus $q - p = \epsilon$. 
  Let $B$ denote the event that 
  $w$ does not contain two consecutive Catalan slots in $y$. 
  We would like to bound $\Pr_w[B]$ from above.

  Define the process $W = (W_t : t \in \NN), W_t \in \{\pm 1\}$ as $W_t = 1$ if and only if $w_t = \A$. 
  Let $S = (S_t : t \in \NN), S_t = \sum_{i \leq t} W_i$ be the position of the particle at time $t$. 
  Thus $S$ is a random walk on $\ZZ$ with $\epsilon$ negative (i.e., downward) bias. 
  By convention, set $W_0 = S_0 = 0$. 

  \paragraph{Case 1: $x$ is an empty string.} 
  In this case, we write $w = yz$ so that $|y| = k$. 
  Let $m_t$ denote the probability that 
  $t$ is the first index so that both $t$ and $t+1$ are Catalan slots in $w$, 
  with $m_0 = 0$, and consider the probability generating function 
  $\{m_t\} \SeqGF \gfM(Z) = \sum_{t = 0}^\infty m_t Z^t$. 
  Controlling the decay of the coefficients $m_t$ suffices
  to give a bound on $\Pr[B]$, i.e., 
  the probability that 
  $y$ \emph{does not} contain two consecutive Catalan slots, 
  because this probability is at most 
  $
    1 - \sum_{t =0}^{k-1} m_t 
      = \sum_{t = k}^\infty m_t
      % \,.
  $. 
  % It seems challenging to give a closed-form algebraic expression for
  % the generating function $\gfM$; 
  To this end, we develop a
  closed-form expression for a related probability generating function
  $\gfMhat(Z) = \sum_t \hat{m}_t Z^t$ which stochastically
  dominates $\gfM(Z)$. 
  Recall that this means that for any $k, \sum_{t \geq k} m_k \leq \sum_{t \geq k} \hat{m}_k$. 
  Finally, bound the latter sum  
  by using the analytic properties of $\gfMhat(Z)$. 


  % \paragraph{Generating function for the first instance of consecutive Catalan slots.}
  Recall the ``first ascent'' and ``first descent'' 
  generating functions $\gfA(Z)$ and $\gfD(Z)$ 
  from the proof of Bound~\ref{bound:unique-honest-catalan}. 
  We wish to devise the generating function for 
  the first occurrence of a left-Catalan slot 
  immediately followed by a right-Catalan slot. 
  To that end, 
  note that $\gfD(Z)$ is the generating function for 
  the first left-Catalan slot. 
  The generating function for the first right-Catalan slot 
  can be devised as follows. 
  Consider the walk $S$ starting at the origin. 
  With probability $q(1 - p/q) = \epsilon$, the walk will 
  immediately descend a step and never return to the origin. 
  But this means $S_1 \leq S_t, t \geq 2$ and hence 
  the first slot is a right-Catalan slot and we are done. 
  Otherwise, i.e., with probability $1 - \epsilon$, 
  the walk makes a (guaranteed) return to the origin in future. 
  In this case, we will have to restart our search 
  for the next consecutive Catalan slots but, 
  before that, 
  we will have to ensure that we are in a ``safe position.'' 
  In particular, we can safely restart our search if 
  Specifically, if the current position (i.e., level) of the walk is at its historical minimum, 
  we can restart our search by applying $\gfD(Z)$ to find the next left-Catalan slot.
  Thus an ``epoch'' begins with a guaranteed return and 
  ends when the walk descends to a new level for the first time. 
  Let $\gfE(Z)$ be the generating function of an epoch. 
  Thus we can write 
  \begin{align}\label{eq:gfM}
    \gfM(Z) 
    &= \gfD(Z) \cdot \{\epsilon + (1-\epsilon)\gfE(Z)\gfM(Z) \} \nonumber \\
    &= \frac{\epsilon \gfD(Z)}{1 - (1 - \epsilon) \gfE(Z) }
    \,.
  \end{align}

  
  An epoch can have two shapes. 
  If an epoch starts with an up-step (i.e., an ``up'' shape), 
  it is easy to see that the epoch ends as soon as the walk 
  returns to the origin from above and, importantly, 
  that the walk will (eventually) return to the origin with probability one. 
  However, if the epoch starts with a down-step (i.e., a ``down'' shape), 
  we have to ``remember'' the lowest level $\ell$ touched 
  by the walk in its way to its (sure) ascent to the origin 
  and then descend $\ell$ levels to end the epoch. 
  In particular, we have to ensure that we return to the origin with probability one. 
  
  A generating function of a stopping time of a random walk 
  is ill suited to ``remember'' its historical minimum/maximum. 
  However, it can remember the length of the walk for free. 
  Thus, instead of working directly with $\gfE(Z)$, 
  we work with a generating function $\gfEhat(Z)$ 
  which is identical to $\gfE(Z)$ for the up shape 
  but differs in the down shape. 
  Specifically, in the down shape, 
  the walk represented by $\gfEhat(Z)$ descends as many levels 
  as the number of steps it took to return to the origin. 
  Clearly, $\gfE \DominatedBy \gfEhat$ where 
  \[
      \gfEhat(Z) \triangleq p Z \gfD(Z) + q Z \gfA(Z \gfD(Z) )/\gfA(1)
      \,.
  \] 
  Here, the first term denotes the ``return to origin from above'' shape. 
  An individual term in $\gfA(Z \gfD(Z)) = \sum_t a_t Z^t \gfD(Z)^t$ 
  has the interpretation 
  ``if the first ascent took $t$ steps then follow it by descending $t$ levels.''
  Since $\gfA(Z)$ is not a probability generating function, 
  we have to normalize it by $\gfA(1)$ to denote that 
  the ascent happens with certainty. 
  This implies, 
  \[
      \gfM(Z) 
          \DominatedBy \gfMhat(Z) 
          \triangleq \frac{\epsilon \gfD(Z)}{1 - (1 - \epsilon) \gfEhat(Z) }
  \]

  % \paragraph{Convergence of $\gfM(Z)$.}
  It remains to establish a bound on the radius of convergence of
  $\gfMhat$. 
  A sufficient condition for the convergence of
  $\gfMhat(z)$ for some $z \in \RR$ is 
  that all generating functions appearing in the definition of
  $\gfMhat$ converge at $z$ and 
  that $(1-\epsilon) \gfEhat(Z) \neq 1$. 

  By retracing our footsteps as in the proof of Bound~\ref{bound:unique-honest-catalan}, 
  we can see that $\gfD(z), \gfA(z)$, and $\gfA(z \gfD(z))$ converge 
  when $|z|$ satisfies~\eqref{eq:roc-AZDZ}. 
  Moreover, since $\gfD(Z)$ is a probability generating function, 
  it follows that $\gfEhat(Z)$ is stochastically dominated by 
  $p Z \gfD(Z) + q Z \gfA(Z \gfD(Z) )/\gfA(1) \cdot \gfD(Z)$.
  Therefore, when $\gfEhat(z)$ converges for some $z$, it satisfies 
  \begin{align*}
      \gfEhat(z)
      &\leq pz\gfD(z) + (q/p) (q z\gfD(z))\gfA(z\gfD(z)) \\
      &< 1/2 + (q/p)/2
  \end{align*}
  since $\gfA(1) = p/q, pz\gfD(z) < 1/2$, 
  and $qx\gfA(x) < 1/2$ for any $z, x$ so that $\gfA(x)$ and $\gfD(z)$ converge, respectively. 
  Therefore, $(1-\epsilon)\gfEhat(z) = 2p \gfEhat(z) < p + q = 1$. 
  It follows that 
  $\gfMhat(z)$ converges for
  $|z| < 1 + \epsilon^3/2 + O(\epsilon^4) \leq \exp(\epsilon^3/2 + O(\epsilon^4))$. 
  Recall that if the radius of convergence of
  $\gfMhat$ is $\exp(\delta)$ then 
  $\Pr[B]$ is 
  $O(1) \cdot e^{-\delta k}$. 
  We conclude that
  \begin{align}
    \Pr_w[B] 
      &\leq O(1) \cdot e^{-\epsilon^3(1 + O(\epsilon))k/2} \,.
  \label{eq:prob_two_catalan_gf}
  \end{align}


  \paragraph{Case 2: $x$ is non-empty.}
  This part of the proof is the same as the $|x| \geq 1$ case 
  in the proof of Bound~\ref{bound:unique-honest-catalan}. 
  The only difference is that 
  $\gfChat(Z)$ and $\gfCtilde(Z)$ would be replaced by 
  $\gfMhat(Z)$ and $\gfMtilde(Z)$, respectively, where 
  \[
    \gfMtilde(Z)\DominatedBy \sum_{h = 0}^\infty \mathcal{X}_\infty(h) \gfD(Z)^h \gfMhat(Z)
    \,.
  \]
  We conclude that the bound
  in~\eqref{eq:prob_two_catalan_gf} holds when $|x| \geq 0$. 
  \hfill$\qed$
 % \end{proof}
