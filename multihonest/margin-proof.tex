

The proof of Theorem~\ref{thm:relative-margin} is presented in two parts. 
Let $w \in \{\h, \H, \A\}^*$.
First, for a given decomposition $w = xy$, 
 % and $b \in \{\h, \H\}$, 
we prove an upper bound on $\mu_x(y)$. 
Next, 
considering the fork $F \Fork w$ built by the strategy $Adversary^*$ 
(see Figure~\ref{fig:adv-opt-mh}), 
we show that 
for every decomposition $w = xy$, 
$\mu_x(F)$ is at least as large as the upper bound proven in the first part; 
thus $F$ is canonical. 

As a warm-up, we start with the following claim.

\begin{claim}\label{claim:rho-mu-A}
  $\rho(\varepsilon) = 0$. 
  For any $x, y \in \{\h, \H, \A\}^*$, 
  $\mu_x(\varepsilon) = \rho(x)$, 
  $\rho(xy\A) = \rho(xy) + 1$, 
  and $\mu_x(y\A) = \mu_x(y) + 1$.
\end{claim}
\begin{proof}
  The only possible fork for the empty string $\varepsilon$ 
  contains a single honest vertex 
  with reserve and gap both zero; 
  hence $\rho(\varepsilon) = 0$.

  Let $F$ be a closed fork for the characteristic string $xy$. 
  Let $t_\rho, t_x \in F$ be the two tines that witness $\mu_x(F)$, 
  i.e., $\reach(t_\rho) = \rho(F), \reach_F(t_x) = \mu_x(F)$, 
  and $t_\rho, t_x$ are disjoint over $y$. 
  % Let $\hat{t}$ be the longest tine in $F$.


  In the base case, where $y=\varepsilon$, 
  observe that any two tines of $F$ 
  are disjoint over $y$. 
  Moreover, a single tine $t \in F$ 
  is disjoint with itself over the empty suffix $\varepsilon$. 
  Therefore, the relative margin $\mu_x(\varepsilon)$ must be at least $\rho(x)$. 
  As $\mu_x(F)$ can be no more than $\rho(x)$, it follows that 
  $\mu_x(\varepsilon) = \rho(x)$.


  Now consider a pair of closed forks $F\vdash xy$ and $F'\vdash xy\A$ 
  such that $F \fprefix F'$ and $x,y\in\{\h, \H, \A\}^*$. 
  We must have $F' = F$ since $F'$ is closed. 
  In addition, for any tine $t \in F$, 
  $\reach_{F'}(t) = \reach_F(t) + 1$ 
  since the reserve has increased by one 
  but the gap is unchanged 
  (as no new tine is added). Therefore, 
  $\rho(xy\A) = \rho(xy) + 1$ and 
  $\mu_x(y\A) = \mu_x(y)+1$.
\end{proof}


\section{An upper bound on relative margin}

\begin{proposition}\label{prop:mu-upperbound}
  Let $w,x, y \in \{\h, \H, \A\}^*$ 
  and $b \in \{\h, \H\}$, 
  Then 
  \begin{equation}
    \rho(xyb) \leq \begin{cases}
      0 & \text{if $\rho(xy) = 0$}\,, \\
      \rho(xy) - 1 & \text{otherwise.}
    \end{cases}
    \label{eq:rho-upperbound}
  \end{equation}
  Furthermore,
  \begin{equation}
    \mu_x(yb) \leq \begin{cases}
      0 & \text{if $\rho(xy) > \mu_x(y)=0$}\,, \\
      0 & \text{if $\rho(xy) = \mu_x(y) = 0$ and $b = \H$}\,, \\
      \mu_x(y)-1 & \text{otherwise.}
    \end{cases}
    \label{eq:mu-upperbound}
  \end{equation}
\end{proposition}
\begin{proof}
  Suppose $F'\vdash xyb$ is a closed fork such that 
  $\rho(xyb)=\rho(F')$ and $\mu_x(yb)=\mu_x(F')$. 
  Let $t_\rho, t_x \in F'$ be a pair of $y$-disjoint tines such that $\reach_{F'}(t_\rho)=\rho(F')$ and $\reach_{F'}(t_x)=\mu_x(F')$. 
  (If there are multiple candidates for $t_\rho$ or $t_x$, 
  select the one with the smallest $\leq_\pi$ rank.)
  Let $F\vdash xy$ be the unique closed fork such that $F\fprefix F'$.  
  Note that while $F'$ is obtained from one or more extensions 
  of $F$-tines, 
  these extensions are not necessarily conservative. 
  Recall that $\reach_{F'}(t) \leq 0$ for any tine $t \in F', \ell(t) = |xy| + 1$.

  \paragraph{Proving ~\eqref{eq:rho-upperbound}.} 
  Let $A$ be the set of all $F'$-tines with label $|xy| + 1$.
  Let $\sigma \in A$ be the first tine in the $\leq_\pi$ ordering so that $\reach(\sigma) = \max_{t \in A}\{\reach_{F'}(t)\}$.
  By Fact~\ref{fact:reach-fork-ext-mh}, 
  $\reach_{F'}(\sigma) \leq 0$ and, 
  in addition, for any $t \in F$, 
  $\reach_{F'}(t) \leq \reach_F(t) - 1$.   
  Let $\hat{t}$ be the maximum-reach tine in $F$ 
  with the smallest $\leq_\pi$ rank.

  If $\rho(F) = 0$ then 
  $\reach_{F'}(t) < 0$ for all $t \in F$. 
  Hence $t_\rho = \sigma$ and, consequently, 
  $\rho(xyb) \leq 0$. 
  If $\rho(F) \geq 2$ then $t_\rho \in F$
  % $\reach_{F'}(\hat{t}) \geq 1$ but $\reach_{F'}(\sigma) \leq 0$; 
  % thus $t_\rho = \hat{t}$ 
  and, therefore, 
  $\rho(xyb) = \reach_{F'}(t_\rho) \leq \rho(F) - 1 \leq \rho(xy) - 1$. 
  If $\rho(F) = 1$ and $t_\rho \in F$ then, 
  as before, 
  $\rho(xyb) = \reach_{F'}(\hat{t}) = \reach_{F}(\hat{t}) - 1 = \rho(F) - 1 \leq \rho(xy) - 1$.
  If $\rho(F) = 1$ and $t_\rho \not \in F$ then, as we have seen before, 
  $\rho(xyb) = \reach_{F'}(\sigma) \leq 0 = \rho(F) - 1\leq \rho(xy) - 1$.
  Thus we have proved~\eqref{eq:rho-upperbound}.


  \paragraph{Proving ~\eqref{eq:mu-upperbound}.} 
  If $\ell(t_\rho) = |xy| + 1$ then we are done: 
  by our preceding argument, $\reach_{F'}(t_\rho) \leq 0$. 
  On the other hand, 
  Note that $t_\rho \not \in F$ since, by Fact~\ref{fact:reach-fork-ext-mh}, reach of any $F$ tine can only decrease
  $t_\rho$ must have been an extension of a maximum-reach $F$-tine.

  \paragraph{Case 1: $\rho(xy)>0$ and $\mu_x(y)=0$.} 
    We wish to show that $\mu_x(yb) \leq 0$.
    Suppose (toward a contradiction) that $\mu_x(yb) > 0$. 
    Then neither $t_\rho$ nor $t_x$ is a conservative extension because, as we proved in Claim ~\ref{claim:ex-mh}, conservative extensions have reach zero. This means that $t_\rho$ and $t_x$ existed in $F$, and their $F$-reach was strictly greater than their $F'$-reach (by Claim ~\ref{claim:nex-mh}). 
    Because $t_\rho$ and $t_x$ 
    % have been implicitly 
    are 
    disjoint over $y0$, they must also be disjoint over $y$; therefore, $\mu_x(F)$ must be at least $\min(\reach_F(t_\rho),\reach_F(t_x))$. 
    It follows that 
    $0 
    = \mu_x(y) 
    \geq \min(\reach_F(t_\rho),\reach_F(t_x))
    > \min(\reach_{F'}(t_\rho),\reach_{F'}(t_x))
    = \mu_x(F') = \mu_x(yb)
    $. 
    The last term is strictly positive by assumption and hence, a contradiction ensues.

  % \paragraph{Case 2: $\rho(xy)=0$.}
  \paragraph{Case 2: $\rho(xy)=0$.}
    We wish to show that 
    \begin{enumerate*}[label=(\textit{\roman*})]
      \item $\mu_x(yb) \leq 0$ if $b = \H$ and $\mu_x(y) = 0$, and 
      \item $\mu_x(yb) \leq \mu_x(y) - 1$ otherwise.
    \end{enumerate*}
    First, we claim that $t_\rho$ must arise from an extension. 
    Suppose, toward a contradiction, that $t_\rho$ is not an extension, 
    i.e., $t_\rho \in F$. 
    The fact that $t_\rho$ achieves the maximum reach in $F'$ 
    implies that 
    $t_\rho$ has a non-negative reach 
    since the longest honest tine always achieves reach zero. 
    Furthermore, 
    Claim ~\ref{claim:nex-mh} states that 
    all $F$-tines see their reach decrease. 
    Therefore, $t_\rho \in F$ must have had a strictly positive reach. 
    But this contradicts the central assumption of the case, i.e., 
    that $\rho(xy)=0$. 
    Therefore, we conclude that $t_\rho \in F' \setminus F$.

    Let $s \in F$ be the tine-prefix of $t_\rho \in F'$ so that 
    $t_\rho$ is an extension of $s$. 
    % Since $\reach_{F'}(t_\rho) = \rho(xy0) = 0$ by~\eqref{eq:rho-recursive-mh}, 
    Observe that $\reach_F(s)$ must be non-negative since 
    otherwise, $s$ could not have been extended. 
    In fact, our assumption $\rho(xy)=0$ implies that 
    $\reach_F(s) = 0$. 
    In addition, since $t_x$ and $t_\rho$ are disjoint over $yb$, 
    so are $t_x$ and $s$. 
    \begin{description}[font=\normalfont\itshape\space]
      % \item[If $t_x \in F$,] 
      \item[If $b = \h$,] 
      $t_\rho$ is the only extension in $F'$ and hence 
      $t_x$ must be in $F$. 
      Consequently, \\      
      $\min( \reach_F(s),\reach_F(t_x) ) \leq \mu_x(y)$. 
      Because $\reach_F(s)=0$ and $\reach_F(t_x) \leq \rho(xy)=0$, it follows that $\reach_F(t_x) \leq \mu_x(y)$. 
      Finally, since $t_x \in F$, 
      Claim ~\ref{claim:nex-mh} tells us that 
      $\reach_{F'}(t_x) < \reach_F(t_x)$. 
      Taken together, these two inequalities show that 
      $\mu_x(yb) = \reach_{F'}(t_x) < \reach_F(t_x) \leq \mu_x(y)$. 
      The last inequality follows since $s$ and $t_x$ are disjoint over $y$ and $\reach_F(s) = 0 = \rho(xy)$. 
      We conclude that $\mu_x(yb) \leq \mu_x(y) - 1$.

      \item[If $b = \H$ and $\mu_x(y) < 0$,] 
      % $F$ cannot contain a single maximum-reach tine and its
      we claim that $t_x \in F$. 
      To see why, note that as $t_x$ is $yb$-disjoint with $t_\rho$, 
      it must extend some $F$-tine $t$ that is $y$-disjoint with $t_\rho$. 
      However, as $\mu_x(y) < 0$, $t$ must have negative reach and hence cannot be extended into $t_x$; this is a contradiction. 
      Therefore, $t_x \in F$ and we can apply the argument in the ``$b = \h$'' case above 
      to conclude that $\mu_x(yb) \leq \mu_x(y) - 1$.

      \item[If $b = \H$ and $\mu_x(y) = 0$,] 
      then there are two alternatives depending on 
      whether $t_x$ is an extension. 
      If $t_x$ is not an extension, we can apply the argument in the ``$b = \h$'' case above and conclude that 
      $\mu_x(yb) \leq \mu_x(y) - 1 = -1$.
      On the other hand, if $t_x \not\in F$, 
      both $t_x$ and $t_\rho$ are extensions and, 
      by Fact~\ref{fact:reach-fork-ext-mh}, 
      $\max(\reach_{F'}(t_x), \reach_{F'}(t_\rho) ) \leq 0$. 
      In addition, Fact~\ref{fact:reach-fork-ext-mh} states that for all $t \in F$, 
      $\reach_{F'}(t) < \reach_{F}(t) \leq \rho(xy) = 0$. 
      We conclude that $\mu_x(yb) \leq 0$.

    \end{description}



  \paragraph{Case 3: $\rho(xy)>0$ and $\mu_x(y)\neq0$.}
    We wish to show that $\mu_x(yb) \leq \mu_x(y) - 1$ 
    or, equivalently, that $\mu_x(yb) < \mu_x(y)$. 
    % Note that by~\ref{eq:rho-recursive-mh}, 
    % $\rho(xy0) = \rho(xy) - 1 \geq 0$.
    We will break this case into two sub-cases. 
    \begin{description}[font=\normalfont\itshape\space]
      \item[If both $t_\rho, t_x \in F$,] 
      then $\mu_x(yb) = \reach_{F'}(t_x) < \reach_{F}(t_x) \leq \mu_x(y)$. 
      Here, the first inequality follows from Fact~\ref{fact:reach-fork-ext-mh} 
      and the second inequality follows from the fact that 
      $t_x, t_\rho$ is $y$-disjoint and 
      $\reach(t_x)$ is at most $\reach(t_\rho)$ by design.

      \item[Otherwise,] 
      at least one of $t_x, t_\rho$ arose from an extension. 
      Since $\reach_{F'}(t_x) \leq \reach_{F'}(t_\rho)$ by design, 
      it follows that $\reach_{F'}(t_x) \leq 0$ 
      as the reach of an extension is at most zero.
      If $\mu_x(y) > 0$ then we are done: $\mu_x(yb) \leq 0 < \mu_x(y)$.  
      On the other hand, suppose $\mu_x(y) < 0$. 
      Recall the tine $s$ mentioned before.
      As $t_x$ is $y$-disjoint with $s$ and 
      $\mu_x(y)$ is negative by assumption, 
      $\reach_F(t_x)$ is at most $\mu_x(y)$. 
      We conclude that 
      $
      \mu_x(yb) = \reach_{F'}(t_x) 
      < \reach_F(t_x) 
      \leq \mu_x(y)
      $ 
      where the inequality follows from Fact~\ref{fact:reach-fork-ext-mh}.


      % \item[If either $t_\rho \not \in F$ or $t_x \not \in F$.]
      % It must be true that $\reach_{F'}(t_x)\leq 0$, because either $t_x$ is the extension (and therefore has reach exactly 0) or $t_\rho$ is the extension and we have $\reach_{F'}(t_x)=\mu_x(y0)\leq\rho(xy0)=\reach_{F'}(t_\rho)=0$. Recall that we have assumed $\mu_x(y)\neq0$. If $\mu_x(y)>0,$ we are done: certainly $\mu_x(y0)\leq0<\mu_x(y)$. If, however, $\mu_x(y)<0$, there is more work to do. 
      % In this case, we claim that $t_x \in F$, i.e., $t_x$ did not arise from an extension. 
      % To see why, consider the following: if $t_x$ arose from extension, then there must be some $s \in F$ 
      % so that $s \Prefix t_x$ and $\reach_F(s) \geq 0$. Additionally, by our claim about non-extended tines, we see that 
      % $\reach_F(t_\rho)>\reach_{F'}(t_\rho) = \rho(xy0) \geq 0$. 
      % Therefore, 
      % $\mu_x(y) \geq \min\{\reach_F(t_\rho), \reach_F(s)\} \geq 0$, 
      % contradicting our assumption that $\mu_x(y) < 0$. 
      % Thus $t_x \in F$. 

      % The only remaining scenario is the one in which 
      % $\mu_x(y)<0$ and $t_\rho$ arises from an extension 
      % of some tine $s \in F, \reach_F(s) \geq 0$. 
      % In this scenario, $t_x$ cannot have been the extension 
      % (since there is only one). 
      % By Claim~\ref{claim:nex-mh}, 
      % $\reach_F(t_x) > \reach_{F'}(t_x)$. 
      % Using a now-familiar line of reasoning, note that 
      % the two tines
      % $t_x$ and $s$ are disjoint over $y$ 
      % and, therefore, 
      % $\mu_x(y) \geq \min\{\reach_F(s), \reach_F(t_x)\}$. 
      % Since, 
      % $\mu_x(y) < 0$ by assumption and $\reach_F(s) \geq 0$, 
      % it follows that 
      % $\mu_x(y) \geq \reach_F(t_x) > \reach_{F'}(t_x)=\mu_x(y0)$, as desired. \qedhere
    \end{description}
\end{proof}






\section{Proofs of Theorem~\ref{thm:relative-margin} and Theorem~\ref{thm:opt-adversary-canonical}}
\label{sec:relative-margin-thm-proof}\label{sec:opt-adversary-thm-proof}



\paragraph{Proof of Theorem~\ref{thm:relative-margin}.}
Let $w \in \{\h, \H, \A\}^*$. 
If $w = \varepsilon$ then, by Claim~\ref{claim:rho-mu-A}, 
$\rho(\varepsilon) = 0$. 
If $|w| \geq 1$,~\eqref{eq:rho-recursive-mh} is implied by 
the combination of 
Claim~\ref{claim:rho-mu-A},~\eqref{eq:rho-upperbound} and~\eqref{eq:rho-lowerbound}. 


Let $w = xy$ be an arbitrary decomposition. 
We proceed by induction on $|y|$. 
If $|y| = 0$ then 
Claim~\ref{claim:rho-mu-A} implies that $\mu_x(\varepsilon) = \rho(x)$. 
Otherwise,~\eqref{eq:mu-relative-recursive-mh} 
is implied by the combination of 
Claim~\ref{claim:rho-mu-A},~\eqref{eq:mu-upperbound} and~\eqref{eq:mu-lowerbound}.

% The ``additionally'' part of Theorem~\ref{thm:relative-margin} 
% is proven in Theorem~\ref{thm:opt-adversary-canonical}.
\hfill\qed


\paragraph{Proof of Theorem~\ref{thm:opt-adversary-canonical}.}
The proof is by induction on $|w|$. 
If $w$ is the empty string $\varepsilon$, 
the only fork $F \Fork \varepsilon$ is the trivial fork 
containing a single (honest) root vertex. 
By Claim~\ref{claim:rho-mu-A}, 
$F$ satisfies $\rho(\varepsilon) = 0$ 
and 
$\mu_\varepsilon(\varepsilon) = \rho(\varepsilon) = 0$.

Now, let $n$ be a non-negative integer and 
let $w$ be a characteristic string of length $n+1$. 
Assume that Theorem~\ref{thm:opt-adversary-canonical} 
holds for all characteristic strings of length $0, 1, \ldots, n$. 
% In particular, this means that the fork $F_{n}$ 
% built by $\Adversary^*$ for $w_1 \ldots w_{n}$ is canonical, 
% i.e., for every decomposition $w_1 \ldots w_{n} = xy$, 
% $\mu_x(F_{n}) = \mu_x(y)$.
Note that this assumption satisfies 
the premise in Proposition~\ref{prop:mu-lowerbound}. 
A combined application of 
Claim~\ref{claim:rho-mu-A}, Proposition~\ref{prop:mu-upperbound}, 
and Proposition~\ref{prop:mu-lowerbound} 
implies 
Theorem~\ref{thm:opt-adversary-canonical} 
for $|w| = n + 1$.
\hfill\qed









