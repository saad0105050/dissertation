
In this chapter, 
we build upon the exposition in Section~\ref{sec:model-settlement} 
and develop some important combinatorial notions. 
These notions help us quantify the ``power'' of the adversary 
in terms of causing a settlement violation. 
In particular, we show that the notion of ``relative margin'' is just as expressive 
as the Catalan slots for characterizing the UVP. 
An added advantage is that we can obtain a recurrence relation 
for relative margin as a function of the characteristic string; 
see Theorem~\ref{thm:relative-margin}. 
This leads to an algorithm which can calculate explicit error estimates by converting 
this recurrence relation into a Markov chain; see \Section~\ref{sec:exact-prob-multihonest}.
% The main results in this chapter are Theorem~\ref{} Theorem~\ref{thm:relative-margin}


\section{Closed forks, reach, and extensions}
\begin{definition}[Closed fork]
  A fork $F$ is \emph{closed} if every leaf is honest. For convenience, we say the trivial fork is closed.
\end{definition}
Closed forks have two nice properties that make them especially useful in reasoning about the view of honest parties.
First, 
all honest observers will select a unique longest tine from this fork 
(since all longest tines in a closed fork are honest, 
honest parties are aware of all previous honest blocks, 
they observe the longest chain rule, and they employ the same consistent tie-breaking rule).  
Second, 
% recalling our description of the
% settlement game, 
closed forks intuitively capture decision points for the adversary.
The adversary can potentially show many tines to many honest parties, 
but once an honest node has been placed on top of 
a tine, any adversarial blocks beneath it are part of the public record and are visible to all honest parties. 
For these
reasons, we will often find it easier to reason about closed forks than arbitrary forks. % (without loss of generality).

The next few definitions are the start of a general toolkit for reasoning about an adversary's capacity to build highly diverging paths in forks, based on the underlying characteristic string.
%current state of a fork.

%%%%Reach and margin
\begin{definition}[Gap, reserve, and reach]\label{def:gap-reserve-reach-mh}
For a closed fork $F \vdash w$ and its unique longest tine $\hat{t}$, we define the \emph{gap} of a tine $t$ to be $\gap(t)=\length(\hat{t})-\length(t)$.
Furthermore, we define the \emph{reserve} of $t$, denoted $\reserve(t)$, to be the number of adversarial indices in $w$ that appear after the terminating vertex of $t$. More precisely, if $v$ is the last vertex of $t$, then
\[
  \reserve(t)=|\{\ i \mid w_i=1 \ and \ i > \ell(v)\}|\,.
  \]
These quantities together define the \emph{reach} of a tine: $
\reach(t)=\reserve(t)-\gap(t)$.
\end{definition}

The notion of reach can be intuitively understood as a measure of
the resources available to our adversary in the settlement
game. Reserve tracks the number of slots in which the adversary has
the right to issue new blocks.  When reserve exceeds gap (or
equivalently, when reach is nonnegative), such a tine could be
extended---using a sequence of dishonest blocks---until it is as long
as the longest tine. Such a tine could be offered to an honest player
who would prefer it over, e.g., the current longest tine in the
fork. In contrast, a tine with negative reach is too far behind to be
directly useful to the adversary at that time.

\begin{definition}[Maximum reach]
For a closed fork $F\vdash w$, we define $\rho(F)$ to be the largest reach attained by any tine of $F$, i.e., 
\[
\rho(F)=\underset{t}\max \ \reach(t)\,.
\]
Note that $\rho(F)$ is never negative (as the longest tine of any fork always has reach at least 0). We overload this notation to denote the maximum reach over all forks for a given characteristic string: 
\[
\rho(w)=\underset{\substack{F\vdash w\\\text{$F$ closed}}}\max\big[\underset{t}\max \ \reach(t)\big]\,.
\]
\end{definition}

Reach of vertices is always non-increasing as we move down a tine. 
That is, if $B_1, B_2, \ldots$ are vertices on the same tine in the root-to-leaf order, then 
$\reach(B_i) \leq \reach(B_{i+1})$. 
The inequality is strict if $B_{i + 1}$ is honest. 
Consequently, the reach of an adversarial tine is no more than 
the reach of the last honest vertex in that tine. 
In any fork, the reach of a maximum-length tine is always non-negative. 
Hence, an honest tine with the maximum length over all honest tines 
will always have a non-negative reach. 
Thanks to the monotonicity of the honest-depth function $\hdepth(\cdot)$, 
if there are multiple honest tines 
having the (same) maximum length among all honest tines, 
they must have the same label. 
Therefore, if $h$ is the last honest slot in $w$ and 
$t$ a maximum-length honest tine with label $h$,  
then $\reach(t) \geq 0$. 



\paragraph{Non-negative reach, $\A$-heaviness, and viable adversarial extensions.}
Let $w \in \{\h, \H, \A\}^T$,   
$s \in [T + 1]$, and 
$F \Fork w_1 \ldots w_{s - 1}$ an arbitrary fork. 
Let $B \in F$ be an honest vertex 
and $t$ a maximum-length tine in $F$.
Consider the following statements: 
\begin{enumerate}[label=(\alph*)]
  \item \label{fact-reach-part:viable-adv-ext} $B$ has an adversarial extension viable at the onset of slot $s$.
  \item \label{fact-reach-part:nonneg-reach} $\reach_{F}(B)$ is non-negative.
  \item \label{fact-reach-part:Aheavy} The interval $I = [\ell(B) + 1, s - 1]$ is $\Aheavy$. 
  \item \label{fact-reach-part:conservative} $\length(t) = \#_\h(I) + \#_\H(I) + \length(B)$.     
\end{enumerate}

\begin{proposition}\label{prop:fork-structure-reach}
    ~\ref{fact-reach-part:viable-adv-ext} $\Longrightarrow$
    \ref{fact-reach-part:nonneg-reach} $\Longrightarrow$
    \ref{fact-reach-part:Aheavy}.
    In addition, if we assume~\ref{fact-reach-part:conservative}, then 
    ~\ref{fact-reach-part:Aheavy} $\Longrightarrow$ 
    ~\ref{fact-reach-part:nonneg-reach} $\Longrightarrow$
    ~\ref{fact-reach-part:viable-adv-ext}.
\end{proposition}
Proposition~\ref{prop:fork-structure-reach} can be seen as 
a refinement of Proposition~\ref{prop:fork-structure} 
when $F$ is a closed fork. 
\begin{proof}~
  \begin{description}[font=\normalfont\itshape\space]
    \item[\ref{fact-reach-part:viable-adv-ext} implies~\ref{fact-reach-part:nonneg-reach}.]
      An adversarial extension of $B$ 
      contains only adversarial vertices from $I$. 
      If this extension is viable at the onset of slot $s$, 
      $\#_\A(I)$ must be at least $\gap_{F}(B)$.
      Since $\reserve_{F}(B) = \#_\A(I)$, we have 
      $\reach_{F}(B) = \reserve_{F}(B) - \gap_{F}(B) \geq 0$.  
      
    \item[\ref{fact-reach-part:nonneg-reach} implies~\ref{fact-reach-part:Aheavy}.]
      By assumption, $\reach_{F}(B)  = \reserve_{F}(B) - \gap_{F}(B) \geq 0$.               
      $t$ contains at least $\#_\h(I) + \#_\H(I)$ vertices from 
      the interval $I$; 
      hence, $\gap_{F}(B) \geq \#_\h(I) + \#_\H(I)$. 
      Since $\reserve_{F}(B) = \#_\A(I)$, 
      it follows that $\#_\A(I) \geq \#_\h(I) + \#_\H(I)$. 

    \item[\ref{fact-reach-part:conservative} and~\ref{fact-reach-part:Aheavy} implies~\ref{fact-reach-part:nonneg-reach}.]
      Since $I$ is $\Aheavy$, 
      $\reserve_F(B) = \#_\A(I) \geq \#_\h(I) + \#_\H(I)$. 
      However, since~\ref{fact-reach-part:conservative} holds, 
      the latter quantity equals $\length(t) - \length(B) = \gap_F(B)$. 
      It follows that $\reach_F(B) = \reserve_F(B) - \gap_F(B) \geq 0$. 

    \item[\ref{fact-reach-part:conservative} and~\ref{fact-reach-part:nonneg-reach} implies~\ref{fact-reach-part:viable-adv-ext}.]
      $I$ contains at least $\gap_F(B)$ adversarial slots. 
      We can use these slots augment $B$ 
      into an adversarial tine $t'$ 
      of length at least $\length(t)$. 
      Thus $t'$ will be viable at the onset of slot $s$.
  \end{description}  
\end{proof}


Observe that for any characteristic string $x$, 
one can \emph{extend} (i.e., augment) a closed fork prefix $F \Fork x$ 
into a larger closed fork $F' \Fork x0$ so that $F \ForkPrefix F'$. 
A \emph{conservative extension} is a minimal extension in that 
it consumes the least amount of reserve (cf. Definition~\ref{def:gap-reserve-reach-mh}), 
leaving the remaining reserve to be used in future.
Extensions and, in particular, conservative extensions 
play a critical role in the exposition that follows. 

\begin{definition}[Extensions]\label{def:extension-mh}  
  Let $w \in \{\h, \H, \A\}^*$ be a characteristic string 
  and $F$ a closed fork for $w$. 
  Let $F'$ be a closed fork for $wb, b \in \{\h, \H\}$ 
  so that $F \ForkPrefix F'$. 
  We say that \emph{$F'$ is an extension of $F$} if 
  every honest vertex in $F'$ either belongs to $F$ or has label $|w| + 1$. 
  Let $\sigma \in F'$ be an honest vertex with $\ell(\sigma) = |w| + 1$ 
  and let $s$ be the longest honest prefix of $\sigma$. 
  (Necessarily, $s \in F$.)
  We say that \emph{$\sigma$ is an extension of $s$}. 
  The new tine $\sigma$ is a \emph{conservative extension} if 
  % $\height(F) + 1 = \max_{t \in S} \length(t)$.  
  $\height(F') = \height(F) + 1$.  
\end{definition} 
Since $F'$ is closed, all longest tines in $F'$ are honest and they have label $|w| + 1$.
Consider a longest honest tine $\sigma \in F'\setminus F$ and 
let $s \in F$ be the longest honest prefix of $\sigma$.
It follows that 
$\length(\sigma) 
\geq 1 + \height(F) 
= 1 + \length(s) + \gap_F(s)$ 
Therefore, the root-to-leaf path in $F^\prime$ 
that ends at $\sigma$ 
contains at least $\gap_F(s)$ adversarial vertices $u \in F'$ 
so that $\ell(u) \in [\ell(s) + 1, |w|]$ and 
$u \not \in F$. 
If $\sigma$ is a conservative extension, 
the number of such vertices is exactly $\gap_F(s)$. 


\begin{proposition}[Extensions and reach]\label{prop:reach-fork-ext-mh}
  Let $b \in \{\h, \H\}$. 
  Let $F \Fork w$ and $F^\prime \Fork wb$ be closed forks so that 
  $F \ForkPrefix F^\prime$ and 
  $F^\prime$ is obtained from $F$ via one or more extensions 
  $\sigma \in F^\prime, \ell(\sigma) = |w| + 1$.
  Then $\reach_{F^\prime}(t) \leq \reach_F(t) - 1$ for every $t \in F$. 
  If all these extensions are conservative, then 
  $\reach_{F^\prime}(t) = \reach_F(t) - 1$ for every $t \in F$. 
  Furthermore, a conservative extension $\sigma$ satisfies 
  $\reach_{F^\prime}(\sigma) = 0$.
\end{proposition}
The proposition above follows from the claims below.
\begin{claim}\label{claim:nex-mh}
  Let $b \in \{\h, \H\}$. 
  Consider a closed fork $F\vdash w$ and some closed fork $F'\vdash wb$ such that $F\fprefix F'$. 
  If $t \in F$ then 
  $\reach_{F'}(t)\leq \reach_{F}(t) - 1$. 
  The inequality becomes and equality 
  if $F'$ is obtained via 
  conservative extensions from $F$.
\end{claim}
\begin{proof}
  We know that $\reach_{F'}(t)=\reserve_{F'}(t)-\gap_{F'}(t).$ From $F$ to $F'$, the length of the longest tine increases by at least one, and the length of $t$ does not change. 
  It follows that $\gap_{F'}(t) \geq \gap_{F}(t) + 1$. 
  The inequality becomes an equality 
  if $F'$ is obtained from $F$ via only conservative extensions. 
  The reserve of $t$ does not change, because there are no new $\A$s in the characteristic string. Therefore, 
  $
    \reach_{F'}(t)
    =\reserve_{F'}(t)-\gap_{F'}(t)
    \leq \reserve_{F}(t)-\gap_{F}(t) - 1
    =\reach_{F}(t) - 1
  $. 
\end{proof}
\begin{claim}\label{claim:ex-mh}
  Conservative extensions have reach zero.
  % Let $b \in \{\h, \H\}$. 
  % Consider closed forks $F\vdash w, F'\vdash wb$ 
  % such that $F\fprefix F'$. 
  % If a tine $t$ of $F'$ is a conservative extension 
  % then $\reach_{F'}(t)=0$.
\end{claim}
\begin{proof}
  Let $b \in \{\h, \H\}$. 
  Consider closed forks $F\vdash w, F'\vdash wb$ 
  such that $F\fprefix F'$. 
  Let $t \in F'$ be a conservative extension. 
  This means $t$ is honest, $\ell(t) = |w| + 1$, 
  and 
  $t$ is a longest tine in $F'$. 
  The last statement implies $\gap_{F'}(t)=0$. 
  Since $\reserve_{F'}(t)=0$, it follows that 
  $\reach_{F'}(t)=\reserve_{F'}(t)-\gap_{F'}(t) = 0$. 
\end{proof}


\section{Relative margin}
\begin{definition}[The $\sim_x$ relations]
  For two tines $t_1$ and $t_2$ of a fork $F$, we write $t_1 \sim t_2$
  when $t_1$ and $t_2$ share an edge; otherwise we write
  $t_1 \nsim t_2$. We generalize this equivalence relation to reflect
  whether tines share an edge over a particular suffix of $w$: for
  $w = xy$ we define $t_1 \sim_x t_2$ if $t_1$ and $t_2$ share an edge
  that terminates at some node labeled with an index in $y$;
  otherwise, we write $t_1 \nsim_x t_2$ (observe that in this case the
  paths share no vertex labeled by a slot associated with $y$).  We
  sometimes call such pairs of tines \emph{disjoint} (or, if
  $t_1 \nsim_x t_2$ for a string $w = xy$, \emph{disjoint over
    $y$}). Note that $\sim$ and $\sim_\varepsilon$ are the same
  relation.
\end{definition}

\begin{definition}[Margin]\label{def:margin}
The \emph{margin} of a fork $F\vdash w$, denoted $\mu(F)$, is defined as 
\begin{equation}\label{eq:margin-absolute}
\mu(F)=\underset{t_1\nsim t_2}\max \bigl(\min\{\reach(t_1),\reach(t_2)\}\bigr)\,,
\end{equation}
where this maximum is extended over all pairs of disjoint tines of
$F$; thus margin reflects the ``second best'' reach obtained over all
disjoint tines. In order to study splits in the chain over particular portions of a
string, we generalize this to define a ``relative'' notion of margin:
If $w = xy$ for two strings $x$ and $y$ and, as above, $F \vdash w$,
we define
\[
  \mu_x(F)=\underset{t_1\nsim_x t_2}\max \bigl(\min\{\reach(t_1),\reach(t_2)\}\bigr)\,.
\]
Note that $\mu_\varepsilon(F) = \mu(F)$.

For convenience, we once again overload this notation to denote the
margin of a string. $\mu(w)$ refers to the maximum value of $\mu(F)$
over all possible closed forks $F$ for a characteristic string $w$:
\[
\mu(w)=\underset{\substack{F\vdash w,\\ \text{$F$ closed}}}\max \, \mu(F)\,.
\]
Likewise, if $w = xy$ for two strings $x$ and $y$ we define
\[
\mu_x(y)=\underset{\substack{F\vdash w,\\ \text{$F$ closed}}} \max \, \mu_x(F)\,.
\]
%(Cf.~\cite{KRDO17}, which defined and studied the ``absolute'' version
%$\mu(\cdot)$ of this quantity of~\eqref{eq:margin-absolute}.)
\end{definition}
Note that, at least informally, 
disjoint tines with large reach are of natural
interest to an adversary who wants to build an $x$-balanced fork, 
since such a fork contains two (partially disjoint) long tines.
It is easy to see that 
if $w = xx'y$ and 
$\mu_{xx'}(y)$ is negative then $\mu_x(x'y)$ is negative as well.

% \begin{fact}\label{fact:neg-margin}
%   Let $w$ be a characteristic string 
%   and let $w = xx'y$ be an arbitrary decomposition. 
%   If $\mu_{xx'}(y) < 0$ then $\mu_x(x'y) < 0$.
% \end{fact}


\section{A recurrence relation for relative margin}\label{sec:margin-recurrence}
The theorem below shows how to recursively compute $\mu_x(y)$ 
for a given decomposition $w = xy$.


\begin{theorem}\label{thm:relative-margin}
  Let $\varepsilon$ be the empty string 
  and $b \in \{\h, \H\}$. 
  Then $\rho(\varepsilon) = 0$ 
  and, for all nonempty strings $w \in \{\h, \H, \A\}^*$ 
  \begin{equation}
    \rho(w\A) = \rho(w) + 1\,, \qquad\text{and}\qquad
    \rho(wb) = \begin{cases} 0 & \text{if $\rho(w) = 0$,}\\
      \rho(w)-1 & \text{otherwise.}
    \end{cases}
    \label{eq:rho-recursive-mh}
  \end{equation}


  Furthermore, for any strings $x, y \in\{\h, \H, \A\}\text{\emph{*}}$,
  $\mu_x(\varepsilon) =\rho(x)$, 
  \begin{equation}
    \mu_x(y\A)= \mu_x(y)+1\,,\qquad\text{and}\qquad
    \mu_x(yb)= \begin{cases}
      0 & \text{if $\rho(xy) > \mu_x(y)=0$}\,, \\
      0 & \text{if $\rho(xy) = \mu_x(y) = 0$ and $b = \H$}\,, \\
      \mu_x(y)-1 & \text{otherwise.}
    \end{cases}
    \label{eq:mu-relative-recursive-mh}
  \end{equation}  
\end{theorem}

The proof of Theorem~\ref{thm:relative-margin} is given in \Section~\ref{sec:margin-proof-multihonest}. 
Let $w$ be a characteristic string and 
let $m, k \in \NN$ so that $m + k \leq |w|$. 
Let $x \Prefix w, |x| = m-1$ and $xy \PrefixEq w, |xy| \geq m + k$.
If the symbols in $w$ are independent and identically distributed, 
the recursive formulation in~\eqref{eq:mu-relative-recursive-mh} implies an algorithm --- which takes time and space $O(|w|^3)$ --- 
for computing the probability that $\mu_x(y) \geq 0$. 
But this is exactly the probability that slot $m$ is not $k$-settled, 
according to~\eqref{eq:settlement-uvp} 
and Lemma~\ref{lemma:uvp-margin} below. 
In \Section~\ref{sec:exact-prob-multihonest}, 
we describe this algorithm in more detail and 
compile some explicit values for this probability.


In \Section~\ref{sec:exact-prob-multihonest}, 
we describe an algorithm which explicitly computes 
consistency violation estimates using Theorem~\ref{thm:relative-margin}.


\section{Balanced forks, settlement violations, and relative margin}
%Informally, $t_1\sim_x t_2$ indicates that when we restrict our view of history to only blocks \emph{after}
%the prefix $x$, $t_1$ and $t_2$ share an edge (and thus agree on at least one block after that point).

A natural structure we can use to reason about settlement times 
(see Definition~\ref{def:settlement-mh}) 
is that of a ``balanced fork.''

\begin{definition}[Balanced fork]\label{def:balanced-fork} A
  fork $F$ is \emph{balanced} if it contains a pair of tines $t_1$ and
  $t_2$ for which $t_1\nsim t_2$ and
  $\length(t_1)=\length(t_2)=\height(F)$. We define a relative notion
  of balance as follows: a fork $F \vdash xy$ is \emph{$x$-balanced}
  if it contains a pair of tines $t_1$ and $t_2$ for which
  $t_1 \not\sim_x t_2$ and $\length(t_1) = \length(t_2) = \height(F)$.
\end{definition}

Thus, balanced forks contain two completely disjoint, maximum-length
tines, while $x$-balanced forks contain two maximum-length tines that
may share edges in $x$ but must be disjoint over the rest of the
string. 
See Figures~\ref{fig:balanced-mh} and~\ref{fig:x-balanced-mh} 
for examples of balanced forks.
\begin{figure}[ht]
  \centering
  \begin{minipage}{0.45\textwidth}\centering


    \begin{tikzpicture}[>=stealth', auto, semithick,
      honest/.style={circle,draw=black,thick,text=black,double,font=\small},
     malicious/.style={fill=gray!10,circle,draw=black,thick,text=black,font=\small}]
      \node at (0,-2) {$w =$};
    \node at (1,-2) {$\h$}; \node[honest] at (1,-.7) (b1) {$1$};
    \node at (2,-2) {$\A$}; \node[malicious] at (2,.7) (a1) {$2$};
    \node at (3,-2) {$\h$}; \node[honest] at (3,.7) (a2) {$3$};
    \node at (4,-2) {$\A$}; \node[malicious] at (4,-.7) (b2) {$4$};
    \node at (5,-2) {$\h$}; \node[honest] at (5,-.7) (b3) {$5$};
    \node at (6,-2) {$\A$}; \node[malicious] at (6,.7) (a3) {$6$};
      \node[honest] at (-1,0) (base) {$0$};
    \draw[thick,->] (base) to[bend left=10] (a1);
        \draw[thick,->] (a1) -- (a2);
        \draw[thick,->] (a2) -- (a3);
    \draw[thick,->] (base) to[bend right=10] (b1);
        \draw[thick,->] (b1) -- (b2);
        \draw[thick,->] (b2) -- (b3);
      \end{tikzpicture} 
    \caption{A balanced fork}
    \label{fig:balanced-mh}
  % \end{figure}
  \end{minipage}
  \hfill
  \begin{minipage}{0.45\textwidth}\centering

  % \begin{figure}[ht]
  %   \centering
    \begin{tikzpicture}[>=stealth', auto, semithick,
      honest/.style={circle,draw=black,thick,text=black,double,font=\small},
     malicious/.style={fill=gray!10,circle,draw=black,thick,text=black,font=\small}]
      \node at (0,-2) {$w =$};
    \node at (1,-2) {$\h$}; \node[honest] at (1,0) (ab1) {$1$};
    \node at (2,-2) {$\h$}; \node[honest] at (2,0) (ab2) {$2$};
    \node at (3,-2) {$\h$}; \node[honest] at (3,.7) (a3) {$3$};
    \node at (4,-2) {$\A$}; \node[malicious] at (4,-.7) (b3) {$4$};
    \node at (5,-2) {$\h$}; \node[honest] at (5,-.7) (b4) {$5$};
    \node at (6,-2) {$\A$}; \node[malicious] at (6,.7) (a4) {$6$};
      \node[honest] at (-1,0) (base) {$0$};
    \draw[thick,->] (base) -- (ab1);
    \draw[thick,->] (ab1) -- (ab2);
    \draw[thick,->] (ab2) to[bend left=10] (a3);
      \draw[thick,->] (a3) -- (a4);
    \draw[thick,->] (ab2) to[bend right=10] (b3);
      \draw[thick,->] (b3) -- (b4);
      \end{tikzpicture} 
    \caption{An $x$-balanced fork, where $x=\h\h$}
    \label{fig:x-balanced-mh}

  \end{minipage}
\end{figure}


% \paragraph{Balanced forks and settlement time.}
A fundamental question arising in typical blockchain settings is how
to determine \emph{settlement time}, the delay after which the
contents of a particular block of a blockchain can be considered
stable. The existence of a balanced fork is a precise indicator for
``settlement violations'' in this sense. Specifically, consider a
characteristic string $xy$ and a transaction appearing in a block
associated with the first slot of $y$ (that is, slot $|x| + 1$). One
clear violation of settlement at this point of the execution is the
existence of two chains---each of maximum length---which diverge
\emph{prior to $y$}; in particular, this indicates that there is an
$x$-balanced fork $F$ for $xy$. Let us record this observation below.\footnote{
  A balanced fork in~\cite{LinearConsistency} 
  had the property that 
  at least one maximum-length tine was adversarial. 
  But this is not true in our setting since we allow multiply honest slots.
}


\begin{observation}\label{obs:settlement-balanced-fork-mh}
  Let $s, k \in \NN$ be given and 
  let $w$ be a characteristic string. 
  Slot $s$ is not $k$-settled for the characteristic string $w$ 
  if 
  there exist a decomposition $w = xyz$, 
  where $|x| = s - 1$ and $|y| \geq k+1$, 
  and an $x$-balanced fork for $xy$. 
\end{observation}

In particular, to provide a rigorous $k$-slot settlement
guarantee---which is to say that the transaction can be considered
settled once $k$ slots have gone by---it suffices to show that with
overwhelming probability in choice of the characteristic string
determined by the leader election process (of a full execution of the
protocol), no such forks are possible. Specifically, if the protocol
runs for a total of $T$ time steps yielding the characteristics string
$w = xy$ (where $w \in \{0,1\}^T$ and the transaction of interest
appears in slot $|x| + 1$ as above) then it suffices to ensure that
there is no $x$-balanced fork for $x\hat{y}$, where $\hat{y}$ is an
arbitrary prefix of $y$ of length at least $k + 1$. 
% see Corollary~\ref{cor:main} in Section~\ref{sec:estimates}.  
Note that
for systems adopting the longest chain rule, this condition must
necessarily involve the \emph{entire future dynamics} of the
blockchain. We remark that our analysis below will in fact let us take
$T = \infty$.


% \paragraph{An alternative proof of Lemma~\ref{lemma:uvp-margin} via balanced forks.}
Let $w$ be a characteristic string. 
Writing $w = xy$, 
consider any tine-pair $(t_x, t_\rho)$ in a fork $F \Fork w$ so that 
$\reach_F(t_\rho) = \rho(F)$ and $t_x$ is $y$-disjoint with $t_\rho$.
Observe that if $\mu_x(y) < 0$ then $\reach_F(t_x) < 0$. 

% In a similar Fact 1 in~\cite{LinearConsistencySODA}, 
\begin{fact}\label{fact:margin-balance}
  Let $xy \in \{\h, \H, \A\}^*$ be a characteristic string. 
  There is no 
  $x$-balanced fork for $xy$ 
  if and only if 
  $\mu_x(y) < 0$.
\end{fact}
\begin{proof}[Proof sketch.]  
  If a fork $F \Fork xy$ 
  satisfies $\mu_x(F) \geq 0$, 
  it contains two $y$-disjoint tines $t_1, t_2$, 
  each with a non-negative reach, 
  so that $\min(\reach(t_1), \reach(t_2)) = \mu_x(F)$. 
  As $\reserve(t_i) \geq \gap(t_i)$ for $i \in \{1,2\}$, 
  we can extend these tines using only new adversarial vertices 
  so that both these extensions 
  have the maximum length 
  in the augmented fork. 
  Thus the augmented fork is $x$-balanced.

  On the other hand, if a fork $F \Fork xy$ is $x$-balanced, 
  there must be two $y$-disjoint 
  maximum-length tines $t_1, t_2 \in F$. 
  As the gap of a maximum-length tine is zero, 
  we must have $\reach(t_i) = \reserve(t_i) \geq 0$ 
  for $i \in \{1, 2\}$. 
  It follows that $\mu_x(y) \geq \mu_x(F) \geq \min_i \reach(t_i) \geq 0$.
\end{proof}

% Let us focus on proving Lemma~\ref{lemma:uvp-margin} 
% using the above fact. 
% Let $x\Prefix w$. 
% It is easy to see that 
% if the slot $s = |x| + 1$ has the UVP 
% then there can be no $x$-balanced fork for $xx'$ 
% where $xx' \PrefixEq w$;
% consequently, $\mu_x(x') < 0$. 
% On the other hand, if $s$ does not have the UVP 
% then there must be a prefix $xx' \PrefixEq w$ 
% and a fork $F \Fork xx'$ 
% containing two maximum-length (and hence viable) 
% tines $t_1, t_2$ 
% diverging prior to slot $s$. 
% Hence $F$ is $x$-balanced.
% \hfill\qed


\section{Expressing UVP in terms of relative margin}
Let $w$ be a characteristic string. 
Recall that in Theorem~\ref{thm:unique-honest}, 
we showed that whether a slot has the UVP in $w$ --- a 
structural property of the forks for $w$ --- is 
characterized by the ``Catalan-ness'' of the said slot. 
Below, we show that relative margin 
has the same expressive power 
as the Catalan slots 
in terms of characterizing the UVP. 


  
% \begin{definition}[Margin-critical slot]\label{def:margin-critical-slot}
%     Let $w \in \{\h,\H,\A\}^T$ be a characteristic string, 
%     $s \in [T]$ be a slot in $w$, 
%     and $x$ be a prefix of $w$ so that $|x| = s - 1$. 
%     Slot $s$ is called \emph{margin-critical} 
%     if, 
%     for all decompositions 
%     $w = xyz$ so that $|y| \geq 1$ and $|z| \geq 0$, 
%     we have $\mu_{x}(y) < 0$.
%     % for all decompositions $w = xyz$ so that $|x| < s, |xy| \geq s$.    
% \end{definition}
% Since $\mu_x(\varepsilon) \geq 0$ and $|y| \geq 1$ in the above definition, 
% it follows that a margin-critical slot (i.e., the first slot in $y$) must be honest.




% Below, we show that a margin-critical honest slot has the bottleneck property as well. 
\begin{lemma}\label{lemma:uvp-margin}
  Let $T \in \NN, w \in \{\h, \H, \A\}^T$, and 
  $s \in [T]$ so that $w_s = \h$. 
  Let $x = w_1 \ldots w_{s-1}$.
  Slot $s$ has the UVP in $w$ 
  if and only if 
  for every prefix $xy \PrefixEq w$, 
  $\mu_x(y) < 0$. 
  % A a uniquely honest slot is margin-critical 
  % if and only if 
  % it has the UVP. 
\end{lemma}


% \begin{corollary}\label{coro:catalan-margin-critical}
%   Let $w \in \{\h, \H, \A\}^T$ be a characteristic string 
%   and let $x \PrefixEq$ 
%   A uniquely honest slot in $w$ 
%   is Catalan 
%   if and only if 
%   % it is margin-critical. 
%   for every prefix $xy \PrefixEq w$, 
%   $\mu_x(y) < 0$. 
% \end{corollary}

\begin{proof}~
  % Recall from Theorem~\ref{thm:unique-honest} 
  % that a uniquely honest slot is Catalan in $w$ 
  % if and only if 
  % it has the UVP in $w$. 

  \begin{description}[font=\normalfont\itshape\space]
    \item[The $\Longleftarrow$ direction.]
      Suppose that 
      for every prefix $xy \PrefixEq w$ where $|y| \geq 1$, 
      we have $\mu_x(y) < 0$. 
      We wish to show that $s$ has the UVP in $w$.

      Let $F$ be any fork for $xy$ 
      and let 
      $t \in F, \ell(t) \leq s - 1$ be an honest tine. 
      Since it is disjoint with any tine in $F$ over the suffix $y$, 
      $\reach(t) < 0$ and, by Proposition~\ref{prop:fork-structure-reach}, 
      $t$ does not have an adversarial extension $t' \in F, t \Prefix t'$ that is 
      viable at the onset of slot $|xy| + 1$. 
      Therefore, if a tine in $F$ 
      is viable at the onset of slot $|xy| + 1$, 
      it must contain an honest vertex with label at least $s$. 
      However, since an honest vertex builds only on top of a viable tine, 
      it follows that any viable tine must contain 
      the unique honest vertex with label $s$.

    \item[The $\Longrightarrow$ direction.]
      Suppose $s$ has the UVP in $w$.
      Let $k \in [s, T]$ be an integer and 
      write $w = xyz$ with $|xy| = k$. 
      (Note that $y_1 = w_s$.)
      We wish to show that $\mu_x(y) < 0$.

      Let $F$ be any fork for $xy$.
      % all tines in $F$ viable at the onset of 
      % % slot $|xy| + 1$ 
      % slot $|xy| + 1$ 
      % have, as their common prefix, 
      % {\color{blue}the unique honest tine} $\tau, \ell(\tau) = s$. 
      Since slot $s$ belongs to $y$, 
      $F$ cannot contain two tines 
      such that 
      \begin{enumerate*}[label=(\roman*)]
        \item both tines are viable at the onset of slot $|xy| + 1$ 
        and, at the same time, 
        \item disjoint over the length of $y$ 
        since they must contain the unique vertex with label $s$. 
      \end{enumerate*}
      % As any $x$-balanced fork for $xy$ requires two maximally long tines 
      % that are disjoint over $y$, 
      In particular, 
      $F$ cannot be $x$-balanced. 
      As $F$ was an arbitrary fork for $xy$, 
      no fork for $xy$ can be $x$-balanced for our choice of $k$.
      We use Fact~\ref{fact:margin-balance} 
      to conclude that 
      $\mu_x(y)$ must be negative.
      % Since our $k \in [s, T]$ is arbitrary, 
      % the above conclusion applies to 
      % all decompositions $w = xyz$ where $|x| = s - 1$ and $|xy| \geq s$.
      % Therefore, slot $s$ is margin-critical in $w$.

  \end{description}
\end{proof}



