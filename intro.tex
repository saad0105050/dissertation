Blockchain technology is a distributed computing primitive 
for implementing a decentralized, immutable ledger; 
such a ledger has important applications ranging 
from cryptocurrency to governance, to industrial supply chain management, 
to power networks, and so on.

Recently, the Proof-of-Stake (PoS) blockchain technology 
has emerged as an alternative 
to the Proof-of-Work (PoW) technology. 
While the latter enjoys wide popularity, 
it consumes a vast amount of wasteful energy as well. 
The case in point is Bitcoin: its global energy consumption in 2019 
is on par with that of Israel and Ireland.
It is, therefore, 
worthwhile to investigate 
whether the PoS technology can be made as functional and secure 
as the PoS technology.

In this dissertation, 
we study 
the security properties of 
Nakamoto-style PoS blockchain protocols. 
Not only have these protocols been deployed  
by leading cryptocurrency companies, 
but they have also garnered a keen interest from 
cryptography, algorithms, and distributed computing communities. 


A distributed, immutable ledger must provide 
\emph{consistency} and \emph{bias resistance}. 
Here, consistency means that the records written in the ledger cannot be erased. 
The privilege of writing a record in the ledger is 
a rare commodity by design; 
a user becomes an eligible writer by winning a verifiable lottery. 
However, a malicious coalition of users may conspire to 
undermine the consistency of the ledger once they become writers. 
They can also conspire to bias the lottery process in their favor 
so that they are more likely to become writers in the future. 

The aim of this research is to understand how
Nakamoto-style PoS protocols can achieve both consistency and bias-resistance 
while facing formidable attacks on its integrity.
We approach this problem by developing a simple yet powerful modeling and analysis framework.
The contributions from our research have 
improved the state-of-the-art security guarantees for these protocols. 
As a consequence, protocol designers can select 
strong security parameters---orders of magnitude stronger than 
what previous research had suggested---without
making any dent to the performance. 

% If all honest scribes attempt to write a given data into the ledger then, within a stipulated period of time, 
% it indeed gets written; this is called \emph{liveness}. 
% If something is ever written in the ledger, it will remain there (i.e., become ``settled'') with overwhelming probability; 
% this is \emph{persistence}. 

% blockchains have many important theoretical and practical applications. 


% There are many important theoretical and practical applications of blockchains 
% in decentralized computing, including cryptocurrencies such as Bitcoin.
% Like any distributed computing paradigm, 
% a central goal of blockchains is to provide liveness and persistence 
% while simultaneously achieving 
% performance (e.g., how long to settle?), 
% scale (e.g., how many users?), and 
% security (e.g., how likely is an error?). 


\section{Background}\label{sec:background-intro}
% \InlineCases{
%   \item PoW \item PoS \item Consistency and Settlement
%   \item Epochs, slots
%   \item Leader election, honest majority
%   \item Probabilistic statements
%   \item Consistency problem; double spending attack
%   \item Informal k-CP
%   \item Importance of the consistency error bound

%   \item Abstraction
%   \item Honest majority assumptions

%   \item Beacons, nonce, aggregate
%   \item Grinding problem, grinding power
%   \item Ouroboros MPC, Praos beacon
% }


The aim of this dissertation 
is to develop new understanding 
of the consistency and bias-resistance properties of 
Nakamoto-style PoS blockchains. 
Below, we introduce some important related concepts.




\Paragraph{Immutable ledgers.}
The term \emph{blockchains} refers to a family of protocols and algorithms 
in distributed computing 
that can be used to implement an ``immutable ledger'' 
which maintains an ordered list of records; 
this is why it is called a ``chain'' and a new record is called a ``block.''
Users of this ledger can reach agreement without needing to trust each other. 
The holy grail of blockchain algorithms is to achieve a good balance between 
the performance, security, and the scale of deployment. 
Of these, the trade-off between the performance and security 
is \emph{the} most important issue for both protocol designers and the users.

An immutable ledger must guarantee both \emph{liveness} and \emph{persistence}.
If all honest scribes attempt to write a given data into the ledger then, within a stipulated period of time, 
it indeed gets written; this is called liveness. 
If something is ever written in the ledger, it will remain there (i.e., become ``settled'') with overwhelming probability; 
this is persistence. 

Record settlement takes place in a varying speed in 
different families of blockchain protocols.
In the \emph{eventual consensus} model, 
the record persistence happens over time: 
as time goes by and more and more records get written in the ledger, 
the earlier records become less and less likely to be erased. 

\Paragraph{The leader election lottery.}
Since blockchain algorithms make it unnecessary 
for the participants to trust one another, 
it relies heavily on how one can prove any claims. 
For example, if a user attempt to write a record on the ledger, 
he should supply a proof that he has the privilege to write. 
These privileges are, by design, a rare commodity. 
In fact, 
at every round, 
every participant performs a lottery to determine 
whether he has this authority to write a record at any; 
if he wins, he becomes a \emph{leader} and he writes in the ledger. 


Depending on whether a dishonest leader has been elected, 
we can qualitatively describe a round as either honest or dishonest. 
Thus, we can speak of ``honest rounds'' and ``dishonest rounds'' 
where, in an honest round, the leaders adhere to the protocol but 
in a dishonest round, the leaders may arbitrarily deviate from the protocol. 
Furthermore, an honest round can be \emph{uniquely honest} if it has a single leader, 
or \emph{multihonest} if it features multiple leaders.


\Paragraph{Proof-of-Work vs. Proof-of-Stake lotteries.}
In blockchains in the \emph{Proof-of-Work} (PoW) paradigm, 
the lottery is computation-intensive: 
the winning probability is proportional to the computational power of the user. 
It not surprising, therefore, that a large PoW network (such as Bitcoin) 
consumes a gargantuan amount of energy.

In contrast, the \emph{Proof-of-Stake} (PoS) is a paradigm where 
the lottery does not incur the steep cost of heavy computation. 
Instead, the winning probability is proportional to the \emph{relative stake} of the user; 
in PoS, each user is assigned a number, called his ``stake,'' 
which acts as proxy of how much he has invested in the system. 
An important parameter in PoS security is the relative stake held by the dishonest players; 
we denote this adversarial relative stake by $\alpha$.

If PoS blockchains can provide the same (or similar) level of 
functionality, security, performance, and scale as the PoW counterparts, 
PoS could be a viable alternative to PoW blockchains 
and may emerge as the right technology for an energy-efficient future.

% But there is a difficulty: 
% in the PoW model, 
% a lottery winner can write at most one record 
% but 
% in the PoS model, a dishonest lottery winner may write 
% multiple records with impunity. 


\Paragraph{Nakamoto-style PoS and the consistency problem.}
% This comes in contrast to the ``exact consensus'' model 
% where a record achieves persistence as soon as it is written. 
This dissertation focuses on \emph{Nakamoto-style PoS}; 
these are PoS blockchains in the eventual consensus model where the participants 
always adopt a maximum-length blockchain from their local views. 
Examples of Nakamoto-style PoS are Ouroboros~\cite{Ouroboros}, Praos~\cite{Praos}, 
Genesis~\cite{Genesis}, SnowWhite~\cite{SnowWhite}, 
and Sleepy Consensus~\cite{Sleepy}.
However, the honest players face some obstacles toward reaching consensus: 
dishonest players are \emph{Byzantine}, i.e., 
they can arbitrarily deviate from the protocol, 
often acting in concert, 
showing multiple longest chains to the honest players;
and honest players may experience network errors. 
If two honest players disagree about a record that was 
written more than $k$ rounds in the past;
this is called a \emph{$k$-consistency violation}. 
This $k$ is called the \emph{settlement delay parameter} of the protocol
and it acts as the security parameter as well.
Since consistency is central to an eventual consensus protocol, 
any improvements in regards to the consistency implies 
improvements in many other areas of the protocol as well.


The lottery in the leader election implies that 
whether a leader is honest (or dishonest) is a probabilistic event, 
and the sequence of leaders during protocol execution 
influences the probability of a $k$-consistency violation.
The heart of a PoS security analysis is to argue that $k$-consistency violations are rare; 
concretely, this means the bad probability should decrease as we increase $k$. 



\Paragraph{Bias-resistant randomness beacons and the grinding problem.}
Conducting the leader election in an unbiased and verifiable way 
requires that the players have access to a common random string. 
However, as more and more elections are made, 
this string becomes less and less random, possibly introducing bias in the election. 
To mitigate this behavior, 
new common randomness is introduced in regular intervals; 
the process is called the \emph{randomness beacon}.

Specifically, the blockchain protocol execution is punctuated into \emph{epochs} of length $O(k)$; 
each epoch constitutes a predetermined number of rounds 
and, during an epoch, 
the players take part in a \emph{collective coin-flipping protocol} 
to compute the random string for the next epoch; 
this string is called the \emph{beacon output} for this epoch.
However, adversarial leaders can control (to some extent) 
which blockchain prefix becomes settled within an epoch and, 
if the beacon output depends on this prefix,  
they exert some control over the beacon output as well. 
As this output is used in the leader elections in the next epoch, 
a bias in the beacon output inflates the chance of a consistency violation. 
This is the so-called \emph{grinding attack} on the blockchain; 
the amount of inflation in the error probability is called the \emph{grinding power}.


\Paragraph{The interplay between consistency and bias-resistance.}
Although we should take a large $k$ to 
reduce the likelihood of a $k$-consistency error, but 
we cannot do so unilaterally. 
Increasing $k$ implies that the records 
would take longer to be declared ``settled'' 
and hence, it may not always be practical.
However, increasing $k$ 
increases the epoch length as well, 
executing the beacon protocol over longer periods of time and 
thereby increasing the strength of a grinding attack. 

Therefore, 
if we want to make the Nakamoto-style PoS blockchains to be 
both secure and fast, 
we need a tight bound on the $k$-consistency error probability
as well as 
a tight bound on the grinding power (as a function of $k$). 
Moreover, 
for the Nakamoto-style PoS technology  
to become a viable alternative to the PoW technology, 
we want these security bounds to be comparable (at least asymptotically) 
with the bounds in the PoW setting.


% \Paragraph{The longest-chain rule.}
% We study the behavior of the elementary \emph{longest-chain rule}
% algorithm, carried out by a collection of participants. 
% \begin{itemize}
%   \item In each round,
%   each participant collects all valid blockchains from the network; if a
%   participant is a leader in the round, he adds a block to the longest
%   chain and broadcasts the result.
% \end{itemize}




% A prevalent algorithm for eventual consensus blockchains 
% (in both PoW and PoS) is \emph{the longest-chain rule}.







\section{Research focus}\label{sec:intro-focus}
We focus on the following aspects of the consistency and grinding problems in 
Nakamoto-style PoS. 

\Paragraph{Consistency error bounds in PoS vs. PoW.}
% In the previous section, 
% we have pointed out how important a consistency bound is. 
% Consider the $\kCP$ property.
% The first ever PoS consistency bounds, 
% put forth by~\citet{Ouroboros} and~\citet{SnowWhite}, 
% had the form $e^{-\Omega(\sqrt{k})}$. \footnote{
%   $\Omega(g(k))$ is the class of functions $f(k)$ 
%   which, loosely speaking, 
%   ``stay above $g(k)$'' for large inputs $k$.
%   Similarly, $O(g(k))$ is the class of functions $f(k)$ 
%   which, loosely speaking, 
%   ``stay below $g(k)$'' for large inputs $k$.
% }
% However, the $\kCP$ error bound in the PoW setting is $e^{-\Omega(k)}$. 
Suppose that we are willing to tolerate a fixed amount of consistency error. 
Suppose, in addition, that the corresponding settlement delay Nakamoto-style PoW is $k$.
The state-of-the-art bounds for the Nakamoto-style PoS, however, 
suggests that the settlement delay be $k^2$. 
It was not sure whether this quadratic gap (i.e., $k^2$ vs. $k$)
is due to an inherent drawback of the PoS paradigm, 
or whether it was an artifact of the analysis. 
Be what it may, 
it was an urgent open question to close this gap. 


\Paragraph{Leader election lottery.}
The security guarantees critically rely on the 
statistical properties of the leader election process. 
The existing PoS security analyses put a penalty on multihonest slots although, intuitively, 
these should be treated as good events for the analysis. 
In reality, a multihonest slot is treated as a neutral event~\cite{SnowWhite,Sleepy} 
or worse, as an adversarial event~\cite{Ouroboros,Praos,Genesis}. 
Such a conservative modeling approach weakens the consistency error bounds 
by discounting the contribution from the honest leaders. 

Multihonest slots are a frequent outcome 
in a broad swath of practical leader election schemes as well. 
The above modeling approach precludes the PoS blockchains 
from using this important class of leader election schemes. 
In particular, under the existing analyses, 
we cannot use a leader election scheme 
where a uniquely honest slot is less likely than an adversarial slot.


\Paragraph{Grinding problem in existing protocols.}
Many PoS protocols, such as~\cite{Ouroboros,Etherium}, 
avoid the grinding problem 
by employing computation- and time-intensive 
beacon protocols that may involve specialized hardware as well. 
Some recent PoS protocols~\cite{Praos,SnowWhite}  
have instead opted for simple beacon protocols, 
conceding some grinding power to the adversary.
These security guarantees assume that this grinding power be ``bounded'' 
without putting forth an analysis of the grinding power.

There are two problems with this approach.
First, since there is no analysis of the grinding power in the literature, 
we do not understand how it grows. 
A large grinding power can seriously weaken the consistency error bound and 
send repercussions in every part of the protocol design. 
Second, 
assuming a bound on the grinding power is similar, in spirit, to 
assuming a bound on the hashing power of the adversary. 
However, putting this PoW-style hashing-power assumption 
in the middle of the PoS security guarantee 
is deeply problematic.

Finally, the literature does not contain an analysis of 
any other simple beacon for Nakamoto-style PoS. 
It is important to fill this gap 
since a simple, more secure beacon 
would readily improve the security guarantees of existing PoS protocols.




% \Paragraph{Simple, secure beacons.} 
% Recall that Ouroboros precludes grinding 
% by employing a multi-party protocol which relies on heavy cryptography. 
% In contrast, Praos and SnowWhite use simple beacon protocols 
% but must withstand potential grinding. 
% Algorand~\cite{Algorand} uses a non-MPC beacon 
% but it is not an eventual consensus protocol 
% (it is an expected constant-time BFT protocol).









\section{Research objectives}\label{sec:intro-objectives}
The observations above can be congealed into the following objectives:

\newcounter{ConsistencyObjectives}
\begin{description}[font=\normalfont\itshape\space]
  \item[Consistency Problem:]~
  \begin{enumerate}
    \item Derive tight asymptotic consistency bounds for PoS blockchains, 
    treating multihonest slots as good events.

    \setcounter{ConsistencyObjectives}{\value{enumi}}
  \end{enumerate}

  \item[Grinding Problem:]~
  \begin{enumerate}
    \setcounter{enumi}{\value{ConsistencyObjectives}}

    \item Quantify the grinding power in the beacon in Praos and SnowWhite 
    as a function of the consistency parameter $k$ and the adversarial stake $\alpha$.
    
    \item Propose and analyze new, simple, secure beacons.
  \end{enumerate}
\end{description}





\section{The value of this research}
\InlineCases{
  \item Fast settlements
  \item Reduce wasted efforts
  \item Explicit probabilities
  \item Cardano
  \item Quantify grinding in deployed protocols
  \item Trade-off between consistency and grinding
  \item New simple beacons
}


This research adds some compelling value to the extant scientific knowledge. 

For the first time in the literature, 
this work shows that 
PoS blockchains can achieve the same asymptotic dependence on $k$ 
as in the PoW setting; namely, we improve the state-of-the-art bound 
from $e^{-\Omega(\sqrt{k})}$ to $e^{-\Omega(k)}$~\cite{LinearConsistencySODA}. 
As discussed before, this immediately implies that 
PoS blockchains settle much faster than what we thought before. 
Thus, deployed protocols can safely select stronger security parameters. 
We also provide algorithms and software tools 
for practitioners to explicitly compute consistency error probabilities. 
A leading PoS cryptocurrency technology company, \emph{Cardano}~\cite{Cardano}, 
is actively using the results and recommendation from this dissertation.

In addition, our results in the ``simple honest stake majority'' setting~\cite{Multihonest}. 
This readily improves the guarantees of several existing analyses, 
namely Praos, Genesis, and SnowWhite. 
These results allow for leader election schemes 
for which a uniquely honest slot, on average, is less likely than 
an adversarial slot. 
No previous analysis can handle this parameter regime.
We also introduce new combinatorial notions which 
greatly simplifies consistency proofs.
Moreover, this work establishes connections between 
combinatorial notions and ideas from two 
different schools of PoS analysis.

Shifting focus to PoS beacons, 
our analysis of the Praos beacon is a first in the literature.
Importantly, we characterize the growth of the grinding power 
by studying its first two moments. 
Our results show that when the adversarial stake ratio is below $9\%$, 
this grinding power is, in fact, $\Poly(k, \alpha)$. 
On the flip side, this means the quality of the beacon degrades fast 
as the adversarial stake ratio increases beyond $9\%$. 

Thus, we need more secure yet simple PoS beacons. 
The current work delivers in this front by 
analyzing two new beacons. 
The first one works with a Praos-like private leader election; 
it is secure for $\alpha \leq 0.41$. 
The second one works with Ouroboros-like public leader election 
and it is secure for as long as $\alpha \in (0, 1/2)$. 


\Paragraph{Some insights and highlights from this dissertation.}
\begin{enumerate}
  \item Longest-chain PoS protocols have the same asymptotic 
  consistency error bounds, namely $e^{-\Omega(k)}$. 
  The constants hidden by the $\Omega$ notation differs, though: 
  let $\epsilon \in (0,1)$ so that 
  the adversarial stake is $\alpha = (1-\epsilon)/2$. 
  Then longest-chain PoW incurs a $\kCP$ violation with probability $e^{-\Omega(\epsilon^2)k}$, 
  whereas the error probability for longest-chain PoS is $e^{-\Omega(\epsilon^3)k}$.

  \item There is a program to explicitly compute these probabilities.

  \item Longest-chain PoS can operate securely as long as on average, an honest slot 
  (i.e., the said slot is associated with only honest leaders) 
  is more likely than an adversarial slot.

  \item For longest-chain PoS blockchains, 
  there is an optimal adversarial strategy 
  which can produce an execution 
  where all slots are \emph{maximally unsettled}, meaning that 
  if it is possible to unsettle a slot -- any slot -- 
  this adversary \emph{will} unsettle this slot. 
  We prove our security against this adversary.

  \item If we are willing to adopt a \emph{consistent} 
  longest-chain tie-breaking rule (see Section~\ref{sec:lcr-model}), 
  we can allow $\Pr[\text{uniquely honest slot}] = 0$ 
  and still remain secure.

  \item The consistency guarantee for Praos is valid 
  as long as the adversarial stake $\alpha < 0.0955$; 
  in that regime, the grinding power is $\Poly(k)$ 
  where $k$ comes from the $\kCP$ property of Praos. 
  Beyond that, the grinding power of the adversary grows exponentially in $k$.

  \item There are simple beacon protocols 
  whose grinding power grows at a much slower rate than that of Praos.
\end{enumerate}




\section{Outline of this dissertation}
To do







