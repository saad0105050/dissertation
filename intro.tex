Let us start by deconstructing the title of this work, 
\emph{Security of Proof-of-Stake Blockchains.}

The term \emph{blockchains} refers to a family of technology, systems, and algorithms 
in distributed computing 
that enable 
many participants to reach agreement without needing to trust each other. 
As an algorithmic black-box, blockchains can be used to implement an immutable ledger: 
If all honest scribes attempt to write a given data into the ledger then, within a stipulated period of time, 
it indeed gets written; this is called \emph{liveness}. 
If something is ever written in the ledger, it will remain there (i.e., become ``settled'') with overwhelming probability; 
this is \emph{persistence}. 
There are many important theoretical and practical applications of blockchains 
in decentralized computing, including cryptocurrencies such as Bitcoin.
Like any distributed computing paradigm, 
a central goal of blockchains is to provide liveness and persistence 
while simultaneously achieving 
performance (e.g., how long to settle?), 
scale (e.g., how many users?), and 
security (e.g., how likely is an error?). 

Since blockchain algorithms make it unnecessary 
for the participants to trust one another, 
it relies heavily on how one can prove any claims. 
For example, if I attempt to write a record on the ledger, 
I should supply a proof that I have the privilege to write. 
These privileges are, by design, a rare commodity. 
In fact, the participants continually perform a lottery to determine 
who has this authority. 
In blockchains in the \emph{Proof-of-Work} (PoW) paradigm, 
the participants need to 
perform heavy computation to participate in the lottery. 
Thus, a PoW network, as a whole, consumes 
a gargantuan amount of energy.

In contrast, the \emph{Proof-of-Stake} (PoS) is a paradigm where 
the lottery does not incur the steep cost of heavy computation. 
If we can prove that the PoS blockchains can provide the same (or similar) level of 
functionality, security, performance, and scale as the PoW counterparts, 
PoS could be a viable alternative to PoW blockchains 
and may emerge as the right technology for an energy-efficient future.
% But there is a difficulty: 
% in the PoW model, 
% a lottery winner can write at most one record 
% but 
% in the PoS model, a dishonest lottery winner may write 
% multiple records with impunity. 

In this dissertation, 
we delve deep into the trade-off between the performance and the security of PoS blockchains 
in the \emph{eventual consensus} model. 
In this model, 
the persistence happens over time: 
as time goes by and more and more records get written in the ledger, 
the earlier records become less and less likely to be erased. 
This comes in contrast to the ``exact consensus'' model 
where a record achieves persistence as soon as it is written. 
Examples of eventual consensus PoS protocols include Ouroboros, Praos, Genesis, SnowWhite, 
and Sleepy Consensus.






\section{Background}
To do








\section{Research focus}
To do





\section{Research objectives}
To do




\section{The value of this research}
To do




\section{Contributions}
To do



\section{Outline of this \Dissertation}
To do




