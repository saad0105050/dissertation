%=======================

\subsection{Option sets of size at most two\texorpdfstring{; the $(\ell, n, \alpha)$-Bernoulli game}{}}

    Let us consider a uniform XOR target game where an option set 
    contains at most two elements. 
    From Definition~\ref{def:xor-game-explicit}, 
    we conclude that in this case, 
    an option set is either $\{\star\}$ 
    or $\{0^\kappa, s\}$ where $s$ is uniform in $\{0,1\}^\kappa$.


    \begin{definition}[$(\ell, n, \alpha)$-Bernoulli game]
        \label{def:static-game}\label{def:Bernoulli-game}
        Let $\alpha \in (0, 1/2)$. 
        Let $\Bernoulli(\alpha)$ denote the distribution of a Bernoulli random variable 
        $B \in \{0,1\}$ where $\Pr[B = 1] = \alpha$.
        An $(\ell, n, \Bernoulli(\alpha))$-game 
        is called the \emph{$(\ell, n, \alpha)$-Bernoulli game}.
    \end{definition}
    The following is an immediate corollary from Theorem~\ref{thm:xor-game-lookahead}.

    \begin{theorem}\label{thm:xor-game-playornot}
        Let $\alpha \in (0, 1/2)$ and $\lambda \in (0, \log_2(1/\alpha)]$. 
        Let $U$ be an $(\ell, n, \alpha)$-Bernoulli game. 
        Let $g$ be the grinding power of $U$. 
        Let $A$ be the event that 
        at least one of the option sets $P_{\ell + i}, i \in [n]$ is $\{\star\}$; 
        condition on $A$. 
        Then 
        $$
            \Exp [g^\lambda \mid A] \leq 2^{\lambda(\ell + 1)} n/(1+\alpha)
            \,.
        $$
    \end{theorem}

    % \begin{lemma}[Grinding power moments in the $(\ell, n, \alpha)$-Bernoulli game]
    % \label{lemma:xor-game-bernoulli-moment}
    % \end{lemma}
    \begin{proof}
        We retrace the proof of Lemma~\ref{lemma:adaptive-moment} 
        using $S = B$ and $r_i = 1 - B_i, i \in [\ell + n]$; 
        hence $|P_i| = 1 + B, i \in [\ell + n]$. 
        In particular, the random variables $H, r_i, i \in [\ell + n]$ 
        and the 
        events $A, A_h, h \in [n]$ remain the same. 
        % Let $H \defeq \max \{i : \text{$i \in [\ell+1, \ell+n]$ and $B_i = 0$} \}$. 
        % By our assumptions, $H \in [\ell+1, \ell+n]$ is well-defined, 
        % at least one of the $B_i$s is zero, and hence, 
        % the leading product in~\eqref{eq:xor-game-lookahead-gp} 
        % vanishes. 
        % Condition on the event $H = h$; 
        % this means $B_{\ell + h} = 0$ and 
        % $B_i = 1$ for $i \geq \ell + h+1$. 
        % Thus the additive terms for $k \leq h - 1$ 
        % in~\eqref{eq:xor-game-lookahead-gp}
        % vanishes as well.
        % Since $|P_i| \in \{1, 2\}$, 
        % the expression in~\eqref{eq:xor-game-lookahead-gp} becomes 
        Thus $\beta = 1 - \alpha$ and 
        $\Pr[H = h] = (1 - \alpha) \alpha^{n - h}$ for $h \in [n]$. 
        Condition on $A_h$. 
        Since $1 + S \leq 2$ and $|P_i| = 2$ for $i \geq \ell + h + 1$, 
        we can upper bound the right-hand side in~\eqref{eq:ell-n-d-game-bound-gh} as 
        \begin{align*}
        % \label{eq:xor-game-playornot-gp}
            g_h
            &\leq (1 + B)^\ell \sum_{k = h}^n 2^{n - k}
            % \sum_{k=h}^n (1 - B_k) \, 
            %     \left(  \prod_{i=k}^{\ell+k-1}      (1 + B_i)   \right)   
            %     \left(  \prod_{i=\ell+k+1}^{\ell+n} 2 B_i       \right)    
            %     \\        
            % &\leq 
            %     \sum_{k=h}^{n }
            %     (1 - B_k)\,
            %     2^\ell\,
            %     2^{n - k}            
            \leq 
                2^{\ell} \cdot ( 1 + 2 + \cdots + 2^{n - h} ) 
            = 2^{n + \ell - h + 1}
        \,.
        \end{align*}
        It follows that
        $
            \Exp [g_h^\lambda \mid A_h]
            \leq 
                2^{\lambda(n + \ell - h + 1)} 
        $ 
        since 
        the quantities $|P_{i}|$ in the game $U$ are independent. 
        Since $\Pr[A] = 1 - \alpha^n$, we have 
        \begin{align*}
            \Exp [g^\lambda \mid A]
                &=
                \sum_{h = 1}^n \Pr[A_h \mid A] 
                    \Exp [g_h^\lambda \mid A_h] 
                =
                \sum_{h = 1}^n 
                    \frac{(1 - \alpha) \alpha^{n-h}}{1 - \alpha^n}\,
                    \Exp [g_h^\lambda \mid A_h] 
                \leq 
                    \frac{2^{\lambda(\ell + 1)}}{1+\alpha}
                    \sum_{h=1}^n 
                    \left( \alpha 2^\lambda \right)^{n-h} 
        \end{align*}
        since $(1 - \alpha)/(1 - \alpha^n) \leq 1/(1 + \alpha)$ for any $n \geq 2$. 
        By assumption, $\lambda \leq \log_2(1/\alpha)$ or, 
        equivalently, $\alpha 2^\lambda \leq 1$.
        It follows that 
        $
            \Exp [g^\lambda \mid A] \leq  n \cdot 2^{\lambda(\ell + 1)}/(1+\alpha)
        $.
    \end{proof}


    \begin{lemma}\label{lemma:xor-game-Bernoulli-gamma}
        Let $\EpsP \in (0,1), \ell, n \in \NN, n \geq 2$ and 
        $\alpha \in (0, 1/2)$.
        Let $U$ be a $(\ell, n, \alpha)$-Bernoulli game. 
        The grinding power $g = g(U)$ 
        satisfies $\Pr[g \geq \gamma] \leq \alpha^n + \EpsP$ where 
        \begin{align}\label{eq:bernoulli-game-gamma}
            % {\gamma = n 2^{\ell+1 + q/\log_2(1/\alpha)}}
            \gamma = 2^{\ell + 1} \left(
                \frac{n}{(1+\alpha)\EpsP}
            \right)^{1/\log_2(1/\alpha)}
            % \quad\text{and}\quad q = \log_2(1/\EpsP)
            \,.
        \end{align}
    \end{lemma}
    \begin{proof}
        Let $\gamma$ be a positive integer whose value will be set later.
        Let $A$ be the event that 
        at least one of the option sets $P_{\ell + i}, i \in [n]$ in $U$ is $\{\star\}$. 
        (Note that $\Pr[\overline{A}] = \alpha^n$.)
        Theorem~\ref{thm:xor-game-playornot} states that 
        for any $\lambda \in (0, \log_2(1/\alpha)]$, 
        $\Exp [g^\lambda \mid A]$ is at most $n 2^{\lambda(\ell + 1)}$. 
        Since $g$ is non-negative, 
        Markov's inequality implies that 
        $$
        \Pr[g \geq \gamma] 
        \leq \Pr[\overline{A}] + \Pr[g \geq \gamma | A]
        \leq \alpha^n  + \Exp[g^\lambda \mid A]/\gamma^\lambda
        \leq \alpha^n  + \frac{n}{1+\alpha} 
            \left(\frac{2^{\ell + 1}}{\gamma}\right)^\lambda
        \,.$$ 
        Writing $a = 2^{\ell + 1}/\gamma$, 
        the derivative of this upper bound with respect to $\lambda$ is 
        $n a^\lambda \ln(a)$ . 
        This quantity is non-positive as long as $a \leq 1$ or, equivalently, 
        $\gamma \geq 2^{\ell + 1}$. 
        Thus we get the tightest upper bound on $\Pr[g \geq \gamma]$ 
        by taking the largest permitted value of $\lambda$ 
        as long as $\gamma$ is at least $2^{\ell + 1}$. 
        
        Set $\lambda = \log_2(1/\alpha)$ and take any $\EpsP \in (0,1)$. 
        For any real positive $x$, 
        the inequality $(n/(1+\alpha)) (2^{\ell + 1}/x)^\lambda \leq \EpsP$ 
        is equivalent to saying that 
        $x \geq 2^{\ell + 1} (n/(1+\alpha)\EpsP)^{1/\log_2(1/\alpha)}$. 
        Let us take 
        $\gamma \defeq 2^{\ell + 1} (n/(1+\alpha)\EpsP)^{1/\log_2(1/\alpha)}$ 
        so that this inequality is satisfied by setting $x = \gamma$. 
        This implies two things. 
        First, 
        since $\alpha > 0, \EpsP < 1$, and $n \geq 2$, 
        the quantity  $(n/(1+\alpha)\EpsP)^{1/\log_2(1/\alpha)}$ is at least one and, consequently, 
        $\gamma$ is clearly larger than $2^{\ell + 1}$. 
        Second, 
        the quantity 
        $\left( \Exp[g^\lambda | A]/\EpsP \right)^{1/\lambda}$ is at most $\gamma$ 
        and, using Fact~\ref{fact:grinding-max}, 
        $\Pr[g \geq \gamma | A]$ is at most $\EpsP$. 
        It follows that $\Pr[g \geq \gamma] \leq \alpha^n + \EpsP$.
    \end{proof}



\subsection{When a slot has exactly one nonce leader}\label{sec:coin-tossing-single-leader}
\newcommand{\CoinTossingPublic}{\Pi_\mathsf{public}^{\Players(\alpha),k,s,d}}

% \begin{minipage}[t]{\textwidth}
\begin{figure}
  \begin{framed}    
    \begin{center}
        \textbf{Coin-tosssing protocol $\CoinTossingPublic$}
    \end{center}
    \begin{enumerate}
      \item 
      (Nonce leaders.) 
      For each nonce-emitting slot $t$, use $\eta_0$ to 
      select a single player $u \in \Players$ 
      with probability $\sigma_{u}$. 
      Set $u_t = u$ and call $u_t$ the \emph{nonce leader at slot $t$}.
      
      \item 
      (Nonce announcement.)
      At a nonce-emitting slot $t$, the designated nonce leader 
      broadcasts a random string $y_t \in \{0,1\}^\kappa$ 
      to all nonce players and inserts $y_t$ in $Q$. 

      \item 
      (Nonce relay.) 
      At slot $r$, suppose there is a nonce player $u$ who 
      has previously heard about a nonce $y$, 
      emitted from some slot $t \geq r - k$, 
      which appears neither in $Q$ nor in his view of $\Chain$. 
      Then $u$ re-inserts $y$ in $Q$ with timestamp $r$.

      
      % \item 
      % % (Nonce recording.) 
      % At any slot $r$, 
      % a designated slot leader in $\Pi$ 
      % favors a new chain $\Chain$ over his currently-held chain $\Chain'$ 
      % and appends a block $B$ on top of $\Chain$. 
      % (It is possible that $\Chain = \Chain'$.) 
      % In $B$, he includes every nonce $y \not \in \Chain$ 
      % that satisfies one of the following: 
      % \begin{enumerate*}[label=(\textit{\roman*})]
      %   \item $y \in \Chain'$, or 
      %   \item $y$ was emitted from a slot $t \geq s - k$.
      % \end{enumerate*}
      
      \item 
      (Output.)
      At any slot $\tau \geq T + 2k$, 
      every user $u \in \Players$ examines $\Chain_u$ 
      and computes the \emph{output} $\displaystyle \eta = \eta_0 \oplus \bigoplus_{y \in \Chain_u} y$ 
      where the exclusive-OR runs over 
      all nonces $y$ recorded in $\Chain_u$. 
    \end{enumerate}
  \end{framed}
  \caption{Coin tossing with a single nonce leader per slot}
  \label{alg:coin-tossing-single-leader}
\end{figure}
% \end{minipage}

\begin{proposition}\label{prop:coin-tossing-security-public}
  Let $\alpha \in (0, 1/2)$ and $d, k, s, T, \tau \in \NN$ 
  so that 
  $s \leq k$ and 
  $d$ divides both $k$ and $T$. 
  Let $n = (T - k)/d$ and $\ell = k/d$. 
  Suppose that the coin-flipping protocol $\Pi = \CoinTossingPublic$ 
  uses an $(2^{-\tau}, k, s)$-secure ledger protocol $\Blockchain$.
  Then $\Pi$ is $(\alpha^n + 2^{-\tau + 1}, \rho)$-secure 
  where 
  \begin{equation}\label{eq:minentropy-loss-Bernoulli}
    % \rho = 1 + \ell + \log_2 n + \frac{\log_2(1/\EpsP)}{\log_2(1/\alpha)}
    % \rho = \ell + \log_2 n + \tau/\log_2 (1/\alpha) + 1
    \rho = \ell + \frac{\tau + \log_2 n - \log_2 (1 + \alpha)}{\log_2 (1/\alpha)}    \,.
  \end{equation}
\end{proposition}
\begin{proof}
  Let $\EpsP = 2^{-\tau}$. 
  Recall the $(\ell, n, \alpha)$-Bernoulli game from Definition~\ref{def:Bernoulli-game}.
  \begin{claim}\label{claim:correspondence-Bernoulli-game}
    Let $k, d, T, \kappa \in \NN$ and set $\ell = k/d$ and $n = (T - k)/d$. 
    Let $\alpha \in (0, 1/2)$ and  
    let $\Players = \Players(\alpha)$ be the set of nonce players 
    participating in protocol $\Pi = \CoinTossingPublic$. 
    Then the output distribution of $\Pi$ 
    is identical to 
    the output distribution of 
    an $(\ell, n, \alpha)$-Bernoulli game $U$ with dimension $\kappa$.
  \end{claim}

  Assume the claim for the moment. 
  It follows 
  that the 
  min-entropy of the output distribution of 
  protocol $\Pi$ 
  is equal to 
  the min-entropy of the output distribution of $U$. 
  However, 
  Lemma~\ref{lemma:xor-game-Bernoulli-gamma} and 
  Lemma~\ref{lemma:xor-game-minentropy} together 
  states that 
  except with probability $\EpsP  +\alpha^n$, 
  the latter quantity is at least $\kappa - \log_2 \gamma$ 
  where $\gamma = n\,2^{\ell + 1} (1/\EpsP)^{1/\log_2(1/\alpha)}$.
  Since the game $U$ is played on top of 
  the ledger protocol $\Blockchain$, 
  any guarantee provided by $U$ is 
  further conditioned on the (external) event that 
  $\Blockchain$ is $(\EpsP, k, s)$-secure. 
  Using Definition~\ref{def:coin-flipping-security}, 
  it follows that $\Pi$ is $(\alpha^n + 2 \EpsP, \rho)$-secure 
  where $\rho = \log_2 \gamma$; 
  this $\rho$ is given in~\eqref{eq:minentropy-loss-Bernoulli}. 
  It remains to prove Claim~\ref{claim:correspondence-Bernoulli-game}.

  \paragraph{Proof of Claim~\ref{claim:correspondence-Bernoulli-game}.}

    Consider the protocol $\Pi$. 
    First, observe that every honest nonce will 
    appear in the output calculation of $\Pi$. 
    {\color{red}Why?}
    % Specifically, 
    % the nonce relay step in $\Pi$ 
    % guarantees that an honestly-emitted nonce w
    % suppose a nonce $y_t$ is announced by an honest player at slot $t \in [T]$. 
    % Let $r \geq t + 1$ be the first slot after $t$ 
    % so that $r$ is an honest slot in the ledger protocol $\Blockchain$. 
    % Let $u$ be an honest nonce player and let $\Chain_u$ 
    % be his view of the ledger at slot $r$. 
    % % $r \geq T + 2k + 1$. 
    % Then either $y_t \in \Chain_u$, 
    % Since $\Blockchain$ is $(\EpsP, k, s)$-secure, 
    % it follows that $r - t \leq s \leq k$ except with probability $\EpsP$. 
    % Hence, at slot $r$, 
    % either $y_$is announced no later than slot $t + k - 1$, 
    % it would appear in the calculation of the output $\eta$. 


    % Spcifically, 
    % consider a nonce player $u$ and his view of the ledger $\Chain_u$ 
    % at some slot $r \geq T + 2k + 1$. 
    % % Let $\Chain_u[0:T]$ denote the prefix of $\Chain_u$ 
    % % corresponding to slots $0, 1, \ldots, T$. 
    % If a nonce $y$ does not appear in $\Chain_u[0:T]$ 
    % then the nonce relaying step ensures that 
    % it will be in $Q$ at slot $T$. 
    % Due to the $\sECQ[k]$ and $\kSlotCP$ properties, 
    % it is guaranteed to appear in $\Chain_u[0: T + 2k]$. 


    Every nonce-emitting slot $t$ yields an option set $P_t$, 
    for the nonce leader $u_t$, 
    which contains all possible values for the nonce $y_t$ 
    including a uniformly random string $\eta_t$. 
    \begin{description}[font=\normalfont\itshape\space]
      \item[If the nonce-leader at slot $t$ is honest,] 
      he will immediately announce his only option $y_t = \eta_t$; 
      this corresponds to the case $P_t = \{\star\}$. 
      The symbol $\star \in P_{\ell + 1}$ hides 
      $\eta_{\ell  + 1}$ 
      from the view of the adversarial players 
      before it is announced. 
      When the $\star$ in the game 
      is converted into a uniformly random lookahead string $x$, 
      it mimics the situation that the adversary gets to see 
      $\eta_{\ell + 1}$ for the first time. 

      \item[If the nonce-leader $u_t$ at slot $t$ is dishonest,] 
      his option set $P_t$ equals $\{\eta_t, 0^\kappa\}$. 
      The lookahead parameter $\ell$ in the XOR target game 
      corresponds to the number of nonce-messages that 
      $u_t$ may observe before deciding 
      which option $y_t \in P_t = \{\eta_t, 0^\kappa\}$ to announce. 
      Here, the option $y_t = 0^\kappa$ corresponds to the situation 
      that $u_t$ does not announce at all (or announces late). 
    \end{description}


    Thus the output distribution of $\Pi$ is identical to the output distribution of $U$.
\end{proof}


%==========================================================


\begin{corollary}\label{coro:beacon-Bernoulli}
  Let $\tau, a \in \RR_+, a > 1, \tau \in (0, \infty)$. 
   % and $\EpsP = 2^{-\tau}$. 
  Let $d, k, T \in \NN$ 
  so that $d$ divides both $k$ and $T$ and, 
  writing $n = (T - k)/d$ and $\ell = k/d$, 
  satisfy $a n \geq \tau$.
  Set $\alpha = 2^{-a}$ and 
  suppose that the coin-flipping protocol $\Pi = \CoinTossingPublic$ 
  uses a $(2^{-\tau}, k)$-secure ledger protocol $\Blockchain$.
  An $(L, \Pi)$-composition is $(3 L n \epsilon, \rho)$-secure where
  $$
    \epsilon = 2^{- \tau(1 - 1/a) + \ell + 1}
    \quad\text{and} \quad
    \rho = \ell + \log_2 n + \tau/a + 1
    \,.
  $$
  % $
  %   p 
  %   % = 3 L \EpsP 2^\rho
  %   % = 3 L 2^{\rho - \tau}
  %   = 3 L 2^{- \tau(1 - 1/a) + \ell + \log_2 n + 1}
  %   % \,.
  % $ and 
  % $\rho = \ell + \log_2 n + \tau/a + 1$.
\end{corollary}
\begin{proof}
  Apply  
  Lemma~\ref{prop:coin-tossing-security-public} into 
  Lemma~\ref{lemma:composition}.
\end{proof}
