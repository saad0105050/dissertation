

Let us record the following fact. 

\begin{fact}\label{fact:sigma}
    Let $\alpha \in (0, 1), n \in \NN$ and $\sigma = (\sigma_1, \ldots, \sigma_n) \in \RR^n$ 
    where each $\sigma_i \in (0, \alpha)$ and in addition, $\sum_i \sigma_i = \alpha$. 
    Then 
    (i.) $\prod_i (1 - \sigma_i) \geq 1 - \alpha$ and 
    (ii.) $\sum_i \sigma_i/(1 - \sigma_i) \geq \alpha$.
\end{fact}
\begin{proof}
    The proof is via induction on $n \in \NN$.
    Define $f_n(\alpha) \defeq \prod_i (1 - \sigma_i)$ and
    $g_n(\alpha) \defeq \sum_i \sigma_i/(1 - \sigma_i)$.

    \begin{description}
        \item[Part (i.)]
        For $n = 1$, we have $f_1(\alpha) = 1 - \alpha$ since $\sigma_1 = \alpha$. 
        Our inductuve hypothesis is that $f_{n-1}(\alpha) \geq 1 - \alpha$.
        Let $\sigma_n = \delta$ for 
        an arbitrary $\delta \in (0, \alpha)$. 
        Observe that 
        \[
        f_n(\alpha) 
        = f_{n-1}(\alpha - \delta) \cdot (1-\delta)
        \geq (1 - \alpha + \delta)(1 - \delta)
        = (1 - \alpha) + \delta(\alpha - \delta)
        > 1 - \alpha
        \]
        since $0 < \delta < \alpha$. 
        We used the inductive hypothesis at the first inequality.

        \item[Part (ii.)]
        \emph{(First proof.)}
        Fix an $i \in [n]$ and note that $\sigma_i/(1 - \sigma_i) \geq \sigma_i$. 
        Summing both sides of this inequality 
        for $i \in [n]$ yields the claim.

        \emph{(Second proof.)}
        For $n = 1$, we have $g_1(\alpha) = \alpha/(1 - \alpha) > \alpha$ since 
        $\sigma_1 = \alpha$ and $\alpha \in (0, 1)$. 
        Our inductuve hypothesis is that $g_{n-1}(\alpha) \geq \alpha$.
        Let $\sigma_n = \delta$ for 
        an arbitrary $\delta \in (0, \alpha)$. 
        Observe that 
        \[
        g_n(\alpha) 
        = g_{n-1}(\alpha - \delta) + \frac{\delta}{1-\delta}
        \geq \alpha - \delta +\frac{\delta}{1-\delta}
        = \alpha + \delta \left(-1 + \frac{1}{1 - \delta}\right)
        = \alpha + \delta \sum_{k \geq 1} \delta^k
        \geq \alpha
        \,.
        \]
        We used the inductive hypothesis at the first inequality.
    \end{description}
\end{proof}




\subsection{Proof of Lemma~\ref{lemma:geom-dominates-poisson}}



% \begin{proof}
    $Y$ stochastically dominates $Z$ if and only if $\Exp Z^\lambda \leq \Exp Y^\lambda$ 
    for all $\lambda > 0$. 
    By definition, if $Y$ stochastically dominates $Z$ then the right tail of $Y$ 
    has more volume than the right tail of $Z$.
    Thus we need to prove that for all $k = 0, 1, 2, \ldots$, $\Pr[Z \geq k] \leq \Pr[Y \geq k]$.

    Before we proceed, note that the case $k = 0$ is trivial since 
    both $Y$ and $Z$ has the same domain $0, 1, 2, \ldots$\ . 
    We shall use the easy-to-check fact $\Pr[Y \geq k] = \alpha^k$ for integers $k \geq 0$. 

    \begin{description}
        \item[Case: $k \geq 3$.]
        {\color{red} Why is $k$ at least $3$?} 
        We can compute a tail bound on $Z$ as follows: 
        for any positive real $\theta$, 
        % satisfying $| \sigma_i(e^\theta - 1)| \leq 1$ for all $i$, we have

        \begin{align*}
            \Pr[Z \geq k] 
            &= \Pr[e^{\theta Z} \geq e^{\theta k}] \leq e^{-\theta k} \Exp e^{\theta Z} \\
            &=  e^{-\theta k} \Exp e^{\theta \sum_i X_i} =  e^{-\theta k}\Exp \prod_i e^{\theta X_i} \\
            &=  e^{-\theta k} \prod_i \Exp e^{\theta X_i} 
            =  e^{-\theta k} \prod_i \left(1 - \sigma_i + \sigma_i e^{\theta }\right) \\
            &\leq e^{-\theta k} \prod_i e^{\sigma_i(e^\theta - 1)} = e^{- \theta k} \cdot e^{ \alpha (e^\theta - 1)} \\
            &= e^{- \theta k + \alpha ( e^\theta - 1)}
            \,.
        \end{align*}
        Here, we have used 
        the monotonicity of the exponential function (the first equality),
        the Markov inequality (the first inequality), 
        the independence of the $X_i$s (fourth inequality), 
        the fact $e^x \geq 1 + x$ for all real $x$ (second inequality), 
        and finally, the assumption $\alpha = \sum_i \sigma_i$ (the penultimate equality).
        % Note that we have to later verify that $\sigma_i (e^\theta - 1)$ is indeed at most $1$.

        We want to minimize the above upper bound in terms of $\theta$. 
        That is, we want to solve 
        \begin{align*}
                    &\dfrac{d}{d\theta} \left[ e^{- \theta k + \alpha (e^\theta - 1)} \right]= 0 \\
            \Or&    (-k + \alpha e^\theta) e^{- \theta k + \alpha (e^\theta - 1)} = 0 \\
            \Or& k = \alpha e^\theta \\
            \Or& \theta = \log(k/\alpha)
        \end{align*}
        since $e^x \neq 0$ for $x \in \RR$. 
        Using this $\theta$, we get
        \begin{equation}\label{eq:Z_tail}
            \Pr[Z \geq k] \leq e^{- k \log(k/\alpha) ) - \alpha + k} 
            \,.
        \end{equation}
        Since $\Pr[Y \geq k] = \alpha^k = e^{k \log(\alpha)}$, we can check that 
        $k \log(\alpha) - \left( k  - k \log(k/\alpha)  - \alpha \right)
        = k (\log k - 1) + \alpha
        $.
        Hence as long as $k \geq e$, 
        the quantity $k (\log k - 1) + \alpha$ is always positive. 
        Thus $\Pr[Z \geq k] \leq \alpha^k = \Pr[Y \geq k]$ for $k = 3, 4, \ldots$.


        \item[Case: $k = 1$ and $k = 2$.]
        Let us write $f(\alpha) = \prod_i (1 - \sigma_i)$ and $g(\alpha) = \sum_i \sigma_i/(1 - \sigma_i)$. 
        For the case $k = 1$, 
        observe that $\Pr[Z \geq 1] = 1 - \Pr[Z = 0] = 1 - f(\alpha)$ 
        By Fact~\ref{fact:sigma}, this is at most $1 - (1-\alpha) = \alpha = \Pr[Y \geq 1]$.

        For the case $k = 2$,
        Note that $\Pr[Z = 1] = \sum_i \sigma_i \prod_{j \neq i} (1 - \sigma_j) = f(\alpha) g(\alpha)$. 
        Hence $\Pr[Z \geq 2] = 1 - \Pr[Z = 0] - \Pr[Z = 1] = 1 - f(\alpha) - f(\alpha) g(\alpha) 
        = 1 - f(\alpha) (1 + g(\alpha) )$. 
        Using Fact~\ref{fact:sigma}, this probability is at most 
        $1 - (1-\alpha)(1 + \alpha) = \alpha^2 = \Pr[Y \geq 2]$. 
    \end{description}

    \hfill$\qed$

    \begin{proposition}\label{prop:poisson-moment}
        Let $\alpha \in (0.41, 1/2)$ and 
        let $P \sim \Poisson(\alpha)$. 
        Then $\Exp (1 + P)^\lambda \leq 1 + 2 \alpha$ and, 
        in particular, $\alpha \Exp (1 + P)^\lambda \leq 1$.
    \end{proposition}
    \begin{proof}
        Let $\lambda \in (1, 2)$. 
        Define $f_\lambda(\alpha) = \Exp (1 + P)^\lambda = \sum_{k \geq 0} T_k$, where 
        \begin{align*}
            T_k &= (1 + k)^\lambda \cdot e^{-\alpha} \alpha^k/k!
            \,.
        \end{align*} 
        Notice that $f_\lambda(\alpha)$ is convex and increasing in both $\alpha$ and $\lambda$. 
        (In particular, $T_k$  is convex and increasing since 
        the map $\alpha \mapsto \alpha^k$ is convex and increasing for positive $k$, 
        and the map $c \mapsto c^\lambda$ is convex and increasing for $\lambda \geq 1$ and $c \geq 1$.)

        Define $r_k = T_{k+1}/T_k$ so that 
        $$
            r_k = \frac{\alpha}{1+k} \left(\frac{2 + k}{1 + k}\right)^\lambda
            \,.
        $$ 
        For a fixed $\alpha$ and $\lambda$, observe that 
        $$\frac{r_k}{r_{k+1}} 
        = \frac{2 + k}{1 + k} \left(\frac{(2 + k)^2}{(1+k)(3 + k)} \right)^\lambda
        = \frac{2 + k}{1 + k} \left(\frac{4 + 4k + k^2}{3 + 4k + k^2} \right)^\lambda
        > 1
        \,.
        $$
        Therefore, $r_k > r_{k+i}$ for all $i = 1, 2, \ldots$\ .
        It follows that for any $K \in \NN$, 
        $$
            F_\lambda(\alpha, \lambda; K) 
            \triangleq \sum_{k = 0}^{K-1} T_k + \frac{T_K}{1 - r_K} 
            \geq f_\lambda(\alpha)
        $$
        and, in addition, that $F_\lambda(0.5, \lambda; K) \geq f_\lambda(\alpha)$ for any $\alpha \in (0, 1/2)$.
        % Since the terms in the sum-expression of $f_\lamnda(\alpha)$ decays quickly (in $k$), 
        % it suffices to take $K = 3$. 
        Furthermore, since $f_\lambda(\alpha)$ is convex and increasing in $\alpha$,  
        the straight line segment $L(\alpha)$ connecting the points 
        $$
        a(\lambda) = \left(0.41, F_\lambda(0.41; K) \right)
        \quad\text{and}\quad
        b(\lambda) = \left(0.5, F_\lambda(0.5; K) \right)
        $$ is an upper bound on $F_\lambda(\alpha; K)$. 

        
        For a given $K$, one can optimize for $\lambda$ that minimizes the gap $1/\alpha - F(0.5, \lambda; K)$. 
        However, we take a more relaxed approach and simply observe that 
        using $\lambda = 3/2$ is good enough. 
        In particular, 
        using $K = 3$ and $\lambda = 3/2$, we can directly calculate 
        $$
        F_\lambda(0.41, 3/2; 3) = 1.79428
        \quad\text{and}\quad
        F_\lambda(0.5; 3/2) = 1.98073 < 1/(1/2)
        \,.
        $$
        Thus the line segment joining the points $a(3/2)$ and $b(3/2)$ 
        is given by 
        $$L(\alpha) 
        = a(3/2) + (b(3/2) - a(3/2))\cdot (\alpha - 0.41)
        = 1.79428 + 1.98073 (\alpha - 0.41)
        \leq 1 + 2 \alpha.
        $$
        It suffices to check that $L(0.41) = 1.82 > F_\lambda(0.41, 3/2; 3) = 1.79428$ and 
        $L(0.5) = 2 > F_\lambda(0.5, 3/2; 3) = 1.98073$.
    \end{proof}

% \end{proof} 

