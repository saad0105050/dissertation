
\subsection{Static corruption model}

\begin{minipage}[c]{\textwidth}
	\begin{center}
		\textbf{Blockchain Protocol $\BlockchainProtocolPublicSchedule$}
	\end{center}
	\begin{framed}
		% \Hrule
		\begin{description}[labelindent=0cm,leftmargin=0cm]

			\item[Participants.]
			$\Players$ is the (static) set of protocol participants.

			\item[Parameters.]~
			\begin{description}[labelindent=0.5cm, font=\normalfont\itshape\space]
				\item[Unknown to the protocol:]
				$\alpha, \EpsP \in (0, 1), k, s, \gamma, \Delta \in \NN$. 
				% $\gamma$ is an upper bound on the total adversarial hash queries. 
				$\Delta$ is the \emph{network delay}.

				\item[External, known to the protocol:]
				$\kappa \in \NN$, the length of the common random string. 
				$\eta_0 \in \{0,1\}^\kappa$, the initial common random string. 
				$\sigma : \Players \rightarrow [0,1]$, the \emph{stake function} 
				which is also a probability function; it satisfies the 
				Requirement (i.) below.

				\item[Internal, known to the protocol:]  
				$R \in \NN$, the \emph{epoch length}.
			\end{description}

			\item[Assumptions.] ~
			\begin{enumerate}[label=(\roman*.),labelindent=0cm]
				\item (Synchrony.) The protocol proceeds in slots $t = 1, 2, \ldots$\,.

				\item (Message delivery.) Any message oroginating in a slot $t$ 
				is delivered prior to the slot $t + \Delta$.
				
				\item (Adversarial stake.) Suppose the adversary controls a subset $A$ of players 
				which is unknown to the protocol.
				Let $\alpha$ denote the \emph{adversarial stake ratio}, i.e., 
				$\alpha \defeq \sum_{\Player \in A} \sigma(\Player)$. 
				Then $\alpha \in [0, 1/2)$ throught the execution of the protocol. 

				\item (Corruption). The adversary cannot immediately corrupt a player. 
				He has to wait at least $R$ slots before corrupting someone.
			\end{enumerate}


			\item[Guarantees.] 
			% ~
			% \begin{enumerate}[label=(\roman*.)]
				% \item 
				The protocol satisfies $\kSlotCP$ and $\sECQ$ except with probability at most 
				% $\gamma \EpsP$
				$\EpsP$
				. 
			% \end{enumerate}
			

			\item[Exposed functionalities.]~
			\begin{itemize}[label=-,labelindent=0cm,leftmargin=0.5cm]
				\item \textsf{announce}: Any player can announce a message 
				to all other players.  

				\item \textsf{add\_messages\_to\_block}: When a player becomes a slot leader, he may add any message to the newly minted block.
			\end{itemize}
		\end{description}


		\dotfill

		% \begin{center}
		% 	Internal
		% \end{center}
		\begin{description}
			\item[Epochs and the beacon.]
			The execution of the protocol can be partitioned into \emph{epochs} $e = 1, 2, \ldots$\,, 
			each epoch spanning $R$ slots. 
			A sub-protocol $\BeaconProtocol$ takes $\eta_0$ as input and emits, for $i \in \NN$,
			a random string $\eta_i$ at the end of the epoch $e_i$.

			\item[Slot leaders.]
			For slots $t = 1, 2, \ldots$, the execution of the protocol 
			determines an ordered list $\Leaders = \{L_1, L_2, \ldots\}$ 
			where each set $L_t$ contains a single player $\Player \in \Players$; 
			this player is called the ``slot leader for slot $t$.'' 
			At slot $t$, only the leader for that slot 
			is allowed to add a block to an existing blockchain. 
			Considering the $i$th epoch $E_i = \{iR + 1, iR + 1, \ldots, (i+1)R\}$, 
			the ordered list $\{L_t\}, t \in E_i$ is the ``slot leader schdule'' for that epoch.
			This schedule is publicly available at the onset of the epoch; 
			thus anyone can verify if a block is issued by a legitimate slot leader.

			\item[Leader election.]
			Consider the $i$th epoch $E_i = \{iR + 1, iR + 1, \ldots, (i+1)R\}$ comprised of $R$ slots.
			At the onset of this epoch, the sets $L_t, t \in E$ are determined as follows:
			Independently for each slot $t \in E_i$, the random string $\eta_{i-1}$ is used to publicly draw 
			a single slot leader $\Player \in \Players$ using the distribution $\sigma$. 			
			i.e., every player $\Player$ has probability $\sigma(\Player)$ 
			to be in $L_t$. 
			$L_t = \{\Player\}$ and in particular, $|L_t| = 1$.

		\end{description}

	\end{framed}
\end{minipage}

% \begin{remark}
% 	% That protocol would rely only on the guarantees given 
% 	% by this blockchain prtocol. 
% 	% This is this protocol does not mention \emph{network delay}; 
% 	% it is subsumed by the guarantees it provides. 
% 	For a fixed $\alpha$, 
% 	the error probability $\EpsP$ may depend exponentially 
% 	on the network delay $\Delta$ as well as 
% 	linearly on other parameters such as $k$ and $s$. 
% \end{remark}


\subsection{Adaptive corruption model}
\begin{minipage}[c]{\textwidth}
	\begin{center}
		\textbf{Blockchain Protocol $\BlockchainProtocolPrivateSchedule$}
	\end{center}
	\begin{framed}
		The protocol is identical to $\BlockchainProtocolPublicSchedule$ except at the following steps.

		\begin{description}

			\item[Assumptions.]~
			\begin{itemize}[labelindent=0cm]
				\item (Corruption). The adversary can immediately corrupt a player. 
			\end{itemize}

			\item[Slot leaders.]
			For slots $t = 1, 2, \ldots$, the execution of the protocol 
			determines an ordered list $\Leaders = \{L_1, L_2, \ldots\}$ 
			where each $L_t$ is a (possibly empty) subset of $\Players$. 
			The players in a set $L_t$ are called the ``slot leaders for slot $t$.''
			At slot $t$, each $\Player \in L_t$ is allowed to add a block to an existing blockchain. 
			Considering the $i$th epoch $E_i = \{iR + 1, iR + 1, \ldots, (i+1)R\}$, 
			the ordered list $\{L_t\}, t \in E_i$ is the ``slot leader schdule'' for that epoch.
			Whether a player is a leader for slot $t$ is not known a priori; 
			it is known only after he issues a block with a proof of his legitimacy.
			Using this proof, anyone can verify whether a block is issued by a legitimate slot leader.


			\item[Leader election.]
			Consider the $i$th epoch $E_i = \{iR + 1, iR + 1, \ldots, (i+1)R\}$ comprised of $R$ slots.
			The sets $L_t, t \in E_i$ are determined as follows.
			Independently for each slot $t$, each player $\Player \in \Players$ 
			uses the random string $\eta_{i-1}$ to make 
			an independent and private test to determine whether he is 
			in $L_t$; 
			a player $\Player$ becomes a slot leader
			with probability proportional to $\sigma(\Player)$. 
			Thus $L_t$ may contain zero or more players. 

		\end{description}

	\end{framed}
\end{minipage}
