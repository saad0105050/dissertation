
In this section, we show how to use our beacon \emph{inside} an existing blockchain protocol. 
We would see that this would endow the blockchain protocol with improved resilience against 
grinding attacks.

We note that both blockchain protocols are in the static stake model.


\subsection{A blockchain beacon in the static corruption model}
In this case, we may take the leader schedule to be public. 
The corruption delay will be longer than the epoch length $R$; 
in this way, by the time the adversary 
gets a chance to corrupt a slot leader, 
the epoch will pass and he will not be a legitimate slot leader anymore. 

It is obvious that a beacon made by a composition of the coin-tossing protocol 
with the one-speaks-every-$d$-slots schedule is a feasible choice. 
It remains to study the relevant blockchain parameters.

\begin{itemize}
    \item (Blockchain protocol parameters.) 
    Suppose we have a blockchain protocol where each block settles within $k$ slots 
    except with probability $\EpsP$. 
    Fix the adversarial stake ratio $\alpha \in (0, 1/2)$ and 
    choose arbitrary values for the nonce-size $\kappa \in \NN$, 
    lifetime (in epochs) $L \in \NN$. 
    Set $\lambda = \log_2(1/\alpha)$ and choose 
    % $\delta \in (0,1 - 1/\lambda)$
    $\delta \in (0,1)$.
    Set $q = \log_2(1/\EpsP)$, $n = q/\lambda$, 
    $\ell = \lfloor q(1 - \delta - 1/\lambda) - (3 + \log_2n ) \rfloor$, and
    $t = \lfloor k/\ell \rfloor$. 
    % and $\ell = \lfloor k/t \rfloor$.
    Thus the failure probability of the beacon is at most $L \EpsP^\delta$. 
    Let us set the the blockchain protocol's epoch length 
    $R = \ell t+(q/\lambda)t+k+k = 3k+qt/\lambda$. 

    % \item (Public schedule of slot-leaders.) 
    % At the outset of each epoch, 
    % a roster $\mathcal{L} = \{v_1, \ldots, v_{R}\}$ of \emph{slot leaders} is published. 
    % Specifically, for each $i \in [R]$, 
    % a protocol participant $v_i$ is selected with probability proportional to his stake. 
    % This player is designated to create the block at slot $i$. 

    % \item (Public schedule of nonce-players.)
    % At the outset of each epoch, 
    % a roster $\mathcal{N} = \{u_1, \ldots, u_{n+\ell}\}$ of \emph{nonce players} is published. 
    % Specifically, for each $i \in [n+\ell]$, 
    % a protocol participant $u_i$ is selected with probability proportional to his stake. 
    % This player is designated to contribute, at slot $(i - 1)t + 1$, 
    % a uniformly random element $\eta_i \in \{0, 1\}^\kappa$.
    
    % \item (Nonce announcement.) 
    % A nonce player at slot $i$ generates his nonce and announces it to the leaders of the slots $j \in [i+1, i+k-1]$.
    
    % \item (Nonce recording.) 
    % At an honest slot $j$, let $Q$ be the set of nonces that 
    % the slot-leader has received so far via the nonce-channel. 
    % When he adds his block to some chain $\Chain$, he includes all the nonces in $Q$ except 
    % (i.) those that are already present in $\Chain$ and 
    % (ii.) those that were issued at slots $i \leq j - k$.
\end{itemize}

\begin{remark}
Nonces in the first $k$ slots in an epoch serves as the initial lookahead region. 
The last $k$ slots in an epoch act as a waiting period so that 
the blocks from the previous $k$ slots (i.e., slots $R - 2k+1, \ldots, R-k$) 
become part of the persistent view of all honest players via the CP axiom. 
This will ensure that the last nonce, if heard by an honest recipient, 
must be recorded in a persistent block. 
In addition, note that $R$ is at most $4k$; 
the nonce generation is completed within first $2k$ slots
and the trailing $2k$ slots are used as the ``stabilizing region''
\end{remark}

\paragraph{Comparison with Ouroboros.}
\textbf{To do}

% Can a nonce be ever lost? The answer is no.

% \begin{claim}[]
% An announced nonce is never lost.
% \end{claim}
% \begin{proof}
% Suppose the nonce for slot $j$ is lost. 
% By the honest chain growth axiom, the chain held by an honest player at the end of the epoch 
% must contain at least one honest block from the slots $j+1, \ldots, j+k$. 
% However, by the broadcast axiom, this honest block must contain the nonce from slot $j$, 
% contradicting our assumption.
% \end{proof}


\subsection{A blockchain beacon in the dynamic corruption model}
In this setting, it is necessary to elect slot leaders (as well as nonce-players) in private. 
It is obvious that a beacon made by a composition of the coin-tossing protocol 
with the all-speaks-every-$d$-slots schedule is a feasible choice. 
It remains to study the relevant blockchain parameters.

\paragraph{Comparison with Praos.}



