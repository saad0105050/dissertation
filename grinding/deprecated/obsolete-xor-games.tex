
\section{Superceded material}


\subsection{The Basic XOR target game}\label{sec:generic-xor-game}


Our analysis of grinding power reduces to an analysis of the following
$\NumSets$-round one-player \emph{XOR target game}. 
The game is determined by four parameters: 
a dimension $\kappa \in \NN$, 
an ``initial'' value $\eta_0 \in \{0,1\}^\kappa$, 
a ``target'' subset $T \subset \{0,1\}^\kappa$, 
and a sequence of ``selection'' sets $\mathcal{S} = (S_1, \ldots, S_\NumSets)$, where each
$S_i \subset \{0,1\}^\kappa$. 
Informally, the game $G = G_\kappa = (\eta_0, T;  \mathcal{S})$ is
inductively defined as follows:
\begin{itemize}
\item If $\NumSets = |\mathcal{S}| = 0$, the player wins if and only if $\eta_0 \in T$.
\item Otherwise, the player is
  presented with a uniformly random value $r \in \{0,1\}^\kappa$ and must
  select an element $s \in S_1 \Union \{r\}$. The game is then
  begun anew with the initial value $\eta_0 \oplus s$ and the remaining
  selection sets $S_2, \ldots, S_\NumSets$. (That is, the player
  commences play of the game
  $G' = (\eta_0 \oplus s, T; \mathcal{S}_{-1})$.)
\end{itemize}
When it is clear from the context, we drop the parameter $\kappa$ from the notation of $G$.

More formally, we may associate with a game $G = G_\kappa = (\eta_0, T;  \mathcal{S})$ 
the quantity $\alpha(G)$, inductively defined so that
\begin{align*}
  \alpha(\eta_0, T;  \EmptyList) &= \begin{cases} 1 & \text{if $\eta_0 \in T$,}\\
    0 & \text{otherwise, and}
  \end{cases}\\
  \alpha(\eta_0, T;  \mathcal{S}) &= \Exp_r\left[
        \max_{s \in \{r\} \Union S_1} 
            \alpha(\eta \oplus s, T;  \mathcal{S}_{-1})
    \right]\,.  
\end{align*}
The quantity $\alpha(G)$ corresponds to the probability of success of
the informal game above under optimal play. It is convenient to
similarly consider a version of these games where the initial value
$\eta_0$ is selected uniformly at random. In this case, the game is
determined by $\kappa$, $T$, and the sequence of selection sets $\mathcal{S}$;
we then overload our notation, writing $G = (T;  \mathcal{S})$ and defining
\[
  \alpha(G) = \Exp_{\eta}[\alpha(\eta, T; \mathcal{S})]
\]
We also define the random variable $X_\Adversary$ as the set of strings 
that are tested for membership in the target set 
by a player $\Adversary$ playing the game with an uniform initial string.
It will be clear from context whether we are considering games with
predetermined or uniform initial values.

\begin{theorem}\label{thm:xor-game-basic}
  Let $G = (T;  \mathcal{S})$ be an XOR
  game with uniform initial value and let $\NumSets = |\mathcal{S}|$. Then
  \begin{equation}\label{eq:basic-XOR-bound}
    \alpha(T;  \mathcal{S}) \leq \frac{|T|}{2^\kappa}\, \GrindingPower\,,
    \qquad\text{where}\qquad
    \GrindingPower = 1 + \sum_{t=1}^{\NumSets}
      \prod_{i = t}^{\NumSets} |S_i|
  \end{equation}
  with the convention that $\GrindingPower = 1$ when $\NumSets = 0$.
  The quantity $\GrindingPower$ is an upper bound on $|X_\Adversary|$ and it is independent of $T$.
\end{theorem}

\begin{proof}
  For a game $G = (T;  \mathcal{S})$ with a uniform initial value $\eta_0$, define
  \begin{align*}
    \Lambda(G) &= \Lambda(T; \mathcal{S}) = \{ \eta_0 \in \{0,1\}^\kappa \Given \alpha(\eta_0, T; \mathcal{S}) = 1\}\,,\\
    \lambda(G) & = \lambda(T; \mathcal{S}) = \Pr_{\eta_0}[\eta_0 \in \Lambda] = \frac{|\Lambda|}{2^\kappa}\,,\quad\text{and}\\
    \mu(G) &= \mu(T; \mathcal{S}) = \max_{\eta_0 \not\in \Lambda} \alpha(\eta_0, T;  \mathcal{S})\,.
  \end{align*}
  Thus $\lambda(T; \mathcal{S})$ is the probability that $\eta_0$
  is a ``sure success'' for the player (given $T$ and $\mathcal{S}$); in particular, he 
  does not need to use/see the random strings $r_i, i \in [\NumSets]$ presented in the game. 
  If $\eta_0$ is not a sure success, however, 
  it may still be possible for the player to ``steer'' it to the set $\Lambda$ using 
  the random strings; this probability is $\mu(T; \mathcal{S})$. 
  In addition, observe that $\alpha(G \Given \eta_0 \in \Lambda)$ is in fact $\lambda(G)$ 
  and that $\alpha(G \Given \eta_0 \not \in \Lambda)$ is at most $\mu(G)$.
  
  
  The initial string $\eta_0$ either belongs to the set $\Lambda$ or lies outside it. 
  Conditioned on the former event, which happens with probability $\lambda(G)$, $\alpha(G)$ equals $1$. 
  In the complementary event, however, 
  $\alpha(G)$ is just $\mu(G)$.
  Therefore, 
  \[
    \alpha(G) \leq \lambda(G)\cdot 1 + (1 - \lambda(G)) \cdot \mu(G) \leq \lambda(G) + \mu(G)\,,
  \]
  so it suffices to establish appropriate upper bounds on the two
  quantities $\lambda(G)$ and $\mu(G)$. 
%   These quantities satisfy a simple
%   inductive relationship that directly yields the
%   bound~\eqref{eq:basic-XOR-bound}. First, note that if $\NumSets = 0$, so
%   that $\mathcal{S} = \epsilon$, then
%   \[
%     \lambda(T;  \epsilon) = |T|/2^\kappa \qquad\text{and}\qquad \mu(T;  \epsilon) = 0\,.
%   \]
%   In general, we will establish the inductive relations
%   \begin{align*}
%     \lambda(T;  \mathcal{S}) &\leq |S_1| \cdot \lambda(T; \mathcal{S}_{-1})\,,\quad\text{and}\\
%     \mu(T;  \mathcal{S}) &\leq \mu(T;  \mathcal{S}_{-1}) + \lambda(T; \mathcal{S}_{-1})\,.
%   \end{align*}
  
  Our immediate goal is to establish a recursive relation describing $\mu$ and $\lambda$ in terms of 
  the length of the list $\mathcal{S}$. 
  For the sake of clarity, let us write $\lambda_t \triangleq \lambda(T; \mathcal{S}[\NumSets - t + 1:\NumSets])$
  for $t \in [\NumSets]$. Analogously, let us write $\mu_t$. 
  We designate the special cases $\lambda_0 = \lambda(T; \EmptyList)$ and $\mu_0 = \mu(T; \EmptyList)$.
  We start off by observing that if $\NumSets = 0$, so that $\mathcal{S} = \EmptyList$, then
  \[
    \lambda_0 = |T|/2^\kappa \qquad\text{and}\qquad \mu_0 = 0\,.
  \]
  For a fixed $T$ and $\mathcal{S}$, 
  let $\Lambda_t \triangleq \mathcal{S}[\NumSets - t + 1:\NumSets] \oplus T$ for $n \in [\NumSets]$, and 
  we use the convention that $\Lambda_0 = T$.
  Note that $\Lambda_\NumSets = \Lambda(G)$ and in addition, that 
  $|\Lambda_t| \leq |S_{\NumSets - t + 1}| \cdot |\Lambda_{t-1}|$ for $t \in [\NumSets]$. 
  Since $\eta_0$ is uniform, so is $\eta_0 + s$ for any $s \in S_1$.
  This implies $\Pr[\eta_0 \in \Lambda_\NumSets] \leq |S_1|\, \Pr[(\eta_0 + s) \in \Lambda_{\NumSets - 1}]$, 
  which gives us the recursive relation 
  \[
    \lambda_\NumSets \leq |S_1| \cdot \lambda_{\NumSets-1}
    \,.
  \]
  By ``unrolling'' the inductive expression for $\lambda_\NumSets$, 
  we get 
  \[
  \lambda_t 
  = \lambda_0 \, \prod_{i=0}^{t-1}{|S_{\NumSets - i}|} 
%   = \lambda_0 \, \prod_{i = \NumSets - t + 1}^{\NumSets}{|S_{i}|}
  \qquad\text{for $t \in [\NumSets]$}
  \, .
  \] 
  

  Conditioning on the event that $\eta_0 \not \in \Lambda_\NumSets$, 
  \begin{align*}
    \mu_\NumSets
    &= \alpha(G \Given \eta_0 \not \in \Lambda_\NumSets) \\
    &= \Exp_r\, \max_{s \in S_1 \Union\{r\} } \alpha(\eta_0 + s, T; \mathcal{S}_{-1} \Given \eta_0 \not \in \Lambda_\NumSets) \\
    &= \Pr[(\eta_0 + r) \in \Lambda_{\NumSets-1}] \cdot 1 +
    \Pr[(\eta_0 + r) \not \in \Lambda_{\NumSets - 1}] \cdot \max_{s \in S_1 } \alpha(\eta_0 + s, T; \mathcal{S}_{-1} \Given \eta_0 \not \in \Lambda_\NumSets) \\
    &= \lambda_{\NumSets - 1} \cdot 1 + (1 - \lambda_{\NumSets - 1}) \cdot \mu_{\NumSets - 1}\\
    &\leq \lambda_{\NumSets - 1} + \mu_{\NumSets - 1}
    \, .
  \end{align*}
  By unrolling this recursive relation,
  we have $\mu_1 = \lambda_0 + \mu_0 = \lambda_0$, $\mu_2 = \lambda_1 + \lambda_0$ 
  and in general, $\mu_\NumSets = \sum_{i=0}^{\NumSets-1}{\lambda_i}$. 
  This means, 
  \begin{align*}
    \alpha(G) &\leq \mu_\NumSets + \lambda_\NumSets 
    \leq \sum_{t=0}^\NumSets{ \lambda_t } 
    = \lambda_0 \left( 1 + \sum_{t=1}^\NumSets{ \prod_{i=0}^{t-1}{|S_{\NumSets - i}|} } \right) 
    % &= \lambda_0 \left( 1 + \sum_{t=1}^\NumSets{ \prod_{i=t}^{n}{|S_{i}|} } \right) 
    \,.
  \end{align*}
 The bound in the claim follows by reversing the indexing in the sum and using $\lambda_0 = |T|/2^\kappa$.
\end{proof}
\begin{remark}
Let $H = \{i \in [\NumSets] \SuchThat |S_i| = 0 \}$ and suppose $|H| > 0$.
Let $h \triangleq \max(H)$.
If $\NumSets \geq 1$, the expression in Theorem~\ref{thm:xor-game-basic} reduces to 
\[
    \alpha(G) \leq |T|/{2^{\kappa}} \cdot \sum_{t=h+1}^\NumSets \prod_{i=t}^{\NumSets}{|S_{i}|}
    \,.
\]
\end{remark}


%==========================================================
\subsection{The $\BlockLength$-block XOR target game}\label{sec:xor-game-block}
We are given a list of sets $\mathcal{S}$ of length $\NumSets$ and an integer $\BlockLength$ 
assuming $\BlockLength$ divides $\NumSets$, let $\NumSync = \NumSets/\BlockLength$.
We define the $\NumSync$-round \emph{$\BlockLength$-block XOR target game}
\[
    G = G_{\kappa, \BlockLength}(\eta_0, T; \mathcal{S})
\]
as follows where $\eta_0$ is the initial string and $T$ is the target set. 
The game maintains the invariant 
that $|\mathcal{S}|$ is either $0$ or a multiple of $\BlockLength$. 
\begin{itemize}
    \item If $| \mathcal{S} | = 0$, the player wins if and only if $\eta_0 \in T$.
    \item Otherwise, if $|\mathcal{S}| = \BlockLength$, 
    the player wins if and only if $\left( \eta_0 \oplus S_1 \oplus \cdots \oplus S_\BlockLength \right) \Intersect T \neq \emptyset$.
    \item Otherwise, the player is presented with 
    $\BlockLength$ uniformly random strings $r_i \in \{0,1\}^\kappa$ for $i \in [\BlockLength]$
    and he must select $\BlockLength$ strings 
    $s_i \in S_i \Union \{r_i\}$ for each $i = 1, 2, \cdots, \BlockLength$.
    Finally, the game begins anew as 
    \[
        G(\eta, T; \mathcal{S}_{-\BlockLength} )
    \]
    where $\eta = \eta_0 \oplus s_1 \oplus \cdots \oplus s_\BlockLength$.
\end{itemize}
As before, we define a quantity $\alpha(G)$ associated with this game as follows. 
% Let $\mathcal{S} \triangleq (F_1 \Union \{r_1\}, \cdots, F_\BlockLength \Union \{r_\BlockLength\})$.
If $|\mathcal{S}| = 0$, define
\begin{align*}
    \alpha(G) &= 
    \begin{cases}
        1\,, & \quad\text{if}\quad \eta_0 \in T\, , \\
        0\,, & \quad\text{otherwise}\, .
    \end{cases}
    % \,, \qquad\text{if $|\mathcal{F}| = 0$}
    % \,,
    % \\
\end{align*}
Otherwise, i.e., if $|\mathcal{S}| > 0$, define
\begin{align*}
    \alpha(G)&=
    \Exp_{\substack{r_1, \cdots, r_\BlockLength \\ r_i \leftarrow_\$ \{0,1\}^\kappa}} \left[ 
        \max_{\substack{s_i \in S_i \Union \{r_i\} \\ i \in [\BlockLength]} } 
        \alpha\left((\eta_0 \oplus s_1 \oplus \cdots s_\BlockLength), T; \mathcal{S}_{-\BlockLength} \right) 
    \right]
    % \,, \qquad\text{otherwise}
    \,.
\end{align*}
As before, we are interested in the situation where 
$\eta_0$ is uniform in $\{0, 1\}^\kappa$. 
In that case, we overload the notation $G$ and write
$G = G(T;\mathcal{S})$ and similarly, write
\[
    \alpha(T;\mathcal{S}) = \Exp_{\eta_0 }\, \alpha(\eta_0, T;\mathcal{S})
    \,.
\]
We also define the random variable $X_\Adversary$ as the set of strings 
that are tested for membership in the target set 
by a player $\Adversary$ playing the game with an uniform initial string.


% For the theorem below, we define the notation $\mathcal{S}[t;\BlockLength]$ as follows. 
% Given a list $\mathcal{S}$ of length $\NumSets$ and assuming $\BlockLength$ divides $\NumSets$, 
% let $\mathcal{S}[t; \BlockLength]$ denote the $t$-th block of $\mathcal{S}$, 
% with block-size $\BlockLength$, for $t \in [\NumSets/\BlockLength]$, i.e., 
% \[
%     \mathcal{S}[t; \BlockLength] 
%     = \mathcal{S}[(t-1)\BlockLength +1\, : \, t \BlockLength] 
%     = (S_{(\NumSets/\BlockLength - t)\BlockLength +1}, \ldots, S_{(\NumSets/\BlockLength - t + 1) \BlockLength} )
%     \,.
% \]
\begin{theorem}\label{thm:xor-game-block}
  Let $G = (T; \mathcal{S})$ be an $\BlockLength$-block XOR
  game with uniform initial value. Let $\NumSets = |\mathcal{S}|$.
  Then
% \frac{|T|}{2^\kappa}\,\underbrace{\sum_{t=1}^{\NumSets}
    %   \prod_{i = t}^{\NumSets} |S_i|}_{(\dagger)}  
  
  \begin{align*}%\label{eq:XOR-bound-block}
    \alpha(T; \mathcal{S}) \leq \frac{|T|}{2^\kappa}\, \GrindingPower\,,
    \qquad\text{where}\qquad
    \GrindingPower =
        \prod_{i=1}^{\NumSets} |S_i| \cdot
            \sum_{t=1}^{\NumSets/\BlockLength}\,
                % \prod_{S \in \mathcal{S}[t; \BlockLength]} (|S| + 1)
                \prod_{j=1}^{\BlockLength} \left(|S_{ (t-1) \BlockLength + j}| + 1 \right)
  \end{align*}
  with the convention that $\GrindingPower = 1$ when $|\mathcal{S}| = 0$.
  The quantity $\GrindingPower$ is an upper bound on $|X_\Adversary|$ and it is independent of $T$. 
  Moreover, if at least one of the sets $S_1, \ldots, S_n$ is empty, 
  \begin{align}\label{eq:xor-game-block-gp}
      \GrindingPower &= 
        \prod_{j=\NumSets-h\BlockLength +1}^{\NumSets - h \BlockLength + \BlockLength}{(|S_j| + 1)} \cdot
        \prod_{i = \NumSets - h \BlockLength + \BlockLength + 1}^{\NumSets} |S_i|
        \,,
  \end{align}
  where $h = \min_{i \in [n]}\{\lceil (n - i + 1)/ \ell \rceil \SuchThat |S_i| = 0\}$.
\end{theorem}
\begin{proof}
Let $\NumSync = \NumSets/\BlockLength$ and 
consider a round $t$ of the game where $t = m, (m-1), \ldots, 1$. 
For a fixed $t$, define 
\[
    p = (\NumSync - t)\BlockLength +1\,, 
    \qquad \text{and}\qquad 
    q = (\NumSync - t + 1)\BlockLength
    \,.
\]
Fix $T$ and $\mathcal{S}$, and consider the Cartesian product 
\[
    P_t = \left(S_{p} \Union \{r_1\}\right) \times \cdots \times 
    \left(S_{q}  \Union \{r_\BlockLength\}\right)
    \,,
\] 
where $S_i \in \mathcal{S}[(t-1)\BlockLength +1\, : \, t \BlockLength]$ for $p \leq i \leq q$ and the 
random strings $r_1, \ldots, r_\BlockLength$ were presented at round $t$ of the game.
For the exposition below, let $a_t$ denote the total number of tuples in $P_t$, 
$b_t$ denote the number of tuples with no random component, 
and $c_t$ denote the number of tuples with at least one random component. 
It is easy to see that 
\begin{align}\label{eq:a-b-c-tuplecount}
    a_t = \prod_{i=p}^q (|S_i| + 1)\,,\qquad 
    b_t = \prod_{i=p}^q {|S_i|}\,,\qquad\text{and}\qquad
    c_t = a_t - b_t
    \,.
\end{align}
Define 
\[
    \Lambda_t \triangleq 
    T \oplus \left( 
        S_{p} \oplus 
        S_{p + 1} \oplus 
        \cdots \oplus 
        S_{\NumSets} \right)
        \,.
\]
Thus $\Lambda_\NumSync = \Lambda(G)$ and we use the convention that $\Lambda_0 = T$. 
Now it is easy to see that $|\Lambda_t| \leq b_t \cdot |\Lambda_{t-1}|$. 
Next, let us define $\lambda_t \triangleq \lambda(T; \mathcal{S}_{-t \BlockLength})$. 
Define $\mu_t$ in a similar way. 
As before, if $\NumSets = 0$, so that $\mathcal{S} = \EmptyList$, then
\[
    \lambda_0 = \lambda(T; \EmptyList)=|T|/2^\kappa 
    \qquad\text{and}\qquad 
    \mu_0 = \mu(T; \EmptyList) = 0
    \,.
\]


% the game proceeds in rounds $t = 1, 2, \cdots, \NumSets/\BlockLength$.
The proof follows the same rationale as the proof of Theorem~\ref{thm:xor-game-basic} 
and yields a similar recursive relation for $\lambda$ and $\mu$. 
Specifically, the current game will proceed in rounds $1, \ldots, \NumSync$ whereas the previous game 
used to proceed in rounds $1, \ldots, \NumSets$.
For $\lambda_\NumSync$ in our case, the player has $b_\NumSync$ non-random choices 
which has the same semantics as the $|S_1|$ factor in the previous game. 
Similarly, for $\mu_\NumSync$ in our case, the player has $c_\NumSync$ random choices 
(which can potentially steer $\eta_0$ into the set $\Lambda_{\NumSync - 1}$) along with some 
non-random choices, each giving him (by definition) at most $\mu_{\NumSets - 1}$ success probability.
(Previously, there was only one random choice, $r$.)
Concretely, by proceeding as we did in the previous game, we get
\begin{align*}    
    \lambda_\NumSync
    &\leq 
        b_\NumSync \cdot 
        \lambda_{\NumSync - 1}
    \,, \\
    \mu_\NumSync
    &\leq 
        c_\NumSync \cdot \lambda_{\NumSync - 1} + \mu_{\NumSync - 1} 
    \,,
\end{align*}
with the base cases $\lambda_0 = |T|/2^\kappa$ and $\mu_0 = 0$. 
The factor $c_\NumSync$ above represents a union bound over all random choices.
By unrolling as before, this becomes 
\[
    \lambda_\NumSync \leq \lambda_0\,\prod_{t=1}^\NumSync b_t\,,\qquad 
    \mu_\NumSync \leq \sum_{t=1}^{\NumSync}c_t \lambda_{t-1}
    \,.
\]
Writing $\alpha(G)$ as $\alpha_m$,
\begin{align}
\alpha_m
&\leq \lambda_m + \mu_m 
= \lambda_0 \prod_{t=1}^\NumSync b_t + \lambda_0 \sum_{t=1}^\NumSync c_t\, \prod_{u = 1}^{t - 1}b_u \label{eq:gp-block-xor-general}\\
&\leq \lambda_0 \prod_{t=1}^\NumSync b_t + \lambda_0 \prod_{u = 1}^{\NumSync - 1} b_u \,\sum_{t=1}^\NumSync c_t 
= \lambda_0 \prod_{t=1}^\NumSync b_t + \lambda_0 \prod_{u = 1}^{\NumSync - 1} b_u \,\sum_{t=1}^\NumSync (a_t - b_t) \nonumber \\
&\leq \lambda_0 \prod_{u = 1}^{\NumSync} b_u \,\sum_{t=1}^{\NumSync-1} a_t 
= \frac{|T|}{2^\kappa}\cdot \prod_{i = 1}^{\NumSets} |S_i| \cdot \sum_{t=1}^{\NumSync-1} \prod_{S \in \mathcal{S}_t} (|S| + 1)
\nonumber \,.
\end{align}

For the ``moreover'' part, condition on the event that there exists some $i \in [n]$ so that $|S_i| = 0$. 
This starts off a cascade of events: since $b_h = 0$, we have $\lambda_t = 0, \mu_t = \mu_h$ for $t \geq h$, 
and $\mu_h = c_h \lambda_{h-1}$. 
Thus the expression in~\eqref{eq:gp-block-xor-general} simplifies as
\begin{align*}
    \alpha_\NumSync 
    &\leq \mu_h \leq c_h \lambda_{h-1} = \lambda_0\, c_h\, \prod_{t = 1}^{h - 1}b_t 
    = \frac{|T|}{2^\kappa}\cdot 
        \prod_{j=\NumSets-h\BlockLength +1}^{\NumSets - h \BlockLength + \BlockLength}{(|S_j| + 1)} \cdot
        \prod_{i = \NumSets - h \BlockLength + \BlockLength + 1}^{\NumSets} |S_i|
\end{align*}
by using the upper bound $c_h \leq a_h$ and using~\eqref{eq:a-b-c-tuplecount}.
% Since $\Pr[|S_i| = 0] = 1-\alpha$, it follows that $\Pr[H = b] \leq (1 - \alpha^\ell)\alpha^{(m - b)\ell} $


\end{proof}


\subsection{Proof of Theorem~\ref{thm:xor-game-lookahead}}
In what follows, we will be using the script notation $\mathcal{S}$ to denote 
an \emph{ordered list of sets} $\mathcal{S} = (S_1, S_2, \ldots)$. 
$\EmptyList$ denotes an empty list. 
For any list $\mathcal{S}$ and positive integers $a, b$ satisfying $1 \leq a \leq b \leq |\mathcal{S}|$, 
we denote a segment of $\mathcal{S}$ using the notation 
$\mathcal{S}[a:b] = (S_a, S_{a+1}, \ldots, S_b)$. 
In particular, $\mathcal{S}[a,a] = S_a$ and $\mathcal{S}[a,b] = \EmptyList$ for improper indices $a,b$.
We use a negative subscript $\mathcal{S}_{-t}$ as a shorthand for the segment $\mathcal{S}[t+1 : |\mathcal{S}|]$.
In particular, $\mathcal{S}_{-1} = (S_2, \ldots, S_{|\mathcal{S}|})$, $\mathcal{S}_0 = \mathcal{S}$, and 
$\mathcal{S}_{|\mathcal{S}|} = \EmptyList$.
We also extend the binary operator $\oplus$ for strings, sets, and lists of sets. 
Specifically, for two sets $A,B \subset \{0,1\}^\kappa$, a string $\eta \in \{0,1\}^\kappa$, 
and a list $\mathcal{S} = (S_1, S_2, \ldots)$, we write
$\eta \oplus A = \{\eta \oplus a \SuchThat a \in A\}$, 
$A \oplus B = \{a \oplus b \SuchThat a \in A, b \in B\}$, 
$\eta \oplus \mathcal{S} = \eta \oplus S_1 \oplus \cdots \oplus S_{|\mathcal{S}|}$, and so on.




\begin{proof}
\renewcommand{\S}{\mathcal{S}}
\renewcommand{\P}{\mathcal{P}}
\renewcommand{\SS}[1]{S_{#1}, \ldots, S_{\BlockLength}}
\renewcommand{\PP}[1]{P_{#1}, \ldots, P_{\NumSets}}
\renewcommand{\HatPP}[2]{\hat{P}_{#1}, \ldots, \hat{P}_{#2}}
\renewcommand{\Sfull}{ S_1, \ldots, S_\BlockLength}
\renewcommand{\Pfull}{ P_1, \ldots, P_\NumSets}

For a game $G = (T; \Sfull; \Pfull )$ with a uniform initial value $\eta_0$, define
\begin{align*}
    \Lambda(G) &= \Lambda(T ; \Sfull; \Pfull) = \{ \eta_0 \in \{0,1\}^\kappa \Given \alpha(\eta_0, T; \Sfull; \Pfull) = 1\}\,,\\
    \lambda(G) & = \lambda(T; \Sfull; \Pfull) = \Pr_{\eta_0}[\eta_0 \in \Lambda] = \frac{|\Lambda|}{2^\kappa}\,,\quad\text{and}\\
    \mu(G) &= \mu(T; \Sfull; \Pfull) = \max_{\eta_0 \not\in \Lambda} \alpha(\eta_0, T;  \Sfull; \Pfull)\,.
\end{align*}
Thus $\lambda(T; \Sfull; \Pfull)$ is the probability that $\eta_0$
is a ``good initial string'' for the player in that
\[
    \eta_0 \oplus 
    (s_1 \oplus \cdots \oplus s_\BlockLength) 
    (p_1 \oplus \cdots \oplus p_\NumSets)
    \in T
    \,,
\]
where $s_i \in S_i$ and $p_i \in P_i$.
A good initial string guarantees success for the player because
he does not need to use or see the random strings $r_i, i \in [\NumSets]$ presented during the game. 
If $\eta_0$ is not a good string, however, 
it may still be possible for the player to ``steer'' it to the set $\Lambda$ by utilizing 
the random strings in a future step; by definition, 
his chance of success with this $\eta_0$ is at most $\mu(G)$. 
% In other words, $\alpha(G \Given \eta_0 \in \Lambda)$ is in fact $\lambda(G)$ 
% and $\alpha(G \Given \eta_0 \not \in \Lambda)$ is at most $\mu(G)$.


% The initial string $\eta_0$ either belongs to the set $\Lambda$ or lies outside it. 
Conditioned on the event that $\eta_0 \in \Lambda(G)$, 
which happens with probability $\lambda(G)$, 
$\alpha(G)$ equals $1$ by definition. 
In the complementary event $\eta_0 \not \in \Lambda(G)$, however, 
$\alpha(G)$ is at most $\mu(G)$ by definition.
Therefore, 
\begin{align}\label{eq:alpha-lambda-mu-lookahead}
    \alpha(G) \leq \lambda(G)\cdot 1 + (1 - \lambda(G)) \cdot \mu(G) \leq \lambda(G) + \mu(G)\,,
\end{align}
so it suffices to establish appropriate upper bounds on the two
quantities $\lambda(G)$ and $\mu(G)$. 


Our immediate goal is to establish a recursive relation describing $\mu(G)$ and $\lambda(G)$ in terms of 
the the lists $\Sfull$ and $\Pfull$. 
First, let us condition on the event that $\eta_0 \in \Lambda(G)$. 
There must be some $s \in S_1$ such that $(\eta_0 + s) \in \Lambda(T; \SS{2}, P_1; \PP{2})$.
% \[
%     \eta_0 \oplus s \oplus
%     (S_2 \oplus \cdots \oplus S_\BlockLength) 
%     \oplus s \oplus 
%     (P_1 \oplus \cdots \oplus P_\NumSets)
%     \Intersect T \neq \emptyset
%     \,.
% \]
% The optimal player always selects this $s$. 
Since there $|S_1|$ choices for $s$, we deduce that
\[
    \Lambda(T; \Sfull; \Pfull) \,\subset\, S_1 \oplus \Lambda(T; \SS{2}, P_1; \PP{2})
    \,,
\]
or equivalently, the recursive definition
\[
    \lambda(T; \Sfull; \Pfull) \leq |S_1| \cdot \lambda(T;\, \SS{2}, P_1\,;\, \PP{2})
\]
with the base case 
\[
    \lambda(T; \Sfull;\, \EmptyList) = \prod_{i=1}^\BlockLength |S_i|
    \,.
\]
By unrolling the recursion of $\lambda$ above, we conclude that
\begin{equation}\label{eq:lambda-expr-lookahead}
    \lambda(T; \Sfull; \Pfull) 
    \leq \prod_{i=1}^\BlockLength |S_i| \cdot 
        \prod_{i=1}^{\NumSets}|P_i| \cdot 
        \frac{|T|}{2^\kappa}
        \,.
\end{equation}


Conditioning on the event that $\eta_0 \not \in \Lambda(G)$, 
we observe that the player can in fact be lucky. 
Specifically, after seeing the random string $r$, 
an optimal player can check if the uniform string $\eta_0 + r$ 
is a good initial string for the game $G^\prime = G(\eta_0 \oplus r\,;\, T;\,\SS{1}; \PP{2})$. 
If so, (i.) he selects the same $s \in S_1$ in the current game as well as in $G^\prime$, 
and (ii.) in future---when he would be committing to some $s^\prime \in P_1 \Union \{r\}$---he selects $r$. 
% (This is possible since the XOR operation is symmetric in its arguments.)
In this lucky event, his success probability is at most $\lambda(T;\SS{1}; \PP{2})$. 
Barring this lucky event, however, it does not matter which $s \in S_1$ the player chooses: 
his success probability, by definition, is at most $\mu(T;\,\SS{2}, P_1 \Union \{r\}\,;\, \PP{2})$.
In addition, the probability of this unlucky event is at most one. 
Therefore, using the law of total probability, 
\begin{align}\label{eq:mu-expr-lookahead}
    \mu(T; \Sfull; \Pfull)
    &\leq \lambda(T\,;\,\SS{1}\,;\, \PP{2})\cdot 1 \nonumber\\
    &\qquad +\,1\cdot \mu(T\,;\,\SS{2}, P_1 \Union \{r\}\,;\, \PP{2}) \nonumber\\
    &=      \sum_{k=1}^\BlockLength \lambda(T\,;\,\HatPP{1}{k-1}, \SS{k}\,;\, \PP{k+1}) \nonumber \\
    &\quad +  \sum_{k=\BlockLength + 1}^\NumSets \lambda(T\,;\,\HatPP{k-\BlockLength}{k-1}\,;\, \PP{k+1})
\, ,
\end{align}
with the convention that any list $S_a, \ldots, S_b$ with $a > b$ is an empty list 
and $\hat{P_i} = P_i \Union \{r_i\}$ where $r_i$ is the random string associated with $P_i$. 
To arrive at the expression above, 
we observed that when $\NumSets = 0$, we have $\mu(T; \Sfull; \EmptyList) = 0$ 
since in that case, there is no way the player can win this game. 

The claim in Theorem~\ref{thm:xor-game-lookahead} follows by 
using the upper bounds~\eqref{eq:lambda-expr-lookahead} and ~\eqref{eq:mu-expr-lookahead}
in the expression for $\alpha(G)$ in~\eqref{eq:alpha-lambda-mu-lookahead}.


% \begin{align*}
%     A
%     &= \sum_{k=1}^\BlockLength \lambda(T\,;\,\HatPP{1}{k-1}, \SS{k}\,;\, \PP{k+1}) \\
%     &= \sum_{k=1}^\BlockLength \prod_{i=k+1}^\NumSets |P_i| \cdot \left(\prod_{i=1}^{k-1}(1 + |P_i|)\right) \cdot \prod_{i=k}^\BlockLength |S_i| \cdot \frac{|T|}{2^\kappa}\\
%     &\leq \ell \cdot \prod_{i=1}^\NumSets |P_i| \cdot \left(\prod_{i=1}^{\BlockLength-1}(1 + |P_i|)\right) \cdot \prod_{i=1}^\BlockLength |S_i| \cdot \frac{|T|}{2^\kappa}
%     \,,\quad\text{and}\\
%     B 
%     &= \sum_{k=\BlockLength + 1}^\NumSets \lambda(T\,;\,\,;\,\HatPP{k-\BlockLength}{k-1}\,;\, \PP{k+1}) \\
%     &= \sum_{k=\BlockLength+1}^\NumSets \prod_{i=k+1}^\NumSets |P_i| \cdot \left(\prod_{i=k - \BlockLength + 1}^{k}(1 + |P_i|)\right) \cdot \frac{|T|}{2^\kappa}\\
%     &\leq (\NumSets - \BlockLength)\cdot \prod_{i=1}^\NumSets |P_i| \cdot \max_{k \geq \BlockLength + 1} \prod_{i = k - \BlockLength + 1}^{k}(1 + |P_i|) \cdot \frac{|T|}{2^\kappa}
%     \,.
% \end{align*}
\end{proof}


% The expressions of the factor $\GrindingPower$ in the XOR target games described above 
% depends only on the \emph{cardinality} of the sets 
% involved. 
% Thus it makes sense to consider the special case when the size of these sets
% are independent and identically distributed which gives a simplified expression 
% for $\GrindingPower$.




% \subsection{Grinding power in an XOR target game}
% An upper bound on $\GrindingPower$---the cardinality of the set $X_\Adversary$---would provide 
% a lower bound on $H_\infty(X_\Adversary)$. 
% We call this quantity the ``grinding power'' of $\Adversary$.

% \begin{definition}[Grinding set and grinding power in the XOR target games]\label{def:grinding-power-xor}
% Suppose a player $\Adversary$ plays an XOR target game with an uniform initial string.
% Let the random variable $X_\Adversary$ be the set of strings 
% which could be tested by $\Adversary$ for membership in the target set. 
% $X_\Adversary$ is called the \emph{grinding set} of $\Adversary$, 
% and its cardinality $\GrindingPower \defeq |X_\Adversary|$ is called the \emph{grinding power} of $\Adversary$.
% \end{definition}

% A central goal of this paper is to give tight upper bounds for the grinding power 
% in different settings of interest.

% We define the ``grinding power'' of the player in the first XOR target game 
% $G_\kappa(\eta_0, T; \mathcal{S})$ as
% \[
%     g = g(\mathcal{S}) = \prod_{i=1}^{|\mathcal{S}|} (|S_i| + 1)
%     \,.
% \]
% Similarly, we define the grinding power of the player in the $\BlockLength$-block XOR target game 
% $G_{\kappa, \BlockLength}(\eta_0, T; \mathcal{S})$ as
% \[
%     g = g(\mathcal{S}) = \max_{t \in [\NumSync]} \prod_{i=1}^{|\mathcal{S}|} (|S_i| + 1)
%     \,.
% \]

