
\newcommand{\FuncBeacon}{\mathcal{F}_\mathsf{beacon}}
\newcommand{\ProtocolBeacon}{\mathcal{\Pi}_\mathsf{beacon}}
% \newcommand\Litem[1]{\item{\bfseries #1\enspace}}




\subsection{Collective coin-flipping}

    A \emph{coin-flipping protocol $\Pi$ of dimension $\kappa$} 
    takes, as input, a uniformly random $\kappa$-bit string $\eta_0$
    and outputs a $\kappa$-bit string $\eta$. 
    The goal of the protocol is to make the distribution of $\eta$ 
    as close to uniform as possible.
    
    \begin{definition}[Coin-flipping with $(\epsilon, \rho)$-security]\label{def:coin-flipping-security}
        Let $\kappa \in \NN, \rho, \epsilon \in \RR, \rho \in [0, \kappa]$, and $\epsilon \in (0,1)$.
        Let $\Pi$ be an coin-flipping protocol with dimension $\kappa$ 
        and output $\eta$. 
        We say that \emph{$\Pi$ is $(\epsilon, \rho)$-secure} if 
        $\MinEntropy(\eta) \geq \kappa - \rho$ 
        except with probability $\epsilon$. 
        Here, $\MinEntropy$ is the min-entropy function.
    \end{definition}
    
    Suppose a coin-flipping protocol $\Pi$ is $(\epsilon, \rho)$-secure. 
    Then 
    the min-entropy loss of its output distribution is at most $\rho$. 
    On the other hand, let $T$ be an arbitrary set of $\kappa$-bit Boolean strings. 
    The probability that $\eta \in T$ is at most $2^\rho \cdot |T| \cdot 2^{-\kappa}$. 
    The amplification factor $2^\rho$ can be interpreted as 
    the adversarial participants 
    choosing the designated output $\eta$ from 
    a set containing $2^\rho$ uniformly random strings; 
    {\color{red} strings in this set may be correlated with each other}. 
    An unbiased protocol will necessarily have $\rho = 0$.


    \paragraph{Protocol participants.} 
    Let $\Players$ be the set of protocol participants. 
    With each party $u \in \Players$ is associated a positive real $\sigma_u \in [0, 1]$ 
    so that $\sum_{u \in \Players} \sigma_u = 1$. 

    \paragraph{The adversary.} 
    Let $\alpha \in [0,1]$. 
    The adversary can corrupt any player instantly subject to 
    a budget $\alpha$: that is, 
    at any moment, the set of players $A \subset \Players$ controlled by the adversary 
    always satisfy $\sum_{u \in A} \sigma_u \leq \alpha$. 
    This adversary is called an \emph{$\alpha$-dominated adversary}. 
    If a player is ever corrupted by the adversary, 
    he remains under his control for the rest of the protocol. 
    The adversary is Byzantine: 
    once he controls a player, he can send arbitrary messages on behalf of the player.

    \paragraph{Communication model.} 
    We assume a synchronous communication where 
    the adversary is responsible for delivering messages in any order he chooses.

    \begin{definition}[$(\ell, n)$-coin-flipping with dimension $\kappa$]\label{def:coin-flipping}
        Let $\ell, n, \kappa \in \NN$ and 
        let $\eta_0 \in \{0,1\}^\kappa$ be a uniformly random \emph{initial value}. 
        The protocol has $r = \ell + n$ rounds. 
        \begin{itemize}
            \item (Committees.) 
            With every round $i \in [r]$ is associated 
            a \emph{committee} $C_i$ of players. 
            (The committee election scheme is left unspecified at this moment. 
            See Figure~\ref{fig:leader-election-schemes} below.)
            The committee membership is private unless 
            a member discloses it. 

            \item (Nonces.) 
            Before the protocol commences, 
            for every $i \in [r]$, 
            each member $u \in C_i$ is given a \emph{nonce} $y_{i,u} \in \{0,1\}^\kappa$. 
            The nonces of the honest members are uniformly random. 
            The adversarial members may choose their nonces arbitrarily at this point 
            but once the protocol commences, 
            they cannot change the value of their nonces. 

            \item (Adversarial knowledge and the lookahead.) 
            The adversarial players know, in advance, 
            about all adversarial committe members and 
            their nonces across all rounds.  
            They also know which rounds contain an honest committee member.
            Furthermore, at round $i$, 
            adversarial players can see 
            the honest nonces for rounds $i, i+1, \ldots, \min\{i + \ell - 1, r\}$.

            \item (Announcement.)
            With each round $i$ is associated 
            a publicly visible set $Y_i$, initially empty. 
            At round $i$, 
            an honest committee member inserts his nonce into $Y_i$. 
            An adversarial committee member $u$ may strategically choose to insert or not. 

            \item (Output.)
            At the end of round $r$, 
            let $m_i$ 
            be the lexicographic minimum of all strings recorded in $Y_i, i \in [r]$.
            (If $Y_i$ is empty then set $m_i = 0^\kappa$.)
            The \emph{output} of the protocol is 
            $
                \eta = \eta_0 \oplus m_1 \oplus \cdots m_{\ell + n}
                % \,.
            $. 
        \end{itemize}
    \end{definition}

    Since the adversary cannot change the nonces, 
    all he can do to bias the output 
    is to strategically vary the size of the sets $Y_i, i \in [\ell + n]$.

    \begin{figure}[h]
        \begin{framed}
            \begin{center}
                Two committee election schemes                
            \end{center}
            \begin{enumerate}[label=\textbf{Scheme \Alph*:},ref=\Alph*,leftmargin=6em]
                \item \label{lottery:public}
                (\textbf{Singleton committee.})
                All players use a common random string to 
                publicly sample, 
                independently for each round $i \in [\ell + n]$, 
                a single player $u_i \in \Players$   
                with probability $\sigma_u$ and 
                set $C_i = \{u_i\}$. 

                \item \label{lottery:private}
                (\textbf{Private lottery.})
                Independently for each round $i \in [\ell + n]$, 
                player $u \in \Players$ 
                independently inserts himself 
                into the committee $C_i$ 
                with probability $\sigma_u$. 
            \end{enumerate}
        \end{framed}
        \caption{Two committee election schemes for an $(\ell, n)$-coin-flipping protocol}
        \label{fig:leader-election-schemes}
    \end{figure}


    \begin{theorem}[Main theorem]
        Let $\kappa, \gamma, \ell, n \in \NN$ and $\epsilon \in (0,1)$.
        Let $\Pi$ be an $(\ell, n)$-coin-flipping protocol with dimension $\kappa$ 
        and output $\eta$ whose committees are elected using Scheme A.
        Then $\Pi$ is $(\EpsP, \gamma)$-secure where {\color{red} to do}.
    \end{theorem}

\subsection{Eventual consensus PoS blockchains using the longest-chain rule}
    Let $\Blockchain$ be an eventual consensus PoS blockchain protocol 
    under the longest-chain rule. 
    The protocol $\Blockchain$ advances in discrete rounds 
    which we call \emph{slots}.
    Every participant $u$ in $\Blockchain$ 
    maintains a blockchain $\Chain_u$ 
    and updates it at every slot using the following simple rule: 
    \begin{enumerate}
      \item If a longer blockchain $\Chain$ is available, 
      $u$ sets $\Chain_u \leftarrow \Chain$.

      \item If $u$ is assigned to create a block at this slot, 
      $u$ adds a new block to $\Chain_u$ and broadcasts immediately.
    \end{enumerate}
%     \paragraph{Time, slots, and synchrony}

%     \paragraph{Transaction ledger properties.} 
%     A blockchain can implement a \emph{transaction ledger} in which ``transactions'' 
%     can be divided into linearly ordered ``blocks,'' giving a partial order in the transactions. 
%     A ledger must satisfy the following properties:
%     \begin{description}[labelindent=0.5cm]
%         \item[Persistence with parameter $k \in \NN$:] A block is declared \emph{stable} 
%         if it is at least $k$ blocks deep from the end of the ledger. 
%         If an honest node (at slot $t$) reports a block $B$ at slot $b$ to be stable, 
%         all honest nodes, queried at any slot $t^\prime \geq t$, must also report $B$ to be stable at slot $b$.

%         \item[Liveness with parameters $u,k \in \NN$:] Suppose a transaction is issued at slot $t$ and 
%         all honest nodes attempt to include it in the ledger for $u$ consecutive slots. 
%         Then the transaction will be stable in the views of all honest nodes at slot $t^\prime > t+u + 1$.
%     \end{description}

%     The above properties can be derived from the followoing elementary properties of the blockchain 
%     that implements the ledger.
%     However, it would be convenient for us to lay down some notation first. 
    Consider a blockchain $\Chain$ and suppose its most recent block is issued from some slot $s \in \NN$. 
    The ``trimmed chain'' $\Chain\TrimSlot{k}$ is defined as 
    the blockchain obtained from $\Chain$ by deleting all blocks (from $\Chain$) 
    corresponding to the last $k$ slots, i.e., slots $s, s - 1, \ldots, s - k + 1$. 
    In addition, we use the expression $\Chain_1 \Prefix \Chain_2$ to mean that 
    the chain $\Chain_1$ is a prefix of chain $\Chain_2$. 
    Furthermore, given a blockchain $\Chain$ and two slots $t_1$ and $t_2 \geq t_1$, 
    $\Chain[t_1 : t_2]$ denotes the chain segment containing all blocks from $\Chain$ 
    that are issued from slots $t \in [t_1, t_2]$.

    \begin{definition}[Common Prefix property with parameter $k \in \NN$]\label{def:cp}        
        Let $\Chain_1$ and $\Chain_2$ be two blockchains adopted by two honest players 
        at the onset of rounds (i.e., slots) $r_1$ and $r_2$, respectively, with $r_1 \leq r_2$. 
        Then $\Chain_1\TrimSlot{k} \Prefix \Chain_2$. 
    \end{definition}
    We use the shorthand $\kSlotCP$ for referring to the Common Prefix property defined above. 
    Observe that we defined this property in terms of elapsed time, i.e., slots; 
    traditionally (cf. \cite{C:GarKiaLeo17}), it has been defined in terms of the number of deleted blocks. 


    \begin{definition}[Existential Chain Quality property with parameter $s \in \NN$]\label{def:ECQ}        
        Consider slots $t_1, t_2, t$ satisfying $t_1 + s \leq t_2 \leq t$. 
        Let $\Chain$ be the blockchain held by an honest party at slot $t$. 
        Then $\Chain[t_1:t_2]$ contains at least one 
        honestly gneerated block.
    \end{definition}
    We use the shorthand $\sECQ$ for referring to the ECQ property defined above. 

    \begin{definition}[Blockchain protocol with $(\EpsP, k,s)$-security]\label{def:blockchain-security}
        Let $\EpsP \in \RR$ and $k, s \in \NN$. 
        A PoS blockchain protocol is $(\epsilon, k, s)$-secure if 
        it satisfies $\kSlotCP$ and $\sECQ$ property 
        except with probability at most $\epsilon$.
    \end{definition}


% \subsection{Coin-tossing game}
%     Consider the following decentralized coin-tossing game. 
     
%     \begin{description}
%         \item[Setup:] \begin{description}
%             \item (Parameters.) Let $\eta_0 \in \{0,1\}^\kappa$ be uniformly random. 
%             Let $N = \ell + n$ and let $\Players$ be the set of players.
            
%             \item (The adversary.) A subset of the players $\Players$ are controlled by the adversary, $\Adversary$. 
%             In addition, let $T \subset \{0,1\}^\kappa$ be an arbitrary \emph{target set} chosen by $\Adversary$.

%             \item (Public roster for nonce players.) For $t \in [N]$, a player $U_t \in \Players$ is selected uniformly at random 
%             using the random string $\eta_0$. 
%             The list $\NoncePlayers = \{U_1, \ldots, U_N\}$ is the roster of the \emph{nonce players}.
            
%             \item (Nonces.) Every nonce player $U_t$ is privately given a uniformly random \emph{nonce} 
%             $x_t \subset \{0,1\}^\kappa$. 
%             However, the adversarial nonce players may disclose their nonces among themselves.

%             % \item (Options sets.) Every nonce player $U_t$ is given an \emph{option set} 
%             % $P_t \subset \{0,1\}^\kappa$ where each string $x \in P_t$ is selected uniformly at random. 
%             % Moreover, (i.) the option set for an honest player contains exactly one element and 
%             % (ii.) the option set belonging to an adversarial player is known to all adversarial players.
            
%             \item (Announced nonces.) Let $X = (X_1, \ldots, X_N)$ be a publicly visible list of \emph{announced nonces} where each $X_t \in \{0,1\}^\kappa$. 
%             Initially, set $X_t = \ZeroString$ for $t \in [N]$.
%         \end{description}
        
        
        
%         \item[Rounds:] the game progresses in rounds $t = 1, 2, \ldots, N$.
        
%         \item[Honest moves:] An honest nonce player $U_t$ sets $X_t = x_t$ at round $t$.
            
%         \item[Adversarial moves:] An adversarial player $U_t$ 
%         makes a move at round $s = \min(t+\ell, N)$ after observing 
%         the annoucned nonces $X_1, \ldots, X_s$. 
%         Then he either sets $X_t = x_t$ or does nothing; 
%         in the latter case, $X_t$ remains the all-zero string, $\ZeroString$.

%         \item[Final outcome:] The string $\eta = \eta_0 \oplus X_1 \oplus \cdots \oplus X_N$ is the outcome of the coin-tossing game. 
        
%         \item[Win:] $\Adversary$ wins if $\eta \in T$.
%     \end{description}
%     Set up: 
%     using the random string $\eta_0$.
%     The game progresses in rounds $t = 1, \ldots, N$.

% \subsection{Randomness beacon}

% \begin{definition}[
% % $\FuncBeacon$: 
% Decentralized beacon 
% % ideal functionality 
% ]
% \textbf{Parameters:} 
% $\kappa, N \in \NN$, $\gamma \in \RR_+$, and $\EpsP \in (0,1)$. 
% % $H : \{0,1\}^\kappa \rightarrow \{0,1\}^\kappa$, $\Players$.

% Let $\eta_0 \in \{0,1\}^\kappa$ be uniformly random. 
% For periods $t = 1, 2, \ldots$, a series of strings $\eta_1, \eta_2, \ldots$ are announced
% where each $\eta_t$, called \emph{the $t$th output}, is in $\{0,1\}^\kappa$. 
% In addition, the following properties are met:
% \begin{itemize}

%     \item Structure: For $n \in \NN$, 
%     let $X_n = \{x_1, \ldots, x_N\}$ 
%     where each $x_i$ is uniform in $\{0,1\}^\kappa$. 
%     Define $\eta_n = x_1 \oplus \cdots \oplus x_N$.
    
%     \item Unpredictability: For any algorithm $\Adversary$ and all $n \in \NN$, 
%     \[
%         \Pr[\Adversary = \eta_n \Given \eta_0, \ldots, \eta_{n-1}] = 2^{-\kappa}
%         \,.
%     \]
    
%     \item Small bias: For any algorithm $\Adversary$, all $n \in \NN$, and arbitrary $T \subset \{0,1\}^{\kappa}$, 
%     \[
%         \Pr[\eta_n \in T \Given \text{$\eta_{n-1}$ is uniform}] \leq \gamma |T|/2^{\kappa}
%         \,.
%     \]
    
%     \item Liveness: For all $n \in \NN$ and any algorithm $\Adversary$, there exists at least one $x_i \in X_n$ such that 
%     \[
%         \Pr[\Adversary = x_i \Given \eta_0, \ldots, \eta_{n-1}, X_n / \{x_i\}] = 2^{-\kappa}
%         \,.
%     \]
    
%     \item Public verifiability: For all $n \in \NN$, every $\eta_n$ is accompanied with a proof 
%     $\pi_n$ showing 
%     (i.) that $\eta_{n-1}$ is the correct input for $\Epoch_n$, and 
%     (ii.) that $x_i$ is indeed uniform in $\{0,1\}^\kappa$.
    
%     \item Guaranteed output delivery: The protocol always makes progress, i.e., 
%     \[
%         \Pr[\text{beacon outputs $\eta_n$} \Given \eta_{n-1}] = 1
%         \,.
%     \]
    
% \end{itemize}
% \end{definition}

% % When the above functionality is implemented by a protocol, 
% % we expect some additional properties from this protocol.
% % \begin{definition}[$\ProtocolBeacon$: Decentralized beacon protocol]
% % \textbf{Parameters:} $\Players$ are the set of eligible protocol participants.


% % \begin{itemize}
% %     \item Nonce players: At $\Epoch_n$, only a subset 
% %     $\NoncePlayers_t = \{u_1, \ldots, u_N\} \subset \Players$ 
% %     of the eligible players participate to produce $\eta_n$. 
% %     In particular, the player $u_i$ submits $x_i$. 
    

% %     \item Liveness: For all $n \in \NN$, there exists at least one $x_i \in X_n$ such that 
% %     \[
% %         \Pr[\Adversary = x_i \Given \eta_0, \ldots, \eta_{n-1}, X_n / \{x_i\}] = 2^{-\kappa}
% %         \,.
% %     \]
% % \end{itemize}

% % \end{definition}

% \paragraph{A simplified proof-of-stake blockchain.}
% A proof-of-stake (PoS) blockchain protocol $\Protocol$ is run by a set of players $\Players$. 
% To do: 
% \begin{enumerate}
%     \item Promises: eventual consensus
%     \item Adversary: rushing
%     \item Network: Synchronous. Can reduce partially-synchronous setting to a synchronous one
%     \item Mechanisms: leader election, nonce generation, longest-chain rule
%     \item Block, chain, slot, epoch
%     \item Blockchain axioms (genesis, increasing slots, single honest block per slot, 
%     total order of honest blocks)
%     \item Leader election and characteristic strings
%     \item Nonce generation by hashing nonces
% \end{enumerate}
 

% A subset of $\Players$ participate in the beacon protocol $\Beacon$. 
% Both protocols are synchronous. 
% In particular, their clocks advance in discrete units called \emph{slots} 
% and every $R$ consecutive slots make up one \emph{epoch} where the same
% $R \in NN$ is used as a protocol parameter in both protocols. 
% In addition, the two protocols are synchronoized, i.e., they start together and progress together. 

% \paragraph{The beacon inside a blockchain protocol.}
% Every epoch $\Epoch_t$ in the blockchain protocol has its own \emph{epoch randomness} $\eta_t \in \{0,1\}^\kappa$ where $\kappa \in \NN$ is a protocol parameter. 
% In particular, the events in $\Epoch_{t}$ completely determine the next epoch randomness $\eta_{t+1}$.
% The blockchain protocol has a \emph{public leader schedule}. 
% That is, at the outset of $\Epoch_t$, everyone can use the random string $\eta_t$ 
% (and other protocol-defined public information) to uniquely determine an ordered list 
% $\Leaders_t = \{u_1, \ldots, u_R\} \subset \Players^R$ associated with this epoch. 
% The player $u_i$ is designated to issue a block at the $i$th slot of this epoch.

% The players participating in the beacon protocol are called the \emph{nonce players}. 
% Let $\NoncePlayers_t \subset \Players$ be the nonce players for $\Epoch_t$.
% In traditional PoS blockchain protocols such as [Ouroboros, SnowWhite], 
% $\NoncePlayers_t$ is always equal to $\Leaders_t$. 
% When a leader issues a block, he includes a nonce in that block along with its signature. 
% The dynamics of the beacon protocol is thus inseparable from the dynamics of the blockchain protocol. 
% In particular, in a given epoch, biasing the output of the beacon is equivalent to 
% creating a fork in the blockchain. 
% An analysis of this scenario is presented in Section~\ref{sec:grinding-praos}.

% Alternatively, we can choose $\NoncePlayers_t$ independently of $\Leaders_t$ and 
% record the nonces in the blockchain via an ``input endorsement'' mechanism. 
% It turns out that this method has some attractive features compared to the traditional method.
% In what follows, we formally describe the model of this alternative setting first, 
% in Section~\ref{sec:model-xor-games}, 
% and the model for the more traditional setting is given in Section~\ref{sec:model-blockchain}.

% \subsection{Randomness beacon via off-chain nonces}\label{sec:model-xor-games}
% In this model, we expect that blockchain protocol to satisfy the following axioms.

% \begin{quote}

% \begin{axiom}[\textbf{Common Prefix (CP); with parameters $k \in \NN$ and $\EpsP \in (0, 1)$}]\label{axiom:cp}
%     At any slot $j$, the honest players agree about the blocks issued at slots $i = 1, \ldots, j - k$, except with probability $\EpsP$.
% \end{axiom}


% \begin{axiom}[\textbf{Frequent Honest Blocks (FHB); with parameters $k \in \NN$ and $\EpsP \in (0, 1)$}]\label{axiom:hcg}
%     %Suppose an honest player adopts the chain $\Chain$ at the onset of slot $j$. 
%     % Let $S_i= \{i, \ldots, i+ k - 1\}$. 
%     % Except with probability $\EpsP$, the following holds: 
%     % for every $i = 1, \ldots, j - k + 1$, 
%     % $\Chain$ contains at least one honest block from $S_i$.
    
%     Let $\Chain$ be a chain held by an honest player,  
%     $h$ be the number of honest blocks in $\Chain$, 
%     and $s_1 < s_2 < \ldots < s_h$ be the slots corresponding to the honest blocks. 
%     Then $s_{i+1} - s_{i} \leq k - 1$For every $i \in [h - 1]$
% \end{axiom}

% \begin{axiom}[\textbf{Broadcast; with parameter $b \in \NN$}]\label{axiom:broadcast}
%     Messages initiated at slot $i$ must be delivered no later than slot $i+b$.
% \end{axiom}

% \begin{axiom}[\textbf{Public Nonce Schedule (PNS); with parameter $N \in \NN$}]\label{axiom:nonce-schedule}
%     At the outset of every epoch $E$, the blockchain protocol must produce a schedule 
%     $\mathcal{N} = \left\{ (u_i, s_i) \right\}_{i=1}^N$ with the following semantics: 
%     (i.) $\mathcal{N}$ is public, 
%     (ii.) for every $i \in [N]$, the unique player $u_i$ is selected independently 
%     with probability proportional to his stake (which is determined by the blockchain protocol), and
%     (iii.) only the player $u_i$ may announce (and sign) a nonce at slot $s_i \in E$. 
% \end{axiom}

% \end{quote}

% The CP and HCG axioms are implied the three classical blockchain properties, namely 
% Common Prefix (CP), Chain Quality (CQ), and Chain Growth (CG). 
% These properties, introduced in the [GKL] paper, imply the liveness and persistence of blockchains 
% in both Proof-of-Work [GKL, PassShiShelat] and Proof-of-Stake [Ouroboros family, Snow White, Algorand] settings. 

% Keeping the CP axiom in mind, 
% the broadcast axiom says that a message originated at slot $i$ 
% reaches its recipients before it is too late, 
% i.e., before everyone has already agreed about the block issued at slot $i$. 
% Hence $b$ must be at most $k$ where $k$ is the parameter in the CP axiom.
% The broadcast axiom is satisfied in the synchronous setting with $b = 1$, 
% while in the semi-synchronous setting, 
% $b$ is an upper bound on the network delay $\Delta$ which is unknown to the protocol participants.

% % The PNS axiom helps isolate the description of the randomness beacon from the description of the rest of the blockchain protocol. 


% \subsection{Randomness beacon with on-chain nonces}\label{sec:model-blockchain}